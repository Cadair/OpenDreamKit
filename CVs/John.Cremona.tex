\begin{participant}[type=leadPI,PM=3,gender=male]{John E. Cremona}
  Professor of Mathematics.  DPhil (Oxford, 1981) under Birch.
  Previous posts: Michigan, Dartmouth (US), Exeter, and Nottingham (as
  chair and Head of Pure Mathematics). Cremona has around 50
  publications, including a book and papers in Compositio and Crelle.
  He has held grants from EPSRC and other UK sources worth \pounds2.5m
  as well as \euro2.5m from the EU for Marie-Curie Research Training
  Networks in 2000-2004 and 2006-2010.  He was a Scientist in Charge
  of one of twelve teams in both of these networks, and leader of the
  work package ``Effective Cohomology Computations'' in the second,
  responsible for several deliverables.  He has been on the Scientific
  Committee of 30 international conferences (including several \Sage
  Days), and given many invited lecture series.  He co-organised
  semester-long research programmes at IHP Paris (2004) and MSRI
  (2011).  He has been an editor for five journals including the LMS
  Journal of Computation and Mathematics and the Journal of the
  Foundations of Computational Mathematics (FoCM).  He has supervised
  16 PhD students, a dozen Masters students, two EU-funded Marie-Curie
  fellows and currently has three EPSRC-funded postdoctoral research
  assistants working for the LMFDB project.  Cremona has given over 30
  invited conference addresses and seminars in 9 countries in the last
  10 years; most recently he was a Plenary Speaker at the 2014 FoCM
  meeting in Montevideo, where he spoke about the LMFDB project to a
  wide international audience of computational mathematicians.

  Cremona's research includes areas of particular relevance to the
  current project.  His methods for systematically enumerating
  elliptic curves, which are the subject of a book and numerous
  papers, have been used to compile a definitive database of elliptic
  curves which is very widely cited, and now forms part of the LMFDB.
  Cremona's experience in managing such computations and the
  management, publication and electronic dissemination of the
  resulting large datasets set a standard which large-scale
  number-theoretical database projects such as the LMFDB now seek to
  match.  Cremona's experience and reputation in this field have been
  important for the LMFDB project.

  Cremona has been a leading computational number theorist in the UK
  since his PhD thesis in 1981, following in the tradition of Birch
  and Swinnerton-Dyer.  He has written thousands of lines of code in
  his \software{C++} library \software{eclib} (one of the standard packages included in \Sage
  since its inception) which includes his widely-used program
    \software{mwrank} for computing ranks of elliptic curves.  As well as
  writing thousands of lines of new \Python code for \Sage, he has also
  contributed to the active number-theoretical packages \PariGP and
  \Magma.
\end{participant}
%%% Local Variables:
%%% mode: latex
%%% TeX-master: "../proposal"
%%% End:
