\begin{participant}[type=leadPI,PM=12,salary=1500,gender=male]{Marcin Kostur}
is an assistant Professor at the Institute of Physics, author of  over 40
publication cited over 1000 times in the field of statistical physics,
solid state physics (Josephson Junction dynamics), microfluidics and
biophysics. He is experienced in application of GPU architecture to
numerical simulations of stochastic processed in physics. His recent
computiational interests are focused at the Open Source project
``Sailfish'' - HPC implementation of Lattice Boltzmann Method on GPU.

He is leader of two projects incorporating computations to the science education:

\begin{itemize}
\item Computing in high school science education - iCSE4schools,
  project funded by Erasmus+, Key Action 2 - Strategic Partnerships'',
  (budget: EUR 263.320, 2014-2017)
\item ``Computers in Science Education: iCSE'' http://icse.us.edu.pl
  (budget: c.a. EUR 1 milion, funded by EFS, 2011-2014 )
\end{itemize}

He is also co-Author and a task coordinator of PAAD (Platforma Analiz i
Archiwizacji Danych) founded by POIG program for 2014-2015 with a total budget
c.a. of EUR 4 million. The task ``Interactive HPC services for science''
main goal is to provide interactive interface to HPC infrastructure
(heterogenous cluster of 48 nodes, including 24 GPU and 24 Xeon Phi)
using innovative technology of ``web notebook'' interface.  




\end{participant}
%%% Local Variables:
%%% mode: latex
%%% TeX-master: "../proposal"
%%% End:
