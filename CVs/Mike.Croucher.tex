\begin{participant}[type=R, PM=6, salary=7500]
Michael Croucher is a Research Software Engineer at The University of Sheffield. He received his bachelor's degree in Theoretical Physics from The University of Sheffield in 1999 and completed his Theoretical Physics PhD there in 2005. He subsequently took up a research-support post in The University of Manchester's IT Services department before being appointed as Head of Application Software Support for the Faculty of Engineering and Physical sciences.
 
Michael is passionate about improving the quality of research software. He enables researchers to ask larger and more complex research questions by improving the software they develop. By teaching and demonstrating fundamental software engineering principles, he assists academic colleagues in producing robust, reproducible, fast and correct software.
 
He achieves these aims via a number of means:
 
Consultancy: He works directly on research code written in various languages. For a sample of recent client testimonials, see http://www.walkingrandomly.com/?page_id=5122
 
Outreach: He is the author of \url{http://www.walkingrandomly.com/} - a blog focused on mathematics and scientific computing with over 400,000 unique visitors a year. The associated twitter account, @walkingrandomly, has almost 3000 followers.  He is a fellow of the Software Sustainability Institute, an organisation that promotes and supports research software engineering.
 
Education: He has taught programming and Software Carpentry classes to hundreds of researchers and uses his contacts with education and industry to arrange specialised teaching events relating to research software.
 
Mentoring: He acts as a 'code-coach' to new researchers, providing code reviews and private tutorials.
 
Vendor liaison: He has strong relationships with several vendors of scientific software including Mathworks, Wolfram Research, Maplesoft and NAG. These relationships have led to many fruitful collaborations between them and academic colleagues.
 
High Performance Computing: He has been involved with teams that develop and support Institutional HPC systems such as the Manchester University Condor Pool - a 3000+ CPU core facility made by utilising spare time on hundreds of desktop PCs. In this team, his primary role was to assist researchers in transitioning their workflow from the desktop to the Condor system.
\end{participant}
