\paragraph{Paul-Olivier Dehaye}

% months = 6
% 
%
% Fair evaluation of the number of months you will be spending on this
% specific project along the four years

% salary=YYY
%
% Approximate monthly gross salary (in term of total cost for the
% employer). If you are uncomfortable having this information in a
% public file, you can alternatively send the information to Nicolas,
% or to your institution leader if the latter will be willing to fill
% in himself the budget forms on the eu portal.

% The above information will be used to evaluate the cost of the
% project for the institutions. You may remove the above comments once
% you have filled in the months= and salary= lines.

Paul-Olivier Dehaye is a Swiss National Science Foundation Assistant Professor 
at the University of Zurich. After his Phd at Stanford (2006), he has also worked in Oxford, 
at the Institut des Hautes Etudes Scientifiques and at ETH Zurich. He currently has 13 
papers published in international peer-reviewed journals. He is currently supervising three PhD students and one post-doc.

His main research is at the intersection of Number Theory and Combinatorics, and in 
particular in Random Matrix Theory conjectures. He has additional interests in FLOSS, 
semantic tools, massive online education and crowdsourcing, all with the view of 
enabling larger scale mathematical and scientific collaborations. He is also member of the program committee of CICM 2015 (Conference on Intelligent Computer Mathematics).

He is a contributor to the \Sage, \LMFDB and \OpenEdX projects, and has organised two 
conferences relating to these projects. The first was held in 2013 in Edinburgh, and organised
jointly with Nicolas Thiery. Its official title was  \emph{Online databases: from L-functions to combinatorics}, 
and it served as a precursor to some aspects of this grant, by bringing the \SageCombinat and \LMFDB communities together. 
The second was held in June 2014 in Zurich and organised jointly with Stanford. It aimed at building a community around the open 
source python-based MOOC platform \OpenEdX, and opened a series of conferences now held twice annually. 

Dehaye has also taught for two years now a python course using \OpenEdX, which aims to bring 
first year students to the level of potential contributor to \Sage. This course also has a 
project-based component. It is now run locally for a small audience, but could be scaled up 
in various ways. 

