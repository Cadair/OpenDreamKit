\begin{participant}[type=PI,PM=2,gender=female]{Ursula Martin}

  Professor Ursula Martin has recently joined the University of Oxford, where she holds a
  Professorship, in conjunction with a Senior EPSRC Fellowship, on a joint arrangement
  between the Department of Computer Science and the Mathematical Institute. Her current
  research concerns social and computational techniques for creating mathematics, building
  on a significant track record at the interface of mathematics and computing. Prior to
  this she worked at Queen Mary University of London, where as Vice Principal for Science
  and Engineering she led strategic change, and was active in knowledge transfer
  activities and developing young staff.

Her work is characterised by strongly interdisciplinary collaboration in new
problem domains at the interface of mathematics and computer science,
identifying novel interactions between theory and practice, with real-world
problems inspiring scientific advance. Major achievements include results
linking randomness and symmetry, new unifying explanations of the power of
computational logic, and new practical techniques for using computational logic
and algebra in industry.

The work to be undertaken in the Work Package 5 (Social Aspects) fits very well
into her current project, which concerns crowdsourced mathematics: the overarching goal is
to understand and extend the human and computer creation of mathematics. 
It is mostly funded by her
2014 EPSRC Advanced Fellowship (EPSRC awards only one or two of these annually
in Computer Science) is a partnership of industry, government and international
academia.  
\end{participant}
%%% Local Variables:
%%% mode: latex
%%% TeX-master: "../proposal"
%%% End:
