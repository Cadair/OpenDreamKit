\documentclass[12pt]{amsbook}

\usepackage{hyperref}
\usepackage[utf8]{inputenc}
\usepackage{ae,aecompl,aeguill}	% pour utiliser << et >>
\usepackage{times}
\usepackage[babel=true,kerning=true]{microtype}
\usepackage{xspace}


\newcommand{\software}[1]{\textsc{#1}\xspace}
\newcommand{\Sage}{\software{Sage}}

\title{ODK in joint H2020 e-infrastructure projects document}


\begin{document}

\maketitle

\section{One-pager document to do for 15th January}
\subsection{High level service catalogue}
Existing services and services under development


OpenDreamKit - Open Digital Research Environment Toolkit for the Advancement of Mathematics - aims at enabling research groups and communities to set up their own Virtual Research Environments (VREs) thanks to the flexible toolkit that OpenDreamKit is currently developing.
OpenDreamKit is indeed composed of mathematicians and of computer scientists who are both the developers and the potential users of this toolkit. Furthermore, as the whole project is in opensource, further communities and research groups will be able to use this toolkit according to their own needs.

The toolkit that the project is building is based on several already existing popular opensource sofwares: LinBox, MPIR, Sage(sagemath.org), GAP, PariGP,LMFDB, and
Singular.
However these softwares need to be modified in order to work together as smoothly as possible when combined to create a VRE: this is what OpenDreamKit participants will be working on until mid-2019. Component architecture, user interfaces, high performance in mathematical computing,  dissemination and community building,  interoperability between systems (Data, Knowledge, Softwares),  as well as social aspects in the development of mathematical software and knowledge, compose the research themes on which OpenDreamKit participants are working on.

VREs based on OpenDreamKit will be able to combine symbolic mathematics, automatic code generation, numerical computation, data bases, post-processing and visualisation in a single collaborative workspace. The basic units are executable documents, i.e., data- and code- driven narratives that combine live code, equations, text, interactive dashboards and other rich media. Potential applications include active scientific logbooks, papers, lecture notes, etc., covering the whole lifecycle of a mathematical research project.
This will enable step changes in effective research, research communication, and reproducibility in computational mathematics and science. It will further provide end-to-end toolchains that link fundamental mathematics to domain specialised computation, thus bridging the gap between fundamental research and technology, and paving the way towards faster application, exploitation and commercialisation of basic research.



\subsection{Project positioning with respect to:}
\begin{description}
\item[Similar activities]
\item[National and regional e-infrastructures]
The OpenDreamKit project was labelled by the Free and Open Source Software Work Group of the Systematic Cluster.

\end{description}

\section{Use cases and testimonials}
\subsection{Use cases}
Propose 1 or several case studies, which must involve services provided by 3 projects or more
\subsection{Testimonials}
Interview of real researchers or innovators, related to one or more of the use-cases


\section{Documentation paragraphs for inspiration}
\subsection{ODK summary}
Page 86 Grant agreement


OpenDreamKit will deliver a flexible toolkit enabling research groups to set up Virtual Research Environments,
customised to meet the varied needs of research projects in pure mathematics and applications and supporting
the full research life-cycle from exploration, through proof and publication, to archival and sharing of data and
cod. OpenDreamKit will be built out of a sustainable ecosystem of community-developed open software, databases,
and services, including popular tools such as LinBox, MPIR, Sage(sagemath.org), GAP, PariGP,LMFDB, and
Singular. We will extend the Jupyter Notebook environment to provide a flexible UI. By improving and unifying
existing buildingblocks, OpenDreamKit will maximise both sustainability and impact,with beneficiaries extending
to scientific computing, physics,chemistry, biology and more and including researchers, teachers, andindustrial
practitioners.We will define a novel component-based VRE architecture and the adaptexisting mathematical software,
databases, and UI components to workwell within it on varied platforms. Interfaces to standard HPC and  grid
services will be built in. Our architecture will be informed by recent research into the sociology of mathematical
collaboration, so as to properly support actual research practice. The ease of set up, adaptability and global impact
will be demonstrated in a variety of demonstrator VREs. We will ourselves study the social challenges associated
with large-scale open source code development and of publications based on executable documents, to ensure
sustainability. OpenDreamKit will be conducted by a Europe-wide demand-steered collaboration, including leading
mathematicians, computational researchers, and software developers long track record of delivering innovative open
source software solutions for their respective communities. All produced code and tools will be open source.

\subsection{ODK description}

Page 162 Grant agreement
\begin{itemize}

\item{\textbf{Mathematics in today's world}}

Improvements of the economy, ecology, health care, security and society overall are driven by innovation. The key tools for
innovation are mathematical knowledge and algorithms. Our global positioning system (GPS) needs relativistic mathematics,
our mobile phones are allocated frequencies through combinatorial optimisation, the combinatorics of our genome yields clues
to curing rare diseases, the privacy of our communications depends on cryptographic protocols steeped in number theory,
and our national security is reliant on the mathematical analysis of increasingly complex networks. Engineering, Science and
Business innovations that enrich society and mankind are made possible through mathematical foundations which are often
developed long before their potential applications. Reciprocally, modern mathematical research is increasingly accelerated
by and enabled through collaborative tools, computational environments and online databases. These digital tools have the
potential to revolutionise the way research is conducted.

\item{\textbf{Goal of ODK}}

In this project, we will provide mathematicians and scientists with a generic unified toolkit, the Open Digital Research
Environment Toolkit for the Advancement of Mathematics (OpenDreamKit), that allows building of specific Virtual Research
Environments (VREs) for particular tasks and communities.
We will achieve this by focusing on a toolkit of software components from which tailored VREs can be assembled flexibly to
cater for the diversity and evolution of needs in mathematics, science and engineering. We are at a critical point providing an
opportunity to do so: collaborative tools for code sharing (e.g. github) now allow us to bring together very large communities
of open source code developers.

\item{\textbf{Last decade opensource softwares and the evolution ODK leads}}

Simultaneously the last decade has witnessed the emergence of fundamental open source building blocks, at the forefront
of which are computational components such as the general purpose mathematical software system SAGE and the interactive
computing environment JUPYTER (successor of IPYTHON). Throughout this project we will reuse and extend open source
code, and OpenDreamKit will benefit from future open source contributions during and beyond the lifetime of the project.
By unifying tools with overlapping functionality, such as JUPYTER and SAGE with their notebooks, we focus the effort of the
computational community onto OpenDreamKit, producing additional economies of scale. Finally, thanks to the “by users for
users” model, the development will be steered by the actual needs of the community.

\item{\textbf{Composition of VRES based on ODK}}

In more detail, VREs based on OpenDreamKit can combine symbolic mathematics, automatic code generation, numerical
computation, data bases, post-processing and visualisation in a single collaborative workspace. The basic units are executable
documents, i.e., data- and code- driven narratives that combine live code, equations, text, interactive dashboards
and other rich media. Potential applications include active scientific logbooks, papers, lecture notes, etc., covering the whole
lifecycle of a mathematical research project.
This will enable step changes in effective research, research communication, and reproducibility in computational mathematics
and science. It will further provide end-to-end toolchains that link fundamental mathematics to domain specific
specialised computation, thus bridging the gap between fundamental research and technology, and paving the way towards
faster application, exploitation and commercialisation of basic research.
As part of this project, we will also study the social challenges associated with large-scale open source code development
and publications based on executable documents, and implement demonstrator VREs based on OpenDreamKit.*


\item{\textbf{Impact of ODK in differenf fields of science}}

The OpenDreamKit team is a Europe-wide collaboration that brings together a leading body of mathematicians and transdisciplinary
computational researchers, with an extensive track record of delivering innovative open source software solutions.
By focusing on a toolkit rather than a monolithic VRE, and by concentrating the efforts on improving and unifying existing
general purpose building blocks, and in the forefront JUPYTER, OpenDreamKit will simultaneously maximise sustainability
and broad impact. Indeed, though the primary target users are researchers in mathematics, the set of beneficiaries extends
to workers in scientific computing, physics, chemistry, biology, engineering, medicine, earth sciences and geography, social
sciences and finance, and includes researchers as well as teachers and practitioners in the industry. OpenDreamKit will
further foster development models that are mutually beneficial to academia and highly innovative SMEs.

\end{itemize}


\end{document}