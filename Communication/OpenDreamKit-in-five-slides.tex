\documentclass[
  usenames,svgnames, %xcolor options
  compress,
  %handout,
  ]{beamer}

\mode<presentation> {
  \usetheme{Singapore}
  \useinnertheme[shadow]{rounded}
  \usefonttheme[onlymath]{serif}
  \setbeamertemplate{navigation symbols}{\insertframenumber{}/\inserttotalframenumber}
}

\usepackage{xspace}
\newcommand{\TheProject}{\texttt{OpenDreamKit}\xspace}

\usepackage{hyperref}
\usepackage{listings} % For typesetting source code (verbatim, etc.)
\usepackage{graphicx}
\usepackage{tikz}

\usepackage{xcolor}
\definecolor{blue}{rgb}{0.2,0.2,0.7} 
\def\blue{\color{blue}}
\def\green{\color{green}}
\def\red{\color{red}}
\def\purple{\color{purple}}
\def\gray{\color{gray}}
\def\grey{\color{gray}}

% \guillemotleft and \guillemotright or just use >> or <<.
\usepackage[T1]{fontenc}
\usepackage[utf8]{inputenc}
\usepackage{xspace}

%% Sage
\newcommand{\SAGEp}{Sage}
\newcommand{\SAGE}{Sage\xspace}
\newcommand{\SAGEplus}{\SAGEp$^{++}$}
\newcommand{\Sage}{\texttt{Sage}\xspace}
\newcommand{\mupad}{\texttt{MuPAD}\xspace}
\newcommand{\maple}{\texttt{Maple}\xspace}
\newcommand{\magma}{\texttt{Magma}\xspace}
\newcommand{\python}{\texttt{Python}\xspace}
\newcommand{\cython}{\texttt{Cython}\xspace}
\newcommand{\mathematica}{\texttt{Mathematica}\xspace}
\newcommand{\starcombinat}{\texttt{$\ast$-Combinat}\xspace}
\newcommand{\mupadcombinat}{\texttt{MuPAD-Combinat}\xspace}
\newcommand{\sagecombinat}{\texttt{Sage-Combinat}\xspace}

\colorlet{structure}{red!65!black}

\title{{\huge \color{red} OpenDreamKit}\\
  Open Digital Research Environment Toolkit\\
  for the Advancement of Mathematics\\}

\subtitle{A project funded by the Horizon 2020\\
  European Research Infrastructures Work Programme}

\begin{document}
\begin{frame}
  \maketitle

  \Huge
  \centerline{\url{OpenDreamKit.org}}
\end{frame}


\begin{frame}{Context}
  \begin{block}{Emergence in the last decade(s) of a vibrant ecosystem
      of open source software for pure mathematics}

    \begin{itemize}
    \item Specialized libraries: GAP, Linbox, Pari/GP, MPIR, Singular,
      ...
    \item General purpose systems: Sage, ...
    \item Online databases: LMFDB, ...
    \item Interactive computing environments:\\
      IPython/Jupyter, SageMathCloud, ...
    \item Together with the wider Scientific Python ecosystem
    \end{itemize}
  \end{block}
  \pause
  \bigskip

  \begin{block}{Viable open source alternatives to Maple, Mathematica,
      Matlab, Magma, ...}
    \begin{itemize}
    \item Research
    \item Education (in France: lycée $\longrightarrow$ agrégation, ...)
    \item Industry?
    \end{itemize}
  \end{block}
\end{frame}


\begin{frame}
  \begin{block}{A successful development model ...}
    \begin{itemize}
    \item By users, for users
    \item Indirect funding via research grants
    \item Large international developer communities (300 for Sage)
    \end{itemize}
  \end{block}
  \pause

  \begin{block}{... with some limitations}
    \begin{itemize}
    \item Some highly technical tasks are lagging behind:
      \begin{itemize}
      \item Hard to justify work on them for a researcher
      \item Hard to justify work on them on a research grant
      \end{itemize}
    \item Impeding the wide adoption of those systems
    \end{itemize}
  \end{block}
  \pause

  \begin{block}{A need for funding for:}
    \begin{itemize}
    \item A few full time developers
    \item Training and development workshops, developer meetings, ...
    \end{itemize}
  \end{block}
\end{frame}

\begin{frame}{OpenDreamKit (2015-2019)}
  \begin{block}{Open Digital Research Environment Toolkit\\
    for the Advancement of Mathematics}
    \begin{itemize}
    \item H2020 European Research Infrastructures Work
      Programme\\
      Call: Virtual Research Environments
    %\item Submitted: January 2015, Accepted: May 2015
    \item Budget: 7.6 million euros
    \item 15 sites
    \item 50 participants
    \end{itemize}
  \end{block}
\end{frame}
% \begin{block}{50 participants on 15 sites}
%     \begin{itemize}
%     \item France: Paris Sud, Versailles Saint-Quentin, Bordeaux
%       (CNRS), Grenoble, Logilab
%     \item Germany: Kaiserslautern, Bremen
%     \item Great Britain: Oxford, Southampton, Sheffield, St Andrews, Warwick
%     \item Norway: Simula (Oslo)
%     \item Poland: Silesia
%     \item Switzerland: U Zürich
%     \end{itemize}
%   \end{block}
% \end{frame}

\begin{frame}{OpenDreamKit (2015-2019)}
  \begin{block}{Aims}
    \begin{itemize}
    \item Foster the ecosystem of open source software for pure
      mathematics and beyond
    \item Deliver a flexible Virtual Research Environment toolkit
      supporting collaborative work of soft, data, and knowledge
    \end{itemize}
  \end{block}

  \begin{block}{Main tasks}
    \begin{itemize}
    \item Modularization and interfaces between systems
    \item Build, documentation, tests systems
    \item Portability, distribution, deployment
    \item High Performance
    \item Interactive collaborative computing environments
    \item Mathematical databases
    \item % TODO: expand More semantic in computation, data, ..
    \item Research on social aspects of math soft development
    \item Community building and training
    \end{itemize}
  \end{block}
\end{frame}

\renewcommand{\logo}[1]{\raisebox{-.8ex}{\includegraphics[height=3ex]{logos/#1}}}

\begin{frame}
  \begin{block}{The \TheProject partners}
    \begin{itemize}
    \item France: Bordeaux \logo{LaBRI.png} CNRS, Grenoble (UJF),
      Paris \logo{UPSud.png} Logilab, Versailles Saint-Quentin
      \logo{uvsq.jpg}
    \item Germany: Bremen \logo{jacobs-logo.png}, Kaiserslautern
    \item Great Britain: Oxford \logo{ox_brand_cmyk_pos.pdf},
      Sheffield, Southampton \logo{University-of-Southampton.png}, St
      Andrews \logo{USTAN_logo.png}, Warwick
    \item Norway: Oslo \logo{simula.png}
    \item Poland: University of Silesia \logo{US}
    \item Switzerland: Zürich Universität
    \end{itemize}

    In close collaboration with the international community!
  \end{block}
  
  \begin{block}{\color{red} We recruit full time developers!}
    \centerline{\large \url{http://OpenDreamKit.org/joinus}}
  \end{block}

\end{frame}


\end{document}

