\begin{sitedescription}{UB}
% PIC: 
% see: http://ec.europa.eu/research/participants/portal/desktop/en/orga

% See ../proposal.tex, section Members of the Consortium for a
% complete description of what should go there

\begin{itemize}
\item INRIA, LaBRI, IMB
\item journal de th\'eorie des nombres
\item Plafrim and a mesocentre Avakas
\item Several softwares developped in Bordeaux: pari/GP, tulip, etc
\end{itemize}


\subsubsection*{Curriculum vitae}

% Curriculum of the personnel at this institution

\begin{participant}[type=leadPI,PM=12,salary=4700,gender=male]{Vincent Delecroix}
CNRS researcher at the LaBRI (Bordeaux, France) since october 2013, Vincent
Delecroix is a junior researcher in Dynamical Systems with strong links with
Combinatorics and Number Theory. He published 7 articles in international
journals with several collaborators around the world (England, Mexico,
United-States).

Since 2010 he is an important contributor to \Sage with 30 tickets authored and
around 50 reviewed. He organised several Sage days and Sage workshops in
Bordeaux, Marseille, Orsay, Perpignan, Bobo Dioulasso (Burkina Faso),
Saint-Louis (S\'en\'egal).
\end{participant}
%%% Local Variables:
%%% mode: latex
%%% TeX-master: "../proposal"
%%% End:


%\begin{participant}[type=R,PM=6,gender=male]{Adrien Boussicault}
  Maître de Conférences at the LaBRI (Laboratoire Bordealais de Recherche en 
  informatique), Adrien Boussicault is a young researcher in Algebraic and 
  Enumerative Combinatorics. He has 8 papers in international journals. 
  His contributions to \Sage include writing 3 tickets to implement 
  combinatorial objects and co-organising \SageCombinat Days 57.
\end{participant}
%%% Local Variables:
%%% mode: latex
%%% TeX-master: "../proposal"
%%% End:


\begin{participant}[PM=12]{Karim Belabas}
  Karim is one of the main pari developers.
\end{participant}
%%% Local Variables:
%%% mode: latex
%%% TeX-master: "../proposal"
%%% End:

\paragraph{Bill Allombert}

% months=6
% salary=YYY

CNRS Ing\'enieur de Recherche. One of the main pari developer.


%?? others ??


\subsubsection*{Publications, products, achievements}

Some recent Publications :
\begin{enumerate}
\item 
Belabas, Karim; Friedman, Eduardo; Computing the residue of the Dedekind
zeta function.  Math. Comp. 84 (2015), no. 291, 357–369. 

\item
The PARI Group; PARI/GP version 2.7.0, Bordeaux, 2014,
http://pari.math.u-bordeaux.fr/.

\item
Belabas, Karim et al. Explicit methods in number theory. Rational points and
Diophantine equations, 179 pages, Panoramas et Synthèses 36, 179p., 2012.

\item
Allombert, Bill; Bilu, Yuri and Pizarro-Madariaga, Amalia;) CM-Points on
Straight Lines , to appear in "Analytic Number Theory" (dedicated do H. Maier),
Springer. 
\end{enumerate}



\subsubsection*{Previous projects or activities}

Current grants:
\begin{enumerate}
\item
 ANR PEACE (2012-2015)
    Goal: The discrete logarithm problem on algebraic curves is one of the rare
    contact points between deep theoretical questions in arithmetic geometry and
    every day applications. On the one side it involves a better understanding,
    from an effective point of view, of moduli space of curves, of abelian
    varieties, the maps that link these spaces and the objects they classify.
    On the other side, new and efficient algorithms to compute the discrete
    logarithm problem would have dramatic consequences on the security and
    efficiency of already deployed cryptographic devices. 

\item
ERC starting grant ANTICS (2011-2016) 
    Goal: "Rebuild algorithmic number theory on the firm grounds of theoretical
    computer science".
    Challenges: complexity (how fast can an algorithm be?), reliability
    (how correct should an algorithm be?), parallelisation.
\end{enumerate}

\subsubsection*{Significant infrastructure}
\TOWRITE{VD}{this still needs to be done}

Two center for computations: Plafrim and Avakas.
\end{sitedescription}

%%% Local Variables:
%%% mode: latex
%%% TeX-master: "../proposal"
%%% End:

%  LocalWords:  sitedescription th eorie des nombres Plafrim mesocentre Avakas developped
%  LocalWords:  subsubsection Belabas Synthèses Allombert Bilu Pizarro-Madariaga Maier
%  LocalWords:  parallelisation TOWRITE
