\begin{sitedescription}{JU}
Jacobs University is a private Anglo-Saxon style research university.  It opened in 2001
and has an international student body (1320 students from 115 nations as of 2011).  The
KWARC (KnoWledge Adaptation and Reasoning for Content~\cite{KWARC:online}) Group headed by
{\emph{Prof.\ Dr.\ Michael Kohlhase}} specializes in knowledge management for STEM.
Formal logic, natural language semantics, and semantic web technology provide the
foundations for the research of the group.

\subsubsection*{Curriculum vitae}
% Curriculum of the personnel at this institution
\begin{participant}[type=leadPI,PM=6,gender=male]{Michael Kohlhase}
  Dr. Michael Kohlhase is full professor for Computer Science at Jacobs University Bremen
  and an associate adjunct professor at Carnegie Mellon University.

  He studied pure mathematics at the Universities of T\"ubingen and Bonn (1983 - 1989) and
  continued with computer science, in particular higher-order unification and automated
  theorem proving (Ph.D. 1994, Saarland University).

  His current research interests include knowledge representation for mathematics,
  inference-based techniques for natural language processing, and computer-supported
  education. He has pursued these interests during extended visits to Carnegie Mellon
  University, SRI International, and the Universities of Amsterdam, Edinburgh, and
  Auckland.

  Michael Kohlhase is recipient of the dissertation award of the Association of German
  Artificial Intelligence Institutes (AKI; 1995) and of a Heisenberg stipend of the German
  Research Council (DFG 2000-2003). He was a member of the Special Research Action 378
  (Resource-Adaptive Cognitive Processes), leading projects on both automated theorem
  proving and computational linguistics. Michael Kohlhase was trustee of the Conference on
  Automated Deduction (CADE), Mathematical Knowledge Management (MKM), and the CALCULEMUS
  conference, he is a member of the W3C Math working group, president of the OpenMath
  Society, and the general secretary of the Conference on Intelligence Computer
  Mathematics (CICM).
\end{participant}

%%% Local Variables:
%%% mode: latex
%%% TeX-master: "../proposal"
%%% End:


 



\begin{participant}[PM=6,type=PI]{Florian Rabe}
  is a post-doctoral research fellow at Jacobs University Bremen.  He completed his PhD in
  2008 and his habilitation in 2014 and holds the venia legendi.

  He has worked on the formal representation and management of mathematical knowledge for
  10 years.  He was a lead researcher in the LATIN project (2009-2012), which produced a
  highly modular and integrated library of formal languages for knowledge representation.
  He is currently a principal investigator in the OAF project, which builds on LATIN to
  produce an archive of libraries of formal mathematical knowledge.

  He is the creator and main developer of the MMT language and system, which are the
  backbone of both LATIN and OAF.  MMT has been developed for 8 years with contributions
  from $>10$ people and currently consists of $>30,000$ lines of Scala code.

  He served in the organization committee of 2 and the program committee of 6
  international conferences (2 as track chair) on intelligent computer mathematics, and
  has organized 4 international workshops on module systems and libraries for mathematical
  knowledge.  He has authored 65 research papers (11 in international journals) and has
  supervised 17 undergraduate and graduate theses.
\end{participant}

%%% Local Variables:
%%% mode: latex
%%% TeX-master: "../proposal"
%%% End:

\begin{participant}[PM=36]{Christian Maeder}
  Dr. Christian Maeder is a research developer at Jacobs University. He has extensive
  experience in designing and implementing logic-based software system. He is the lead
  implementor of the HETS system.
\end{participant}

%%% Local Variables:
%%% mode: latex
%%% TeX-master: "../proposal"
%%% End:


 



\begin{participant}[type=JRes,PM=24,gender=male]{Mihnea Iancu}
  Mihnea Iancu is a third-year doctoral student at the KWARC group. He is the lead
  implementor of the \software{MathHub.info} system. He has worked extensively on the representation
  for formal and informal mathematical knowledge in the \software{MMT} system.
\end{participant}

%%% Local Variables:
%%% mode: latex
%%% TeX-master: "../proposal"
%%% End:


 




\paragraph{Relevant previous experience:}

The KWARC group is the lead implementor of the OMDoc (Open Mathematical Document) format
for representing mathematical knowledge \cite{Kohlhase:OMDoc1.2} and redeveloped its
formal core in the OMDoc/MMT format~\cite{RabKoh:WSMSML13}. The latter has been
implemented in the MMT system~\cite{MMTSVN:on,RabKoh:WSMSML13} which provides efficient
implementations of the computational primitives such as type checking, flattening, and
presentation at a logic/foundation-independent level.  The group has developed services
powered by such semantically rich representations, different paths to obtaining them, as
well as platforms that integrate both aspects.  \emph{Services} include the adaptive
context-sensitive presentation framework provided by the MMT API and the semantic search
engine MathWebSearch\cite{KohSuc:asemf06,ProKoh:mwssofse12}. 

Semantic services can be integrated into the documents generated from OMDoc/MMT
representations, making them into ``active documents'', i.e. documents that are
interactive and adaptive to the user and situation.  For \emph{obtaining} rich content,
the group investigates assisted manual editing \cite{JucKoh:sidesc10:biblatex} as well as
automatic annotation using linguistic techniques \cite{GinJucAnc:alsaacl09}.  Finally,
KWARC has developed the \textsf{MathHub.info} portal a community-based library and
knowledge management system for flexiformal libraries, which can be used for semantic
publishing and eLearning~ \cite{KohDavGin:psewads11,MathHub:on,IanJucKoh:sdm14}.

The \textsf{OMDoc/MMT} knowledge representation format and the \textsf{MathHub.info}
system will an important basis for the developments Work Packages 4 and 6.

Michael Kohlhase has initiated and led the CALCULEMUS! IHP-Research and Training Network
and participated in the FP6 IST MoWGLI (Mathematics on the Web: Get it by Logic and
Interfaces) project, the FP6 CSA Once-CS (Open Network of Centres of Excellence in Complex
Systems), The FP7 EDC project WebALT (Web Advanced Learning Technologies).

\paragraph{Specific expertise:}
\begin{itemize}
\item Modeling formal structures of mathematical knowledge in a web-scalable way
\item Transforming large collections of legacy scientific publications to semantically
  structured markup
\item Designing user interfaces for authoring and interacting with mathematical knowledge
\end{itemize}
\end{sitedescription}


\begin{draft}
\vspace{1cm}\TOWRITE{MK}{Complete check list below -- delete completed items if you wish}

\begin{verbatim}
- [x] checked that sum of person months put into finance request is
  the same as sum of person months associated with the Work Packages
  (in proposal.tex, as defined as part of the \begin{workpackage}"
  command.
  
  Take into account person months associated with work package 1, time
  of all staff to be hired and work on the project (including
  investigators). Figure 5 helps with a quick check of the sums over
  different work packages.

- [x] completed site specific resource summary in resources.tex,
  including table of non-staff costs. This is compulsory (EU
  regulations) if the non-staff cost exceed 15% of the total cost, and
  is likely to be the case for most of the partners. We ask everybody
  to do it, to be consistent and show transparently how we have
  planned our total budget.

- [x] Have all our tasks a designated lead institution? Check in the
  Work Packages that all the tasks you are involved in have a
  dedicated lead party. If the lead party is "USO", then use:
  \begin{task}[lead=USO]

- [x] Have all our deliverables a designated lead institution [using
  the 'lead=' key]?

- [ ] In the "Members of the consortium section", have we addressed "a
  description of the legal entity and its main tasks, with an
  explanation of how its profile matches the tasks in the
  proposal"? See Entry for Paris Sud and Southampton as examples.

- [x] In the Members of the consortium section, have we given
  descriptions of all the people we intend to hire (even if we don't
  know who that is yet). 
  
- [ ] Do all our tasks include us in the list of sites involved?
\end{verbatim}
\end{draft}

%KEY-MORE-TODOS


%%% Local Variables: 
%%% mode: latex
%%% TeX-master: "../proposal"
%%% End: 

% LocalWords:  site-jacu.tex clange sitedescription emph compactitem pn semmath RabKoh
% LocalWords:  prosuming-flexiformal KohSuc asemf06 GinJucAnc alsaacl09 StaKoh ProKoh
% LocalWords:  tlcspx10 KohDavGin psewads11 ednote Radboud Bia ystok CALCULEMUS textsf
% LocalWords:  textbf keypubs OntoLangMathSemWeb uwb Deyan Ginev Stamerjohanns mwssofse12
% LocalWords:  searchability krasm Iacob KohKoh skmfe08 HilKohSta copmem06 ako IanJucKoh
%  LocalWords:  subsubsection sdm14 Centres
