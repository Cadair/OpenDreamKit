\begin{sitedescription}{UJF}

% PIC: UJF == 969011959
% see: http://ec.europa.eu/research/participants/portal/desktop/en/orga

% See ../proposal.tex, section Members of the Consortium for a
% complete description of what should go there

%\TOWRITE{JGD/CP}{Description of UJF}
The Université de Grenoble  partner gathers two teams form the Laboratoire Jean Kuntzmann
(CASYS team, with Jean-Guillaume Dumas and Laurent Fousse) and the Laboratoire d’Infor-
matique de Grenoble (MOAIS, with Thierry Gautier, Clément Pernet and Jean-Louis Roch). The
CASYS team is specialized in Algebraic computations, cryptology, codes and hybrid symbolic-
numeric dynamical systems. The MOAIS team is specialized in programming and scheduling
design on distributed resources for applications based on interactive simulation. The software
developed by this partner is significant. It includes K AAPI , L IN B OX and also G IVARO (C++
library for arithmetic and algebraic computations), PAC (Parallel Algebraic Computations),
Galet (Matrix multiplication schedule generator), FFSpMV (sparse matrix-vector product over
finite fields), PaloAlto (cryptology on curves and cryptanalysis toolbox), Taktuk (tool for de-
ployment of parallel remote executions of commands to a large set of remote nodes), etc.
\subsubsection*{Curriculum vitae}

% Curriculum of the personnel at this institution

\begin{participant}[type=PI,PM=6,salary=6500,gender=male]{Jean-Guillaume Dumas}
  Professor at the Laboratoire Jean Kuntzmann, Jean-Guillaume Dumas is a senior
  researcher in Computer Algebra with 40 papers published in international
  journals or refereed international conferences.  Among other things, he is
  vice-president of ACM Special interest group on symbolic and algebraic
  manipulations (SIGSAM), department chair within his Laboratoire (6 research
  teams, 130 members) and has collaborators in USA, Canada, Ireland, Germany and
  Luxembourg; he has also co-organized fifteen international conferences.

  Computer Algebra is a field at the frontier between mathematics and computer
  science, with heavy needs for computer exploration.  Jean-Guillaume Dumas is
  the main developer of the LinBox and Givaro C++ libraries (libgivaro1,
  libgivaro-dev, libgivaro-doc, liblinbox0, liblinbox-dev in Debian) used, e.g.,
  by Sage respectively as its exact linear algebra and its finite fields.

  Along the way, he coauthored part of the proposal for NSF-INRIA grant QOLAPS
  on Quantfier elimination, Optimization, Linear Algebra and Polynomial Systems
  and he is the director of the French ANR program on High-Performance Algebraic
  Computations.
\end{participant}

%%%%%%%%%%%%%%%%%%%%%%%%%%%%%%%%%%%%%%%%%%%%%%%%%%%%%%%%%%%%
%%% Local Variables:
%%% mode: latex
%%% mode: flyspell
%%% ispell-local-dictionary: "american"
%%% TeX-master: "../proposal"
%%% fill-column: 80
%%% End:

\begin{participant}[PM=12,salary=4500]{Cl\'ement Pernet}
  Associate Professor at the joint Inria-LIG research group MOAIS, Cl\'ement Pernet is a
  junior researcher in Computer Algebra, parallel computing and coding theory with 16
  papers published in international journals or refereed international conferences. He is
  associate editor of the ACM transactions on Mathematical Software and has co-organized
  10 conferences, including 2 sage-days and the 2012 edition of ISSAC, the leading
  conference in computer algebra.

  Since he was a post-doc at University of Washington, under the supervision of William
  Stein, head of the Sage project, he has had many contributions to Sage on the exact
  linear algebra and the symbolic computation tools. He co-authored the book ``Calcul
  Mathématique avec Sage'' with the chapter on Linear algebra.  Cl\'ement Pernet is the
  founder and lead developper of the fflas-ffpack library, kernel for dense linear algebra
  over a finite field, delivering high performance computation to LinBox and Sage. He is a
  core contributor to the LinBox library and contributed to the m4ri library.

% \begin{description}
%   \item[Personal Data]\ 

% \begin{tabular}{ll}
% Gender:& male \\
% Nationality:  & French  \\
% Address:        & LIP-AriC, ENS de Lyon\\
%                 & 46, All\'ee d'Italie\\
%                 & F-69364 Lyon CEDEX 07\\
%                 & France \\
% Phone:          & +33 437 28 74 75 \\
% Email:          & clement.pernet@imag.fr \\
% URL:            & http://lig-membres.imag.fr/pernet\\ \\
% Status:         & Associate Professor\\
% \end{tabular}

% \item[Scientific Qualification]\ 

% \begin{tabular}{ll}
% PhD 2006& Universit\'e J. Fourier (Grenoble 1) \\
% Habilitation 2014& Universit\'e J. Fourier (Grenoble 1) \\
% \end{tabular}

% \item[Academic Career]\ 

% \begin{tabular}{ll}
%  2008 & Postdoc, University of Washington \\
%  2009 -- present& Associate Professor, Universit\'e J. Fourier (Grenoble 1)\\
%  2013 -- 2014 & CNRS Research leave at LIP, \'ENS de Lyon\\
%  2014 -- 2015 & Inria Research leave at LIP, \'ENS de Lyon\\
% \end{tabular}

% \item[Scientific Service]\ 

% \begin{tabular}{ll}
%  2008 & Pacific Institute for Math. Sciences (PIMS) fellowship award\\
%  2010--2011 & Coordinator of the CNRS PEPS grant \textit{Parallel Computer Algebra}\\
%  2012-2014 & Member of the Inria associate team QOLAPS (with NCSU, USA) \\
%  2012--2015 & Member (70\%) of the ANR HPAC (High Performance Algebraic
%  Computing) project\\ 
% \end{tabular}





% \item[Selected Mathematical Software]\ 

% \begin{tabular}{ll}
%  1997--present & Coauthor of {\sc{Singular}} libraries for adjoint ideals, absolute factorization, \\
%                           & integral bases, invariant theory, parametrization of rational curves, \\
%                           & primary decomposition, normalization, and sheaf cohomology\\

%  2009--present& Head of the {\sc{Singular}} developers group\\
%\end{tabular}
%\end{description}
\end{participant}
%%% Local Variables:
%%% mode: latex
%%% TeX-master: "../proposal"
%%% End:

\begin{participant}[type=R,PM=12,gender=male]{Pierrick Brunet}
  Junior Research and Development Engineer at INRIA Grenoble, Pierrick Brunet is working
  on compilation of C/C++ OpenMP program to C/C++ programs with calls to specific OpenMP
  runtimes.

  With about 25\% of commits in the Pythran~[\ref{pythran-descr}] project, Pierrick is one of the core devs of
  this project which compile a subset of the Python language to native Python modules.

  \TOWRITE{Pierrick Brunet}{add ref to where Pythran is described}
\end{participant}
%%% Local Variables:
%%% mode: latex
%%% TeX-master: "../proposal"
%%% End:

%\input{CVs/First.Last.tex}

\subsubsection*{Publications, products, achievements}

% \begin{enumerate}
% \item Coauthoring of the open source book ``Calcul Mathématique avec
%   Sage'', the first of its kind comprehensive introduction to
%   computational mathematics in Sage for education.
% \end{enumerate}
\begin{description}
\item[Software projects]\
  \begin{description}
 \item[\texttt{fflas-ffpack}:] An open-source \texttt{C++} library offering dense
    linear algebra kernels over a finite field. In the same  spirit as the
    numerical \texttt{BLAS} (Basic Linear Algebra Subroutines), and
    \texttt{LAPACK} libraries, it delivers high performance for the most
    commonly used routines of scientific computing: matrix multiplication,
    solving linear systems, computing echelon forms, determinants,
    characteristic polynomials, etc. This library has set the standard
    approach for high performance exact dense linear algebra. It is currently
    used in \texttt{Sage}, and has inspired the design of similar routines in
    most commercial computer algebra softwares: \texttt{maple}, \texttt{magma}, etc.
  \item[\texttt{LinBox}:] An open-source \texttt{C++} middleware library for
    exact linear algebra. It uses \texttt{fflas-ffpack} for its dense finite
    field linear algebra part and extends its functionalities to other
    computation domains (integers, rationals, polynomial rings) and type
    matrices (sparse and structures matrices, black-box
    matrices). \texttt{LinBox} is integrated in \texttt{Sage}. 
  \item[\texttt{Pythran}:] An open-source \Python-to-\texttt{C++} optimizing compiler
    offering an high performance runtime for Scientific Python kernels. Dynamicity of
    the \Python language is not compliant with static compilation. That's why only
    a subset of the \Python language is supported by \Pythran. Thanks to these
    restrictions, \Pythran generate code up to 3000 faster than original module.
  \end{description}
\item[Selected Publications]\ 
\medskip\noindent
\begin{enumerate}[1.]
\item Coauthoring of the open source book ``Calcul Mathématique avec
  Sage'', the first of its kind comprehensive introduction to
  computational mathematics in Sage for education.

\item Parallel computation of echelon forms (with J-G. Dumas, T. Gautier and Z. Sultan). 
\emph{In Proc. Euro-Par'14}  (2014),  LNCS 499--510. DOI: 10.1007/978-3-319-09873-9\_42.

\item Pythran: Enabling static optimization of scientific python programs
  (Serge Guelton, Pierrick Brunet, Alan Raynaud, Adrien Merlini, and Mehdi Amini.)
\emph{Proceedings of the Python for Scientific Computing Conference (SciPy)} June 2013.

\item Fast Computation of Hermite Normal forms of random integer matrices (with
W. Stein).
\emph{J. of Number Theory} {\bf{130.7}} (2010), 1675--16833. DOI: 10.1016/j.jnt.2010.01.017

\item Dense Linear Algebra over Word-size Prime Fields (with J.-G. Dumas and P. Giorgi). 
\emph{Trans. on Math. Software} {\bf{35.3}} (2008), 1--42. DOI: 10.1145/1391989.1391992.

\item Faster Computation of the Characteristic Polynomial (with A. Storjohann). 
\emph{In Proc. ISSAC'07}  (2007), 307--314. DOI: 10.1145/1277548.
\end{enumerate}

\end{description}

\subsubsection*{Previous projects or activities}

\begin{enumerate}
\item Direction of the ANR program on High-Performance Algebraic
  Computations 2012-2015.
\item Participation to the NSF-Inria associate teams QOLAPS (with NCSU, USA)
\item Coordination of a CNRS PEPS grant (parallel computer algebra)
\item Organization of the ISSAC'12 conference, the main
  international conference in computer algebra, and of PASCO'15 a satelitte
  conference on parallel computer algebra.
\end{enumerate}

\subsubsection*{Significant infrastructure}

\TOWRITE{JGD/CP}{Significant infrastructure in Grenoble (or remove section)}
\end{sitedescription}
%%% Local Variables: 
%%% TeX-master: "../proposal.tex"
%%% End: 
