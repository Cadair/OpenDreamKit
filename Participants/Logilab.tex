\begin{sitedescription}{LL}

Logilab (\url{http://www.logilab.fr/}) is a french SME focused on using the web and free software to help
scientists. It has been in business since 2000 and counts over 20 engineers and
PhDs proficient in software engineering, knowledge representation, design and
management of IT infrastructure, etc.

Logilab invests 15\% of its turnover in research and development and has been
part of several R\&D projects at the national and european levels, always to
provide technical expertise and support to the other partners.

In the context of this project, Logilab will innovate to support the partners
with tools and infrastructure, including open databases to flexibly store
mathematical objects, user interfaces to visualize complex mathematical
properties, fluid workflow tools to ease large-scale collaboration, etc.

%Logilab's PIC number is 948455525.

% See ../proposal.tex, section Members of the Consortium for a
% complete description of what should go there

\subsubsection*{Curriculum vitae}

% Curriculum of the personnel at this institution

\TOWRITE{Logilab}{Check the PM and roles for Florent, Olivier, Julien,
and the participant below; it currently sums up to 46 when the
resources.tex file claims 36.}

\begin{participant}[type=PI,PM=2]{Florent Cayré}
  Engineer with a Master Degree from École Centrale de Paris (top
  French engineering school), Florent Cayré spent six years in SNECMA
  as an engineer conceiving the numeric tools for the turbines design
  and then was head of the group ``Méthodes de conception de turbines
  ; aérothermique et combustion'': collaborative R\&D programs,
  development and integration of various numeric tools (Python, C++,
  C, Fortran). He co-funded the SecondWeb company: development of
  complex Web applications based on CubicWeb platform.

  Head of the science computing department of Logilab since 2012,x
  Florent is responsible for team management, strategic vision,
  projects monitoring, technical expertise, etc. He developed several
  tools for defining and managing computations, and producing enhanced
  result reports through IPython notebooks.
\end{participant}

\begin{participant}{Olivier Cayrol}
Engineer Master Degree from École Centrale de Lyon (top French
engineering school). 3 years at the R&D department of PSA-Peugeot
Citroën, as developer and project manager on the modelling and
simulation of electronic embedded car control devices.

Co-founder and deputy-CEO of Logilab. Design and development of the
system that generates the documents of Logilab from ReST data
sources. This system is based on several free softwares such as Sphinx
or reportlab and defines numerous extensions for answering the specifc
needs.
\end{participant}

\begin{participant}[type=PI,PM=18,gender=male]{Julien Cristau}
  holds Master Degree and PhD from University Paris VII, where he carried out
  mathematical research in automata and linear games. As a \software{Debian}
  developer since 2007, he maintained its key components such as the
  \software{X11} windowing system and acted as a \software{Debian} Release
  Manager since 2011.

  Working as a software engineer in the R\&D department of Logilab since
  2011, he developed software using many different languages and systems,
  helped to release and distribute software on many different platforms,
  maintained parts of the infrastructure and trained other people.
\end{participant}

\begin{participant}[type=R,PM=12,gender=male]{Serge Guelton}

    Serge Guelton holds an egineering degree in Computer Science and
    telecomunication from from Télécom Bretagne and a PhD in compilation and
    parallelism. He's been working as an expert engineering in various INRIA
    teams, and as a lead developer in several strat-ups.

    He's the lead developer of the Pythran project, a Python-to-C++ compiler
    for high-performance scientific kernels.

\end{participant}
%%% Local Variables:
%%% mode: latex
%%% TeX-master: "../proposal"
%%% End:


\begin{participant}[type=res,PM=30]{NN}
  \TOWRITE{Logilab}{Logilab will hire/allocate a full time developer for working on ...}
\end{participant}

\subsubsection*{Publications, products, achievements}

\begin{enumerate}
 \item CubicWeb is a semantic web framework that is available under the LGPL
   license and received a DataConnexion prize from Etalab (the french government
   team dedicated to Open Data)

\item Logilab has been contributing to free software since its creation in 2000
  and is known for it in France and several other countries. It authored PyLint,
  the static Python code checker used worldwide, and has always had at least one
  Debian Developer on staff, thus supporting the largest free software
  distribution used by millions of people.

 \item At OBHM 2013, the 19th Annual Meeting of the Organization for Human Brain
   Mapping, Logilab presented a poster which explains the work done using
   CubicWeb on brain imaging and genetics data in collaboration with INRIA,
   INSERM and the CEA during the Brainomics project co-financed by Agence
   nationale de la Rercherche.

\end{enumerate}

\subsubsection*{Previous projects or activities}

\begin{enumerate}
\item Logilab was a member of the consortium for the ASWAD EU project, with a
  role of software developer. The project demonstrated the interest of free
  software for setting up workflows in public administrations.
\item Logilab was a member of the consortium for the KIDDANET EU project, with a
  role of software developer. The project implemented a proxy to protect kids
  browsing the internet.
\item Logilab was a member of the consortium for the PYPY EU project, with a
  role of software developer. The project implemented a Python interpreter in
  Python, to explore new ways to compile and optimize the execution of Python
  code.
\item Logilab was a member of the consortium for the OpenHPC french project,
  with a role of software developer. The project advanced the state of the art
  of free software for high performance simulation.
\item Logilab was a member of the consortium for the BRAINOMICS french ANR
  project, with a role of software developer. The project advanced the state of
  the art of shared databases for the brain imaging and genetics data.
\item Logilab was a member of the consortium for the THALER french ANR project,
  with a role of software developer. The project advanced the state of the art
  of free software for high performance simulation of molecular dynamics.
\end{enumerate}

\subsubsection*{Significant infrastructure}

Logilab is maintaining its own infrastructure, using virtualisation techniques
and tools such as OpenStack, SaltStack, Docker, etc.

\end{sitedescription}



\begin{draft}
\TOWRITE{NK}{Who is the lead from Logilab? (Enter below in TOWRITE)}\\
\vspace{1cm}\TOWRITE{Logilab}{Complete check list below -- delete completed items if you wish}

\begin{verbatim}
- [ ] checked that sum of person months put into finance request is
  the same as sum of person months associated with the Work Packages
  (in proposal.tex, as defined as part of the \begin{workpackage}"
  command.

  Take into account person months associated with work package 1, time
  of all staff to be hired and work on the project (including
  investigators). Figure 5 helps with a quick check of the sums over
  different work packages.

- [ ] completed site specific resource summary in resources.tex,
  including table of non-staff costs. This is compulsory (EU
  regulations) if the non-staff cost exceed 15% of the total cost, and
  is likely to be the case for most of the partners. We ask everybody
  to do it, to be consistent and show transparently how we have
  planned our total budget.

- [ ] Have all our tasks a designated lead institution? Check in the
  Work Packages that all the tasks you are involved in have a
  dedicated lead party. If the lead party is "USO", then use:
  \begin{task}[lead=USO]

- [ ] Have all our deliverables a designated lead institution [using
  the 'lead=' key]?

- [ ] In the "Members of the consortium section", have we addressed "a
  description of the legal entity and its main tasks, with an
  explanation of how its profile matches the tasks in the
  proposal"? See Entry for Paris Sud and Southampton as examples.

- [ ] In the Members of the consortium section, have we given
  descriptions of all the people we intend to hire (even if we don't
  know who that is yet).
  
- [ ] Do all our tasks include us in the list of sites involved?
\end{verbatim}
\end{draft}

%KEY-MORE-TODOS


%%% Local Variables:
%%% mode: latex
%%% TeX-master: "../proposal"
%%% End:

%  LocalWords:  sitedescription Logilab TOWRITE subsubsection publi Brainomics KIDDANET
%  LocalWords:  fr
