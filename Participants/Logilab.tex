\begin{sitedescription}{LL}

Logilab (\url{http://www.logilab.fr/}) is a french SME focused on using the web and free software to help
scientists. It has been in business since 2000 and employs over 20 engineers and
PhDs proficient in software engineering, knowledge representation, design and
management of IT infrastructure, and other areas.

Logilab invests 15\% of its turnover in research and development and has been
part of several R\&D projects at the national and european levels, always to
provide technical expertise and support to the other partners.

In the context of this project, Logilab will innovate to support the partners
with tools and infrastructure, including open databases to flexibly store
mathematical objects, user interfaces to visualise complex mathematical
properties, fluid workflow tools to ease large-scale collaboration, etc.

%Logilab's PIC number is 948455525.

% See ../proposal.tex, section Members of the Consortium for a
% complete description of what should go there

\subsubsection*{Curriculum vitae}

% Curriculum of the personnel at this institution

\begin{participant}[type=PI,PM=2]{Florent Cayré}
  Engineer with a Master Degree from École Centrale de Paris (top
  French engineering school), Florent Cayré spent six years in SNECMA
  as an engineer conceiving the numeric tools for the turbines design
  and then was head of the group ``Méthodes de conception de turbines
  ; aérothermique et combustion'': collaborative R\&D programs,
  development and integration of various numeric tools (Python, C++,
  C, Fortran). He co-funded the SecondWeb company: development of
  complex Web applications based on CubicWeb platform.

  Head of the science computing department of Logilab since 2012,x
  Florent is responsible for team management, strategic vision,
  projects monitoring, technical expertise, etc. He developed several
  tools for defining and managing computations, and producing enhanced
  result reports through IPython notebooks.
\end{participant}

\begin{participant}{Olivier Cayrol}
Engineer Master Degree from École Centrale de Lyon (top French
engineering school). 3 years at the R&D department of PSA-Peugeot
Citroën, as developer and project manager on the modelling and
simulation of electronic embedded car control devices.

Co-founder and deputy-CEO of Logilab. Design and development of the
system that generates the documents of Logilab from ReST data
sources. This system is based on several free softwares such as Sphinx
or reportlab and defines numerous extensions for answering the specifc
needs.
\end{participant}

\begin{participant}[type=PI,PM=6,gender=male]{David Douard}
  holds a Master Degree from the Ecole Nationale Sup\'{e}rieure de Physique de
  Strasbourg and a PhD in computer science from Universit\'{e} Paris VI.

  As a PhD studend, he developed a graphical user interface for the
  computational system he used for his research as well as lots of processing
  and visualization tools.

  He worked two years at EDF where he was the lead developer of the software in
  charge of evaluating financial risks on the energy market with a specific,
  intensive work on the graphical user interface and on the code driving the
  simulations.

  He has been working at Logilab since 2006, building complex scientific
  applications involving the management and visualization of large amounts of
  data. He has trained tens of engineers and researchers to \software{C/C++},
  \Python, GUI libraries (\software{Tk}, \software{PyQt}, \software{wxPython}),
  \software{Fortran} and scientific computing.
\end{participant}

\begin{participant}[type=PI,PM=18,gender=male]{Julien Cristau}
  holds Master Degree and PhD from University Paris VII, where he carried out
  mathematical research in automata and linear games. As a \software{Debian}
  developer since 2007, he maintained its key components such as the
  \software{X11} windowing system and acted as a \software{Debian} Release
  Manager since 2011.

  Working as a software engineer in the R\&D department of Logilab since
  2011, he developed software using many different languages and systems,
  helped to release and distribute software on many different platforms,
  maintained parts of the infrastructure and trained other people.
\end{participant}

\begin{participant}[type=R,PM=12,gender=male]{Serge Guelton}

    Serge Guelton holds an egineering degree in Computer Science and
    telecomunication from from Télécom Bretagne and a PhD in compilation and
    parallelism. He's been working as an expert engineering in various INRIA
    teams, and as a lead developer in several strat-ups.

    He's the lead developer of the Pythran project, a Python-to-C++ compiler
    for high-performance scientific kernels.

\end{participant}
%%% Local Variables:
%%% mode: latex
%%% TeX-master: "../proposal"
%%% End:


\subsubsection*{Publications, products, achievements}

\begin{enumerate}
 \item CubicWeb is a semantic web framework that is available under the LGPL
   license and received a DataConnexion prize from Etalab (the french government
   team dedicated to Open Data)

\item Logilab has been contributing to free software since its creation in 2000
  and is known for it in France and several other countries. It authored \software{PyLint},
  the static \Python code checker used worldwide, and has always had at least one
  \software{Debian} Developer on staff, thus supporting the largest free software
  distribution used by millions of people.

 \item At OBHM 2013, the 19th Annual Meeting of the Organisation for Human Brain
   Mapping, Logilab presented a poster which explains the work done using
   CubicWeb on brain imaging and genetics data in collaboration with INRIA,
   INSERM and the CEA during the Brainomics project co-financed by Agence
   nationale de la Recherche.

\end{enumerate}

\subsubsection*{Previous projects or activities}

\begin{enumerate}
\item Logilab was a member of the consortium for the ASWAD EU project, with a
  role of software developer. The project demonstrated the interest of free
  software for setting up workflows in public administrations.
\item Logilab was a member of the consortium for the KIDDANET EU project, with a
  role of software developer. The project implemented a proxy to protect kids
  browsing the internet.
\item Logilab was a member of the consortium for the PYPY EU project, with a
  role of software developer. The project implemented a \Python interpreter in
  \Python, to explore new ways to compile and optimise the execution of \Python
  code.
\item Logilab was a member of the consortium for the OpenHPC french project,
  with a role of software developer. The project advanced the state of the art
  of free software for high performance simulation.
\item Logilab was a member of the consortium for the BRAINOMICS french ANR
  project, with a role of software developer. The project advanced the state of
  the art of shared databases for the brain imaging and genetics data.
\item Logilab was a member of the consortium for the THALER french ANR project,
  with a role of software developer. The project advanced the state of the art
  of free software for high performance simulation of molecular dynamics.
\end{enumerate}

\subsubsection*{Significant infrastructure}

Logilab is maintaining its own infrastructure, using virtualisation techniques
and tools such as \software{OpenStack}, \software{SaltStack}, \software{Docker}
and others.

\end{sitedescription}


%KEY-MORE-TODOS


%%% Local Variables:
%%% mode: latex
%%% TeX-master: "../proposal"
%%% End:

%  LocalWords:  sitedescription Logilab TOWRITE subsubsection publi Brainomics KIDDANET
%  LocalWords:  fr
