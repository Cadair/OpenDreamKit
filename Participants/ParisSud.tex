\begin{sitedescription}{PS}

University Paris-Sud is among the 40 top universities worldwide in the
2013 Shanghai ranking, and is one of the two best French research
universities. With about 27000 students, 1800 permanent teaching staff
and 1300 permanent research scientists from national research
organisations (CNRS, Inserm, INRA, Inria), it is the largest campus in
France. Since 2006, scientists from the University were awarded two
Fields medals, one Nobel Prize and a number of other international
(European Inventor Award 2013, Wolf Prize 2010, Holweck Prize 2009,
Japan prize 2007) and national prizes.  The Université Paris-Sud has a
complete array of competences, ranging from the purest of exact
sciences to clinical practices in medicine, covering life and health
sciences, legal sciences and economics. Research at the Université
Paris-Sud, an essential part of academic understanding, is
complemented by research activities with a high valorisation
potential. Research contracts and partnership with companies make the
Université Paris-Sud a key actor and a major player in French
research.  The University is located close to the Plateau de Saclay,
the largest cluster of public and private R\&D institutions in France
(with ca. 16000 research staff), and is one of the core members of the
University Paris Saclay – a world class university and a
world-renowned research and innovation hub.

In the context of this project, the Université Paris Saclay is the
home of one of the largest group of Sage developers worldwide. It's a
member of the Open Source Thematic Group of the Systematic Paris
Region Systems and ICT Cluster. The University also hosts a major
research group working on proof assistants (Coq), which naturally
opens the door for reaching toward this neighbor community.

% The main participants have accumulated 15 years of experience of
% collaborative open source software development for mathematics
% leadership, and community animation.

\subsubsection*{Curriculum vitae of the investigators}

\begin{participant}[type=leadPI,PM=6,salary=4200,gender=female]{Viviane Pons}
  Maître de Conférences at the Laboratoire de Recherche en Informatique, Viviane Pons is a
  young researcher in Algebraic Combinatorics. She defended her thesis in 2013 and has 3
  papers in international journals and 3 communications in international
  conferences, including a talk at PyCon US 2015. Before starting research career, 
  she worked for two years in industry as a Java and web developer.

  She discovered \Sage during her first \Sage Days in 2010 and has since been an active user
  and contributor with 10 (co)authored tickets improving the support of combinatorial
  objects in \Sage. She is heavily involved in the promotion of \Sage, participating in
  \Sage Days and running \Sage introduction tutorials or \Sage presentations at various
  conferences. She is also one of the main developers of the project \software{FindStat}
  dedicated to databases in combinatorics.
\end{participant}
%%% Local Variables:
%%% mode: latex
%%% TeX-master: "../proposal"
%%% End:

%\input{CVs/Nathann.Cohen}
\begin{participant}[type=PI,PM=6,salary=9000,gender=male]{Florent Hivert}

  Professor at the Laboratoire de Recherche en Informatique, Florent Hivert is a senior
  researcher in Algebraic Combinatorics with 29 papers in international journals and 15
  communications in international conferences.
% including an invited lecture at FPSAC'10

With 100 \Sage tickets (co)authored and as many refereed, Hivert is himself a core \Sage
developer, with contributions including key components of the \Sage infrastructure
(documentation, automated test, combinatorics infrastructure, parallelism, \ldots)
and specialised research libraries.
\end{participant}
%%% Local Variables:
%%% mode: latex
%%% TeX-master: "../proposal"
%%% End:

\paragraph{Nicolas M. Thiéry}

Professor at the Laboratoire de Recherche en Informatique, Nicolas
M. Thiéry is a senior researcher in Algebraic Combinatorics with an
international recognition. Among other things, he is a member of the
permanent committee of FPSAC, the main international conference of the
domain, and has collaborators in Canada, India, and in the US where he
spent several years; he also coorganized many international workshops,
in particular Sage Days, and a semester long program hosted by ICERM.

Algebraic combinatorics is a field at the frontier between mathematics
and computer science, with heavy needs for computer
exploration. Pioneer in community-developed open source software for
research in this field, Thiéry founded in 2000 the Sage-Combinat
software project; with 50 researchers in Europe and abroad, this
project has grown under his leadership to be one of the largest
organized community of Sage developers, gaining a leading position in
its field, and making a major impact on one hundred publications. At
this occasion, Thiéry gained a strong community building experience,
and coauthored some of NSF Sage-Combinat grant OCI-1147247.

With 150 tickets (co)authored and as many refereed, Thiéry is himself
a core Sage developer, with contributions including key components of
the Sage infrastructure (e.g. categories), specialized research
libraries (e.g. root systems), thematic tutorials, and two chapters of
the book ``Calcul Mathématique avec Sage''.

\begin{participant}[type=R,PM=6,gender=male]{Samuel Lelièvre}

Maître de conférences since 2006 at
Laboratoire de mathématique d'Orsay, Université Paris-Sud,
% PhD Rennes 2004 under Anton Zorich,
% post-doc Warwick with Vladimir Markovic,
Samuel Lelièvre is an established researcher in Dynamics and Geometry,
with 10 papers published in international journals including
Annales scientifiques de l'École normale supérieure,
Crelle, GAFA, Geometry and Topology.
He participated in three ANR projects, and has collaborators
in France, Israel, the UK, the USA.
His research in Dynamics and Geometry
often involves explicit and experimental approaches,
for which he writes code in order to explore
combinatorial objects such as square-tiled surfaces,
translation surfaces, group actions, group presentations.

He uses and actively promotes Sage since 2010.
He is in the top 15 contributors of the Ask-Sage
questions and answers site.
He coorganized six international meetings including two Sage Days,
presented Sage at PyCon-FR-2011 in Rennes,
supervised Sage tutorials twice at the GDR-IM yearly school
for French PhD students at the interface of Mathematics and
Computer Science, and at the CIMPA/ICPAM school Bobo2012
on Discrete Mathematics (Bobo Dioulasso, Burkina Faso, 2012).
\end{participant}
%%% Local Variables:
%%% mode: latex
%%% TeX-master: "../proposal"
%%% End:

%\begin{participant}[type=R,PM=6,gender=male,salary=5600]{Lo\"ic Gouarin}
  Research Engineer since 2005 at CNRS and more specifically since
  2010 at the Laboratoire de Mathématique d'Orsay, Université
  Paris-Sud, Loïc Gouarin develops scientific computing software in
  different fields like Lattice-Boltzmann methods, Stokes solvers for
  fluid particles interaction, ...

  He is also director of the ``GdR Calcul'' and co-director of the
  ``Réseau Calcul''. These two entities form the ``Groupe Calcul'' of
  the CNRS whose role is to animate the scientific and high
  performance computing community in France, in particular by
  organising conferences, meetings, and seminars. In this context, he
  organises himself 3 to 4 training and development workshops per
  year, and promotes the use of \Python for teaching and research in
  France.

  Organisationally, due to purely administrative constraints within CNRS, 
  Loïc Gouarin will be attached to the CNRS Aquitaine.
\end{participant}

%%% Local Variables:
%%% mode: latex
%%% TeX-master: "../proposal"
%%% End:


\TOWRITE{ES}{Specificity of Loic that will be attached to Bordeaux/CNRS}

\begin{participant}[type=res,PM=48,salary=5500]{NN}
  Full time developer: portability, packaging, ...
\end{participant}

\begin{participant}[type=res,PM=36,salary=5500]{NN}
  Full time developer: Sphinx infrastructure, ...
\end{participant}

\begin{participant}[type=res,PM=36,salary=3100]{NN}
  PhD.
\end{participant}

\begin{participant}[type=res,PM=24,salary=3932]{NN}
  Project manager.
\end{participant}

\subsubsection*{Publications, achievements}

\TODO{Il faut être plus formel dans la description des projets
  antérieurs : Acronyme, titre, agence de financement, durée.  Pareil
  pour les publi - auteurs, titre exact, année etc.}

\begin{enumerate}
\item Lead of the \SageCombinat software project.
\item Coauthoring of the open source book ``Calcul Mathématique avec
  Sage'', the first of its kind comprehensive introduction to
  computational mathematics in Sage for education.
\item XXX tickets contributed to Sage.
\end{enumerate}


\subsubsection*{Previous projects or activities}

\begin{enumerate}
\item Home of six one week-long Sage Days workshops.
\item Co-Organizer of \TODO{XXX} Sage Days.
\item Founder and regular organizer of a bimonthly Sage User Group
  meeting in the greater Paris area.
\item Expertise exchanges with Logilab
\item \TODO{XXX}
\end{enumerate}

\subsubsection*{Significant infrastructure}

The Université Paris Sud hosts the lead developers of the open source
cloud infrastructure \texttt{Stratuslab} and its reference
infrastructure (\TODO{XXX cores}). The participants are regular users
of this infrastructure, and in close contact with the developers.

\TODO{Comments by Olivier Chapuis}
Paris Sud also hosts the WILDER platform, an experimental wall-sized
high-resolution interactive touch-screen for conducting research on
collaborative human-computer interaction and the visualization of
large datasets.
\end{sitedescription}
%%% Local Variables: 
%%% mode: latex
%%% TeX-master: "../proposal"
%%% End: 

%  LocalWords:  sitedescription Paris-Sud organisations Inserm Inria Holweck valorisation
%  LocalWords:  Saclay subsubsection faut formel des projets antérieurs Acronyme titre
%  LocalWords:  agence financement durée Pareil les publi année SageCombinat Calcul avec
%  LocalWords:  Mathématique Logilab Sud texttt Stratuslab Chapuis
