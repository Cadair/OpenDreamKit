\begin{sitedescription}{US}

The University of Silesia in Katowice was established in 1968. Now,
with 12 faculties and several interdisciplinary schools and centers,
over 30000 students and over 2000 academic staff the University is one
of the largest in Poland. Students are educated at three educational
levels: Bachelor, Master and Doctoral and their achievement are
accumulated using European Credit Transfer and Accumulation System
(ECTS). Located in the heart of Upper Silesia, Poland’s old industrial
region with distinct history and cultural identity, the university
attracts many scientists and students.

The origins of the {\em Faculty of Mathematics, Fhysics and Chemistry} date
back to the academic year 1968/1969 and coincide with the
establishment of the University of Silesia. One of the largest
university units, the faculty incorporates, as its name indicates,
three separate departments: mathematics, physics and chemistry, each
with several divisions and subdivisions carrying out the research and
educational activities. There are over 1900 students, both full-time
and part-time, educated at three educational levels: Bachelor`s,
Master`s and Doctoral. The Faculty is entitled to grant doctoral
degrees in the natural sciences. The Faculty staff consists of 243
academics who are both teachers and researchers.

\subsubsection*{Curriculum vitae}
% AK - rules say "CV or description of the profile of the persons"
% - can we name it "Track record" instead of CV?

% Curriculum of the personnel at this institution

\begin{participant}[type=leadPI,PM=12,salary=2000,gender=male]{Marcin Kostur}
is an assistant Professor at the Institute of Physics. He is the author of over 40
publication cited over 1000 times in the field of statistical physics,
solid state physics (Josephson Junction dynamics), microfluidics and
biophysics. He is experienced in application of GPU architecture to
numerical simulations of stochastic processed in physics. His recent
computational interests are focused at the Open Source project
\software{Sailfish} -- HPC implementation of Lattice Boltzmann Method on GPU.
He is leader of two projects incorporating computations to the science education:
\begin{itemize}
\item Computing in high school science education - iCSE4schools,
  project funded by Erasmus+, Key Action 2 - ``Strategic Partnerships'',
  (budget: \euro{263}k, 2014-2017)
\item ``Computers in Science Education: iCSE'' http://icse.us.edu.pl
  (budget: \euro{1}m, funded by EFS, 2011-2014)
\end{itemize}
He is also co-author and a task coordinator of PAAD (Platforma Analiz i
Archiwizacji Danych) project funded by POIG program for 2014-2015 with a total budget
of \euro{4}m. The task ``Interactive HPC services for science''
main goal is to provide interactive interface to HPC infrastructure
(heterogenous cluster of 48 nodes, including 24 GPU and 24 Xeon Phi)
using innovative technology of ``web notebook'' interface.  
\end{participant}
%%% Local Variables:
%%% mode: latex
%%% TeX-master: "../proposal"
%%% End:


\begin{participant}[type=PI,PM=12,salary=2500,gender=male]{Jerzy Łuczka}
Prof. Dr. Jerzy Łuczka (\url{http://zft.us.edu.pl/luczka}) is
a full professor of physics at the University of Silesia (Katowice,
Poland) and the Head of the the Department of Theoretical Physics.

He published more than 150 papers in journals (all on ISI Master
Journal List) which have been cited almost 2000 times.

He is an Editor of European Physical Journal B, Chairman of the
Statistical and Nonlinear Physics Division (European Physical
Society), Fellow of the Institute of Physics (United Kingdom) and
Outstanding Referee (American Physical Society). He was Co-director of
the NATO Advanced Research Workshop ``Stochastic Systems. From
randomness to complexity'', 2002, Erice (Italy) and Member of the
Steering Committee of the program : ``Stochastic Dynamics: Fundamentals
and applications'' (European Science Foundation), 2003-2008.  He
received the DAAD research fellowship (Forschungsaufenthalte für
Hochschullehrer und Wissenschaftler) 1995, 2009 and 20012. He was a
leader of several Polish and two German-Polish grants. He has
collaborators in Germany, Italy and Spain. He has also co-organised
international conferences.

Łuczka’s research interests lie in areas of stochastic processes in
physics, quantum open systems, transport phenomena, physical
fundamentals of quantum information. He has teaching experience with
\Sage in physics, biophysics and econophysics.


\end{participant}
%%% Local Variables:
%%% mode: latex
%%% TeX-master: "../proposal"
%%% End:



\begin{participant}[type=PI,PM=12,salary=1400,gender=male]{Jan Aksamit}
got his PhD in 1982 and worked at University of Silesia as research
assistant, lecturer and senior lecturer. His skills combine forty
40-years of experience in teaching algebra, classical and quantum
mechanics, quantum information, statistical physics, and mathematical
methods of physics with proficiency in computing. He has actively
participated in the project iCSE (innovative Computing in Science
Education), where he created Sage enhanced textbook of Linear Algebra
(polish only: http://visual.icse.us.edu.pl/LA) using modern
interactive technologies. In this project his main task will be to
pursue to exploit the vast lecturing experience for creation of
interactive demonstrators of VRE.

\end{participant}
%%% Local Variables:
%%% mode: latex
%%% TeX-master: "../proposal"
%%% End:


%\input{CVs/First.Last.tex}
%\input{CVs/First.Last.tex}

\subsubsection*{Publications, products, achievements}


\begin{enumerate}

\item Sailfish: A flexible multi-GPU implementation of the lattice Boltzmann method
 Computer Physics Communications Vol.181(9), 2350-2368;2014. Web: http://sailfish.us.edu.pl 

\item  M.Januszewski and M.Kostur. ``Accelerating numerical solution of stochastic differential equations
with CUDA'',  Computer Physics Communications, 181(1):183-188, 2010. Web: https://github.com/marcinofulus/CUDASDE.git

\item iCSE - course materials  materials - http://visual.icse.us.edu.pl/iCSE\_main/

\end{enumerate}

\subsubsection*{Previous projects or activities}

Internationalization of research and education is one of the priority
directions of development of the University of Silesia. The University
scientists are actively engaged in research at the international
level, actively participates in European Commission initiatives
focused both on educational and scientific development, and implements
projects within the LLP/Erasmus+ programme, the Research Fund for Coal
and Steel, Framework Programmes, as well as the EU Structural Funds.

The institution has been involved in more than 40 FP7 proposals, of
which 15 have been funded.

The Faculty of Mathematics, Physics and Chemistry was engaged in the
implementation of several FP6 and FP7 projects, i.e.:

\begin{enumerate}
\item HadronPhysics (RII3/CT/2003/506078)
\item FlaviaNet (MRTN-CT-2006-035482)
\item LAGUNA (212343)
\item LHCPhenoNet (612536). 
\end{enumerate}

There are following projects which are directly connected to
infrastructures for virtual research environments:

\begin{enumerate}
\item 2011-2014 - iCSE (innovative Computing in Science Education) -
  grant from European Social Fund, EUR 1 Million, incorporating
  computational perspective in teaching of mathematics, physics and
  chemistry using cloud based Sage system and Python language.
\item 2014-30.11.2015 PAAD (Platform for data analysis and archiving) - EUR
  3.8Million, funded is mostly HPC center for research with
  interactive access based on web based notebook UI.
\item 2014-30.11.2015 CNS: Center of Applied Science - 2nd stage,
  Infrastructure grant includes EUR 500.000 funding for small HPC and
  cloud infrastructure for education. Technically this will be
  extension of research HPC center for educational purposes.
\end{enumerate}



\subsubsection*{Significant infrastructure}

The University of Silesia has finished currently executes grants from
ESF in total c.a. Million 120 Euro for infrastructure, laboratories and
computing centers. New HPC centers are under construction (PAAD and
CNS projects) which will provide necessary hardware for development
and implementation of virtual research environments.
\end{sitedescription}





\begin{draft}
\vspace{1cm}\TOWRITE{MK}{Complete check list below -- delete completed items if you wish}

\begin{verbatim}
- [X] ;-) checked that sum of person months put into finance request is
  the same as sum of person months associated with the Work Packages
  (in proposal.tex, as defined as part of the \begin{workpackage}"
  command.
  
  Take into account person months associated with work package 1, time
  of all staff to be hired and work on the project (including
  investigators). Figure 5 helps with a quick check of the sums over
  different work packages.

- [ ] completed site specific resource summary in resources.tex,
  including table of non-staff costs. This is compulsory (EU
  regulations) if the non-staff cost exceed 15% of the total cost, and
  is likely to be the case for most of the partners. We ask everybody
  to do it, to be consistent and show transparently how we have
  planned our total budget.

- [ ] Have all our tasks a designated lead institution? Check in the
  Work Packages that all the tasks you are involved in have a
  dedicated lead party. If the lead party is "USO", then use:
  \begin{task}[lead=USO]

- [ ] Have all our deliverables a designated lead institution [using
  the 'lead=' key]?

- [ ] In the "Members of the consortium section", have we addressed "a
  description of the legal entity and its main tasks, with an
  explanation of how its profile matches the tasks in the
  proposal"? See Entry for Paris Sud and Southampton as examples.

- [ ] In the Members of the consortium section, have we given
  descriptions of all the people we intend to hire (even if we don't
  know who that is yet). 
  
- [ ] Do all our tasks include us in the list of sites involved?
\end{verbatim}
\end{draft}

%KEY-MORE-TODOS


%%% Local Variables:
%%% mode: latex
%%% TeX-master: "../proposal"
%%% End:

%  LocalWords:  sitedescription Fhysics subsubsection M.Januszewski M.Kostur programme
%  LocalWords:  Programmes FlaviaNet LHCPhenoNet
