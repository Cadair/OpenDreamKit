\begin{sitedescription}{SA}

% PIC: 
% see: http://ec.europa.eu/research/participants/portal/desktop/en/orga
% 

% See ../proposal.tex, section Members of the Consortium for a
% complete description of what should go there

The Centre for Interdisciplinary Research in Computational Algebra (CIRCA)
fosters research at the interface of Mathematics and Computer Science including
abstract and algorithmic algebra and combinatorics, formal languages and
automata, mathematical software and constraint programming. Our success is 
founded on the close integration of theoretical and algorithmic research and 
the development and use of state-of-the-art software.

In 1997, CIRCA became the centre of the development of \GAP after the 
retirement of under~Prof.~J. Neub\"user who initiated the system in 
mid-1980s in Aachen. This move was supported by EPSRC, EU and Leverhulme 
grants. Now GAP and its packages comprise over 1 million lines of C and 
1.25 million lines of GAP code. The system is distributed freely under the GPL2,
and our records show that it has been installed in at least 3000 sites and 
cited in several thousands publications 
(see \url{http://bit.ly/gap_citations} and \url{http://www.gap-system.org/Doc/Bib/bib.html} ). 
GAP was used in landmark computations such as the ``Millennium Project'' 
to classify all groups of order up to 2000 
and the classification of the over $10^{19}$ semigroups of order 10. It is 
designed to be natural to use for mathematicians; to be powerful and flexible 
for experts and to be freely extensible so that it can encompass new mathematics. 
These objectives have been met and GAP was awarded the ACM/SIGSAM Richard D. Jenks
Memorial Prize for Excellence in Software Engineering applied to Computer Algebra in 2008.

Nowadays, as one of the centres of \GAP\ development, CIRCA has
excellent contacts with developers and users worldwide. Particularly 
relevant to this proposal are the Singular and homalg groups at 
Kaiserslautern, Soicher at QMUL, Praeger and others at UWA, Cooperman 
and his students at NEU and the other GAP centres in Aachen, Fort Collins 
and Braunschweig.

CIRCA's output includes first class results in Pure Mathematics and in
Computer Science, recognised by our highly-cited publications in top
international venues in both disciplines and our widely used research
software. Beyond the individual international connections of the 
investigators and research staff, CIRCA as a centre has national 
importance and international standing. CIRCA has been selected to host 
major conferences such as CP 2010, BCC 2009, PP 2007, and the 
``Groups St Andrews'' series in 2005 and 2013.


\subsubsection*{Curriculum vitae}

% Curriculum of the personnel at this institution

\begin{participant}[type=leadPI,gender=male]{Steve Linton}
  is a Professor of Computer Science at St~Andrews. He has worked in computational algebra
  since 1986 and has helped coordinate the development of \GAP\ %\cite{end}{gap}
  since its move from Aachen in 1997. He personally wrote key features of \GAP, such as
  workspaces and exception handling, and has overseen the development and releases of the
  whole system.  He directed CIRCA from~2000--2013. He is an editor of
  AAECC\footnote{Applicable Algebra and Error Correcting Codes}.  He has been PI of four
  major EPSRC grants and coordinated the EU project~SCIEnce. He is the general chair of
  ISSAC 2015, the main conference in computer algebra.
\end{participant}
%%% Local Variables:
%%% mode: latex
%%% TeX-master: "../proposal"
%%% End:

\begin{participant}[type=PI,PM=24,gender=male]{Alexander Konovalov}
  is a Senior Research Fellow in CIRCA and has worked on \GAP\ for more than 10 years.
  After holding the fellowship at the Vrije Universiteit Brussel in 2006, researching
  computational group ring theory, he moved to St~Andrews in 2007 to join EU
  project~SCIEnce. He leads many aspects of the \GAP project, including release
  preparation, regression testing and liaison with package authors. He has authored 38
  papers and 8~\GAP packages, and co-organised a number of events, most recently the
  LMS/EPSRC Short Instructional Course in Computational Group Theory,
%\footnote{\url{http://www-circa.mcs.st-and.ac.uk/cgt2013/}},
the HPC-GAP workshop (2013), and
%\footnote{\url{http://www.gap-system.org/hpcgap2013/}}
the Summer School on Experimental Methodology in Computational Science Research
(2014).
%\footnote{\url{https://blogs.cs.st-andrews.ac.uk/emcsr2014/}}. 
He is an editor of Journal of Software for Algebra and Geometry 
and a Fellow of the Software Sustainability Institute.
\end{participant}
%%% Local Variables:
%%% mode: latex
%%% TeX-master: "../proposal"
%%% End:

\begin{participant}[type=R,PM=48,gender=male]{Markus Pfeiffer}
is a Research Fellow at the University of St Andrews. He is active both in the 
School of Computer Science and the School of Mathematics and Statistics. 
His unusual breadth of knowledge encompasses expertise in formal language 
theory, decidability, and algebra, as well as practical computation and
programming languages.
Since receiving his PhD in 2013 he has been an active researcher in semigroup 
theory as well as contributing to the \GAP system both as a package author, 
and as a core system developer.
\end{participant}
%%% Local Variables:
%%% mode: latex
%%% TeX-master: "../proposal"
%%% End:

%\input{CVs/First.Last.tex}

% TODO(AK): select and add more items below

\subsubsection*{Publications, products, achievements}

\begin{enumerate}
\item 
S.~Linton, K.~Hammond, A.~Konovalov, C.~Brown, P.W.~Trinder, H.-W.~Loidl, 
P.~Horn and D.~Roozemond, Easy Composition of Symbolic Computation Software using 
SCSCP: A New Lingua Franca for Symbolic Computation.
J. Symbolic Computation, 49 (2013), 95--119.
% http://dx.doi.org/10.1016/j.jsc.2011.12.019
\item
V.~Janjic, C.M.~Brown, M.~Neunh{\"o}ffer, K.~Hammond, S.~Linton, H-W.~Loidl. 
Space exploration using parallel orbits: a study in parallel symbolic computing.
in Parallel Computing: Accelerating Computational Science and Engineering (CSE). 
vol. 25, Advances in Parallel Computing, IOS Press, 2013, p. 225--232.
\item 
A.~Konovalov and S.~Linton. 
SCSCP --- Symbolic Computation Software Composability Protocol, 
Version 2.1.4; 2013. GAP package (\url{http://alexk.host.cs.st-andrews.ac.uk/scscp/}).
\item
V.~Slavici, D.~Kunkle, G.~Cooperman, S.~Linton. 
An efficient programming model for memory-intensive recursive 
algorithms using parallel disks. In ISSAC 2012: Proceedings of 
the 37th International Symposium on Symbolic and Algebraic Computation. 
New York, ACM Press, 2012, p. 327--334.
\item 
R.~Behrends, A.~Konovalov, S.~Linton, F.~L{\"u}beck, M.~Neunh{\"o}ffer.
Towards high-performance computational algebra with GAP.
Proceedings of the Third International Congress on Mathematical Software: 
Kobe, Japan, September 13-17, 2010. LNCS 6327, Springer, 2010, p. 58--61.
% http://dx.doi.org/10.1007/978-3-642-15582-6_12 

\end{enumerate}

\subsubsection*{Previous projects or activities}

\begin{enumerate}
\item
\textbf{Multidisciplinary Critical Mass in Computational
Algebra and Applications} EP/C523229 2005--2010, \pounds 1.1m.
Through a range of subprojects this grant developed CIRCA as 
a broad centre of excellence in computational algebra. We 
extended \GAP, developed new algorithms, and opened up new 
applications in AI, combinatorics and physics. The project 
produced over 200 refereed publications in a huge range of 
venues, and delivered, as intended, a sustainable step change 
in the scale and breadth of CIRCA's research.
\item 
\textbf{SCIEnce: Symbolic Computation Infrastructure for Europe}
(FP6 eRII3-CT-026133) 2006--2011, \euro 3.2m, coordinator.
This project tackled the fragmentation of the European community 
of researchers in, and users of, symbolic computation. Among the 
nine partners were four major system developers (of \GAP, \Maple, 
\MuPAD and \KANT), an international research institute (RISC-Linz) 
and other groups with specialist expertise. Project activities 
included symbolic web services, system composability, symbolic 
grid and cloud computing and a program of visits, workshops and 
summer schools. One important outcome was a new protocol \SCSCP, 
now used well beyond the original project.
\item
\textbf{HPCGAP: High Performance Computational Algebra and Discrete Mathematics} 
EP/G055181 2009--2014, \pounds 1.5m, four sites. This project aimed
at reengineering GAP to support simple, safe and efficient parallel 
programming on a range of systems from multicore laptops and desktops, 
through departmental and university clusters to HPC systems. By the 
end of the project, we had adapted a complex system including a 
language runtime and over 400 000 lines of interpreted code to enable 
safe and efficient parallel programs. The proposed project is very timely as
the multi-threaded version of GAP is becoming mainstream, and users and 
package developers need training and support to parallelise their code.
\end{enumerate}

\subsubsection*{Significant infrastructure}

CIRCA provides hosting for the GAP website and ftp-servers, runs the 
automated system to check, fetch and test updates of GAP packages, uses 
the Jenkins continuous integration system to run regression tests for the 
development version and release candidates on its computer infrastructure, 
and manages GAP releases preparation.
\end{sitedescription}
%%% Local Variables:
%%% mode: latex
%%% TeX-master: "../proposal"
%%% End:

%  LocalWords:  sitedescription Neub centres recognised subsubsection Konovalov Roozemond
%  LocalWords:  textbf
