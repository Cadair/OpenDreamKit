\begin{sitedescription}{UV}
% PIC: 999837104

\subsection*{Université de Versailles -- Saint-Quentin-en-Yvelines}

{\bf PRiSM Laboratory.} The research teams of the PRiSM laboratory
(Parall\'elisme, Réseaux, Syst\`emes et Mod\'elisation) are involved
in two main scientific themes of UVSQ: Mathematics and Computer
science on one hand, ``Design, Modelling and Implementation of
Systems'' on the other hand. These two directions are not separated
from each other, as shown by many collaborations with other labs, and
the participation of many PRiSM teams to both directions. Within the
``Mathematics and Computer Science'' theme, the PRiSM teams study
cryptology and security, models for algorithms and operational
research. All the teams also participate to the ``Design, Modelling
and Implementation of Systems'' theme, with a particular focus on
communication systems (networks and telecommunication), embedded
systems, mobile systems, high speed networks, and database systems.

PRiSM is home to the ``Cryptology and Information Security''. In its
research activities, the cryptography team aims at widely covering the
various themes of academic research in cryptology, public key and
secret key cryptography, cryptanalysis, security of implementations,
number theory, multivariate cryptography, hash functions, etc. The
cryptology team brings its specificity in the computer science courses
at UVSQ and, since several years, the university offers several
teaching programs with a part devoted to cryptology and information
security. In particular, the research graduate program ``Applied
Algebra'' offers a full course in cryptology. It has been complemented
by a professional graduate program, called SeCReTS (Security of
Contents, Networks, Telecommunications and Systems). Many activities
of the team, require the use of advanced computer algebra. For this,
the team has a long history of using computer algebra systems (\GAP,
\PariGP, \Maple, \Magma, and others). In recent years, with the arrival of young
researchers, and with the affirmation of \Sage in research and
teaching, the team has moved from a pure user perspective to a
contributor one, taking active part in the development of computer
algebra software.

%%%%%%

\subsubsection*{Curriculum vitae of the investigators}

\begin{participant}[PM=12,type=leadPI,gender=male]{Luca De Feo}
  got his PhD in 2010 at Ecole Polytechnique. He was appointed Maître de Conférences at
  Versailles-St-Quentin-en-Yvelines University in 2011. His research interests cover
  Algorithmic Number Theory, Computer Algebra, Cryptology and Automated deduction, and he
  has already published 8 papers in international journals or refereed international
  conferences.

  He is an active Sage contributor, with a dozen of tickets co-authored and about as much
  reviewed. He is also active in promoting the use of Sage for research and for teaching:
  most of his papers feature a publicly available Sage implementation, he teaches Sage to
  undergraduate and graduate students, he participates and organizes various events for
  the introduction of Sage to beginners and young researchers.
\end{participant}
%%% Local Variables:
%%% mode: latex
%%% TeX-master: "../proposal"
%%% End:

\begin{picv}{Nicolas Gama}
  got got his PhD in 2009 at Ecole Normale Supérieure. He was appointed Maître de
  Conférences at University of Versailles-St-Quentin-en-Yvelines in 2010.  His research
  interests cover Lattice reduction algorithms, Theory of computer sciences, Algorithmic
  Number Theory, and Cryptology. He has already published 12 papers in international
  journals or refereed international conferences.

  He developed a fork of the NTL library to ease the development of parallel lattice
  algorithms, and added various blockwise lattice primitives, tools like high dimensional
  gaussian sampling over lattices and modulo lattices, which can be directly used to
  implement the most recent lattice-based schemes. This fork is scheduled to be merged
  with the main branch of NTL, and the wrapper library for Sage should then be updated
  accordingly.
\end{picv}
%%% Local Variables:
%%% mode: latex
%%% TeX-master: "../proposal"
%%% End:


\subsubsection*{Publications, achievements}

Recent publications:

\begin{compactenum}
\item A.~Becker, N.~Gama and A.~Joux; Solving shortest and closest
  vector problems: The decomposition approach. ANTS 2014.
\item L.~De~Feo, J.~Doliskani, É.~Schost; Fast arithmetic for the
  algebraic closure of finite fields. ISSAC '14. ACM, 2014. pp
  122-129.
\item N.~El~Mrabet and N.~Gama, Efficient Multiplication over
  Extension Fields, WAIFI 2012.
\item L.~De~Feo, É.~Schost; Transalpyne: a language for automatic
  transposition. ACM SIGSAM Bulletin, 2010, 44 (1/2), pp. 59-71.
\end{compactenum}

Software:

\begin{compactenum}
\item \software{newNTL}. It is a high-performance, portable \software{C++} library providing
  data structures and algorithms for manipulating signed, arbitrary
  length integers, and for vectors, matrices, and polynomials over the
  integers and over finite
  fields. \url{http://www.prism.uvsq.fr/~gama/newntl.html}.
\item \software{FAAST}, a \software{C++} library for Fast Arithmetic in Artin-Schreier
  Towers. \url{http://github.com/defeo/FAAST}.
\end{compactenum}

\subsubsection*{Previous projects or activities}

Current grants:

\begin{compactenum}
\item ANR CLE (2013-2017): Cryptography from Learning with Errors.
  The goal is to propose fast and secure symmetric protocols based on
  the LWE problem.
\item DIGITEO project ARGC (2013-2016): ``Fast arithmetic for geometry
  and cryptology''. The project explores fast algorithms and
  implementations for algebraic geometry and curve-based cryptography.
\item DIGITEO project IdealCodes (2014-2016): IdealCodes
  (\url{http://idealcodes.github.io/}) spans the three research areas
  of algebraic coding theory, cryptography, and computer algebra, by
  investigating the problem of lattice reduction.
\end{compactenum}
\end{sitedescription}



\begin{draft}
\vspace{1cm}\TOWRITE{LDF}{Complete check list below -- delete completed items if you wish}

\begin{verbatim}
- [X] checked that sum of person months put into finance request is
  the same as sum of person months associated with the Work Packages
  (in proposal.tex, as defined as part of the \begin{workpackage}"
  command.
  
  Take into account person months associated with work package 1, time
  of all staff to be hired and work on the project (including
  investigators). Figure 5 helps with a quick check of the sums over
  different work packages.

- [X] completed site specific resource summary in resources.tex,
  including table of non-staff costs. This is compulsory (EU
  regulations) if the non-staff cost exceed 15% of the total cost, and
  is likely to be the case for most of the partners. We ask everybody
  to do it, to be consistent and show transparently how we have
  planned our total budget.

- [X] Have all our tasks a designated lead institution? Check in the
  Work Packages that all the tasks you are involved in have a
  dedicated lead party. If the lead party is "USO", then use:
  \begin{task}[lead=USO]

- [X] Have all our deliverables a designated lead institution [using
  the 'lead=' key]?

- [ ] In the "Members of the consortium section", have we addressed "a
  description of the legal entity and its main tasks, with an
  explanation of how its profile matches the tasks in the
  proposal"? See Entry for Paris Sud and Southampton as examples.

- [ ] In the Members of the consortium section, have we given
  descriptions of all the people we intend to hire (even if we don't
  know who that is yet). 
  
- [X] Do all our tasks include us in the list of sites involved?
\end{verbatim}
\end{draft}

%KEY-MORE-TODOS


%%% Local Variables: 
%%% mode: latex
%%% TeX-master: "../proposal"
%%% End: 

%  LocalWords:  sitedescription Saint-Quentin-en-Yvelines Parall elisme Réseaux Syst emes
%  LocalWords:  elisation Modelization subsubsection Gama Joux Feo Doliskani Schost pp
%  LocalWords:  Mrabet Transalpyne Artin-Schreier DIGITEO
