\begin{sitedescription}{UG}

Today Ghent University attracts over 41000 students, with a foreign student population of about 4100.
In 2015, the university invested over 242 million Euros in research projects
on behalf of public and private partners, and employed around 6600 academic staff members.
Ghent University is ranked 71st in the Shanghai and 118th in the Times ranking.
The University has participated in more than 200 research projects in the EU's Sixth Framework Programme (2002--2006)
and in 260 projects in the Seventh Framework Programme, of which 27 ERC grants and 26 Marie Curie Fellowships.
Ghent University coordinated 42 collaborative projects in FP7.
Up till now, Ghent University is under H2020 (the new Framework Programme (2014--2020)) involved in 100 projects, and coordinates 9 of them.
Among these newly acquired projects, Ghent University hosts 15 ERC grant holders and supervises 9 Marie Sk{\l}odowska Curie fellowships.
The university provides excellent training opportunities to both young and experienced researchers,
and awarded 660 PhD degrees in 2015 of which over 30\% went to international young researchers.
The university is one of the fastest growing European universities in terms of research capacity and productivity,
and its commitment to European research excellence is reflected by the recent extension of the `European Office'
in its Research Office, i.e.~the department overseeing, guiding and administering research projects.

\subsubsection*{Curriculum vitae}

\begin{participant}[type=leadPI,PM=30,gender=male]{Jeroen Demeyer} % Should be 30.5
is a post-doctoral assistant at Ghent University (Belgium).
He has broad mathematical interests, mainly in number theory and its
connections with logic.
He received his PhD in mathematics in Ghent in 2007.
Since then, he has been working as a post-doc in Ghent,
except for a stay of 15 months at the Scuola Normale Superiore in Pisa (Italy).
At Ghent University, he introduced \Sage{} in the mathematics education
and teaches two courses with it.

He is a very active contributor to \Sage{} and he has been its
release manager for a period of 3 years, from 2011 to 2013.
He is the main author of the \software{cysignals} package
for interrupt and signal support in \Cython{}.
He also made various contributions to other projects,
such as \PariGP{}, \Cython{}, and \Jupyter{}.
\end{participant}

%%% Local Variables:
%%% mode: latex
%%% TeX-master: "../proposal"
%%% End:


\subsubsection*{Publications, products, achievements}

At the Department of Mathematics of Ghent University,
researchers have always been interested in computational aspects:
Jeroen Demeyer has contributed to \Sage{} and
Jan De Beule and Michel Lavrauw are co-authors
of the \GAP{} package FinInG for finite incidence geometry.

% \subsubsection*{Previous projects or activities}
%
% Nothing significant that I can think of...

\subsubsection*{Significant infrastructure}

Ghent University has a large High Performance Cluster infrastructure,
consisting of 5 distinct clusters for various purposes, totalling 11328 CPU cores.
On top of this, it also hosts the Tier-1 Flemish Supercomputer,
a single cluster with 8448 cores, to be used for all research institutions in Flanders.

Ghent University is using SageMath for teaching in the mathematics education.
To support this, there are two 12-core servers running a Sage Notebook server.
There is also a VM running Jupyter Hub.
Eventually, this should also be used for teaching purposes,
but this is currently in an experimental phase.
\end{sitedescription}
%%% Local Variables:
%%% mode: latex
%%% TeX-master: "../proposal"
%%% End:
