\TOWRITE{ALL}{Proofread WP 3 Component Architecture pass 2}
\begin{draft}
\TOWRITE{UV (Work Package Lead)}{For WP leaders, please check the following (remove items
once completed)}
\begin{verbatim}
- [X] have all the tasks in this Work Package a lead institution?
- [X] have all deliverables in the WP a lead institution?
- [X] do all tasks list all sites involved in them?
- [X] does the table of sites and their PM efforts match lists of sites for each task?
      (each site from the table is listed in all relevant tasks, and no site is listed
      only in the table or only at some task)
\end{verbatim}
\end{draft}

\begin{workpackage}[id=component-architecture,wphases=0-48!.5,
  title=Component Architecture,lead=UV,
  PSRM=50,UVRM=8,SARM=16, USORM=6, UORM=4, LLRM=14, UJFRM=6, UGRM=14]
  % PS: Full time dev: 48 PM, NT: 4 PM

  \begin{wpobjectives}
    The objective of this work package is to develop and demonstrate a
    set of APIs that enable components, such as database interfaces,
    computational modules, separate systems such as \GAP or \Sage, to
    be flexibly combined and run smoothly across a wide range of
    environments (such as Cloud-based, local, and server environments).
  \end{wpobjectives}

  \begin{wpdescription}
    This Work Package focuses on the structure of the components that make
    up a mathematical software and their interactions. Such components
    can be separate modules inside a unique software, or separate
    softwares interacting through library calls and/or through APIs
    (e.g.: web APIs). When combined together, they make up a full VRE.

    The architecture of these software components must be:
    \begin{compactitem}
    \item \textbf{Portable}, to support a wide range of platforms
      (mobile, desktops, cloud, \dots).
    \item \textbf{Modular}, so to ease installing, building, testing,
      and remixing.
    \item \textbf{Flexible}, so to adapt to different use cases:
      personal computation, HPC, parallel platforms, \dots
    \item \textbf{Open}, in the sense of \emph{open source}, but also
      in the sense of clearly documented and open to
      the user who wants to understand its underpinnings and/or
      contribute to it. Indeed we must not forget that the working
      mathematicians and other users need to know what algorithms the software is
      going to run to solve a given problem.
    \end{compactitem}
  \end{wpdescription}

  \begin{tasklist}
  \begin{task}[id=portability,title=Portability,lead=UV,PM=28,partners={PS,UG},wphases=0-36]
    In order to achieve maximum availability and accessibility,
    mathematical software must be developed and tested for a wide range
    of computer architectures and operating systems.  However most of
    open source development happens in POSIX environments (usually
    Linux or OS X), and almost exclusively on x86 platforms.  The vast
    majority of the developers of mathematical software does not have
    the expertise, nor the access to appropriate hardware and software, to insure
    appropriate testing and porting of components.  The best
    incarnation of this issue is the involved installation procedure
    for \Sage on Windows, a major adoption barrier and common source of
    complaints by end-users.

    In this task we will address the common needs of the community in
    terms of portability layers, building and testing infrastructure.

    \begin{compactitem}
    \item Best practices adopted by the larger open source community
      will be investigated and leveraged, and existing expertise will
      be shared between the component developers.
    \item Windows being largely dominant in the desktop/laptop market,
      a specific focus will be placed on the port of \Sage, and
      therefore all the components included in its distribution (in
      particular \PariGP, \GAP, \Singular, \Linbox) to this platform
      (\delivref{component-architecture}{portability-cygwin}).
    \item The deployment of a common infrastructure for multi-platform
      continuous integration (testing, building and distribution) will
      be addressed
      (\delivref{component-architecture}{multiplatform-buildbot}).
    \end{compactitem}

      % Jean-Pierre:
      % Should we mention port to non-x86-64 archs and non-Linuces?
      % 
      % For CPUs:
      % - I guess at least ARM and ppc64 (IBM POWER*) really make sense.
      % - Sparc is less convincing though the latest sparc CPUs
      % are muche more interesting for math computation as the
      % previous ones, e.g. the GMP folk specifically added assembly
      % for them in their latest release.
      % - Itanium is dead, but it can help discovering bugs as any non
      % standard archs.
      % - Supporting any of these would mean buying (potentially very
      % expensive) hardware.
      % 
      % For OSes?
      % - Should we mention OS X which is a pain at each new release?
      % - A BSD variant would be interesting, let's say FreeBSD which
      % is basically (almost) already supported
      % - Solaris? and/or OpenIndiana? Interesting if we mention sparc...
      % - Windows is already included below, my opinion is:
      % * provide live USB, VMs and Cygwin32 first as these three are
      % basically already working solutions
      % * go Cygwin64 as it is still POSIX
      % * explorate a MinGW solution, at least GAP and PARI should be
      % problematic
      % * try to use MSVC
  \end{task}

  \begin{task}[title=Interfaces between systems,id=interface-systems,lead=PS,PM=22,partners={UV,UO,SA,UG},wphases=0-36]
    In this task we will investigate patterns to share data,
    ontologies, and semantics across computational systems, possibly
    connected remotely.  We will leverage the well established
    semantics used in mathematics (categories, type systems, \dots) to
    give powerful abstractions on computational objects.
    
    We will build upon the work already done in the EU FP6 project
    26133 ``SCIEnce -- Symbolic Computation Infrastructure for
    Europe'' (\url{http://www.symbolic-computing.org/}) on the Symbolic Computation
    Software Composability Protocol (SCSCP). SCSCP is a remote
    procedure call protocol by which a computer algebra system (CAS)
    may offer services to a variety of possible clients, including
    e.g.  another CAS running on the same computer system or remotely;
    another instance of the same CAS (in a parallel computing
    context); a simplistic SCSCP client
    (e.g. C/C++/Python/etc. program) with a minimal SCSCP support
    needed for a particular application; a Web server which passes on
    the same services as Web services, etc.  A distinctive feature of
    the protocol is that both instructions and data are represented in
    the OpenMath format (\url{http://www.openmath.org/}; previously
    supported by the EU JEM Thematic Network; EU project 24969
    ``ESPRIT'' and other projects); moreover, OpenMath support is not
    limited by existing official OpenMath content dictionaries --
    private encodings may be easily embedded into SCSCP messages.
    
    SCSCP is already supported by a number of computer algebra
    systems, including \GAP, Macaulay2, Maple, TRIP and others. We
    will extend support for SCSCP to other relevant systems involved
    in \TheProject (\delivref{component-architecture}{scscp-sage}).
    Through its API, we will enable discovery of subsystems,
    functionality, documentation and computational resources. The user
    interfaces shall be enabled to automatically choose the best
    available algorithms and resources to perform a required
    computation, as well as clearly and intuitively present the
    available choices to the expert user.

    As a first concrete test bed, we will consider the \Sage interface
    to \GAP, or more precisely \libGAP.  Like most \Sage interfaces,
    this uses the now classical \emph{handle} design pattern, whereby
    one can manipulate from \Sage an object created and stored in
    \GAP, through a \emph{handle} (a.k.a. \emph{remote objects}).  By
    mapping \GAP's categories to \Sage's categories, in
    \delivref{component-architecture}{semantic-interface-sage-gap} we
    will:
    \begin{compactitem}
    \item Implement a modular infrastructure for adapters, based on
      SCSCP, in order to let the implementation of adapters scale to a
      large variety of objects.
    \item Refactor the existing adapters, using this infrastructure to
      generalise their features. This step by itself will provide
      adapters for larger categories like semigroups or monoids.
    \item Merge the adapters into the handles, so that a handle to a
      \GAP group will \emph{automatically} behave like a native \Sage
      group.
      % This will remove much back-conversion burden from the
      % delegating methods
    \end{compactitem}
    A specific challenge will be performance; indeed low level method
    adapters, e.g. for arithmetic, need to be compiled when most of
    the interface infrastructure is dynamic by nature.

    % When different algorithms are available, some of them coded in
    % \Sage, some of them in \GAP, the interface shall offer an easily
    % navigable interface for the expert user to choose among them.
  \end{task}

  \begin{task}[title=Modularisation and packaging,id=mod-packaging,lead=UV,PM=20,partners={PS,LL,UG},wphases=0-48]
    % TODO: logilab can contribute to producing VM images
    % using its proven worflow based on saltstack and packer.io PM=2
    % TODO: logilab can help to define the metadata to be provided
    % by software authors to facilitate the packaging of their
    % components. Beware not the reinvent the wheel^H packaging system.
    % PM=2
    % TODO: if needed, logilab can develop a package index similar
    % to PyPI using CubicWeb: html UI for browsing and web service
    % for registration. Maybe an instance of PyPI is enough? PM=6
    % TODO: logilab can help with debian packages PM=6
    In this task we will investigate best practices for composing,
    sharing and interfacing computational components and data for
    connected mathematical systems.

    We will start with a comparative study of the practices adopted in
    various open source projects, both inside and outside of
    \TheProject. This will include reviewing non-mathematical systems,
    e.g.: operating systems, platforms, web frameworks, cloud and HPC
    infrastructures.  In particular, pushed by cloud computing,
    containerisation \cite{Docker} and virtualisation
    \cite{Virtualbox} have become a major trend for distributing
    complex software, thanks to their ease of installation and
    configuration. We shall experiment with these technologies by
    building and distributing virtual appliances for the major
    components of \TheProject
    (\delivref{component-architecture}{virtual-machines}).

    Once the initial study will have identified the present
    shortcomings, we will promote a new generation of mathematical
    software that is capable of scaling to large code bases, large
    datasets, and massively distributed infrastructures. This task
    also needs to consider the results of work
    package~\WPref{social-aspects} on social issues regarding
    distributed development, community management, acknowledging
    contributions, etc.

    As an example, \Sage has a long history of integrating and
    distributing large mathematical libraries/software as a whole,
    with relatively few attention given to defining and exposing
    interfaces. Component re-usability is not a main focus for the
    \Sage community, at the same time the non-standard and relatively
    underused package system discourages writing and maintaining
    autonomous libraries. These factors have contributed to make the
    \Sage distribution what is usually described as a ``monolith''
    (\Sage library code alone, not counting included libraries, makes
    up for 1.5M lines of code and documentation), hard to distribute,
    to maintain, to port, and to develop with.

    On the other hand, \GAP has been distributing
    community-developed ``\GAP packages'' for a long time, but faces
    now fragmentation issues, at the code and at the community
    level. The rudimentary package system adds more technical
    difficulties to \GAP's development model.

    Both models reach the limits of their scalability, and a synthesis
    is very much needed.  Our first experiment will be to enhance
    \Sage's package system
    (\delivref{component-architecture}{sage-repository}), enough to
    support an open repository of user-contributed code, in the same
    spirit of modern systems such as Julia
    (\url{http://pkg.julialang.org/}) or PyPI
    (\url{https://pypi.python.org/}).  Once \emph{internal} packaging
    has been dealt with, the route will be paved to further modularise
    the \Sage distribution, and make sure that the major Linux
    distributions have standard packages for it
    (\delivref{component-architecture}{sage-distribution}).

  \end{task}

\begin{task}[id=simulagora-dev,title=Simulagora integration,PM=4,lead=LL,wphases=0-48]
  To deliver every six month a new Simulagora VM image containing all the software
  components released over the period. The goal is to prove that the project is
  improving the component architecture by measuring the time it takes to
  integrate them.
\end{task}


  \begin{task}[title=Component architecture for High Performance Computing and Parallelism,id=component-for-HPC,PM=12,wphases=36-48,lead=UJF,partners=SA]
    As in all other areas of science, properly supporting massively
    parallel architecture is a major challenge. Many of the
    computational components have already gone a long way in this
    direction, and further work will be carried out in
    WorkPackage~\WPref{hpc}.

    In this task we will investigate and implement
    parallelism-friendly ways of combining components together, so
    that calling components can benefit from the parallelism features
    of called components, with self-adaptation to the environment and
    cooperative sharing of resources. We will use \Sage and its
    components as a test-bed, by producing an HPC-enabled distribution
    (\delivref{component-architecture}{hpc-configure}).
  \end{task}

  \begin{task}[title=Document and modularise \SMC's codebase,id=extract-smc,lead=PS,PM=10,partners={UV,UG},wphases=0-24]
    From its inception in 2013, \SMC\ (see Section~\ref{linked-projects}, page~\pageref{sec:SMC-page}) 
    has quickly developed into a full
    featured VRE.  Because of the tight, partly closed source
    development cycles, \SMC's codebase has quickly grown in size,
    with its documentation not always keeping the pace. As a result,
    it is at the moment very hard for a newcomer to set up a clone
    service of \SMC just from its sources.

    Now that \SMC is
    \href{https://twitter.com/sagemath/status/544939872294014977}{fully
      open source}, we need to go through its codebase, understand and
    document it
    (\delivref{component-architecture}{smc-documentation}), isolate
    components that might be reused by other software (e.g.:
    \Jupyter), and make it as portable as possible.

    The ultimate goal of this task is to produce a \emph{personal}
    version of \SMC, to be shipped along with \Sage, that a user can
    run on his own personal computer
    (\delivref{component-architecture}{personal-smc}).
  \end{task}

  \begin{task}[title=Improving the development workflow in mathematical software,id=workflow,lead=UV,PM=10,partners={PS,LL,UG},wphases=6-24]
    Truly open software must enable any actor to easily contribute his
    work to the community. Be it an experienced developer, or a
    student. Be it for a major software component or for a piece of
    translation. All the systems involved in \TheProject have
    developed their own workflows for contributing back, but those are
    almost exclusively geared toward experienced developers working on
    large components. When these workflows eventually reach their
    scalability limits, software development stagnates and major
    features are delayed. A well known example is given by \Sage's TRAC
    server, where tickets can stay in ``needs review'' state for a
    long time before entering the code base.  \emph{Upstream} bug
    reporting and fixing is another major factor of slow development.

    This task will seek new ways of accepting contributions to
    mathematical software in a scalable way. For example we will
    experiment with integrating bug reporting and contributing
    features right in the VRE (e.g., in \SMC:
    \delivref{component-architecture}{smc-trac}).).

    % TODO: logilab would like to enhance the existing forges to publish
    % linked open data and ease the sharing of information about
    % package/version/issues across systems. See for example
    % https://packages.qa.debian.org/p/python-defaults.html and the link to RDF
    % meta-data on the right
    % https://packages.qa.debian.org/p/python-defaults.ttl
    % see also https://wiki.debian.org/RDF
    % PM=6 up to 12
  \end{task}


\begin{task}[lead=USO,id=oommf-python-interface,title=Python interface for OOMMF micromagnetic simulation library,PM=6,wphases=7-13,partners={SA}]
  % 6 person months
  In this task, we create a Python interface
  for the open source Object
  Oriented MicroMagnetic Framework (OOMMF \cite{OOMMF-url}).
  %which is the most widely used micromagnetic simulation package
  %\cite{OOMMF-citations-url}. 
  As a result, the OOMMF library will be fully accessible and usable
  from a Python interface and become a component in the
  Python/\Jupyter eco system of computational tools and in
  \TheProject. We make use of this component architecture in
  \taskref{UI}{oommf-py-ipython-attributes} to build the micromagnetic
  VRE demonstrator
  (Sect.~\ref{sec:introduction-micromagnetic-vre-demonstrator}).

  In more detail, we will first identify the best option for interfacing
  from Python to OOMMF core (C++) routines. The technical options
  include CTypes, Cython, Swig, and Boost-Python, all with particular
  advantages/disadvantages. Following analysis of the current OOMMF
  code layout and compilation model, we will use the most suitable
  tool, bearing in mind our ambition not to modify the OOMMF code so
  that the python interface we create remains functional and
  maintainable with minimal effort while the OOMMF core code is
  developed further by the OOMMF authors. 
  The interface will expose the C++ objects in Python, providing
  an architecture component that provides full access to OOMMF's
  raw capabilities. For clarity, we will refer to this interface as
  \texttt{OOMMF-py-raw}. %Creation of this \texttt{OOMMF-py-raw} is
  %technically doable as OOMMF had been written allowing to do this
  %from Tcl. The \texttt{OOMMF-py-raw} library for Python provides
  %access to the OOMMF functionality but requires some care when being
  %used.

  Secondly, we will create a user friendly Python library
  \texttt{OOMMF-py} that combines the \texttt{OOMMF-py-raw}
  capabilities in an object orientated and safe-to-use Python library
  targeting researchers in the magnetic materials community. We will
  follow the design of the well-received high level Python interface
  in the Nmag micromagnetic simulation package \cite{Fischbacher2007a}
  interface \cite{Nmag-url}. Unit and regression tests for both
  component interfaces \texttt{OOMMF-py-raw} and \texttt{OOMMF-py} are
  simultanously developed.

  %Once this is completed, several new features will be available to
  %OOMMF users: (i) ability to drive OOMMF from Python, (ii)
  %computational steering, and (iii) combination of OOMMF simulation
  %with the existing Python eco-system of computational tools.

  %Can remove the next paragraph if we are pushed for space.

  %We illustrate the advantage of (iii) through an example: to solve
  %the micromagnetic standard problem 3
  %\cite{Micromagnetic-Standardproblem-3}, traditionally multiple OOMMF
  %simulation runs would have to be conducted, and for each of those a
  %new configuration file as to be written. Between these the size of
  %the simulated geometry needs to be modified until two particular
  %values of energy are the same. Given the new interface developed in
  %this work package, this whole process can be replaced by one Python
  %script that creates multiple OOMMF simulations, combined with a root
  %finding method for the automatic iterative determination of the
  %required simulation geometry.

  %For all tasks relating to
  %\OOMMFNB, documentation and tests are created simultaneously with
  %the code. All codes, tests and documentation will be made available as open source.

  %We anticipate to start this task \localtaskref{oommf-python-interface}
  %in month 4, leading to deliverable \delivref{UI}{oommf-py}.
\end{task}





\end{tasklist}

  \begin{wpdelivs}
    % \begin{wpdeliv}[due=6,miles=startup,id=portability-cygwin32,dissem=PU,nature=OTHER,lead=UV]
    %   {One-click install \Sage distribution for Windows with Cygwin 32bits}
    %   % JPF: this should take a few months of work
    %   % This 32bits version would work right away on Windows 64 bits with
    %   % Cygwin 32 bits; more work would be required for a version working on
    %   % a 64 bits of Cygwin.
    %   % JPF: I agree.
    % \end{wpdeliv}%


    \begin{wpdeliv}[due=6,miles=startup,id=virtual-machines,dissem=PU,nature=OTHER,lead=UV]
      {Virtual images and containers} Creation and distribution of
      preconfigured cloud oriented virtual machines/containers
      (e.g. Docker images) for \PariGP, \Sage, \SMC, \dots
      % Requires: licenses
      \TOWRITE{??}{Make this deliverable shorter.}
    \end{wpdeliv}
    \begin{wpdeliv}[due=18,miles=startup,id=smc-documentation,dissem=PU,nature=R,lead=PS]
      {Understand and document \SMC backend code.}
    \end{wpdeliv}%

    \begin{wpdeliv}[due=12,miles=startup,id=scscp-sage,dissem=PU,nature=OTHER,lead=SA]
      {Support for the \href{http://www.symbolic-computing.org/}{SCSCP} interface protocol
        in all relevant components (\Sage, \GAP, etc.) distribution}
    \end{wpdeliv}

    \begin{wpdeliv}[due=24,miles=proto1,id=personal-smc,dissem=PU,nature=OTHER,lead=PS]
      {\emph{Personal} \SMC: single user version of \SMC distributed
        with \Sage.}
    \end{wpdeliv}%

    \begin{wpdeliv}[due=24,miles=proto1,id=smc-trac,dissem=PU,nature=OTHER,lead=UV]
      {Integration between \SMC and \Sage's TRAC server}
    \end{wpdeliv}
    \begin{wpdeliv}[due=24,miles=proto1,id=sage-repository,dissem=PU,nature=OTHER,lead=UV]
      {Open package repository for \Sage} Refactor \Sage package and
      build system to support community-contributed packages
      installable via the web.
    \end{wpdeliv}

    \begin{wpdeliv}[due=24,miles=proto1,id=portability-cygwin,dissem=PU,nature=OTHER,lead=PS]
      {One-click install \Sage distribution for Windows with Cygwin 32bits and 64bits}

      % Participants involved: Paris Sud, Kaiserslautern, Saint Andrews, Bordeaux
      % Comments on this by Bill Hart
      % The big problems you will have on Windows 64 on Cygwin include:
      % 
      % * anything with assembly language -- the ABI is different on Windows, so
      % it'll need rewriting, or you can incur a performance penalty by using
      % generic C fallback code
      % * the memory allocator on Windows is not so great
      % * bugs exposed due to being on a different platform, e.g. segfaults due to
      % off-by-one errors that were masked by the granularity of malloc on Linux
      % * build issues, due to identifying Cygwin and using the correct header
      % files, which are often different on Cygwin than linux
      % * issues with PATH vs LD_LIBRARY_PATH
      % * Windows has a case insensitive file system
      % * EOL issues
      % * Windows is not able to rapidly create and delete files, which some
      % libraries (esp. test code) calls for
      % * memory limitations (many people using Windows are using laptops with
      % limited memory, only a portion of which is realistically available to
      % Cygwin)
      % * autotools versions that don't support Windows (usually autotools has a
      % release that is used in all the distributions, which doesn't work correctly
      % on Windows, and this is followed up by a version which has all the Windows
      % patches)
      % * building takes forever on Windows. Mingw2 has now gotten parallel build
      % working on Windows and the speed is within a factor of 5 of Linux. But I'm
      % not sure the improvements have propagated to Cygwin yet.
      % * Cygwin 64 is new, contains quite a few bugs still, and things keep
      % changing with every version as they try to get things right.
      % * Although projects will likely accept patches for Windows, they are less
      % likely to maintain support themselves. I would like to think Singular would
      % be an exception to this. And obviously flint and MPIR work on Windows (even
      % with MSVC as of the next version of flint -- or now if you use our bleeding
      % edge repo version).
      % 
      % Comments by Jean-Pierre on some of the above and mor:
      % * first things first: I already completely built Sage on Cygwin64, though it
      % was surely not completely functional.
      % * assembly: that's right, note that as far as Sage and it's dependencies are
      % concerned, only a few of them actually use assembler, and yes all of them
      % provide fallback generic C code IIRC
      % * PATH vs LD_...: basically the same problem as for Cygwin32, so it's already
      % been taken care of for the Cygwin32 port
      % case issue: not a problem IIRC
      % * EOL issues: I don't thing so, Cygwin is POSIX like
      % * autotools issues: most of Sage dependencies are now updated, I used to track
      % the few problematic ones in 2013
      % * time to build: not so long, sure longer than on a POWER7 machine, but I do
      % it on a usual x86_64 laptop running Debian within a Windows VM in a few hours!
      % what we actually really need is patch/build bots to test on Cygwin 32/64!
      % * upstream cooperation: I agree Windows is often a low priority issue, but
      % most teams have welcomed my Cygwin patches
    \end{wpdeliv}

    \begin{wpdeliv}[due=36,miles=community,id=multiplatform-buildbot,dissem=PP,nature=DEM,lead=UV]
      {Continuous integration platform for multi-platform build/test.}

      \Sage's \emph{buildbot} is a x86\_64/Linux specific platform for
      continuous building and testing. We will investigate the
      possibility of evolving it towards a multi-platform tool, and
      opening it to other mathematical software.
    \end{wpdeliv}%
    \begin{wpdeliv}[due=36,miles=community,id=semantic-interface-sage-gap,dissem=PU,nature=OTHER,lead=UO]
      {Semantic-aware \Sage interface to \GAP.}
    \end{wpdeliv}


    \begin{wpdeliv}[due=48,miles=eval,id=sage-distribution,dissem=PU,nature=OTHER,lead=UV]
      {Packaging for major Linux distributions} Make sure that \Sage and
      all the components it depends on (including \GAP,
      Linbox, \PariGP, Singular, \dots) have standard packages in the
      main Linux distributions.
    \end{wpdeliv}

    % lmonade has very similar objectives but uses the gentoo prefix whereas Linux distributions use very different packaging systems:
    % \begin{compactitem}
    % \item gentoo prefix (gentoo)
    % \item pacman (arch),
    % \item yum (redhat),
    % \item apt (debian),
    % \item easy\_install
    % \item Python index packaging (pip)
    % \end{compactitem}}
    \begin{wpdeliv}[due=48,miles=eval,id=hpc-configure,dissem=PU,nature=OTHER,lead=UJF]
      {HPC enabled \Sage distribution}
    \end{wpdeliv}

  \end{wpdelivs}

% \begin{verbatim}
% Raw material:

% Component Architecture
% ----------------------

% Recomputation connection belongs here?

% Collaboration with unreliable (or restricted!) networking connections
% (peer-to-peer, opportunistic syncing, 3rd world). This is technically
% interesting, and gets in support for non-networked working. Not sure
% if it belongs here or not.

% - Security concerns

% Goal: Fostering collaborations/integration between components in an open source ecosystem
% =============================================================================

% - How to make systems "cooperate" rather than "predate each other".
% - E.g. reduce the version issues

% - Foster collaboration with upstream libraries by sharing the
%   development and maintenance of the interfaces, typically as
%   standalone upstream Python bindings (e.g. py-Singular).

% - How to make it easy to develop simultaneously two interdependent
%   components (e.g. Sage+Singular)

% - Foster communication

% - Social aspect:
%   Credit, Citations, Recognition, Funding

% Documentation system
% ====================

% In which package?

% Improvements to Sphinx

% Sage heavily customises the Sphinx documentation system, hacking deep
% in it in some cases, with quite some duplication in some cases.
% Refactor the whole thing, generalizing and contributing back upstream
% as much as possible (e.g. parallel compilation).
% \end{verbatim}

\end{workpackage}

%%% Local Variables: 
%%% mode: latex
%%% TeX-master: "../proposal"
%%% End: 

%  LocalWords:  workpackage wphases wpobjectives wpdescription textbf emph tasklist archs
%  LocalWords:  Linbox delivref portability-cygwin delivref Sparc
%  LocalWords:  non-Linuces sparc muche Itanium OSes VMs Cygwin32 Cygwin64 explorate hpc
%  LocalWords:  Composability Macaulay2 WPref deployment-distrib wpdelivs wpdeliv dissem
%  LocalWords:  Sud segfaults autotools autotools Mingw2 mor multiplatform-buildbot Nmag
%  LocalWords:  buildbot scscp-sage lmonade gentoo pacman redhat debian smc-trac logilab
%  LocalWords:  compactitem worflow saltstack packer.io simulagora-dev Simulagora Jupyter
%  LocalWords:  extract-smc smc-documentation personal-smc TOWRITE oommf-python-interface
%  LocalWords:  micromagnetic oommf-py taskref oommf-py-ipython-attributes CTypes Cython
%  LocalWords:  introduction-micromagnetic-vre-demonstrator texttt OOMMF-py-raw texttt
%  LocalWords:  OOMMF-py-raw simultanously Micromagnetic-Standardproblem-3 OOMMFNB
%  LocalWords:  localtaskref Virtualbox
