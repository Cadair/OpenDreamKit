\documentclass[12pt]{article}

\usepackage{hyperref}
\usepackage[utf8]{inputenc}
\usepackage{ae,aecompl,aeguill}	% pour utiliser << et >>
\usepackage{times}
\usepackage[babel=true,kerning=true]{microtype}
\usepackage{xspace}
\usepackage[show]{ed}

\newcommand{\software}[1]{\textsc{#1}\xspace}
\newcommand{\Sage}{\software{Sage}}

\title{Draft Data Management Plan for OpenDreamKit}
\author{Benoît Pilorget (Editor)}

\begin{document}

\maketitle
\tableofcontents\newpage

\section{Introduction}
\ednote{MK@Beno\^it, please write somem intro here what this document wants to do }

\section{Datasets}

\subsection{UPSud}

Most of the data created by UPsud is related to the software \Sage and will be incorporated into the \Sage codebase. There might also be smaller data sets of tutorials, documentation and teaching content independent of the \Sage codebase which will be stored accordingly to the size and needs.
\begin{description}
\item[Data storage and security] All addition to the the \Sage codebase will be stored within the distributed \Sage repository on the trac server \href{http://trac.sagemath.org/}{trac.sagemath.org}. For smaller datasets, we might use other distributed git repositories and store a central clone on platforms such as github. All the present data is public, and there is no concern about
unauthorised access. Through cloud hosting and local clones of
repositories, there are backups and redundancy.
\item[Dissemination] The \Sage codebase is publicly accessible through he trac server \href{http://trac.sagemath.org/}{trac.sagemath.org} and distributed within the \Sage software. For other data, we have an open access and open source policy and will advertise the data sets accordingly.
\item[Preservation and future access] By using the distributed system git to manage most of our data, we assure a local copy of the data within each participant machine. We rely on external platform (trac and github) for public access. If it should happen that those platforms are not available any more, the data can easily be moved away to another platform.
\end{description}


\begin{enumerate}


\item{Example}


\begin{description}
\item[Name of data] Photos from scottish landscapes
\item[Licence]  ??
\item[Nature of data] Image, photo
\item[Reuse of existing data] We used old photos from 50 decades ago as well as paintings
\item[Mean of production] Camera
\item[Data standard] Photos are in black and white and printed with particular paper and ink
\item [Usage for further experiments] Photos will be put on this website so that they can be viewed by the max people. For this they will need to use that software
\item [Link] ??
\end{description}


\item {Dataset 1}


\begin{description}
\item[Name of data] Addition to the \Sage codebase
\item[Nature of data] Software code
\item[Licence] GPL
\item[Reuse of existing data] The data is added to the already large existing \Sage codebase.
\item[Mean of production] Code implementation by UPsud participants.
\item[Data standard] The code is mostly written in Python, also using the Rest syntax for documentation and \Sage coding conventions.
\item [Usage for further experiments] The code is merged in the software and can be distributed and reused through the Software. Through the git history,
one can trace back older versions of the code and re-enable a former state of the software.
\item [Link] \href{http://trac.sagemath.org/}{trac.sagemath.org}
\end{description}


\item{Dataset 2}


\begin{description}
\item[Name of data] The OpenDreamKit website
\item[Nature of data] Text and metadata concerning OpenDreamKit participants and activities
\item[Reuse of existing data]
\item[Mean of production] Written by OpenDreamKit participants
\item[Data standard] Source code is written in Markdown language and converted into html.
\item [Usage for further experiments]
\item [Link] \href{http://opendreamkit.org/}{opendreamkit.org}, \href{https://github.com/OpenDreamKit/OpenDreamKit.github.io}{https://github.com/OpenDreamKit/OpenDreamKit.github.io}
\end{description}

\end{enumerate}

\subsection{CNRS}

This subsection will contain all datasets the CNRS is currently able to describe
\begin{description}
\item[Data storage and security] Quickly explain how data are stored and protected within yout institution
\item[Dissemination] How data can be disseminated -> openaccess etc
\item[Preservation and future access] How data can be preserved and available in the next years
\end{description}


\subsection{Jacobs University}

The data created by the Jacobs University team will be in the form of OMDoc/MMT
flexiformalizations (representations of mathematical knowledge and data at flexible levels
of formality). Most data will be generated by transforming and semantic preloading of
existing data sources (the mathematical data bases from WP6.)

All data will be hosted publically on the MathHub portal (\url{http://mathhub.info}), a
decicated information portal for active documents and data (flexiformal knowledge with
integrated semantic services). MathHub data is stored, versioned, and protected by the
state-of-the art GIT system.

Original data and will be licensed under an open knowledge license (see
)\url{http://opendefinition.org}, transformed data will be licensed as open as the
original license allows it.

\subsection{University of Southampton}

There are no significant data sets associated with the work at Southampton. The most important data is resulting code and associated documentation and tutorials. The details below refer to this data set, and we expect the data set to be fairly small (order of 1 GB).
\begin{description}
\item[Data storage and security] Data Storage: The code is stored in a distributed repository (git at the moment), and a central clone of this repository is stored with Github.com in the cloud. We may use multiple repositories, and store a central copy of each on Github.com.

  Security: All all the code is public, and there no concern about unauthorised access. Through cloud hosting and local clones of repositories, there are backups and redundancy.
\item[Dissemination] Data can be accessed through the public repositories, and the public website (probably this URL: \href{http://joommf.github.io}{http://joommf.github.io}, tbc), providing open access.
\item[Preservation and future access] We rely on provision of the data through \href{github.com}{github.com} but maintain local copies of the repository in case github.com ceases to exist or suffers from catastrophic technology failure. It is likely that other online repository hosting providers would be able to fill the gap (bitbucket.org is an existing alternative). The University of Southampton offers long term storage of small data sets for 10 years -- the repositories would fall into this categories. While the data wouldn't be conveniently accessible, this provides an extra layer of backups, from which accessible repositories and websites could be created easily.
\end{description}



\end{document}

%%% Local Variables:
%%% mode: latex
%%% TeX-master: t
%%% End:
