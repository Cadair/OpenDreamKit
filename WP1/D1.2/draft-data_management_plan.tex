\documentclass[12pt]{amsbook}


\usepackage[utf8]{inputenc}
\usepackage{ae,aecompl,aeguill}	% pour utiliser << et >>
\usepackage{times}
\usepackage[babel=true,kerning=true]{microtype}

\title{Draft Data Management Plan for OpenDreamKit}
\author{Benoît Pilorget}

\begin{document}

\maketitle

\section{Datasets}



\subsection{UPSud}

This subsection will contain all datasets the UPSud is currently able to describe.
\begin{description}
\item[Data storage and security] Quickly explain how data are stored and protected within yout institution
\item[Dissemination] How data can be disseminated -> openaccess etc
\item[Preservation and future access] How data can be preserved and available in the next years
\end{description}


\begin{enumerate}


\item{Example}


\begin{description}
\item[Name of data] Photos from scottish landscapes
\item[Nature of data] Image, photo
\item[Reuse of existing data] We used old photos from 50 decades ago as well as paintings
\item[Mean of production] Camera
\item[Data standard] Photos are in black and white and printed with particular paper and ink
\item [Usage for further experiments] Photos will be put on this website so that they can be viewed by the max people. For this they will need to use that software
\end{description}


\item {Dataset 1} 


\begin{description}
\item[Name of data] 
\item[Nature of data] 
\item[Reuse of existing data] 
\item[Mean of production]
\item[Data standard]
\item [Usage for further experiments] 
\end{description}


\item{Dataset 2}


\begin{description}
\item[Name of data] 
\item[Nature of data]
\item[Reuse of existing data]
\item[Mean of production] 
\item[Data standard]
\item [Usage for further experiments] 
\end{description}

\end{enumerate}

\subsection{CNRS}

This subsection will contain all datasets the CNRS is currently able to describe
\begin{description}
\item[Data storage and security] Quickly explain how data are stored and protected within yout institution
\item[Dissemination] How data can be disseminated -> openaccess etc
\item[Preservation and future access] How data can be preserved and available in the next years
\end{description}


\subsection{Jacobsuni}



This subsection will contain all datasets the Jacobsuni is currently able to describe
\begin{description}
\item[Data storage and security] Quickly explain how data are stored and protected within yout institution
\item[Dissemination] How data can be disseminated -> openaccess etc
\item[Preservation and future access] How data can be preserved and available in the next years
\end{description}


\end{document}
