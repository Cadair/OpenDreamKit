\section{Conclusion}\label{sec:concl}
In this paper we have presented the \ODK project and the ``Math-in-the-Middle'' approach
it explores for solving the system integration problems inherent in combining an ecosystem
of open source software systems into a coherent mathematical virtual research environment.
The MitM approach relies on a central, curated, flexiformal ontology of the mathematical
domains to be covered by the VRE together with system-near interface theories and
interviews to the core ontology that liaise with the respective systems. We have reported
on two case studies that were used to evaluate the approach: an interface the for \LMFDB,
and a more semantic interface between \GAP and \Sage. 

Even though the development of the MitM is still at a formative stage, these case studies
show the potential of the approach. We hope that the nontrivial cost of curating an
ontology of mathematical knowledge and interviews to the interface theories will be offset
by its utility as a resource, which we are currently exploring: the unification of the
knowledge representation components
\begin{compactitem}
\item allows VRE-wide domain-centred (instead of system-centered) documentation;
\item can be leveraged for distributed computation via uniform protocols like the
  SCSCP~\cite{HorRoz:ossp09} and MONET-style service
  matching~\cite{CaprottiEtAl:MathServiceMatching04:tr} (the absence of content
  dictionaries -- MitM theories -- was the main hurdle that kept these from gaining more
  traction);
\item will lead to best practices in mathematical knowledge management be adopted in the
  systems involved -- in fact, this is already happening.
\end{compactitem}
Whether in the end the investment into the MitM will pay off also depends on the quality
and usability of the tools for mathematical knowledge management. Therefore we invite the
CICM community to interact with an contribute to the \ODK project. 

\subsubsection*{Acknowledgements}

The authors gratefully acknowledge the other participants of the St~Andrews workshop, in
particular John Cremona, Luca de Feo, Mihnea Iancu, Steve Linton, Dennis M\"uller, Viviane
Pons, Florian Rabe, and Tom Wiesing, for discussions and experimentation which clarified
the ideas behind the math-in-the-middle approach.

We acknowledge financial support from the OpenDreamKit Horizon 2020 European Research
Infrastructures project (\#676541), from the EPSRC Collaborative Computational Project
CoDiMa (EP/M022641/1) and from the Swiss National Science Foundation grant PP00P2\_138906.

%%% Local Variables:
%%% mode: latex
%%% TeX-master: "paper"
%%% End:

%  LocalWords:  ednote subsubsection Dehaye Mihnea Iancu Konovalov Leli evre Wiesing
%  LocalWords:  concl flexiformal compactitem ossp09
