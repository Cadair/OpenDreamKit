\documentclass{article}
\usepackage[show]{ed}
\usepackage{odksystems}

\title{Deleted Scenes from the OpenDreamKit paper}
\author{all}
\begin{document}
\maketitle


\subsection{Stuff from the Intro}

Computational experiments have led to new conjectures which have had a deep impact on the
future development of mathematics. An outstanding example is the Birch and Swinnerton-Dyer
conjecture (one of the Clay Millennium Problems).  Databases relying on computer
calculations such as the Small Groups Library in GAP, the Modular Atlas in group and rep-
resentation theory, or the LMFDB, provide indispensable tools for researchers. A
constructive way of understanding proofs of deep theorems yields algorithmic tools to deal
with highly abstract concepts. These tools make the concepts available to a broader class
of researchers, with many potential applications. A prominent example from algebraic
geometry is the desingularisation theorem of Hironaka, for which Hironaka won the Fields
Medal, and its algorithmisation by Villamayor.

\subsection{Existing Virtual Research Environments for Mathematics}

\begin{oldpart}{MK: this should shortened, also we must mention Mathematica, Maple, what
    else? What are their common parts?}
  Early VRE's include \SMC, a collaborative online environment where students, teachers
  and researchers can create, customize, and share a project. This project essentially
  consists of a virtual machine, with a simple web-based user interface, and ready-to-use
  software for interactive computations (\emph{e.g.}\ \Sage) and authoring (\emph{e.g.}\
  \LaTeX), with facilities for real-time communication through chat, video, and shared
  editing of documents, programs and worksheets.  For education purposes, course material
  can be provided as worksheets, assignments can be distributed, collected, and returned
  as well.

  Technically speaking, \SMC is a specific open-source cloud-based Virtual Research and
  Teaching Environment for mathematics developed since 2013 under the lead of William
  Stein, with funding from the NSF, and Google's Education Grant program.  \ednote{NT:
    Mention SageMath, Inc.?}  \ednote{SL: 2016: SageMath, Inc.\ accepted in Google startup
    program, see https://twitter.com/wstein389/status/708439137200836608}

  It presently hosts over 250,000 projects and has over 12,000 weekly active users. This
  fast adoption by a wide variety of users demonstrates the relevance and the long-term
  impact this kind of collaborative environments can have.
\end{oldpart}

\end{document}



%%% Local Variables:
%%% mode: latex
%%% TeX-master: t
%%% End:

%  LocalWords:  maketitle Swinnerton-Dyer resentation desingularisation Hironaka Hironaka
%  LocalWords:  algorithmisation Villamayor oldpart emph emph ednote
