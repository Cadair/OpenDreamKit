\subsection{Exploring GAP types}

\subsubsection{Brief introduction to GAP type system}\label{gap-types-intro}

\ednote{TODO: This is just the citation from \url{http://www.gap-system.org/Manuals/doc/ref/chap13.html}}

``Every GAP object has a type. The type of an object is the information 
which is used to decide whether an operation is admissible or possible 
with that object as an argument, and if so, how it is to be performed.

For example, the types determine whether two objects can be multiplied 
and what function is called to compute the product. Analogously, the 
type of an object determines whether and how the size of the object 
can be computed. It is sometimes useful in discussing the type system, 
to identify types with the set of objects that have this type. Partial 
types can then also be regarded as sets, such that any type is the 
intersection of its parts.

The type of an object consists of two main parts, which describe 
different aspects of the object.

The family determines the relation of the object to other objects. 
For example, all permutations form a family. Another family consists 
of all collections of permutations, this family contains the set of 
permutation groups as a subset. A third family consists of all 
rational functions with coefficients in a certain family.

The other part of a type is a collection of filters (actually stored 
as a bit-list indicating, from the complete set of possible filters, 
which are included in this particular type). These filters are all 
treated equally by the method selection, but, from the viewpoint of 
their creation and use, they can be divided (with a small number of 
unimportant exceptions) into categories, representations, attribute 
testers and properties. Each of these is described in more detail below.

A discussion of the type system can be found in \cite{breuer-linton}
(\url{http://dl.acm.org/citation.cfm?id=281540})''

\ednote{TODO (???) Compare and contrast it with the Sage type system}

\subsubsection{Exporting GAP types}\label{gap-types-export}

Information about types of all objects in the GAP workspace after the system
is loaded may be accessed programmatically. It may be helpful for GAP developers,
package authors and users, who may have questions like 
``How can I get all methods installed for a particular operation'' or 
``Which attributes a group may have?''. 

We are developing tools to export this information in several possible ways:
\begin{itemize}
\item accessible in the GAP session
\item exported to a JSON file to be used by external software such as MMT or
SageMath
\item exported for visualisation (examples may be seen at 
\url{https://github.com/OpenDreamKit/OpenDreamKit/issues/165})
\end{itemize}

Prototypes were produced during GAP-Sage Days in St Andrews in January
\ednote{Supported by OpenDreamKit and CoDiMa projects - check that both are acknowledged}
and now were are going to incorporate the necessary functionality into
the next major release of GAP. We hope to enhance these tools with the 
GAP Jupyter interface developed in \url{https://github.com/gap-packages/jupyter-gap}.

\subsubsection{Consistency checker for GAP manuals}\label{gap-types}

Looking at the GAP type system and GAP manuals, we got some extra immediate benefits
helping us to improve the quality of GAP documentation. The main GAP manuals are using
the GAPDoc format, provided by the GAP package with the same name \cite{gapdoc}. It 
allows to specify the type of the documented object (function, operation, attribute,
property, etc.). However, due to the limitations of the semi-automated conversion from 
the GAP manuals in a different format used in GAP 4.4.12 and earlier, types were not
correctly stated for a number of objects. 

During GAP-Sage Days in January 2016 we implemented a manual 
consistency checker which checks and reports incorrect type specifications 
(see \url{https://github.com/gap-system/gap/pull/538}). 
It immediately reported almost 400 errors. The cleanup is ongoing 
(see e.g. \url{https://github.com/gap-system/gap/pull/675}) and by now it
remains less than a hundred of them.

This tool also performs other useful checks, for example, produces a list of
manual sections having no examples. 

%%% Local Variables:
%%% mode: latex
%%% TeX-master: "paper"
%%% End:
