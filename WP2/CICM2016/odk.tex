\section{The \ODK Project}\label{sec:odk}
The project ``Open Digital Research Environment Toolkit for the Advancement of
Mathematics'''~\cite{OpenDreamKit:on} is a European H2020 project funded under the
EINFRA-9 call~\cite{EINFRA-9} whose theme of \emph{Virtual Research Environments} was a
natural fit to seek for manpower and funding for a developer community that have been
working really hard on the items above. The \ODK consortium consists of core European
developers of the aforementioned systems for pure mathematics, and reaching toward the
numerical community, and in particular the \Jupyter community, to work together on joint
needs. The project aims to address the following goals in close collaboration with the
community:
\begin{compactenum}
\item Further improve the productivity of researchers in pure mathematics and applications
  by further promoting collaborations on \emph{Data}, \emph{Knowledge}, and
  \emph{Software}.
\item Make it easy for teams of researchers of any size to set up custom, collaborative
  \emph{Virtual Research Environments} tailored to their specific needs, resources, and
  workflows.
\item Support the entire life-cycle of computational work in mathematical research, from
  \emph{initial exploration} to \emph{publication}, \emph{teaching}, and \emph{outreach}.
\end{compactenum}
The acceptance of the proposal~\cite{ODKproposal:on} in May 2015 was a strong sign of
recognition, at the highest level of funding agencies, of the values of open science and
the strength and maturity of the ecosystem.

The \ODK projects~\cite{ODKproposal:on} will run for four years, starting from September
2015. It involves about 50 people spread over 15 sites in Europe, with a total budget of
about 7.6 million euros. The largest portion of that will be devoted to employing an
average of 11 researchers and developers working full time on the project. Additionally,
the participants will contribute the equivalent of six other people working full time.  By
definition this project will be mostly funding actions in Europe; however those actions
will be carried out, as usual, in close collaborations with the worldwide community
(potential users of the VRE as well as developers outside the \ODK consortium).

The \ODK work plan consists in 58 concrete tasks split in seven work packages, which
include:
\begin{description}
% \item[WP1] Project Management
% \item[WP2: Community Building, Training, Dissemination, Exploitation,
%   and Outreach]: organization of many development and training
%   workshops, writing of tutorials and documentation, ...
\item[WP3: Component Architecture] work on portability -- especially
  on the Windows platform -- modularity, packaging, distribution,
  deployment, standardization and interoperability between components.
\item[WP4: User Interfaces] work on uniform \Jupyter notebook
  interface for all interactive computational components, improvements
  to \Jupyter, 3D visualization, documentation tools, ...
\item[WP5: High Performance Mathematical Computing] work within
  and between ODK's components to improve performance, and in
  particular better exploit multicore / parallel architectures.
\item[WP6: Data/Knowledge/Software-Bases] \emph{e.g.}\ identification and extensions of ontologies
  and standards to facilitate safe and efficient storage, reuse, interoperation and
  sharing of rich mathematical data whilst taking into account of provenance and
  citability; data archiving and sharing in a semantically sound way component
  architecture; integration between computational software and databases.
\item[WP7: Social Aspects] research on social aspects of collaborative
  software/data/knowledge development in mathematics to inform other collaborations.
\end{description}
\ODK will also actively engage in community building and training by organizing workshops
and training materials.

An innovative aspect of the \ODK project is that its preparation and management happens,
as much as is practical and without infringing on privacy, in the open. For example, most
documents, including the proposal itself, are version controlled on public repositories
and progress on tasks and deliverables is tracked using public issues
(see~\cite{OpenDreamKit:on}). This has proven a strong feature to collaborate tightly with
the community and get early feedback.



%%% Local Variables:
%%% mode: latex
%%% TeX-master: "paper"
%%% End:

%  LocalWords:  specialized Arxiv Jupyter IPython ldots compactitem emph compactenum odk
%  LocalWords:  ODKproposal organization standardization visualization citability oldpart
%  LocalWords:  organizing Swinnerton-Dyer resentation desingularisation Hironaka ednote
%  LocalWords:  Hironaka algorithmisation Villamayor
