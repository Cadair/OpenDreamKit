\section{Towards a Mathematical Virtual Research Environment}

From their earliest days, computers have been used in pure mathematics, either to make
tables, to prove theorems (famously the four colour theorem) or, as with the astronomer's
telescope, to explore new theories. Computer aided experiments, and the use of databases
relying on computer calculations such as the Small Groups Library in GAP, the Modular
Atlas in group and representation theory, or the LMFDB, are now part of the standard
toolbox of the pure mathematician, and certain areas of mathematics completely depend on
it. Computers are also increasingly used to support collaborative work and education.

The last decades witnessed the emergence of a wide ecosystem of open-source tools to
support research in pure mathematics. This ranges from specialized to general purpose
computational tools such as \GAP, \PariGP, \Linbox, \MPIR, \Sage, or \Singular, via online
databases like the \LMFDB. Not counting of course online services like the Wikipedia,
\Arxiv, or MathOverflow. A great opportunity is the rapid emergence of key technologies,
and in particular the \Jupyter (previously \IPython) platform for interactive and
exploratory computing which targets all areas of science.

This has proven the viability and power of collaborative open-source development models,
by users and for users, even for delivering general purpose systems targeting a large
public (researchers, teachers, engineers, amateurs, \ldots). Yet some critical long term
investments, in particular on the technical side, are in order to boost the productivity
and lower the entry barrier:
\begin{compactitem}
\item Streamlining access, distribution, portability on a wide range of platforms, including
  High Performance Computers or cloud services.
\item Improving user interfaces, in particular in the promising area of collaborative
  workspaces as those provided by \SMC.
\item Lowering barriers between research communities and promote dissemination. For example
  make it easy for a specialist of scientific computing to use tools from pure
  mathematics, and reciprocally.
\item Bringing together the developers communities to promote tighter collaboration and
  symbiosis, accelerate joint development, and share best practices.
\item Outsourcing as much of the development as possible to larger communities to focus
  the work forces on their core specialty: the implementation of mathematical algorithms
  and databases.
\item And last but not least: Promoting collaborations at all scales to further improve
  the productivity of researchers in pure mathematics and applications.
\end{compactitem}
These can be subsumed by the goal of \emph{Virtual Research Environments} (VRE), that is
online services enabling groups of researchers, typically widely dispersed, to work
collaboratively on a per project basis. This is exactly where the \ODK project kicks in:


\subsection{The \ODK Project}
The project ``Open Digital Research Environment Toolkit for the Advancement of
Mathematics'''~\cite{OpenDreamKit:on} is a European H2020 project funded under the
EINFRA-9 call~\cite{EINFRA-9} whose theme of \emph{Virtual Research Environments} was a
natural fit to seek for manpower and funding for a developer community that have been
working really hard on the items above. The \ODK consortium consists of core European
developers of the aforementioned systems for pure mathematics, and reaching toward the
numerical community, and in particular the \Jupyter community, to work together on joint
needs. The project aims to help address the following aims in close collaboration with the
community:
\begin{compactenum}
\item Further improve the productivity of researchers in pure mathematics and applications
  by further promoting collaborations on \emph{Data}, \emph{Knowledge}, and
  \emph{Software}.
\item Make it easy for teams of researchers of any size to set up custom, collaborative
  \emph{Virtual Research Environments} tailored to their specific needs, resources, and
  workflows.
\item Support the entire life-cycle of computational work in mathematical research, from
  \emph{initial exploration} to \emph{publication}, \emph{teaching}, and \emph{outreach}.
\end{compactenum}
The acceptance of the proposal~\cite{ODKproposal:on} in May 2015 was a strong sign of
recognition, at the highest level of funding agencies, of the values of open science and
the strength and maturity of the ecosystem.

The \ODK projects~\cite{ODKproposal:on} will run for four years, starting from September
2015. It involves about 50 people spread over 15 sites in Europe, with a total budget of
about 7.6 million euros. The largest portion of that will be devoted to employing an
average of 11 researchers and developers working full time on the project. Additionally,
the participants will contribute the equivalent of six other people working full time.  By
definition this project will be mostly funding actions in Europe; however those actions
will be carried out, as usual, in close collaborations with the worldwide community
(potential users of the VRE as well as developers outside the \ODK consortium).

The \ODK work plan consists in 58 concrete tasks split in seven work packages, which
include:
\begin{description}
% \item[WP1] Project Management
% \item[WP2: Community Building, Training, Dissemination, Exploitation,
%   and Outreach]: organization of many development and training
%   workshops, writing of tutorials and documentation, ...
\item[WP3: Component Architecture] work on portability -- especially
  on the Windows platform -- modularity, packaging, distribution,
  deployment, standardization and interoperability between components.
\item[WP4: User Interfaces] e.g. work on uniform \Jupyter notebook
  interface for all interactive computational components, improvements
  to\Jupyter, 3D visualization, documentation tools, ...
\item[WP5: High Performance Mathematical Computing] e.g. work within
  and between ODK's components to improve performance, and in
  particular better exploit multicore / parallel architectures.
\item[WP6: Data/Knowledge/Software-Bases] e.g identification and extensions of ontologies
  and standards to facilitate safe and efficient storage, reuse, interoperation and
  sharing of rich mathematical data whilst taking into account of provenance and
  citability; data archiving and sharing in a semantically sound way component
  architecture; Integration between computational software and databases;
\item[WP7: Social Aspects] research on social aspects of collaborative
  software/data/knowledge development in mathematics to inform the
\end{description}
\ODK will also actively engage in community building and training by organizing workshops
and training materials.

An innovative aspect of the \ODK project is that its preparation and management happens,
as much as is practical and without infringing on privacy, in the open. For example, most
documents, including the proposal itself, are version controlled on public repositories
and progress on tasks and deliverables is tracked using public issues
(see~\cite{OpenDreamKit:on}). This has proven a strong feature to collaborate tightly with
the community and get early feedback.

% The completion of each task is validated by the release or publication
% of deliverables.

% The tasks, deliverables, and progress thereupon are
%tracked publicly on the


%Computational experiments have led to new conjectures which have had a deep impact on the future development of
%mathematics. An outstanding example is the Birch and Swinnerton-Dyer conjecture (one of the Clay Millennium Problems).
%Databases relying on computer calculations such as the Small Groups Library in GAP, the Modular Atlas in group and rep-
%resentation theory, or the LMFDB, provide indispensable tools for researchers. A constructive way of understanding proofs
%%of deep theorems yields algorithmic tools to deal with highly abstract concepts. These tools make the concepts available
%to a broader class of researchers, with many potential applications. A prominent example from algebraic geometry is the
%desingularisation theorem of Hironaka, for which Hironaka won the Fields Medal, and its algorithmisation by Villamayor.

\begin{oldpart}{MK: this should shortened, also we must mention Mathematica, Maple, what
    else? What are their common parts?}
\subsection{Existing Virtual Research Environments for Mathematics}

  Early VRE's include \SMC, a collaborative online environment where students, teachers
  and researchers can create, customize, and share a project. This project essentially
  consists of a virtual machine, with a simple web-based user interface, and ready-to-use
  software for interactive computations (e.g.\ \Sage) and authoring (e.g.\ \LaTeX), with
  facilities for real-time communication through chat, video, and shared editing of
  documents, programs and worksheets.  For education purposes, course material can be
  provided as worksheets, assignments can be distributed, collected, and returned as well.

  Technically speaking, \SMC is a specific open-source cloud-based Virtual Research and
  Teaching Environment for mathematics developed since 2013 under the lead of William
  Stein, with funding from the NSF, and Google's Education Grant program.  \ednote{NT:
    Mention SageMath, Inc.?}  \ednote{SL: 2016: SageMath, Inc.\ accepted in Google startup
    program, see https://twitter.com/wstein389/status/708439137200836608}

  It presently hosts over 250,000 projects and has over 12,000 weekly active users. This
  fast adoption by a wide variety of users demonstrates the relevance and the long-term
  impact this kind of collaborative environments can have.
\end{oldpart}



%%% Local Variables:
%%% mode: latex
%%% TeX-master: "paper"
%%% End:

%  LocalWords:  specialized Arxiv Jupyter IPython ldots compactitem emph compactenum
%  LocalWords:  ODKproposal organization standardization visualization citability oldpart
%  LocalWords:  organizing Swinnerton-Dyer resentation desingularisation Hironaka ednote
%  LocalWords:  Hironaka algorithmisation Villamayor
