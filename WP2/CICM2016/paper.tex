\documentclass{llncs}
\pagestyle{plain}
\usepackage[show]{ed}
% \usepackage{url}
% \usepackage{wrapfig}
% \usepackage{xspace}

\usepackage[style=alphabetic,backend=bibtex]{biblatex}
\addbibresource{kwarc.bib}

\title{The OpenDreamKit Project\\
  Towards Enhanced Interoperability\\
  via a Math-in-the-Middle Approach}
\author{Nicolas M. Thi\'ery\inst{1} Michael Kohlhase\inst{2}}
\institute{Universit\'e Paris-Sud, Paris, France\and
Jacobs University, Bremen, Germany}

\begin{document}
\maketitle
\begin{abstract}
  OpenDreamKit --- ``Open Digital Research Environment Toolkit for the
  Advancement of Mathematics'' --- is an H2020 EU Research
  Infrastructure project that aims at supporting, over the period
  2015--2019, the ecosystem of open-source mathematical software
  systems, and in particular popular tools such as LinBox, MPIR,
  SageMath, GAP, Pari/GP, LMFDB, Singular, MathHub, and the
  IPython/Jupyter interactive computing environment. From that
  ecosystem, OpenDreamKit will deliver a flexible toolkit enabling
  research groups to set up Virtual Research Environments, customised
  to meet the varied needs of research projects in pure mathematics
  and applications.

  An important step in the OpenDreamKit endeavor is to foster the
  interoperability between a variety of systems, ranging from computer
  algebra systems over mathematical databases to front-ends.

  In this paper, we describe the OpenDreamKit project and report on experiments and future
  plans with the \emph{math-in-the-middle} approach.  This information architecture
  consists in a central mathematical ontology that documents the domain and fixes a joint
  vocabulary, combined with specifications of the functionalities of the various
  systems. Interaction between systems can then be enriched by pivoting off this
  information architecture.
\end{abstract}

\section{Towards a Mathematical Virtual Research Environment}
\ednote{MK@NT: we need a general description here, cite \cite{OpenDreamKit:on,ODKproposal:on}}

\section{Integrating Mathematical Software Systems via the Math-in-the-Middle Approach}
\ednote{MK: essentially the St. Andrews stuff}
\section{Conclusion}
\ednote{MK: I am not sure we need this, but we will see}
\subsubsection*{Acknowledgements}
The authors gratefully acknowledge discussions and experimentation at the St.\ Andrews
workshop, which clarified the ideas behind the math-in-the-middle approach;
in particular John Cremona, Paul-Olivier Dehaye, Luca de Feo, Mihnea Iancu, Alexander
Konovalov, Samuel Leli\`evre, Steve Linton, Dennis M\"uller, Markus Pfeiffer, Viviane Pons,
Florian Rabe, and Tom Wiesing.

We acknowledge financial support from the OpenDreamKit Horizon 2020 European Research
Infrastructures project (\#676541).

\printbibliography
\end{document}
%%% Local Variables:
%%% mode: latex
%%% TeX-master: t
%%% End:

%  LocalWords:  maketitle endeavor ednote ODKproposal printbibliography subsubsection
