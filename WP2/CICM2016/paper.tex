\documentclass[orivec]{llncs}
\pagestyle{plain}
\usepackage[show]{ed}
% \usepackage{url}
% \usepackage{wrapfig}
% \usepackage{xspace}

\usepackage[style=alphabetic,backend=bibtex]{biblatex}
\addbibresource{kwarc.bib}

\title{The OpenDreamKit Project}
\author{Nicolas Thi\'ery\inst{1} Michael Kohlhase\inst{2}}
\institute{Universit\'e Paris-Sud, Paris, France\and
Jacobs University, Bremen, Germany}

\begin{document}
\maketitle
\begin{abstract}
  We present the OpenDreamKit Project, and EU infrastructure project that aims at
  supporting and an ecosystem of open source mathematical sortware systems that forms a
  flexible toolkit from which research groups can set up mathematical virtual research
  environments. These can be customised to meet the varied needs of research projects in
  pure mathematics and applications, and supporting the full research life-cycle from
  exploration, through proof and publication, to archival and sharing of data and
  code. 

  An important step in the OpenDreamKit endeavor is to integrate the software systems into
  a toolkit by making them interoperable, we will present the Math-in-the-Middle
  approach. This information architecture complements the various systems -- ranging
  from computer algebra systems over mathematical data bases to front-ends -- with a
  central mathematical ontology that documents the domain, fixes a joint vocabulary, and
  allows to specify the functionalities of the various systems. Interaction between
  systems can then be implemented by pivoting off this ontology.
\end{abstract}

\section{Towards a Mathematical Virtual Research Environment}
\ednote{MK@NT: we need a general description here, cite \cite{OpenDreamKit:on,ODKproposal:on}}

\section{Integrating Mathematical Software Systems via the Math-in-the-Middle Approach}
\ednote{MK: essentially the St. Andrews stuff}
\section{Conclusion}
\ednote{MK: I am not sure we need this, but we will see}
\subsubsection*{Acknowledgements}
The authors gratefully acknowledge discussions and experimentation at the St. Andrews
workshop\ednote{MK: mention all people by name?} which clarified the ideas behind the
Math-in-the-Middle approach. We acknowledge financial support from the OpenDreamKit
Horizon 2020 European Research Infrastructures project (\#676541).

\printbibliography
\end{document}
%%% Local Variables:
%%% mode: latex
%%% TeX-master: t
%%% End:

%  LocalWords:  maketitle endeavor ednote ODKproposal printbibliography subsubsection
