\documentclass{llncs}
\pagestyle{plain}
\usepackage[show]{ed}
\usepackage{listings}
% \usepackage{url}
% \usepackage{wrapfig}
\usepackage{xspace}

\usepackage[style=alphabetic,backend=bibtex]{biblatex}
\addbibresource{kwarc.bib}
\def\pn{OpenDreamKit}
\title{The OpenDreamKit Project:\\
  Towards Enhanced Interoperability\\
  via a Math-in-the-Middle Approach}
\author{Nicolas M. Thi\'ery\inst{1} Michael Kohlhase\inst{2}}
\institute{Universit\'e Paris-Sud, Paris, France\and
Jacobs University, Bremen, Germany}

\newcommand{\OOMMFNB}{OOMMF-NB\xspace}
\newcommand{\VRE}{VRE\xspace}
\newcommand{\VREs}{VRE\xspace}
\newcommand{\software}[1]{\textsc{#1}\xspace}
\newcommand{\GAP}{\software{GAP}}
\newcommand{\HPCGAP}{\software{HPC-GAP}}
\newcommand{\libGAP}{\software{libGAP}}
\newcommand{\Singular}{\software{Singular}}
\newcommand{\Sage}{\software{Sage}}
\newcommand{\SageCombinat}{\software{Sage-Combinat}}
\newcommand{\MuPADCombinat}{\software{MuPAD-Combinat}}
\newcommand{\Sphinx}{\software{Sphinx}}
\newcommand{\SCSCP}{\software{SCSCP}}
\newcommand{\Python}{\software{Python}}
\newcommand{\IPython}{\software{IPython}}
\newcommand{\Jupyter}{\software{Jupyter}}
\newcommand{\Cython}{\software{Cython}}
\newcommand{\Pythran}{\software{Pythran}}
\newcommand{\Numpy}{\software{Numpy}}
\newcommand{\Pari}{\software{PARI}}
\newcommand{\PariGP}{\software{PARI/GP}}
\newcommand{\libpari}{\software{libpari}}
\newcommand{\GP}{\software{GP}}
\newcommand{\GPtoC}{\software{GP2C}}
\newcommand{\Linbox}{\software{LinBox}}
\newcommand{\LMFDB}{\software{LMFDB}}
\newcommand{\OpenEdX}{\software{OpenEdX}}
\newcommand{\Linux}{\software{Linux}}
\newcommand{\LATEX}{\software{\LaTeX}}
\newcommand{\SMC}{\software{SageMathCloud}}
\newcommand{\Simulagora}{\software{Simulagora}}
\newcommand{\KANT}{\software{KANT}}
\newcommand{\Magma}{\software{Magma}}
\newcommand{\Mathematica}{\software{Mathematica}}
\newcommand{\Maple}{\software{Maple}}
\newcommand{\Matlab}{\software{Matlab}}
\newcommand{\MuPAD}{\software{MuPAD}}
\newcommand{\MPIR}{\software{MPIR}}
\newcommand{\Arxiv}{\software{arXiv}}
\newcommand{\Givaro}{\software{Givaro}}
\newcommand{\fflas}{\software{fflas}}
\newcommand{\MathHub}{\software{MathHub}}
\newcommand{\FindStat}{\software{FindStat}}
\newcommand\DKS{\ensuremath{\mathcal{DKS}}\xspace}
\newcommand{\ODK}{\software{OpenDreamKit}}


\begin{document}
\maketitle
\begin{abstract}
  \ODK --- ``Open Digital Research Environment Toolkit for the
  Advancement of Mathematics'' --- is an H2020 EU Research
  Infrastructure project that aims at supporting, over the period
  2015--2019, the ecosystem of open-source mathematical software
  systems, and in particular popular tools such as LinBox, MPIR,
  SageMath, GAP, Pari/GP, LMFDB, Singular, MathHub, and the
  IPython/Jupyter interactive computing environment. From that
  ecosystem, OpenDreamKit will deliver a flexible toolkit enabling
  research groups to set up Virtual Research Environments, customised
  to meet the varied needs of research projects in pure mathematics
  and applications.

  An important step in the OpenDreamKit endeavor is to foster the
  interoperability between a variety of systems, ranging from computer
  algebra systems over mathematical databases to front-ends.

  In this paper, we describe the OpenDreamKit project and report on experiments and future
  plans with the \emph{math-in-the-middle} approach.  This information architecture
  consists in a central mathematical ontology that documents the domain and fixes a joint
  vocabulary, combined with specifications of the functionalities of the various
  systems. Interaction between systems can then be enriched by pivoting off this
  information architecture.
\end{abstract}

\section{Towards a Mathematical Virtual Research Environment toolkit}

\ednote{MK@NT: we need a general description here, cite \cite{OpenDreamKit:on,ODKproposal:on}}

\paragraph{Context}

\ednote{NT}{Needed?}

From their earliest days, computers have been used in pure
mathematics, either to make tables, to prove theorems (famously the
four colour theorem) or, as with the astronomer’s telescope, to
explore new theories. Computer aided experiments, and the use of
databases relying on computer calculations such as the Small Groups
Library in GAP, the Modular Atlas in group and representation theory,
or the LMFDB, are now part of the standard toolbox of the pure
mathematician, and certain areas of mathematics completely depend on
it. Computers are also increasingly used to support collaborative work
and education.

The last decades has witnessed the emergence of a wide ecosystem of
open source tools to support research in pure mathematics. This ranges
from specialized to general purpose computational tools such as \GAP,
\PariGP, \Linbox, \MPIR, \Sage, or \Singular, via online databases
like the \LMFDB, to interactive computational environments like
\Jupyter. Not counting of course online services like the Wikipedia or
arXiv.

\paragraph{\ODK's aims}

\begin{itemize}
\item Improve the productivity of researchers in pure mathematics and
  applications by further promoting collaborations on \emph{Data},
  \emph{Knowledge}, and \emph{Software}
\item Make it easy for teams of researchers of any size to set up custom,
  collaborative \emph{Virtual Research Environments} tailored to their
  specific needs, resources, and workflows
\item Support the entire life-cycle of computational work in
  mathematical research, from \emph{initial exploration} to
  \emph{publication}, \emph{teaching}, and \emph{outreach}
\end{itemize}

\paragraph{Why Virtual Research Environments?}

Promoting collaborations at all scales is key to further improve the
productivity of researchers in pure mathematics and applications.  An
important next step in that direction is that of \emph{Virtual
  Research Environments} (VRE), that is online services enabling
groups of researchers, typically widely dispersed, to work
collaboratively on a per project basis.
% , and supporting the entire
% life-cycle of computational work in mathematical research, from
% \emph{initial exploration} to \emph{publication}, \emph{teaching}, and
% \emph{outreach}.


Early VRE's include \SMC, a collaborative online environment
where students, teachers and researchers can create, customize, and
share a project. This project essentially consists in a virtual
machine, with a simple web-based user interface, and ready-to-use
software for interactive computations (e.g. \Sage) and authoring
(e.g. \LaTeX), with facilities for real-time communication through
chat, video, and shared editing of documents, programs and worksheets.
For education purposes, course material can be provided as worksheets,
assignments can be distributed, collected, and returned as well.

Technically speaking, \SMC is a specific open-source cloud-based
Virtual Research and Teaching Environment for mathematics developed
since 2013 under the lead of William Stein, with funding from the NSF,
and Google's Education Grant program (\ednote{NT}{Mention SageInc?}).
It presently hosts over 100,000 projects and 10,000 weekly active
users. This fast adoption by a wide variety of users demonstrates the
relevance and the long term impact this kind of collaborative
environments can have.


\paragraph{\ODK in practice}

\ednote{NT}{Explain why the VRE aim wraps up other critical needs for
  our ecosystem}
\begin{itemize}
\item Modularity, packaging, portability, ease and use and install;
\item Performances, including better support for parallel
  architectures (HPC, ...);
\item Integration between computational software and databases;
\item Long term sustainability.
\end{itemize}



This work will be backed by research on social aspects of
collaborative software development in mathematics

%Computational experiments have led to new conjectures which have had a deep impact on the future development of
%mathematics. An outstanding example is the Birch and Swinnerton-Dyer conjecture (one of the Clay Millennium Problems).
%Databases relying on computer calculations such as the Small Groups Library in GAP, the Modular Atlas in group and rep-
%resentation theory, or the LMFDB, provide indispensable tools for researchers. A constructive way of understanding proofs
%%of deep theorems yields algorithmic tools to deal with highly abstract concepts. These tools make the concepts available
%to a broader class of researchers, with many potential applications. A prominent example from algebraic geometry is the
%desingularisation theorem of Hironaka, for which Hironaka won the Fields Medal, and its algorithmisation by Villamayor.




The OpenDreamKit project aims to build a nimble and modular toolkit for computational
mathematics, built around open source software. This would enable easy set up of custom
collaborative environments tailored for specific needs, resources and workflows. These
environments should support the entire life cycle of computational work in mathematical
research, from initial exploration to publication, teaching and outreach.

%good sounding introduction themes from proposal:
data archiving and sharing in a semantically sound way component architecture, user
interfaces, deployment frameworks and standardisation and interoperability of software
systems.  collaborations based on software, data and knowledge

update range of existing open source software systems for seamless deployment identify and
extend ontologies and standards to facilitate safe and efficient storage, reuse,
interoperation and sharing of rich mathematical data whilst taking into account of
provenance and citability

% somewhere we should briefly present how GAP, sage, findstat and LMFDB store data

\section{Integrating Mathematical Software Systems via the Math-in-the-Middle Approach}

% Mathematics has a rich notion of data: it can be either numeric or symbolic data;
% knowledge about mathematical objects given as statements (definitions, theorems or
% proofs); or software that computes with these mathematical objects. All this data is
% really a common resource, and should be maintained as such


To achieve the goal of assembling the ecosystem of mathematical software systems in the
\pn project into a coherent mathematical VRE, we have to make the systems interoperable at
a mathematical level. In particular, we have to establish a common meaning space that
allows to share computation, visualization of the mathematical concepts, objects, and
models (COMs) between the respective systems. Building on this we can build a VRE with
classical IDE techniques. 

Concretely, the problem is that the software systems in \pn have their own representations
of and functionalities for the COMs involved. This starts with simple naming issues:
elliptic curves are represented by \lstinline|ec| in the LMFDB, and as
\lstinline|EllipticCurve| in SageMath, persists through the underlying data structures,
and becomes virulent at the level of algorithms, their parameters, and domains of
appliccability.


% *** maybe use this above *** 
    % \begin{omtext}[title=Advantages] well-known Open Source Software
    %   \begin{enumerate}
    %   \item Let the specialists do that they do best and like\lec{and avoid what the don't}
    %   \item collaboration exponentiates results
    %   \item competition fosters innovation \lec{+ no vendor lock-in}
    %   \end{enumerate}
    % \end{omtext}


%   \item 
%     \begin{omtext}[title=Problem]
%       does an elliptic curve mean the same in GAP, SAGE, LMFDB?
%       \begin{itemize}
%       \item otherwise delegating computation becomes unsound
%       \item storing data in a central KB becomes unsafe 
%       \item the user cannot interpret the results in an UI
%       \end{itemize}
%     \end{omtext}
% \item
%     \begin{omtext}[title=Idea]
%       Need a common meaning space for safe distributed computation in a VRE!
%     \end{omtext}

\ednote{MK: essentially the St. Andrews stuff}
% outline:
% - MMT: distributed nature, flexibility, 
% - 
\section{Conclusion}
\ednote{MK: I am not sure we need this, but we will see}
\subsubsection*{Acknowledgements}
The authors gratefully acknowledge discussions and experimentation at the St.\ Andrews
workshop, which clarified the ideas behind the math-in-the-middle approach;
in particular John Cremona, Paul-Olivier Dehaye, Luca de Feo, Mihnea Iancu, Alexander
Konovalov, Samuel Leli\`evre, Steve Linton, Dennis M\"uller, Markus Pfeiffer, Viviane Pons,
Florian Rabe, and Tom Wiesing.

We acknowledge financial support from the OpenDreamKit Horizon 2020 European Research
Infrastructures project (\#676541).

\printbibliography
\end{document}
%%% Local Variables:
%%% mode: latex
%%% TeX-master: t
%%% End:

%  LocalWords:  maketitle endeavor ednote ODKproposal printbibliography subsubsection
