\documentclass{llncs}
\pagestyle{plain}
\usepackage[show]{ed}
\usepackage{lststex}
\lstset{basicstyle=\sf,columns=fullflexible}
% \usepackage{url}
% \usepackage{wrapfig}
\usepackage{tikz,standalone}
\usetikzlibrary{backgrounds,shapes,fit,shadows,mmt}
\usepackage{wrapfig}
\usepackage{xspace}
\usepackage{hyperref}
\usepackage{stex-logo}
\usepackage[style=alphabetic,backend=bibtex]{biblatex}
\addbibresource{kwarc.bib}% do not change
\addbibresource{rest.bib}% add bibs here!
\def\pn{OpenDreamKit\xspace}
\title{The OpenDreamKit Project:\\
  Towards Enhanced Interoperability\\
  via a Math-in-the-Middle Approach}
\author{Michael Kohlhase\inst{1} Nicolas M. Thi\'ery\inst{2} }
\institute{Jacobs University, Bremen, Germany \and Universit\'e Paris-Sud, Paris, France}

\newcommand{\OOMMFNB}{OOMMF-NB\xspace}
\newcommand{\VRE}{VRE\xspace}
\newcommand{\VREs}{VRE\xspace}
\newcommand{\software}[1]{\textsc{#1}\xspace}
\newcommand{\GAP}{\software{GAP}}
\newcommand{\HPCGAP}{\software{HPC-GAP}}
\newcommand{\libGAP}{\software{libGAP}}
\newcommand{\Singular}{\software{Singular}}
\newcommand{\Sage}{\software{Sage}}
\newcommand{\SageCombinat}{\software{Sage-Combinat}}
\newcommand{\MuPADCombinat}{\software{MuPAD-Combinat}}
\newcommand{\Sphinx}{\software{Sphinx}}
\newcommand{\SCSCP}{\software{SCSCP}}
\newcommand{\Python}{\software{Python}}
\newcommand{\IPython}{\software{IPython}}
\newcommand{\Jupyter}{\software{Jupyter}}
\newcommand{\Cython}{\software{Cython}}
\newcommand{\Pythran}{\software{Pythran}}
\newcommand{\Numpy}{\software{Numpy}}
\newcommand{\Pari}{\software{PARI}}
\newcommand{\PariGP}{\software{PARI/GP}}
\newcommand{\libpari}{\software{libpari}}
\newcommand{\GP}{\software{GP}}
\newcommand{\GPtoC}{\software{GP2C}}
\newcommand{\Linbox}{\software{LinBox}}
\newcommand{\LMFDB}{\software{LMFDB}}
\newcommand{\OpenEdX}{\software{OpenEdX}}
\newcommand{\Linux}{\software{Linux}}
\newcommand{\LATEX}{\software{\LaTeX}}
\newcommand{\SMC}{\software{SageMathCloud}}
\newcommand{\Simulagora}{\software{Simulagora}}
\newcommand{\KANT}{\software{KANT}}
\newcommand{\Magma}{\software{Magma}}
\newcommand{\Mathematica}{\software{Mathematica}}
\newcommand{\Maple}{\software{Maple}}
\newcommand{\Matlab}{\software{Matlab}}
\newcommand{\MuPAD}{\software{MuPAD}}
\newcommand{\MPIR}{\software{MPIR}}
\newcommand{\Arxiv}{\software{arXiv}}
\newcommand{\Givaro}{\software{Givaro}}
\newcommand{\fflas}{\software{fflas}}
\newcommand{\MathHub}{\software{MathHub}}
\newcommand{\FindStat}{\software{FindStat}}
\newcommand{\Mongo}{\software{MongoDB}}
\newcommand\DKS{\ensuremath{\mathcal{DKS}}\xspace}
\newcommand{\ODK}{\software{OpenDreamKit}}
\newcommand{\GS}{\textcolor{red}{\fbox{M}}}
\newcommand{\RS}{\textcolor{blue}{\fbox{D}}}
\newcommand{\US}{\textcolor{green}{\fbox{O}}}
\newcommand\defemph[1]{\textbf{#1}}
\newcommand\cn[1]{\ensuremath{\mathsf{#1}}}


\begin{document}
\maketitle
\begin{abstract}
  \ODK --- ``Open Digital Research Environment Toolkit for the
  Advancement of Mathematics'' --- is an H2020 EU Research
  Infrastructure project that aims at supporting, over the period
  2015--2019, the ecosystem of open-source mathematical software
  systems, and in particular popular tools such as LinBox, MPIR,
  SageMath, GAP, Pari/GP, LMFDB, Singular, MathHub, and the
  IPython/Jupyter interactive computing environment. From that
  ecosystem, OpenDreamKit will deliver a flexible toolkit enabling
  research groups to set up Virtual Research Environments, customised
  to meet the varied needs of research projects in pure mathematics
  and applications.

  An important step in the OpenDreamKit endeavor is to foster the
  interoperability between a variety of systems, ranging from computer
  algebra systems over mathematical databases to front-ends.

  In this paper, we describe the OpenDreamKit project and report on experiments and future
  plans with the \emph{math-in-the-middle} approach.  This information architecture
  consists in a central mathematical ontology that documents the domain and fixes a joint
  vocabulary, combined with specifications of the functionalities of the various
  systems. Interaction between systems can then be enriched by pivoting off this
  information architecture.
\end{abstract}

\section*{Suggested restructuring of \emph{Integrating Mathematical Software Systems via the MITM approach} section}

\begin{description}
\item{MMT system}
\item{GAP} 
\begin{itemize}
\item Brief introduction to the GAP type system 
\item Tentative approaches to exporting GAP types 
\item n application: consistency checker for the GAP documentation
\end{itemize}
\item{SAGE} 
\begin{itemize}
\item Short description of Sage's category system 
\item Tentative approach to formalize Sage's categories in MMT 
\item An application: multisystem semantic handle interfaces
\end{itemize}
\item{LMFDB}
\begin{itemize}
\item Short description of LMFDB's goals and implementation 
\item Tentative approach to formalize LMFDB semantics in MMT
\end{itemize}
\end{description}

\section{Introduction}

% \ednote{NT: wondering if the next three paragraph would not fit better
%   in the odk section; the reader more interested by the MitM part may
%   be put off by this long discussion}

From their earliest days, computers have been used in pure mathematics, either to make
tables, to prove theorems (famously the four colour theorem) or, as with the astronomer's
telescope, to explore new theories. Computer aided experiments, and the use of databases
relying on computer calculations such as the Small Groups Library in GAP, the Modular
Atlas in group and representation theory, or the LMFDB, are now part of the standard
toolbox of the pure mathematician, and certain areas of mathematics completely depend on
it. Computers are also increasingly used to support collaborative work and education.

The last decades witnessed the emergence of a wide ecosystem of open-source tools to
support research in pure mathematics. This ranges from specialized to general purpose
computational tools such as \GAP, \PariGP, \Linbox, \MPIR, \Sage, or \Singular, via online
databases like the \LMFDB. Not counting of course online services like the Wikipedia,
\Arxiv, or MathOverflow. A great opportunity is the rapid emergence of key technologies,
and in particular the \Jupyter (previously \IPython) platform for interactive and
exploratory computing which targets all areas of science.

This has proven the viability and power of collaborative open-source development models,
by users and for users, even for delivering general purpose systems targeting a large
public (researchers, teachers, engineers, amateurs, \ldots). Yet some critical long term
investments, in particular on the technical side, are in order to boost the productivity
and lower the entry barrier:
\begin{compactitem}
\item Streamlining access, distribution, portability on a wide range of platforms, including
  High Performance Computers or cloud services.
\item Improving user interfaces, in particular in the promising area of collaborative
  workspaces as those provided by \SMC.
\item Lowering barriers between research communities and promote dissemination. For example
  make it easy for a specialist of scientific computing to use tools from pure
  mathematics, and reciprocally.
\item Bringing together the developers communities to promote tighter collaboration and
  symbiosis, accelerate joint development, and share best practices.
\item Outsourcing as much of the development as possible to larger communities to focus
  the work forces on their core specialty: the implementation of mathematical algorithms
  and databases.
\item And last but not least: Promoting collaborations at all scales to further improve
  the productivity of researchers in pure mathematics and applications.
\end{compactitem}
These can be subsumed by the goal of \emph{Virtual Research Environments} (VRE), that is
online services enabling groups of researchers, typically widely dispersed, to work
collaboratively on a per project basis. This is exactly where the \ODK project kicks in. 

We will introduce the \ODK project Section~\ref{sec:odk} to establish the context for the
``Math-in-the-Middle'' (MitM) integration approach described in
Section~\ref{sec:mitm}. The remaining sections then elucidate the approach by presenting
first experiments and refinements of the chosen integration paradigm:
Section~\ref{sec:lmfdb} details how existing knowledge representation and data structures
can be represented as MitM interface theories with a case study of equipping the \LMFDB
with a MitM-based programming interface.  Section~\ref{sec:gapsage} discusses system
integration between \GAP and \Sage and how this can be routed through a MitM
ontology. Section~\ref{sec:concl} concludes the paper and discusses future work.




%%% Local Variables:
%%% mode: latex
%%% TeX-master: "paper"
%%% End:

%  LocalWords:  specialized Arxiv Jupyter IPython ldots compactitem emph mitm lfmdb concl
%  LocalWords:  gaptypes

\section{The \ODK project (2015-2019)}\label{sec:odk}

The \ODK project runs for four years, starting in September 2015, and
involves about 50 people spread over 15 sites in Europe, with a total
budget of 7.6 million euros. The largest portion of that is devoted to
employing an average of 11 researchers and developers working full
time on the project, while the other participants contribute the
equivalent of six people working full time.

\ODK's funding is directed toward \emph{Virtual Research Environments}
(VRE), that is online services enabling groups of researchers,
typically widely dispersed, to work collaboratively on a per project
basis. Rather than constructing a large monolithic VRE, we have
designed our proposal around the long-term investments listed in the
previous section, working on the large scale yet modular integration
of mathematical software. Our end goal is a modular, interoperable,
and customisable VRE toolkit built out of relatively modest
components, and our approach to work on the grease to make this
work. According to the funding scheme, the project addresses, besides
its technical goals, aspects such as outreach, dissemination, or tools
to support teaching.

% \ednote{NT: The following paragraph is a political message; not sure
%   it fits there; in case we need to save a bit of space, that's one of
%   the first things to cut out}
An innovative aspect of the \ODK project is that its preparation and management happens,
as much as is practical and without infringing on privacy, in the open. For example, most
documents, including the proposal itself, are version controlled on public repositories
and progress on tasks and deliverables is tracked using public issues
(see~\cite{OpenDreamKit:on}). This has proven a strong feature to collaborate tightly with
the community and get early feedback.

In practice, \ODK's work plan consists of several wide breadth work packages: component architecture (modularity, packaging, distribution, deployment), user interfaces (\Jupyter interactive notebook interfaces, 3D visualization, documentation tools), high performance mathematical computing (especially on multicore/parallel architectures), a study of social aspects of collaborative software development, and a package on data/knowledge/software-bases.

The latter package focuses on the identification and extension of ontologies and standards to facilitate safe and efficient storage, reuse, interoperation and sharing of rich mathematical data, whilst taking provenance and citability into account. It will develop a component architecture for semantically sound data archival and sharing, and integrate computational software and databases. The aim is to  enable researchers to seamlessly manipulate mathematical objects across computational engines (e.g. switch algorithm implementations from one computer algebra system to another), front end interaction modes (database queries, notebooks, web, etc) and even backends (e.g. distributed vs.~local).

In this paper, we discuss the general approach chosen to develop this semantically aware component architecture. 

% The \ODK project is committed to working openly. Deliverables are tracked using public GitHub issues (see~\cite{OpenDreamKit:on}), which tightens the loop for early community feedback.

%%% Local Variables:
%%% mode: latex
%%% TeX-master: "paper"
%%% End:

%  LocalWords:  specialized Arxiv Jupyter IPython ldots compactitem emph compactenum odk
%  LocalWords:  ODKproposal organization standardization visualization citability oldpart
%  LocalWords:  organizing Swinnerton-Dyer resentation desingularisation Hironaka ednote
%  LocalWords:  Hironaka algorithmisation Villamayor

\section{Integrating Mathematical Software Systems via the Math-in-the-Middle Approach}\label{sec:mitm}

% Mathematics has a rich notion of data: it can be either numeric or symbolic data;
% knowledge about mathematical objects given as statements (definitions, theorems or
% proofs); or software that computes with these mathematical objects. All this data is
% really a common resource, and should be maintained as such


To achieve the goal of assembling the ecosystem of mathematical software systems in the
\ODK project into a coherent mathematical VRE, we have to make the systems interoperable at
a mathematical level. In particular, we have to establish a common meaning space that
allows to share computation, visualization of the mathematical concepts, objects, and
models (COMs) between the respective systems. Building on this we can build a VRE with
classical IDE techniques. 

\subsection{A Common Meaning Space for Interoperability}
Concretely, the problem is that the software systems in \ODK have their own representations
of and functionalities for the COMs involved. This starts with simple naming issues (\emph{e.g.}\
elliptic curves are named \lstinline|ec| in the \LMFDB, and as \lstinline|EllipticCurve|
in \Sage), persists through the underlying data structures (five-tuple of natural
numbers for the Weierstrass equation in the \LMFDB and \ednote{MK: how in \Sage? Find a
  system where this is really different          POD: Don t know that there is one, but the formats do change: list vs tuple for instance}), and becomes virulent at the level of
algorithms, their parameters, and domains of applicability.

To obtain a common meaning space for a VRE, we have the three well-known approaches in
Figure~\ref{fig:interop}.
\begin{figure}[ht]\centering
  \begin{tabular}{|c|c|c|}\hline
    peer to peer & open standard & industry standard\\\hline
    \documentclass{standalone}
\usepackage{tikz}
\begin{document}
    \begin{tikzpicture}
      \node[draw] (a) at (0,.3) {A};
      \node[draw] (b) at (1,.3) {B};
      \node[draw] (c) at (1.7,1) {C};
      \node[draw] (d) at (1.7,2) {D};
      \node[draw] (e) at (1,2.7) {E};
      \node[draw] (f) at (0,2.7) {F};
      \node[draw] (g) at (-.7,2) {G};
      \node[draw] (h) at (-.7,1) {H};
      \draw (a) -- (b) -- (c) -- (d) -- (e) -- (f) -- (g) -- (h) -- (a);
      \draw (a) -- (c) -- (h) -- (d) -- (g) -- (e);
      \draw (b) -- (h);
      \draw (b) -- (f);
      \draw (b) -- (d);
      \draw (b) -- (g);
      \draw (b) -- (e);
      \draw (h) -- (e);
      \draw (d) -- (f);
      \draw (g) -- (c);
      \draw (a) -- (d);
      \draw (a) -- (e);
      \draw (a) -- (f);
      \draw (a) -- (g);
      \draw (e) -- (c);
      \draw (c) -- (f);
    \end{tikzpicture}
\end{document}
%%% Local Variables: 
%%% mode: latex
%%% TeX-master: t
%%% End: 
 & \documentclass{standalone}
\usepackage{tikz}
\begin{document}
    \begin{tikzpicture}
      \node[draw] (a) at (0,.3) {A};
      \node[draw] (b) at (1,.3) {B};
      \node[draw] (c) at (1.7,1) {C};
      \node[draw] (d) at (1.7,2) {D};
      \node[draw] (e) at (1,2.7) {E};
      \node[draw] (f) at (0,2.7) {F};
      \node[draw] (g) at (-.7,2) {G};
      \node[draw] (h) at (-.7,1) {H};
      \node[circle,fill=blue!30] (m) at (.5,1.5) {S};
      \draw (m) -- (a);
      \draw (m) -- (b);
      \draw (m) -- (c);
      \draw (m) -- (d);
      \draw (m) -- (e);
      \draw (m) -- (f);
      \draw (m) -- (g);
      \draw (m) -- (h);
    \end{tikzpicture}
\end{document}
%%% Local Variables: 
%%% mode: latex
%%% TeX-master: t
%%% End: 
 & \documentclass{standalone}
\usepackage{tikz}
\begin{document}
    \begin{tikzpicture}
      \node[draw,fill=blue!30] (a) at (0,.3) {A};
      \node[draw] (b) at (1,.3) {B};
      \node[draw] (c) at (1.7,1) {C};
      \node[draw] (d) at (1.7,2) {D};
      \node[draw] (e) at (1,2.7) {E};
      \node[draw] (f) at (0,2.7) {F};
      \node[draw] (g) at (-.7,2) {G};
      \node[draw] (h) at (-.7,1) {H};
      \draw (a) -- (b);
      \draw (a) -- (c);
      \draw (a) -- (d);
      \draw (a) -- (e);
      \draw (a) -- (f);
      \draw (a) -- (g);
      \draw (a) -- (h);
    \end{tikzpicture}
\end{document}
%%% Local Variables: 
%%% mode: latex
%%% TeX-master: t
%%% End: 
\\\hline
    $n^2/2$  translations & $2n$ translations & $2n-2$ translations \\
    symmetric & symmetric & asymmetric\\\hline
  \end{tabular}
  \caption{Approaches for Safe Distributed Computation/Storage/UIs}\label{fig:interop}
\end{figure}
The first does not scale to a project with about a dozen systems, for the third there is
no obvious contender in the \ODK ecosystem. Fortunately, we already have a ``standard'' for
expressing the meaning of COMs -- \defemph{mathematical vernacular}: the language of
mathematical communication, and in fact all the COMs supported in the \ODK VRE are documented
in mathematical vernacular in journal articles, manuals, etc.

The obvious problem is that mathematical vernacular is too 
\begin{inparaenum}[\em i\rm)]
\item \emph{ambiguous}: we need a human to understand structure, words, and symbols
\item \emph{redundant}: every paper introduces slightly different notions. 
\end{inparaenum}
Therefore we adopt an approach, where we partially formalize (\defemph{flexiformalize};
see~\cite{Kohlhase:tffm13}) mathematical vernacular to obtain a flexiformal ontology of
mathematics that can serve as an open communication vocabulary. We call the approach the
\defemph{Math-in-the-Middle} (MitM) Strategy for integration and the ontology the MitM
ontology.
\begin{wrapfigure}r{4cm}\vspace*{-1.5em}
  \documentclass{standalone}
\usepackage[mh]{mikoslides}
% this file defines root path local repository
\defpath{MathHub}{/Users/kohlhase/localmh/MathHub}
\mhcurrentrepos{MiKoMH/talks}
\libinput{WApersons}
% we also set the base URI for the LaTeXML transformation
\baseURI[\MathHub{}]{https://mathhub.info/MiKoMH/talks}

\usetikzlibrary{backgrounds,shapes,fit,shadows}
%\libinput{preamble}
\begin{document}
    \begin{tikzpicture}[scale=1.3]
      \tikzstyle{withshadow}=[draw,drop shadow={opacity=.5},fill=white]
      \tikzstyle{system}=[draw]
      \tikzstyle{standard}=[circle,fill=blue!30]
      \tikzstyle{interface}=[circle,fill=purple!30,inner sep = 1pt,]
      \node[system] (a) at (0,.3) {A};
      \node[system] (b) at (1,.3) {B};
      \node[system] (c) at (1.7,1) {C};
      \node[system] (d) at (1.7,2) {D};
      \node[system] (e) at (1,2.7) {E};
      \node[system] (f) at (0,2.7) {F};
      \node[system] (g) at (-.7,2) {G};
      \node[system] (h) at (-.7,1) {H};
      \node[standard] (m) at (.5,1.5) {S};
      \node[interface] (ia) at (0.2,.9) {a};
      \node[interface] (ib) at (.8,.9) {b};
      \node[interface] (ic) at (1.1,1.2) {c};
      \node[interface] (id) at (1.1,1.75) {d};
      \node[interface] (ie) at (.8,2.1) {e};
      \node[interface] (if) at (0.2,2.1) {f};
      \node[interface] (ig) at (-.1,1.75) {g};
      \node[interface] (ih) at (-.1,1.2) {h};
      \draw (m) -- (ia) -- (a);
      \draw (m) -- (ib) -- (b);
      \draw (m) -- (ic) -- (c);
      \draw (m) -- (id) -- (d);
      \draw (m) -- (ie) -- (e);
      \draw (m) -- (if) -- (f);
      \draw (m) -- (ig) -- (g);
      \draw (m) -- (ih) -- (h);
      \begin{pgfonlayer}{background}
        \node[draw,cloud,fit=(ia) (ib) (ic) (id) (ie) (if) (ig) (ih),
                   inner sep=-7pt,withshadow] (st) {};
        \node[fit=(d) (id),ellipse,inner sep=-1pt,rotate=20,draw,dashed,red] (sys) {};
      \end{pgfonlayer}
      \end{tikzpicture}
\end{document}
%%% Local Variables: 
%%% mode: latex
%%% TeX-master: t
%%% End: 
\vspace*{-.5em}
  \caption{Interface theories}\label{fig:interface-theories}\vspace*{-1em}
\end{wrapfigure}
Before we go into any detail about how this ontology looks and how it induces a uniform
meaning space, we have to address another problem: the descriptions in the MitM ontology
must at the same time be system-near, to make interfacing easy for systems, and serve as
an interoperability standard -- \emph{i.e.}\ be general and stable. If we have an ontology system
that allows modular/structured ontologies, we can solve this apparent dilemma by
introducing \defemph{interface theories}~\cite{KohRabSac:fvip11}, \emph{i.e.}\ ontology modules
(the light purple circles in Figure~\ref{fig:interface-theories}) that are at the same
time system-specific in their description of COMs -- near the actual representation of the
system and part of the greater MitM ontology (depicted by the cloud in
Figure~\ref{fig:interface-theories}) as they are connected to the core MitM ontology (the
blue circle) by views we call \defemph{interviews} (see below). The MitM approach
stipulates that interface theories and interviews are maintained and released together with
the respective systems, whereas the core MitM ontology represents the mathematical scope
of the VRE and is maintained with it. In fact in many ways, the core MitM ontology is the
conceptual essence of the mathematical VRE.

\subsection{Realizing and Utilizing a MitM Ontology}

\begin{wrapfigure}r{6cm}\centering\vspace*{-2em}
  \documentclass{standalone}
\usepackage{tikz}
\usetikzlibrary{mmt}
\def\cn#1{\ensuremath{\mathsf{#1}}}
\begin{document}
\providecommand\myyscale{1}
\providecommand\myxscale{.9}
\begin{tikzpicture}[xscale=\myxscale,yscale=\myyscale]
% \draw[view] (-4,2.4) --node[above] {translation} +(2,0);
% \draw[struct] (-4,1.7) --node[above] {import} +(2,0);
% \draw[meta] (-4,1) --node[above] {meta-theory} +(2,0);
\node[thy] (lf) at (0,2.5)  {$\cn{LF}$};
\node[thy] (lfx) at (1.8,2.5)  {$\cn{LF+X}$};
\node[thy] (fol) at (-1,1.5)   {$\cn{FOL}$};
\node[thy] (hol) at (.9,1.5) {$\cn{HOL}$};
\node[thy] (mon) at (-2.5,0) {$\cn{Monoid}$};
\node[thy] (gp) at (-.5,0) {$\cn{CGroup}$};
\node[thy] (rg) at (2,0)  {$\cn{Ring}$};
\node[thy] (zfc) at (-2.8,1.5) {$\cn{ZFC}$};

\draw[meta](lf) -- (fol);
\draw[meta](lf) -- (hol);
\draw[meta](fol) -- (mon);
\draw[meta](fol) -- (gp);
\draw[meta](hol) -- (rg);
\draw[include](lf) -- (lfx);
\draw[view](fol) -- node[above] {\footnotesize$\cn{f2h}$} (hol);
\draw[struct](gp) to[bend right=10] node[above] {\footnotesize$\cn{add}$} (rg);
\draw[struct](mon) to[out=20,in=160] node[above] {\footnotesize$\cn{mult}$} (rg);
\draw[include](mon) -- (gp);
\draw[view] (fol) -- node[above]{\footnotesize$\cn{folsem}$} (zfc);
\draw[view] (mon) -- node[right,near end]{\footnotesize$\cn{mod}$} (zfc);
\end{tikzpicture}
\end{document}
%%% Local Variables:
%%% mode: latex
%%% TeX-master: t
%%% End:
\vspace*{-.5em}
  \caption{OMDoc/MMT Theory Graphs}\label{fig:mmt}\vspace*{-1em}
\end{wrapfigure}
We use the OMDoc/MMT format~\cite{Kohlhase:OMDoc1.2,MMTSVN:on} to represent the MitM
ontology. OMDoc/MMT is an ontology format specialized to representing mathematical
knowledge modularly in a theory graph: \defemph{theories} are collections of declarations
of concepts, objects, and their properties that are connected by truth-preserving mappings
called \defemph{theory morphisms}. The latter come in two forms: \defemph{inclusions} and
\defemph{structures} that essentially correspond to object-oriented inheritance, and
\defemph{view} that connect pre-existing theories -- in these all axioms of the source
theory have be to proven in the target theory. See ~\cite{RabKoh:WSMSML13} for a full
account. Figure~\ref{fig:mmt} shows an example theory graph. It has three layers:
\begin{compactenum}[\em i\rm)]
\item the (bottom) \defemph{domain level}, which specifies mathematical domains as theories; here
  parts of elementary algebra. The hooked arrows are inclusions for inheritance the
  regular arrows are named structures that induce the additive and multiplicative
  structures of a ring.
\item the \defemph{logic level} represents the languages we use for talking about the
  properties of the objects at the domain level -- again as theories: the meta-theories of
  the domain-level ones -- the dotted arrows signify the meta-relation. At this level, we
  also have inclusions and views (the squiggly arrows) which correspond to logic
  translations (\cn{f2h}) and interpretations into \defemph{foundational theories} like
  set theory (here \cn{ZFC}). Incidentally models can be represented as views into
  foundations.
\item The top layer contains theories that act as metalogics, \emph{e.g.}\ the Logical Framework
  \cn{LF} and extensions which can be used to specify logics and their translations.
\end{compactenum}
The theory graph structure is very well-suited to represent heterogeneous collections of
mathematical knowledge, because views at the domain level can be used to connect differing
but equivalent conceptualizations and views at the logic level can be used to bridge the
different foundations of the various systems. The top level is only indirectly used in in
the MitM framework: it induces the joint meaning space via the meta-relation.

\begin{figure}[ht]\centering
  \documentclass{standalone}
\usepackage[mh]{mikoslides}
% this file defines root path local repository
\defpath{MathHub}{/Users/kohlhase/localmh/MathHub}
\mhcurrentrepos{MiKoMH/talks}
\libinput{WApersons}
% we also set the base URI for the LaTeXML transformation
\baseURI[\MathHub{}]{https://mathhub.info/MiKoMH/talks}

\usetikzlibrary{backgrounds,shapes,fit,shadows,mmt}
\begin{document}
\begin{tikzpicture}[xscale=2.6,yscale=.9]
  \tikzstyle{withshadow}=[draw,drop shadow={opacity=.5},fill=white]
   \tikzstyle{database} = [cylinder,cylinder uses custom fill,
      cylinder body fill=yellow!50,cylinder end fill=yellow!50,
      shape border rotate=90,
      aspect=0.25,draw]
   \tikzstyle{human} = [red,dashed,thick]
   \tikzstyle{machine} = [green,dashed,thick]

\node[thy]  (mf) at (.2,5.3) {MathF};
\node[thy,dashed]  (compf) at (0,6) {CompF};
\node[thy,dashed]  (pf) at (-.9,5.5) {PyF};
\node[thy,dashed]  (cf) at (1,5.5) {C\textsuperscript{++}F};
\node[thy,dashed]  (sf) at (-0.9,4.6) {SAGE};
\node[thy,dashed]  (gf) at (1,4.6) {GAP};

\draw[include] (compf) -- (pf);
\draw[includeleft] (compf) -- (cf);
\draw[include] (pf) -- (sf);
\draw[includeleft] (cf) -- (gf);

\node[thy] (kec) at (0,3) {EC};
\node[thy,minimum height=.4cm] (kl) at (0,4) {\ldots};

\node[thy] (sec) at (-1,2) {SEC};
\node[thy,minimum height=.4cm] (sl) at (-1,3) {\ldots};

\node[thy] (gec) at (1,2) {GEC};
\node[thy,minimum height=.4cm] (gl) at (1,3) {\ldots};

\node[thy] (lec) at (-.3,1.2) {LEC};
\node[thy,minimum height=.4cm] (ll) at (.3,1.2) {\ldots};

\node (sc) at (-2,4) {SAGE};
\node[draw] (salg) at (-2,3.35) {Algo};
\node[database,dashed] (sdb) at (-2,2.4) {DB?};
\node[draw] (skr) at (-2,1.7) {KR};
\node[draw,machine] (sac) at (-2,1) {AbsClass};

\node (gc) at (2,4) {GAP};
\node[draw] (galg) at (2,3.35) {Algo};
\node[database,dashed] (gdb) at (2,2.4) {DB?};
\node[draw] (gkr) at (2,1.7) {KR};
\node[draw,machine] (gac) at (2,1) {AbsClass};

\node (lmfdb) at (0,0) {LMFDB};
\node[database] (ldb) at (1,-.4) {Mongo};
\node[draw] (knowls) at (-1,-.4) {Knowls};
\node[draw,machine] (lac) at (0,-.5) {AbsClass};

  \begin{pgfonlayer}{background}
    \node[draw,cloud,fit=(sec) (sl),aspect=.4,inner sep=-3pt,withshadow,purple!30] (st) {};
    \node[draw,cloud,fit=(gec) (gl),aspect=.4,inner sep=-4pt,withshadow,purple!30] (gt) {};
    \node[draw,cloud,fit=(kec) (kl),aspect=.4,inner sep=0pt,withshadow,blue!30] (kt) {};
    \node[draw,cloud,fit=(lec) (ll),aspect=2.5,inner sep=-7pt,withshadow,purple!30] (lt) {};
  \end{pgfonlayer}

\begin{pgfonlayer}{background}
  \node[draw,withshadow,fit=(sc) (skr) (sac) (sdb),inner sep=1pt] {};
  \node[draw,withshadow,fit=(gc) (gkr) (gac) (gdb),inner sep=1pt] {};
  \node[draw,withshadow,fit=(lmfdb) (lac) (ldb) (knowls),inner sep=1pt] {};
\end{pgfonlayer}

\draw[view] (kec) -- (sec);
\draw[view] (kec) -- (gec);
\draw[view] (kec) -- (lec);
\draw[include] (kec) -- (kl);
\draw[include] (gec) -- (gl);
\draw[include] (sec) -- (sl);
\draw[include] (lec) -- (ll);
\draw[view] (kl) -- (sl);
\draw[view] (kl) -- (gl);
\draw[view] (kl) to[bend left=5] (ll);

\draw[meta] (mf)  to [bend right=10] (st);
\draw[meta] (sf) -- (st);
\draw[meta] (mf)  to [bend left=10] (gt);
\draw[meta] (gf) -- (gt);
\draw[meta] (mf) -- (kt);
\draw[meta] (compf) to[bend right=15] (kt);

\draw[human,->] (skr) -- node[above]{\scriptsize edit} (st);
\draw[human,->] (gkr) -- node[above]{\scriptsize edit} (gt);
\draw[human,->] (knowls) -- node[left,near end]{\scriptsize edit} (lt);

\draw[machine,->] (gt) to[bend right=30] node[below,near start]{\scriptsize generate} (gac);
\draw[machine,->] (st) to[bend left=30] node[below,near start]{\scriptsize generate} (sac);
\draw[human,->] (st) to[bend left=20] node[below]{\scriptsize refactor} (kt);
\draw[human,->] (gt) to[bend right=20] node[below]{\scriptsize refactor} (kt);
\draw[human,->] (lt) -- node[right]{\scriptsize refactor} (kt);
\end{tikzpicture}
\end{document}
%%% Local Variables: 
%%% mode: latex
%%% TeX-master: t
%%% End: 

  \caption{The MitM Paradigm in Details}\label{fig:mitm}
\end{figure}
If we apply OMDoc/MMT to the MitM architecture, we arrive at the situation in
Figure~\ref{fig:mitm}, where we drill into the MitM information architecture from
Figure~\ref{fig:interface-theories}, but restrict to three systems from the \ODK
project. In the middle we see the core MitM ontology (the blue cloud) as an OMDoc/MMT
theory graph connected to the interface theories (the purple clouds) via MitM
interviews. Conceptually, the systems in \ODK consist of three main components:
\begin{inparaenum}[\em i\rm)]
\item a \emph{knowledge representation subsystem} that provides data structures for the COMs and
  their properties.
\item a \emph{database component} that provides mass storage for objects, and 
\item a \emph{library of algorithms} that operate on these.
\end{inparaenum}
To connect a system to an MitM-based VRE, the KR subsystem is either refactored so that it
can generate interface theories, or a schema-like description of the underlying data
structures is created manually from which abstract data structures for the system can be
generated automatically -- in this version the interface theories act as an Interface
Description Language.

In this situation there are two ways to arrive at a greater MitM ontology: the \ODK
project aims to explore both: either
\begin{inparaenum}[\em i\rm)] 
\item standardizing a core MitM by refactoring the various interface theories where they
  overlap, or
\item flexiformalizing the available literature for a core MitM ontology.
\end{inparaenum}
For \emph{i}), the MitM interviews emerge as refinements that add system-specific detail
to the general mathematical concepts\footnote{We use the word ``interface theory'' with a
  slightly different intention when compared to the original use
  in~\cite{KohRabSac:fvip11}: There the core MitM ontology would be an interface between
  the more specific implementations in the systems, whereas here we use the ``interface
  theories'' as interfaces between systems and the core MitM ontology. Technically the
  same issues apply.} For \emph{ii}), we have to give the interviews directly. 

To see that this architecture indeed gives us a uniform meaning space, we observe that the
core MitM ontology uses a mathematical foundation (presumably some form of set theory),
whereas the interface theories also use system-specific foundations that describe aspects
of the computational primitives of the respective systems. We have good formalizations of
the mathematical foundations already; first steps towards a computational ones have been
taken in~\cite{KohManRab:aumftg13}.

%%% Local Variables:
%%% mode: latex
%%% TeX-master: "paper"
%%% End:

%  LocalWords:  pn visualization lstinline ec lstinline Weierstrass ednote interop hline
%  LocalWords:  centering tikz fullgraph tikz mstargraph tikz stargraph defemph emph mmt
%  LocalWords:  inparaenum flexiformalize flexiformal wrapfigure vspace mistargraph cn
%  LocalWords:  KohRabSac fvip11 Realizing Utilizing metalogics specialized RabKoh mitm
%  LocalWords:  compactenum conceptualizations kf-paradigm standardizing flexiformalizing
%  LocalWords:  formalizations KohManRab aumftg13 tffm13

\section{Semantics in Sage}

The \Sage library includes 40k functions and allows for manipulating
thousands of different kinds of objects. As usual in such large
systems, it's critical for taming code bloat to
\begin{enumerate}[(i)]
\item identify the core concepts describing common behavior among the
  objects;
\item exploit this to implement generic operations that apply on all
  object having a given behavior, with appropriate specializations
  when performance calls for it.
\item design or choose a process for selecting the best implementation
  available when calling an operation on one or several objects.
\end{enumerate}

Fortunately in mathematics a lot of (i) has already been taken care
off over the centuries, in particular in the context of abstract
algebra. Our running examples of concepts in this paper will be that
of (multiplicative) \emph{magma}: a set $S$ endowed with a binary
operation $\dot: S\times S \mapsto S$ and of \emph{semigroup}: a magma
such that the binary operation is associative. Thanks to
associativity, a semigroup comes endowed with the powering operation
which can be implemented generically from the binary operation.

\ednote{NT: include here the MMT theory for magmas / semigroups}

In general, the concepts involve sets endowed with a certain number of
basic operations (addition, product, coproduct, ...) which satisfying
certain axioms (associativity, commutativity, ...). Typical concepts
include \emph{fields}, \emph{rings}, \emph{groups}, which are
naturally organized into a hierarchy according to the available
operations and axioms (a field is a ring, etc).

In practice, one wants to compute either with the elements of the
sets, with the sets themselves (called \emph{parents} in \Sage,
following the \Magma tradition), or with morphisms. Category theory
provides a convenient language, so we speak of the category of groups,
the category of fields, ...

Many selection processes for (iii) are available, including object
oriented programming with methods and/or multimethods, modular
programming and traits, composition, etc. As early as \ednote{NT: find
  the date}, the \Axiom system has based its selection process upon a
hierarchy of classes which models the hierarchy of categories, with
operations implemented as methods. Followers include the \MuPAD, or
\Fricas systems. \Sage builds on the same tradition, though using a
general purpose object oriented language (\Python). \GAP uses a custom
selection process described in Section~\ref{...}.

\ednote{NT: Could say more here about the fact that the mathematical
  categories are modeled explicitly in Sage, and not only through a
  hierarchy of classes}

Those design choices are largely motivated by another specific aspect
of mathematics: the number of fundamental concepts is actually fairly
small, and all the richness comes from the many ways the concepts can
be combined together.

To summarize, \emph{mathematical knowledge} from abstract algebra is
modeled explicitly in \Sage, and used to support genericity, control
the method selection process, structure the code and documentation,
enforce consistency, and provide generic tests.


%%% Local Variables:
%%% mode: latex
%%% TeX-master: "deleted-scenes"
%%% End:

\section{An application: toward multi-system semantic aware handle interfaces}

\subsection{The handle paradigm in system interfaces}\label{the-handle-paradigm-in-system-interfaces}

The ``handle'' paradigm has become a classic when interfacing two
computational mathematics systems. For example, most of the \Sage
interfaces, including that for \GAP, \Singular, or \Pari use this
paradigm to delegate calculations to those systems.

In this paradigm, when a system \texttt{A} delegates a calculation to a
system \texttt{B}, the result \texttt{r} of the calculation is not
converted to a native \texttt{A} object; instead \texttt{B} just returns
a handle (or reference) to the object \texttt{r}. Later \texttt{A} can
run further calculations with \texttt{r} by passing it as argument to
\texttt{B} functions or methods. Advantages of this approach include:

\begin{itemize}
\item Avoiding the overhead of back and forth conversions between
  \texttt{A} and \texttt{B}.
\item Manipulating objects of \texttt{B} from \texttt{A} even if they
  have no native representation in \texttt{A}.
\end{itemize}

\subsection{Semantic handle interfaces}\label{semantic-handle-interfaces}

Whenever \texttt{A} and \texttt{B} share some common semantic (for example the concept of
group), it's desirable that handles behave as native \texttt{A} objects. For example, if a
group \texttt{G} is constructed in \texttt{B}, one wants to manipulate handles to
\texttt{G} or its elements as if they were native \texttt{A} groups or group elements,
even if there is no corresponding native implementation for \texttt{G} in \texttt{A}.
This can be achieved with the usual \emph{adapter} design pattern. The bulk of the work is
the implementation of adapter methods so that, for example, calling the method
\texttt{h.cardinality()} on a \Sage handle \texttt{h} to a \GAP object \texttt{G},
triggers in \GAP a call to \texttt{Size(G)}.

In \Sage, this has been implemented in a couple special cases. For
examples, \Sage permutation groups or matrix groups are built on top
of handles to \GAP objects. However, this implementation is monolithic
and does not scale. For example, if \texttt{h} is a handle to a set
\texttt{S}, \Sage only knows that \texttt{h.cardinality()} can be
computed by \texttt{Size(S)} in \GAP if \texttt{S} is a group; in fact
if \texttt{h} has been constructed through the
\texttt{PermutationGroup} or \texttt{MatrixGroup}
constructors. Whereas we would want this method to be available as
soon as \texttt{S} is a set.

\subsection{Generic/hierarchical semantic handle interfaces}\label{generichierarchical-semantic-handle-interfaces}

During the \href{http://gapdays.de/gap-sage-days2016/}{first joint
  \GAP-\Sage days}, the last author worked on a prototype of generic
semantic handle \Sage-\GAP interface. The idea is twofold:

\begin{enumerate}
\def\labelenumi{\arabic{enumi}.}
\item Every \Sage category (\emph{e.g.}\ the category of sets, of groups) can
  provide a collection of adapter methods that are valid for every
  handle to a \GAP object in the corresponding mathematical category.
  This applies as well to elements and morphisms.
\item When a handle \texttt{h} to a \GAP object \texttt{S} is created,
  the properties of \texttt{S} (its \GAP categories and properties)
  are explored, and the handle \texttt{h} is then put in the matching
  (or closest matching) \Sage category.
\end{enumerate}

For example, here is the adapter for the cardinality method and some
context around:
\begin{lstlisting}
class Sets: # Everything about sets in Sage
    class GAP: # The adapter methods relevant to Sets in the Sage-Gap interface
         class ParentMethods: # Adapter methods for sets
             def cardinality(self): # The adapter for the cardinality method
                 return self.gap().Size().sage()
         class ElementMethods: # Adapter methods for set elements
             ...
         class MorphismMethods: # Adapter methods for set morphisms
             ...
\end{lstlisting}

At the current stage of the implementation, a handle to a \GAP field
behaves essentially like a native \Sage field. This remains valid for
objects of all subcategories as well, from magmas to rings. The
infrastructure is relatively lightweight, and can be extended by
developers and users as the need for more adapter methods arises.

\subsection{Scaling to multisystem interfaces?}\label{scaling-to-multisystem-interfaces}

A second stage was initiated during the
\href{http://opendreamkit.org/2015/12/08/WP6StAndrewsMeeting/}{Knowledge
representation in mathematical software and databases workshop}
organized at the University of St Andrews, St Andrews, 25th-27th
January, 2016.

The approach described earlier is likely to work well for implementing
an interface between two systems. However it does not scale for
interfacing \texttt{n} systems, as this requires the implementation of
\texttt{n(n-1)} independent adapter interfaces.

The key point here is that implementing an adapter method (or
function) typically requires only some simple abstract information on
the method, namely its signature and its names in both systems.  In
particular, the only things that changes between an \texttt{A->B}
adapter method and the equivalent \texttt{C->D} adapter method are the
names of the methods.

The second stage of this project is therefore to explore whether the
interfaces could be automatically generated from a consistent
formalizations of the systems.

\ednote{NT:Update this paragraph w.r.t. the rest of this section}

Ideally, the mathematical structure and operations would be described
once, \emph{e.g.}\ in the MMT language (the blue blob in Michael's talk) and
then each system would be formalized by specifying how the operations
are implemented (the purple blobs). For example, one would specify in
MMT that a magma is a set with a binary operation denoted by default
\texttt{o}. The relevant category in \Sage is \texttt{Magmas()}, and
the binary operation is implemented by the method \texttt{\_mul\_}.

We experimented with doing this formalization using lightweight
annotations in the \Sage source code such as:
\begin{lstlisting}
@semantic(mmt="sets")
class Sets:
    class ParentMethods:
         @semantic(mmt="o", gap="Size")
         @abstractmethod
         def cardinality(self):
             r"""
             Return the cardinality of ``self``.
             """
\end{lstlisting}
Note: the only additions to the original source code are the \texttt{@semantic} lines.

Several variants of the annotations exist to allow for adding
annotations on existing categories without touching their file, and also
for specifying directly the corresponding method names in other systems
when this has not yet been formalized elsewhere. Similarly, one could
provide directly the signature information in case that is not yet
modelled in MMT.

\subsection{Difficulties}\label{difficulties}

In \Sage and \GAP (and most other systems with some category mechanism) we distinguish
additive magma and multiplicative magma, duplicating all the information, code, etc. In
MMT however, thanks to morphisms which allow to rename operations transparently, there is
no such distinction: there are just Magmas.

Hence, to actually map additive magmas in \Sage to additive magmas in \GAP (and map the
corresponding methods), one need in the intermediate MMT step to keep an extra bit of
information, namely whether the monoid is additive or multiplicative (or something else;
think of the bracket operation of Lie algebras).


%%% Local Variables:
%%% mode: latex
%%% TeX-master: "deleted-scenes"
%%% End:

%  LocalWords:  subsubsection texttt itemize emph labelenumi lstlisting organized ednote
%  LocalWords:  generichierarchical-semantic-handle-interfaces formalizations formalized
%  LocalWords:  scaling-to-multisystem-interfaces formalization mmt

\section{Exploring GAP types}\label{sec:gaptypes} 

\subsection{Brief introduction to GAP types and categories.}\label{gap-types-intro}

%\ednote{MP: I am not sure what I claim below about \Sage is true, it
%  also irks me a bit that we seem to conflate the idea of a type system with
%  the idea of organising mathematical hierarchies. Of course in \GAP
%  system this is intentional, in \Sage, I don't know. In my head \Sage
%  uses whatever python uses as the type system (duck typing?) and then intro-
%  duces a category system on top. We should agree on a level of description
%  that fits.}

While the \Sage type system is object-oriented, the \GAP type
system puts more of an emphasis on \emph{operations} on and between objects.

Breuer and Linton describe the \GAP type system in \cite{breuer-linton}, and
the \GAP documentation \cite{GAP4} also contains an extensive technical
description of the \GAP type system.

A type in \GAP is a pair consisting of a \emph{family} and a \emph{filter}.

Families partition the space of objects in \GAP, so every object lies in exactly one family.

A filter is a set of \emph{elementary filters}, and hence filters form a hierarchy on
objects by the subset relation on filters.
We say that an object is \emph{in a filter $F$} if its type's filter component
contains $F$ as a subset.

\emph{Operations} in \GAP are declared with an arity and for each argument with a
most general filter for which they are applicable. For instance there is an operation
for forming the direct product of two groups.
The programmer can install \emph{methods} for an operation which can carry strictly
more specific filters for the inputs.

At runtime \GAP through a very sophisticated mechanism called \emph{method selection} will
select the most appropriate method for the given arguments to an operation and execute it.

\emph{Categories} are filters that model mathematically similar objets. In terms of
algebraic structures we can think of a category as the signature of the structure.
For instance there are categories called \texttt{IsMagma}, \texttt{IsMagmaWithOne}, and
\texttt{IsMagmaWithInverses}, which we can think of as objects having signature $\{ * \}$,
$\{*,1\}$, and $\{*,^{-1}\}$ respectively. Semigroups, monoids, or groups are not
categories in \GAP.

Once an object is created, the category it is in cannot be altered.

\emph{Representations} are filters that give a way to represent mathematical
objects in different ways. One of the examples from the \GAP library are permutations
which can be represented in 2-bytes acting on at most 65536 points, or 4 bytes, acting
on at most $2^{32}-1$ points. Other examples include matrices in sparse or dense
representation, or finite field elements, where particularly matrices with entries in
the field of order 2 allow a very efficient representation.

A \emph{Property} \texttt{P} is realised by two filters \texttt{P} and \texttt{HasP} and an
operation which is also called \texttt{P}.

This models three possible states for a property: Its value can either be known or unknown,
which is reflected by the filter \texttt{HasP}, and if it is known, then the filter \texttt{P}
says whether the property holds or not.
If the value of the property is unknown, but there are methods installed for the operation
\texttt{P}, then \GAP will be attempt to compute the value of \texttt{P} using that method.

Examples of properties in \GAP are \texttt{IsAssociative}
or \texttt{IsCommutative}. A group in \GAP is an object that is in the filter
\texttt{IsMagmaWithInverses} and \texttt{IsAssociative}. An abelian group will additionally
be in the filter \texttt{IsCommutative}.

An \emph{attribute} in \GAP is a value attached to a \GAP object. There is
a filter attached with each attribute that reflects whether the value
of the attribute is known, and an operation which can be invoked to determine
the value of the attribute if it is not known.
\texttt{Size} or \texttt{Centre} are two attributes that are defined for groups.

The values of attributes and properties can be unknown on creation,
can be computed on demand, and their values can then be stored for later
reuse without the need to be recomputed. Note that in particular the knowledge
accumulated in the type of a \GAP object can influence method selection, so for
example attaining the knowledge that a group is nilpotent will allow for more
efficient methods to be run for finding its centre.

\begin{lstlisting}
gap> IsGroup;
<Filter "(IsMagmaWithInverses and IsAssociative)">
gap> IsMagmaWithInverses;
<Category "IsMagmaWithInverses">
gap> IsAssociative;
<Property "IsAssociative">
gap> IsSet;
<Property "IsSSortedList">
gap> IsFinite;
<Property "IsFinite">
gap> IsSet=IsSSortedList;
true
gap> G := Group((1,2), (2,3,4));
Group([ (1,2) ])
gap> HasSize(G);
false
gap> HasIsCommutative(G);
false
gap> Size(G);
24
gap> HasSize(G);
true
\end{lstlisting}

\ednote{TODO (???) Compare and contrast it with the Sage type system}

\subsection{Tentative approaches to exporting GAP types.}\label{gap-types-export}

Encoded in filters, categories, representations, attributes, and properties in \GAP
there is a wealth of mathematical knowledge. \GAP allows some introspection
of this knowledge after the system is loaded.

Having a clear picture of the relations between different objects is 
very helpful to GAP developers, package authors, and users.
For example one might be interested in the attributes or properties that \GAP can
compute for an object, or how it tries to compute them.

During the OpenDreamKit workshop in St~Andrews in January 2016 we developed
tools to more conveniently access mathematical knowledge encoded in \GAP,
such as introspection inside a running \GAP session, export to JSON to import
to MMT, and export as a graph for visualisation and exploration.
\ednote{picture based on \url{https://github.com/OpenDreamKit/OpenDreamKit/issues/165}?}

We will make these tools available as part of the standard \GAP distribution with the next
major release of \GAP, as they will prove useful in the development of the \GAP Jupyter interface
\url{https://github.com/gap-packages/jupyter-gap}, and possibly to do
internal consistency checks of \GAP types.

As a side-effect of the work outlined above, we fixed a number of bugs in the handling of
special categories in \GAP.

The JSON output of the \GAP object system after loading a default set of packages is currently
around 11 Megabytes in size and takes many hours to import into MMT. We did not yet attempt to
load a collection of \GAP packages that would expose even more data.

The graph exported for visualisation has 540 vertices, 759 edges and 8 connected components, if
packages are loaded this increses to 1616 vertices, 2178 edges and 17 connected components.

\ednote{NT: Do you have anything to say about the GAP-MMT formalization? 
E.g. hints on potential ways this formalization may be written?}
\ednote{MP: I have some ideas as to how I would write an MMT formalisation, unfortunately I do not
  understand MMT well enough yet to know whether my ideas make any sense. I'll think about this a bit
  more and add my comments then\\
       MP: Also, what exactly are we talking about when we talk about MMT formalisation? The description
  of an export, or how to implement structures exported from MMT in GAP?}

\subsection{An application: consistency checker for the GAP
  documentation.}\label{gap-types}

One of the immediate outcomes of the development of the tools described in the
previous section is the consistency checker for the GAP documentation. 

GAP uses special format for its main manuals. It is called GAPDoc and is 
provided by the GAP package with the same name \cite{gapdoc}. Besides main 
manuals, it is adopted by 97 out of 130 packages currently redistributed 
with GAP. Using GAPDoc, one builds text, PDF and HTML versions of the manual
from a common source given in XML.

GAPDoc defines XML constructions to specify the type of the documented object 
(function, operation, attribute, property, etc.). However, due to the 
limitations of the semi-automated conversion of GAP manuals from the \TeX-based
manuals used in GAP 4.4.12 and earlier, a number of objects had their types
stated incorrectly. 

We developed the consistency checker for the GAP documentation, which extracts
type annotations from the documented GAP objects and compares them with their
actual types. It immediately reported almost 400 inconsistencies out of 3674 
manual entries. In the subsequent cleanup, we by now have eliminated about 
75\% of them. The  consistency checker will appear in the next release of
GAP 4.8.3, and will be available via \texttt{make check-manuals}.
It also performs other useful checks: for example, it produces a list of
manual sections having no examples. Thus, the new tool helps to improve
the quality of GAP documentation, and may be useful for the similar checks
of those GAP packages which use GAPDoc-based manuals.

% \url{https://github.com/gap-system/gap/pull/675}
% \url{https://github.com/gap-system/gap/pull/538}

%%% Local Variables:
%%% mode: latex
%%% TeX-master: "paper"
%%% End:

%  LocalWords:  ednote emph breuer-linton texttt texttt itemize Jupyter formalization
%  LocalWords:  gapdoc gaptypes

\section{LMFDB Knowledge and Interoperability}\label{sec:lmfdb}
%\subsection{Short description of LMFDB's goals and implementation}
The \emph{$L$-functions and modular forms database} is a project involving dozens of
mathematicians, who assemble computational data about $L$-functions, modular forms, and
related number theoretic objects. The main output of the project is a website, hosted at
\url{http://www.lmfdb.org}, that presents this data in a way that could serve as a
reference for research efforts and should be accessible at the graduate student level.
The mathematical concepts underlying the \LMFDB are extremely complex and varied, so part
of the effort has been focused on how to relay knowledge (mathematical definitions and their
relationships) to data and software. For this purpose, the \LMFDB has developed so-called
\emph{knowls}, which are a technical solution to present \LaTeX-encoded information
interactively, heavily exploiting the concept of transclusion. The end result is a
very modular and highly interlinked set of definitions in mathematical natural language.

\ednote{NT: here we use ``natural language'' when we use ``vernacular
  language'' in the MitM section; do we want to make this uniform?}

The \LMFDB code is primarily written in \Python, with some reliance on \Sage for
the business logic. The frontend is written in the web framework Flask, while the backend
uses the NoSQL document database system \Mongo \cite{lmfdb-repo}. Again, due to the
complexity of the objects considered, many idiosyncratic encodings are used for the
data. This makes the whole data management lifecycle particularly tricky, and dependent on
different select groups of individuals for each component.

%\subsection{Tentative approach to MMT semantic layers for the LMFDB}
As the LMFDB spans the whole "vertical" workflow, from writing software, to producing new
data, up to presenting this new knowledge, it is a perfect test case for a large scale
case study of the MitM approach. Conversely, a semantic layer would be beneficial to its
activities across data, knowledge and software, which it would help integrate more
cohesively and systematically.

Among the components of the LMFDB, elliptic curves stand
out in the best shape, and a source of best practices for other areas. 
%
% For this reason,
% the \ODK collaboration targeted that relatively small subset of the LMFDB for its
% prototype. We plan to extend coverage in a second phase, and expect it to be relatively
% easy if we manage to demonstrate the added value of a semantic layer for our prototype.
%
We have generated MitM interface theories for LMFDB elliptic curves by (manually)
refactoring and flexiformalizing the {\LaTeX} source of knowls into \sTeX (see
Listing~\ref{stex-ec} for an excerpt), which can be converted into flexiformal OMDoc/MMT
automatically. The MMT system can already type-check the definitions, avoiding circularity
and ensuring some level of consistency in their scope and make it browsable through
\textsf{MathHub.info}, a project developed in parallel to MMT to host such formalisations.
\ednote{NT: should mathhub be instead described in the previous section?}


\ednote{NT: something seems to have gotten wrong with this listing}
\lstinputlisting[language={[sTeX]TeX},label={stex-ec},firstline=21,
   caption= {\protect\stex flexiformalization of an \LMFDB knowl}]
   {examples/elliptic-curve.tex}

   The second step consisted of translating these informal definitions into progressively
   more exhaustive MMT formalisations of mathematical concepts (see
   Listing~\ref{lst:mmt-ec}). The two representations are coordinated via the theory and
   symbol names -- we can see the sTeX representation as a human-oriented documentation of
   the MMT.\ednote{MK: go over the examples to make this true!}

\lstinputlisting[morekeywords={namespace,theory,include},mathescape,
firstline=21,
caption= {MMT formalisation of elliptic curves and their Weierstrass models},
label=lst:mmt-ec]{examples/elliptic-curve.mmt}

Finally, we have to integrate computational data into the interface theories. Based on
recent ongoing efforts \cite{lmfdb-formats} to document the \LMFDB ``data schemata'' we
established OMDoc/MMT theories that linked the database fields to their data types (string
\emph{vs.} float \emph{vs.} integer tuple, for instance) and mathematical types (elliptic
curves or polynomials) -- the latter based on the vocabulary in the interface theories
generated from the \LMFDB knowls. This schema theory is complemented by a theory on
composable \emph{MMT codecs}, which in turn acts as a specification for a collection of
implementations in various programming languages (currently \Python, Scala, and
C\textsuperscript{++} for \Sage, MMT, and \GAP\ednote{NT: respectively?}) which are first instances of a computational
foundation (see Section~\ref{sec:MitM}).  For instance, one could compose two MMT codecs,
say \emph{polynomial-as-reversed-list} and \emph{rational-as-tuple-of-int}, to signify
that the data $[(2,3),(0,1),(4,1)]$ is meant to represent the polynomial $4x^2+2/3$. Of
course, these codecs could be further decomposed (signalling which variable name to use,
for instance). The initial cost of developing these codecs is high, but the clarity
gained in documentation is valuable, they are highly reusable, and they drastically expand
the range of tooling that can be built around data management. Our efforts also fit neatly
alongside similar efforts underway across the sciences to standardize metadata formats
(for instance through the Research Data Alliance's Typing Registry Working
Group\cite{rda-typing}).

\paragraph{A typical application}
Based on these MitM interface theories we can generate I/O interfaces that translate
between the low-level \LMFDB API, which delivers raw \Mongo data in JSON format into MMT
expressions that are grounded in the interface theories. This ties the \LMFDB database
into the MitM architecture transparently. As a side effect, this opens up the \LMFDB to
programmatic queries via the MMT API, which can be queried and can then relay them to the
\LMFDB API directly and transparently.

%  LocalWords:  subsubsection emph knowls textsf lmfdb-repo stex-ec stex knowl mmt-ec odk
%  LocalWords:  Weierstrass emphasizing alltt tp rightarrow vdash doteq vdash doteq lst
%  LocalWords:  polynomial_equation_injectivity a_invariants_factors_injectivity colorbox
%  LocalWords:  minimality_idempotence lmfdb-formats emphasized composable standardize
%  LocalWords:  rda-typing lstinputlisting mathescape elliptic-curve.mmt lmfdb firstline

%%% Local Variables:
%%% mode: latex
%%% TeX-master: "paper"
%%% End:
%  LocalWords:  flexiformal flexiformalization flexiformalizing ednote

\section{Conclusion}
\ednote{MK: I am not sure we need this, but we will see}
\subsubsection*{Acknowledgements}
The authors gratefully acknowledge discussions and experimentation at the St.\ Andrews
workshop, which clarified the ideas behind the math-in-the-middle approach;
in particular John Cremona, Paul-Olivier Dehaye, Luca de Feo, Mihnea Iancu, Alexander
Konovalov, Samuel Leli\`evre, Steve Linton, Dennis M\"uller, Markus Pfeiffer, Viviane Pons,
Florian Rabe, and Tom Wiesing.

We acknowledge financial support from the OpenDreamKit Horizon 2020 European Research
Infrastructures project (\#676541) and from the EPSRC Collaborative Computational Project
CoDiMa (EP/M022641/1).

%%% Local Variables:
%%% mode: latex
%%% TeX-master: "paper"
%%% End:

%  LocalWords:  ednote subsubsection Dehaye Mihnea Iancu Konovalov Leli evre Wiesing


\printbibliography
\end{document}
%%% Local Variables:
%%% mode: latex
%%% TeX-master: t
%%% End:

%  LocalWords:  maketitle endeavor ednote ODKproposal printbibliography subsubsection pn
%  LocalWords:  IPython Jupyter emph itemize specialized Arxiv oldpart organization ec
%  LocalWords:  citability github Swinnerton-Dyer resentation desingularisation Hironaka
%  LocalWords:  Hironaka algorithmisation Villamayor visualization lstinline omtext lec
%  LocalWords:  multisystem-semantic-handle-interfaces.tex presenting-lmfdb.tex Dehaye
%  LocalWords:  Mihnea Iancu Konovalov Leli evre Wiesing odk mitm presenting-lmfdb concl
%  LocalWords:  multisystem-semantic-handle-interfaces
