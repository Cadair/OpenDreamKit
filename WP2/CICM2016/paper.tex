\documentclass{llncs}
\pagestyle{plain}
\usepackage[show]{ed}
\usepackage{listings}
% \usepackage{url}
% \usepackage{wrapfig}
\usepackage{xspace}
\usepackage{hyperref}

\usepackage[style=alphabetic,backend=bibtex]{biblatex}
\addbibresource{kwarc.bib}
\def\pn{OpenDreamKit}
\title{The OpenDreamKit Project:\\
  Towards Enhanced Interoperability\\
  via a Math-in-the-Middle Approach}
\author{Michael Kohlhase\inst{1} Nicolas M. Thi\'ery\inst{2} }
\institute{Jacobs University, Bremen, Germany \and Universit\'e Paris-Sud, Paris, France}

\newcommand{\OOMMFNB}{OOMMF-NB\xspace}
\newcommand{\VRE}{VRE\xspace}
\newcommand{\VREs}{VRE\xspace}
\newcommand{\software}[1]{\textsc{#1}\xspace}
\newcommand{\GAP}{\software{GAP}}
\newcommand{\HPCGAP}{\software{HPC-GAP}}
\newcommand{\libGAP}{\software{libGAP}}
\newcommand{\Singular}{\software{Singular}}
\newcommand{\Sage}{\software{Sage}}
\newcommand{\SageCombinat}{\software{Sage-Combinat}}
\newcommand{\MuPADCombinat}{\software{MuPAD-Combinat}}
\newcommand{\Sphinx}{\software{Sphinx}}
\newcommand{\SCSCP}{\software{SCSCP}}
\newcommand{\Python}{\software{Python}}
\newcommand{\IPython}{\software{IPython}}
\newcommand{\Jupyter}{\software{Jupyter}}
\newcommand{\Cython}{\software{Cython}}
\newcommand{\Pythran}{\software{Pythran}}
\newcommand{\Numpy}{\software{Numpy}}
\newcommand{\Pari}{\software{PARI}}
\newcommand{\PariGP}{\software{PARI/GP}}
\newcommand{\libpari}{\software{libpari}}
\newcommand{\GP}{\software{GP}}
\newcommand{\GPtoC}{\software{GP2C}}
\newcommand{\Linbox}{\software{LinBox}}
\newcommand{\LMFDB}{\software{LMFDB}}
\newcommand{\OpenEdX}{\software{OpenEdX}}
\newcommand{\Linux}{\software{Linux}}
\newcommand{\LATEX}{\software{\LaTeX}}
\newcommand{\SMC}{\software{SageMathCloud}}
\newcommand{\Simulagora}{\software{Simulagora}}
\newcommand{\KANT}{\software{KANT}}
\newcommand{\Magma}{\software{Magma}}
\newcommand{\Mathematica}{\software{Mathematica}}
\newcommand{\Maple}{\software{Maple}}
\newcommand{\Matlab}{\software{Matlab}}
\newcommand{\MuPAD}{\software{MuPAD}}
\newcommand{\MPIR}{\software{MPIR}}
\newcommand{\Arxiv}{\software{arXiv}}
\newcommand{\Givaro}{\software{Givaro}}
\newcommand{\fflas}{\software{fflas}}
\newcommand{\MathHub}{\software{MathHub}}
\newcommand{\FindStat}{\software{FindStat}}
\newcommand\DKS{\ensuremath{\mathcal{DKS}}\xspace}
\newcommand{\ODK}{\software{OpenDreamKit}}


\begin{document}
\maketitle
\begin{abstract}
  \ODK --- ``Open Digital Research Environment Toolkit for the
  Advancement of Mathematics'' --- is an H2020 EU Research
  Infrastructure project that aims at supporting, over the period
  2015--2019, the ecosystem of open-source mathematical software
  systems, and in particular popular tools such as LinBox, MPIR,
  SageMath, GAP, Pari/GP, LMFDB, Singular, MathHub, and the
  IPython/Jupyter interactive computing environment. From that
  ecosystem, OpenDreamKit will deliver a flexible toolkit enabling
  research groups to set up Virtual Research Environments, customised
  to meet the varied needs of research projects in pure mathematics
  and applications.

  An important step in the OpenDreamKit endeavor is to foster the
  interoperability between a variety of systems, ranging from computer
  algebra systems over mathematical databases to front-ends.

  In this paper, we describe the OpenDreamKit project and report on experiments and future
  plans with the \emph{math-in-the-middle} approach.  This information architecture
  consists in a central mathematical ontology that documents the domain and fixes a joint
  vocabulary, combined with specifications of the functionalities of the various
  systems. Interaction between systems can then be enriched by pivoting off this
  information architecture.
\end{abstract}

\section{Towards a Mathematical Virtual Research Environment toolkit}

\ednote{MK@NT: we need a general description here, cite \cite{OpenDreamKit:on,ODKproposal:on}}

\subsection{Context}

\ednote{NT}{Needed?}

From their earliest days, computers have been used in pure
mathematics, either to make tables, to prove theorems (famously the
four colour theorem) or, as with the astronomer’s telescope, to
explore new theories. Computer aided experiments, and the use of
databases relying on computer calculations such as the Small Groups
Library in GAP, the Modular Atlas in group and representation theory,
or the LMFDB, are now part of the standard toolbox of the pure
mathematician, and certain areas of mathematics completely depend on
it. Computers are also increasingly used to support collaborative work
and education.

The last decades has witnessed the emergence of a wide ecosystem of
open source tools to support research in pure mathematics. This ranges
from specialized to general purpose computational tools such as \GAP,
\PariGP, \Linbox, \MPIR, \Sage, or \Singular, via online databases
like the \LMFDB. Not counting of course online services like the
Wikipedia, \Arxiv, or MathOverflow.

A great opportunity is the rapid emergence of key technologies, and in
particular the \Jupyter (previously \IPython) platform for interactive
and exploratory computing which targets all areas of science.

\subsection{Virtual Research Environments}

Promoting collaborations at all scales is key to further improve the
productivity of researchers in pure mathematics and applications. An
important next step in that direction is that of \emph{Virtual
  Research Environments} (VRE), that is online services enabling
groups of researchers, typically widely dispersed, to work
collaboratively on a per project basis.
% , and supporting the entire
% life-cycle of computational work in mathematical research, from
% \emph{initial exploration} to \emph{publication}, \emph{teaching}, and
% \emph{outreach}.

Early VRE's include \SMC, a collaborative online environment
where students, teachers and researchers can create, customize, and
share a project. This project essentially consists in a virtual
machine, with a simple web-based user interface, and ready-to-use
software for interactive computations (e.g. \Sage) and authoring
(e.g. \LaTeX), with facilities for real-time communication through
chat, video, and shared editing of documents, programs and worksheets.
For education purposes, course material can be provided as worksheets,
assignments can be distributed, collected, and returned as well.

Technically speaking, \SMC is a specific open-source cloud-based
Virtual Research and Teaching Environment for mathematics developed
since 2013 under the lead of William Stein, with funding from the NSF,
and Google's Education Grant program (\ednote{NT}{Mention SageInc?}).
It presently hosts over 100,000 projects and 10,000 weekly active
users. This fast adoption by a wide variety of users demonstrates the
relevance and the long term impact this kind of collaborative
environments can have.

\subsection{Motivations}

The \ODK proposal grew out of a reflection on the needs of the (pure)
mathematics community in terms of computational software and
databases. The highly successful development in the last decades of
open source systems has proven the viability and power of
collaborative open source development models, by users and for users,
even for delivering general purpose systems targeting a large public
(researchers, teachers, engineers, amateurs, ...).

Yet some critical long term investments, in particular on the
technical side, are in order to boost the productivity and lower the
entry barrier:

\begin{itemize}
\item Streamline access, distribution, portability on a wide range of
  platforms, including High Performance Computers or cloud services.
\item Improve user interfaces, in particular in the promising area of
  collaborative workspaces as those provided by \SMC.
\item Lower barriers between research communities and promote
  dissemination. For example make it easy for a specialist of
  scientific computing to use tools from pure mathematics, and
  reciprocally.
\item Bring together the developers communities to promote tighter
  collaboration and symbiosis, accelerate joint development, and share
  best practices.
\item Outsource as much of the development as possible to larger
  communities to focus the work forces on their core specialty: the
  implementation of mathematical algorithms and databases.
\end{itemize}

Many people in the community have been working really hard on the
above items but crucially lack manpower or funding.

\subsection{\ODK}

The European H2020 call
\href{http://ec.europa.eu/research/participants/portal/desktop/en/opportunities/h2020/topics/2144-einfra-9-2015.html}{EINFRA-9:
  e-Infrastructure for Virtual Research Environment} was a natural fit
to seek for manpower and funding: putting the emphasis on Virtual
Research Environments nicely wraps up all the above needs in a single
mission.

A consortium was built by gathering core European developers of the
aforementioned systems for pure mathematics, and reaching toward the
numerical community, and in particular the \Jupyter community, to work
together on joint needs.

Together they answered the call with the \ODK proposal --- ``Open
Digital Research Environment Toolkit for the Advancement of
Mathematics'' --- to help address the following aims in close
collaboration with the community.

\begin{itemize}
\item Further improve the productivity of researchers in pure
  mathematics and applications by further promoting collaborations on
  \emph{Data}, \emph{Knowledge}, and \emph{Software}.
\item Make it easy for teams of researchers of any size to set up custom,
  collaborative \emph{Virtual Research Environments} tailored to their
  specific needs, resources, and workflows.
\item Support the entire life-cycle of computational work in
  mathematical research, from \emph{initial exploration} to
  \emph{publication}, \emph{teaching}, and \emph{outreach}.
\end{itemize}

The acceptance of the proposal in May 2015 was a strong sign of
recognition, at the highest level of funding agencies, of the values
of open science and the strength and maturity of the ecosystem.

The \ODK project will run for four years, starting from September
2015. It will provide substantial resources to the open source
computational mathematics ecosystem, and in particular popular tools
such as LinBox, MPIR, SageMath, GAP, Pari/GP, LMFDB, Singular,
MathHub, and the IPython/Jupyter interactive computing environment.

The project involves about 50 people spread over 15 sites in Europe,
with a total budget of about 7.6 million euros. The largest portion of
that will be devoted to employing an average of 11 researchers and
developers working full time on the project. Additionally, the
participants will contribute the equivalent of six other people
working full time.

By definition this project will be mostly funding actions in Europe;
however those actions will be carried out, as usual, in close
collaborations with the worldwide community.

\subsection{Work plan}

\ednote{NT}{Not sure how to write up this section; we want to give
  some concrete idea of what's going to happen}

In practice, the \ODK work plan consists in 58 concrete tasks split in
7 work packages:
\begin{itemize}
\item WP1: Project Management
\item WP2: Community Building, Training, Dissemination, Exploitation,
  and Outreach\\
  This work package includes for example the organization of many
  development and training workshops.
\item WP3: Component Architecture\\
  This package includes work on portability, especially on the Windows
  platform,, on modularity and packaging, etc.
\item WP4: User Interfaces\\
    Example: Task4.1: Uniform \Jupyter notebook interface for all interactive components

\item WP5: High Performance Mathematical Computing\\
  In this work package, many 

\item WP6: Data/Knowledge/Software-Bases
\item WP7: Social Aspects
\end{itemize}
The completion of each task is validated by the release or publication
of deliverables. Here are a brief description of a few of those tasks:

\begin{itemize}
\item Identify and extend ontologies and standards to facilitate safe
  and efficient storage, reuse, interoperation and sharing of rich
  mathematical data whilst taking into account of provenance and
  citability
\item data archiving and sharing in a semantically sound way component
  architecture, user interfaces, deployment frameworks and
  standardisation and interoperability of software systems.
\end{itemize}


This work will be backed by research on social aspects of
collaborative software development in mathematics

\subsection{On open proposal writing and management}

An innovative aspect of the \ODK project is that the proposal writing
and management happens, as much as is practical and without infringing
on privacy, in the open. For example, the git repository for the
proposal, as well as progress trackers for tasks and deliverables are
hosted publicly on github. This has proven a strong feature to get
early feedback from the community.


%Computational experiments have led to new conjectures which have had a deep impact on the future development of
%mathematics. An outstanding example is the Birch and Swinnerton-Dyer conjecture (one of the Clay Millennium Problems).
%Databases relying on computer calculations such as the Small Groups Library in GAP, the Modular Atlas in group and rep-
%resentation theory, or the LMFDB, provide indispensable tools for researchers. A constructive way of understanding proofs
%%of deep theorems yields algorithmic tools to deal with highly abstract concepts. These tools make the concepts available
%to a broader class of researchers, with many potential applications. A prominent example from algebraic geometry is the
%desingularisation theorem of Hironaka, for which Hironaka won the Fields Medal, and its algorithmisation by Villamayor.





\section{Integrating Mathematical Software Systems via the Math-in-the-Middle Approach}

% Mathematics has a rich notion of data: it can be either numeric or symbolic data;
% knowledge about mathematical objects given as statements (definitions, theorems or
% proofs); or software that computes with these mathematical objects. All this data is
% really a common resource, and should be maintained as such


To achieve the goal of assembling the ecosystem of mathematical software systems in the
\pn project into a coherent mathematical VRE, we have to make the systems interoperable at
a mathematical level. In particular, we have to establish a common meaning space that
allows to share computation, visualization of the mathematical concepts, objects, and
models (COMs) between the respective systems. Building on this we can build a VRE with
classical IDE techniques. 

Concretely, the problem is that the software systems in \pn have their own representations
of and functionalities for the COMs involved. This starts with simple naming issues:
elliptic curves are represented by \lstinline|ec| in the LMFDB, and as
\lstinline|EllipticCurve| in SageMath, persists through the underlying data structures,
and becomes virulent at the level of algorithms, their parameters, and domains of
appliccability.


% *** maybe use this above *** 
    % \begin{omtext}[title=Advantages] well-known Open Source Software
    %   \begin{enumerate}
    %   \item Let the specialists do that they do best and like\lec{and avoid what the don't}
    %   \item collaboration exponentiates results
    %   \item competition fosters innovation \lec{+ no vendor lock-in}
    %   \end{enumerate}
    % \end{omtext}


%   \item 
%     \begin{omtext}[title=Problem]
%       does an elliptic curve mean the same in GAP, SAGE, LMFDB?
%       \begin{itemize}
%       \item otherwise delegating computation becomes unsound
%       \item storing data in a central KB becomes unsafe 
%       \item the user cannot interpret the results in an UI
%       \end{itemize}
%     \end{omtext}
% \item
%     \begin{omtext}[title=Idea]
%       Need a common meaning space for safe distributed computation in a VRE!
%     \end{omtext}

\ednote{MK: essentially the St. Andrews stuff}
% outline:
% - MMT: distributed nature, flexibility, 
% - 

\section{An application: toward multi-system semantic aware handle interfaces}

\subsection{The handle paradigm in system interfaces}\label{the-handle-paradigm-in-system-interfaces}

The ``handle'' paradigm has become a classic when interfacing two
computational mathematics systems. For example, most of the \Sage
interfaces, including that for \GAP, \Singular, or \Pari use this
paradigm to delegate calculations to those systems.

In this paradigm, when a system \texttt{A} delegates a calculation to a
system \texttt{B}, the result \texttt{r} of the calculation is not
converted to a native \texttt{A} object; instead \texttt{B} just returns
a handle (or reference) to the object \texttt{r}. Later \texttt{A} can
run further calculations with \texttt{r} by passing it as argument to
\texttt{B} functions or methods. Advantages of this approach include:

\begin{itemize}
\item Avoiding the overhead of back and forth conversions between
  \texttt{A} and \texttt{B}.
\item Manipulating objects of \texttt{B} from \texttt{A} even if they
  have no native representation in \texttt{A}.
\end{itemize}

\subsection{Semantic handle interfaces}\label{semantic-handle-interfaces}

Whenever \texttt{A} and \texttt{B} share some common semantic (for example the concept of
group), it's desirable that handles behave as native \texttt{A} objects. For example, if a
group \texttt{G} is constructed in \texttt{B}, one wants to manipulate handles to
\texttt{G} or its elements as if they were native \texttt{A} groups or group elements,
even if there is no corresponding native implementation for \texttt{G} in \texttt{A}.
This can be achieved with the usual \emph{adapter} design pattern. The bulk of the work is
the implementation of adapter methods so that, for example, calling the method
\texttt{h.cardinality()} on a \Sage handle \texttt{h} to a \GAP object \texttt{G},
triggers in \GAP a call to \texttt{Size(G)}.

In \Sage, this has been implemented in a couple special cases. For
examples, \Sage permutation groups or matrix groups are built on top
of handles to \GAP objects. However, this implementation is monolithic
and does not scale. For example, if \texttt{h} is a handle to a set
\texttt{S}, \Sage only knows that \texttt{h.cardinality()} can be
computed by \texttt{Size(S)} in \GAP if \texttt{S} is a group; in fact
if \texttt{h} has been constructed through the
\texttt{PermutationGroup} or \texttt{MatrixGroup}
constructors. Whereas we would want this method to be available as
soon as \texttt{S} is a set.

\subsection{Generic/hierarchical semantic handle interfaces}\label{generichierarchical-semantic-handle-interfaces}

During the \href{http://gapdays.de/gap-sage-days2016/}{first joint
  \GAP-\Sage days}, the last author worked on a prototype of generic
semantic handle \Sage-\GAP interface. The idea is twofold:

\begin{enumerate}
\def\labelenumi{\arabic{enumi}.}
\item Every \Sage category (\emph{e.g.}\ the category of sets, of groups) can
  provide a collection of adapter methods that are valid for every
  handle to a \GAP object in the corresponding mathematical category.
  This applies as well to elements and morphisms.
\item When a handle \texttt{h} to a \GAP object \texttt{S} is created,
  the properties of \texttt{S} (its \GAP categories and properties)
  are explored, and the handle \texttt{h} is then put in the matching
  (or closest matching) \Sage category.
\end{enumerate}

For example, here is the adapter for the cardinality method and some
context around:
\begin{lstlisting}
class Sets: # Everything about sets in Sage
    class GAP: # The adapter methods relevant to Sets in the Sage-Gap interface
         class ParentMethods: # Adapter methods for sets
             def cardinality(self): # The adapter for the cardinality method
                 return self.gap().Size().sage()
         class ElementMethods: # Adapter methods for set elements
             ...
         class MorphismMethods: # Adapter methods for set morphisms
             ...
\end{lstlisting}

At the current stage of the implementation, a handle to a \GAP field
behaves essentially like a native \Sage field. This remains valid for
objects of all subcategories as well, from magmas to rings. The
infrastructure is relatively lightweight, and can be extended by
developers and users as the need for more adapter methods arises.

\subsection{Scaling to multisystem interfaces?}\label{scaling-to-multisystem-interfaces}

A second stage was initiated during the
\href{http://opendreamkit.org/2015/12/08/WP6StAndrewsMeeting/}{Knowledge
representation in mathematical software and databases workshop}
organized at the University of St Andrews, St Andrews, 25th-27th
January, 2016.

The approach described earlier is likely to work well for implementing
an interface between two systems. However it does not scale for
interfacing \texttt{n} systems, as this requires the implementation of
\texttt{n(n-1)} independent adapter interfaces.

The key point here is that implementing an adapter method (or
function) typically requires only some simple abstract information on
the method, namely its signature and its names in both systems.  In
particular, the only things that changes between an \texttt{A->B}
adapter method and the equivalent \texttt{C->D} adapter method are the
names of the methods.

The second stage of this project is therefore to explore whether the
interfaces could be automatically generated from a consistent
formalizations of the systems.

\ednote{NT:Update this paragraph w.r.t. the rest of this section}

Ideally, the mathematical structure and operations would be described
once, \emph{e.g.}\ in the MMT language (the blue blob in Michael's talk) and
then each system would be formalized by specifying how the operations
are implemented (the purple blobs). For example, one would specify in
MMT that a magma is a set with a binary operation denoted by default
\texttt{o}. The relevant category in \Sage is \texttt{Magmas()}, and
the binary operation is implemented by the method \texttt{\_mul\_}.

We experimented with doing this formalization using lightweight
annotations in the \Sage source code such as:
\begin{lstlisting}
@semantic(mmt="sets")
class Sets:
    class ParentMethods:
         @semantic(mmt="o", gap="Size")
         @abstractmethod
         def cardinality(self):
             r"""
             Return the cardinality of ``self``.
             """
\end{lstlisting}
Note: the only additions to the original source code are the \texttt{@semantic} lines.

Several variants of the annotations exist to allow for adding
annotations on existing categories without touching their file, and also
for specifying directly the corresponding method names in other systems
when this has not yet been formalized elsewhere. Similarly, one could
provide directly the signature information in case that is not yet
modelled in MMT.

\subsection{Difficulties}\label{difficulties}

In \Sage and \GAP (and most other systems with some category mechanism) we distinguish
additive magma and multiplicative magma, duplicating all the information, code, etc. In
MMT however, thanks to morphisms which allow to rename operations transparently, there is
no such distinction: there are just Magmas.

Hence, to actually map additive magmas in \Sage to additive magmas in \GAP (and map the
corresponding methods), one need in the intermediate MMT step to keep an extra bit of
information, namely whether the monoid is additive or multiplicative (or something else;
think of the bracket operation of Lie algebras).


%%% Local Variables:
%%% mode: latex
%%% TeX-master: "deleted-scenes"
%%% End:

%  LocalWords:  subsubsection texttt itemize emph labelenumi lstlisting organized ednote
%  LocalWords:  generichierarchical-semantic-handle-interfaces formalizations formalized
%  LocalWords:  scaling-to-multisystem-interfaces formalization mmt


\section{Conclusion}
\ednote{MK: I am not sure we need this, but we will see}
\subsubsection*{Acknowledgements}
The authors gratefully acknowledge discussions and experimentation at the St.\ Andrews
workshop, which clarified the ideas behind the math-in-the-middle approach;
in particular John Cremona, Paul-Olivier Dehaye, Luca de Feo, Mihnea Iancu, Alexander
Konovalov, Samuel Leli\`evre, Steve Linton, Dennis M\"uller, Markus Pfeiffer, Viviane Pons,
Florian Rabe, and Tom Wiesing.

We acknowledge financial support from the OpenDreamKit Horizon 2020 European Research
Infrastructures project (\#676541).

\printbibliography
\end{document}
%%% Local Variables:
%%% mode: latex
%%% TeX-master: t
%%% End:

%  LocalWords:  maketitle endeavor ednote ODKproposal printbibliography subsubsection
