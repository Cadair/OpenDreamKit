\subsection{Presenting LMFDB data}
\subsubsection{Short description of LMFDB's goals and implementation}
The \emph{$L$-functions and modular forms database} is a project involving dozens of  mathematicians, who assemble computational data about $L$-functions, modular  forms and related number theoretic objects. The main output of the project is a website, hosted at \url{http://www.lmfdb.org}, that presents this data in a way that could serve as a reference for research efforts and should be accessible at the graduate student level.  The mathematical concepts underlying the \LMFDB are extremely complex and varied, so part of the effort has been focused on how to relay mathematical definitions and their relationships to data and software. For this purpose, the \LMFDB has developed so-called \emph{knowls}, which are a technical solution to present \LaTeX-encoded information interactively, heavily exploiting the concept of transclusion. The end result there is a very modular and highly interlinked set of definitions in mathematical natural language. 
 
The \LMFDB backend is written in \textsf{python}, with a heavy reliance on \Sage and the document database system \Mongo \cite{lmfdb-repo}. Again, due to the complexity of the objects considered, many idiosyncratic encodings are used for the data. This makes the whole data management lifecycle particularly tricky, and dependent on different select groups of individuals for each component. There are ongoing efforts to document the formats used \cite{lmfdb-formats}.


\subsubsection{Tentative approach to MMT semantic layers for the LMFDB}
Since the LMFDB spans the whole "vertical" of writing software to produce data up to presenting this new knowledge,  it is a perfect test case for a large scale try-out of the "knowledge in the middle": a semantic layer would be beneficial to its activities across data, knowledge and software, which it would help integrate more cohesively and systematically. Among the components of the LMFDB, elliptic curves stand out in the best shape, and a source of best practices for other areas. For this reason, the OpenDreamKit collaboration targeted that relatively small subset of the LMFDB for its prototype. We plan to extend coverage in a second phase, and expect it to be relatively easy if we manage to demonstrate the added value of a semantic layer for our prototype. 

In practice, our first step was to integrate into the MMT system the mathematical knowledge that had already been made explicit in the knowls. For this, we used the \stex interpreter, which was able to crudely type-check the definitions, avoiding circularity and ensuring some level of consistency in their scope. Knowledge in this format immediately became browsable through \textsf{MathHub.info}, a project developed in parallel to MMT to host such formalisations. 

The second step consisted of translating these informal natural language definitions into progressively more exhaustive MMT formalisations of mathematical concepts. While this certainly required knowledge of the MMT language, the need for a mathematics background was reduced thanks to the very atomic definitions present in the informal layer.  