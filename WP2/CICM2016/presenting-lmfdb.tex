\section{Presenting LMFDB data}
\subsection{Short description of LMFDB's goals and implementation}
The \emph{$L$-functions and modular forms database} is a project involving dozens of  mathematicians, who assemble computational data about $L$-functions, modular  forms and related number theoretic objects. The main output of the project is a website, hosted at \url{http://www.lmfdb.org}, that presents this data in a way that could serve as a reference for research efforts and should be accessible at the graduate student level.  The mathematical concepts underlying the \LMFDB are extremely complex and varied, so part of the effort has been focused on how to relay mathematical definitions and their relationships to data and software. For this purpose, the \LMFDB has developed so-called \emph{knowls}, which are a technical solution to present \LaTeX-encoded information interactively, heavily exploiting the concept of transclusion. The end result there is a very modular and highly interlinked set of definitions in mathematical natural language. 
 
The \LMFDB backend is written in \textsf{python}, with a heavy reliance on \Sage and the document database system \Mongo \cite{lmfdb-repo}. Again, due to the complexity of the objects considered, many idiosyncratic encodings are used for the data. This makes the whole data management lifecycle particularly tricky, and dependent on different select groups of individuals for each component. 

\subsection{Tentative approach to MMT semantic layers for the LMFDB}
Since the LMFDB spans the whole "vertical", from writing software, to producing new data, up to presenting this new knowledge,  it is a perfect test case for a large scale try-out of the "knowledge in the middle": a semantic layer would be beneficial to its activities across data, knowledge and software, which it would help integrate more cohesively and systematically. Among the components of the LMFDB, elliptic curves stand out in the best shape, and a source of best practices for other areas. For this reason, the OpenDreamKit collaboration targeted that relatively small subset of the LMFDB for its prototype. We plan to extend coverage in a second phase, and expect it to be relatively easy if we manage to demonstrate the added value of a semantic layer for our prototype. 

In practice, our first step was to integrate into the MMT system the mathematical knowledge that had already been made explicit in the knowls (see Fig.~\ref{stex-ec}). For this, we used the \stex interpreter, which was able to crudely type-check the definitions, avoiding circularity and ensuring some level of consistency in their scope. Knowledge translated into this format immediately became browsable through \textsf{MathHub.info}, a project developed in parallel to MMT to host such formalisations. 

\lstinputlisting[language={[sTeX]TeX},label={stex-ec},
   caption= {Sample \protect\stex implementation of \LMFDB knowl content for elliptic curves
    and their Weierstrass models.}]{elliptic-curve.tex}

The second step consisted of translating these informal natural language definitions into
progressively more exhaustive MMT formalisations of mathematical concepts (see
Listing~\ref{lst:mmt-ec}). This is possibly the hardest step, as that layer will have to
eventually interoperate with the informal \stex definitions, the data that has been
previously computed, the software to compute more, and the software needed to view all
this information. We found that the best approach was to push for "small theories", again
emphasizing modularity.

\def\RS{\colorbox{blue}{\phantom{l}}}
\def\GS{\colorbox{red}{\phantom{l}}}
\lstinputlisting[morekeywords={namespace,theory,include},mathescape,
caption= {MMT formalisation of elliptic curves and their Weierstrass models},
label=lst:mmt-ec]{elliptic-curve.mmt}


The next step was in integrating computational data into MMT. We first needed to expand
the capabilities of MMT with respect to literals. For this purpose, the MMT core team
constructed an MMT interface with the \LMFDB API, which delivered data in JSON
format. This is not fully satisfactory, and a more generic solution could be found,
interfacing for instance directly with the \Mongo database.  Once JSON documents were
obtained, we still needed to define explicit specifications for them. There are recent
ongoing efforts \cite{lmfdb-formats} to document in natural language the formats used by
the \LMFDB, and this provided an easy starting point. Actually, the scope of those
specifications had to be extended, as they were not focused only on the format used to
store the data fields (string \emph{vs.} float \emph{vs.} integer tuple, for instance),
but also the types these data fields were meant to represent (rational \emph{vs.}
polynomial, for instance). It turns out that many patterns exist there: often different
people will encode the same semantic type with the same data format, following the same
conventions. Therefore, to maximize reusability, we again emphasized modularity and
developed \emph{MMT codecs}. Codecs integrate \emph{some} information about semantics and
some about data formats, and are composable. For instance, one could compose two MMT
codecs, say \emph{polynomial-as-reversed-list} and \emph{rational-as-tuple-of-int}, to
signify that the data $[(2,3),(0,1),(4,1)]$ is meant to represent the polynomial
$4x^2+2/3$. Of course, these codecs could be further decomposed (signalling which variable
name to use, for instance). The initial costs of developing these codecs are high, but the
clarity gained in documentation is valuable, they are highly reusable, and they
drastically expand the range of tooling that can be built around data management. Our
efforts also fit neatly alongside similar efforts underway across the sciences to
standardize metadata formats (for instance through the Research Data Alliance's Typing
Registry Working Group\cite{rda-typing}).

% emphasis on typing systems, which make sense in each individual context and come with
% associated goals and processes for maintenance.  MMT as above that, by necessity

%  LocalWords:  subsubsection emph knowls textsf lmfdb-repo stex-ec stex knowl mmt-ec odk
%  LocalWords:  Weierstrass emphasizing alltt tp rightarrow vdash doteq vdash doteq lst
%  LocalWords:  polynomial_equation_injectivity a_invariants_factors_injectivity colorbox
%  LocalWords:  minimality_idempotence lmfdb-formats emphasized composable standardize
%  LocalWords:  rda-typing lstinputlisting mathescape elliptic-curve.mmt

%%% Local Variables:
%%% mode: latex
%%% TeX-master: "paper"
%%% End:
