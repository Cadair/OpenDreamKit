\section{Semantics in Sage}

The \Sage library includes 40k functions and allows for manipulating
thousands of different kinds of objects. As usual in such large
systems, it's critical for taming code bloat to
\begin{enumerate}[(i)]
\item identify the core concepts describing common behavior among the
  objects;
\item exploit this to implement generic operations that apply on all
  object having a given behavior, with appropriate specializations
  when performance calls for it.
\item design or choose a process for selecting the best implementation
  available when calling an operation on one or several objects.
\end{enumerate}

Fortunately in mathematics a lot of (i) has already been taken care
off over the centuries, in particular in the context of abstract
algebra. Our running examples of concepts in this paper will be that
of (multiplicative) \emph{magma}: a set $S$ endowed with a binary
operation $\dot: S\times S \mapsto S$ and of \emph{semigroup}: a magma
such that the binary operation is associative. Thanks to
associativity, a semigroup comes endowed with the powering operation
which can be implemented generically from the binary operation.

\ednote{NT: include here the MMT theory for magmas / semigroups}

In general, the concepts involve sets endowed with a certain number of
basic operations (addition, product, coproduct, ...) which satisfying
certain axioms (associativity, commutativity, ...). Typical concepts
include \emph{fields}, \emph{rings}, \emph{groups}, which are
naturally organized into a hierarchy according to the available
operations and axioms (a field is a ring, etc).

In practice, one wants to compute either with the elements of the
sets, with the sets themselves (called \emph{parents} in \Sage,
following the \Magma tradition), or with morphisms. Category theory
provides a convenient language, so we speak of the category of groups,
the category of fields, ...

Many selection processes for (iii) are available, including object
oriented programming with methods and/or multimethods, modular
programming and traits, composition, etc. As early as \ednote{NT: find
  the date}, the \Axiom system has based its selection process upon a
hierarchy of classes which models the hierarchy of categories, with
operations implemented as methods. Followers include the \MuPAD, or
\Fricas systems. \Sage builds on the same tradition, though using a
general purpose object oriented language (\Python). \GAP uses a custom
selection process described in Section~\ref{...}.

\ednote{NT: Could say more here about the fact that the mathematical
  categories are modeled explicitly in Sage, and not only through a
  hierarchy of classes}

Those design choices are largely motivated by another specific aspect
of mathematics: the number of fundamental concepts is actually fairly
small, and all the richness comes from the many ways the concepts can
be combined together.

To summarize, \emph{mathematical knowledge} from abstract algebra is
modeled explicitly in \Sage, and used to support genericity, control
the method selection process, structure the code and documentation,
enforce consistency, and provide generic tests.


%%% Local Variables:
%%% mode: latex
%%% TeX-master: "paper"
%%% End:
