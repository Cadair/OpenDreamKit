\documentclass{article}
\usepackage[show]{ed}
\usepackage{odksystems}
\usepackage{listings}
\lstset{basicstyle=\sf,columns=fullflexible}
\usepackage{graphicx}
\usepackage{hyperref}
\usepackage{url}
\usepackage[style=alphabetic,backend=bibtex]{biblatex}
\addbibresource{kwarc.bib}% do not change
\addbibresource{rest.bib}% add bibs here!

\title{Deleted Scenes from the OpenDreamKit paper}
\author{all}
\begin{document}
\maketitle


\subsection{Stuff from the Intro}

Computational experiments have led to new conjectures which have had a deep impact on the
future development of mathematics. An outstanding example is the Birch and Swinnerton-Dyer
conjecture (one of the Clay Millennium Problems).  Databases relying on computer
calculations such as the Small Groups Library in GAP, the Modular Atlas in group and rep-
resentation theory, or the LMFDB, provide indispensable tools for researchers. A
constructive way of understanding proofs of deep theorems yields algorithmic tools to deal
with highly abstract concepts. These tools make the concepts available to a broader class
of researchers, with many potential applications. A prominent example from algebraic
geometry is the desingularisation theorem of Hironaka, for which Hironaka won the Fields
Medal, and its algorithmisation by Villamayor.

\subsection{Existing Virtual Research Environments for Mathematics}

\begin{oldpart}{MK: this should shortened, also we must mention Mathematica, Maple, what
    else? What are their common parts?}
  Early VRE's include \SMC, a collaborative online environment where students, teachers
  and researchers can create, customize, and share a project. This project essentially
  consists of a virtual machine, with a simple web-based user interface, and ready-to-use
  software for interactive computations (\emph{e.g.}\ \Sage) and authoring (\emph{e.g.}\
  \LaTeX), with facilities for real-time communication through chat, video, and shared
  editing of documents, programs and worksheets.  For education purposes, course material
  can be provided as worksheets, assignments can be distributed, collected, and returned
  as well.

  Technically speaking, \SMC is a specific open-source cloud-based Virtual Research and
  Teaching Environment for mathematics developed since 2013 under the lead of William
  Stein, with funding from the NSF, and Google's Education Grant program.  \ednote{NT:
    Mention SageMath, Inc.?}  \ednote{SL: 2016: SageMath, Inc.\ accepted in Google startup
    program, see https://twitter.com/wstein389/status/708439137200836608}

  It presently hosts over 250,000 projects and has over 12,000 weekly active users. This
  fast adoption by a wide variety of users demonstrates the relevance and the long-term
  impact this kind of collaborative environments can have.
\end{oldpart}

\section{An application: toward multi-system semantic aware handle interfaces}

\subsection{The handle paradigm in system interfaces}\label{the-handle-paradigm-in-system-interfaces}

The ``handle'' paradigm has become a classic when interfacing two
computational mathematics systems. For example, most of the \Sage
interfaces, including that for \GAP, \Singular, or \Pari use this
paradigm to delegate calculations to those systems.

In this paradigm, when a system \texttt{A} delegates a calculation to a
system \texttt{B}, the result \texttt{r} of the calculation is not
converted to a native \texttt{A} object; instead \texttt{B} just returns
a handle (or reference) to the object \texttt{r}. Later \texttt{A} can
run further calculations with \texttt{r} by passing it as argument to
\texttt{B} functions or methods. Advantages of this approach include:

\begin{itemize}
\item Avoiding the overhead of back and forth conversions between
  \texttt{A} and \texttt{B}.
\item Manipulating objects of \texttt{B} from \texttt{A} even if they
  have no native representation in \texttt{A}.
\end{itemize}

\subsection{Semantic handle interfaces}\label{semantic-handle-interfaces}

Whenever \texttt{A} and \texttt{B} share some common semantic (for example the concept of
group), it's desirable that handles behave as native \texttt{A} objects. For example, if a
group \texttt{G} is constructed in \texttt{B}, one wants to manipulate handles to
\texttt{G} or its elements as if they were native \texttt{A} groups or group elements,
even if there is no corresponding native implementation for \texttt{G} in \texttt{A}.
This can be achieved with the usual \emph{adapter} design pattern. The bulk of the work is
the implementation of adapter methods so that, for example, calling the method
\texttt{h.cardinality()} on a \Sage handle \texttt{h} to a \GAP object \texttt{G},
triggers in \GAP a call to \texttt{Size(G)}.

In \Sage, this has been implemented in a couple special cases. For
examples, \Sage permutation groups or matrix groups are built on top
of handles to \GAP objects. However, this implementation is monolithic
and does not scale. For example, if \texttt{h} is a handle to a set
\texttt{S}, \Sage only knows that \texttt{h.cardinality()} can be
computed by \texttt{Size(S)} in \GAP if \texttt{S} is a group; in fact
if \texttt{h} has been constructed through the
\texttt{PermutationGroup} or \texttt{MatrixGroup}
constructors. Whereas we would want this method to be available as
soon as \texttt{S} is a set.

\subsection{Generic/hierarchical semantic handle interfaces}\label{generichierarchical-semantic-handle-interfaces}

During the \href{http://gapdays.de/gap-sage-days2016/}{first joint
  \GAP-\Sage days}, the last author worked on a prototype of generic
semantic handle \Sage-\GAP interface. The idea is twofold:

\begin{enumerate}
\def\labelenumi{\arabic{enumi}.}
\item Every \Sage category (\emph{e.g.}\ the category of sets, of groups) can
  provide a collection of adapter methods that are valid for every
  handle to a \GAP object in the corresponding mathematical category.
  This applies as well to elements and morphisms.
\item When a handle \texttt{h} to a \GAP object \texttt{S} is created,
  the properties of \texttt{S} (its \GAP categories and properties)
  are explored, and the handle \texttt{h} is then put in the matching
  (or closest matching) \Sage category.
\end{enumerate}

For example, here is the adapter for the cardinality method and some
context around:
\begin{lstlisting}
class Sets: # Everything about sets in Sage
    class GAP: # The adapter methods relevant to Sets in the Sage-Gap interface
         class ParentMethods: # Adapter methods for sets
             def cardinality(self): # The adapter for the cardinality method
                 return self.gap().Size().sage()
         class ElementMethods: # Adapter methods for set elements
             ...
         class MorphismMethods: # Adapter methods for set morphisms
             ...
\end{lstlisting}

At the current stage of the implementation, a handle to a \GAP field
behaves essentially like a native \Sage field. This remains valid for
objects of all subcategories as well, from magmas to rings. The
infrastructure is relatively lightweight, and can be extended by
developers and users as the need for more adapter methods arises.

\subsection{Scaling to multisystem interfaces?}\label{scaling-to-multisystem-interfaces}

A second stage was initiated during the
\href{http://opendreamkit.org/2015/12/08/WP6StAndrewsMeeting/}{Knowledge
representation in mathematical software and databases workshop}
organized at the University of St Andrews, St Andrews, 25th-27th
January, 2016.

The approach described earlier is likely to work well for implementing
an interface between two systems. However it does not scale for
interfacing \texttt{n} systems, as this requires the implementation of
\texttt{n(n-1)} independent adapter interfaces.

The key point here is that implementing an adapter method (or
function) typically requires only some simple abstract information on
the method, namely its signature and its names in both systems.  In
particular, the only things that changes between an \texttt{A->B}
adapter method and the equivalent \texttt{C->D} adapter method are the
names of the methods.

The second stage of this project is therefore to explore whether the
interfaces could be automatically generated from a consistent
formalizations of the systems.

\ednote{NT:Update this paragraph w.r.t. the rest of this section}

Ideally, the mathematical structure and operations would be described
once, \emph{e.g.}\ in the MMT language (the blue blob in Michael's talk) and
then each system would be formalized by specifying how the operations
are implemented (the purple blobs). For example, one would specify in
MMT that a magma is a set with a binary operation denoted by default
\texttt{o}. The relevant category in \Sage is \texttt{Magmas()}, and
the binary operation is implemented by the method \texttt{\_mul\_}.

We experimented with doing this formalization using lightweight
annotations in the \Sage source code such as:
\begin{lstlisting}
@semantic(mmt="sets")
class Sets:
    class ParentMethods:
         @semantic(mmt="o", gap="Size")
         @abstractmethod
         def cardinality(self):
             r"""
             Return the cardinality of ``self``.
             """
\end{lstlisting}
Note: the only additions to the original source code are the \texttt{@semantic} lines.

Several variants of the annotations exist to allow for adding
annotations on existing categories without touching their file, and also
for specifying directly the corresponding method names in other systems
when this has not yet been formalized elsewhere. Similarly, one could
provide directly the signature information in case that is not yet
modelled in MMT.

\subsection{Difficulties}\label{difficulties}

In \Sage and \GAP (and most other systems with some category mechanism) we distinguish
additive magma and multiplicative magma, duplicating all the information, code, etc. In
MMT however, thanks to morphisms which allow to rename operations transparently, there is
no such distinction: there are just Magmas.

Hence, to actually map additive magmas in \Sage to additive magmas in \GAP (and map the
corresponding methods), one need in the intermediate MMT step to keep an extra bit of
information, namely whether the monoid is additive or multiplicative (or something else;
think of the bracket operation of Lie algebras).


%%% Local Variables:
%%% mode: latex
%%% TeX-master: "deleted-scenes"
%%% End:

%  LocalWords:  subsubsection texttt itemize emph labelenumi lstlisting organized ednote
%  LocalWords:  generichierarchical-semantic-handle-interfaces formalizations formalized
%  LocalWords:  scaling-to-multisystem-interfaces formalization mmt

\section{Semantics in Sage}

The \Sage library includes 40k functions and allows for manipulating
thousands of different kinds of objects. As usual in such large
systems, it's critical for taming code bloat to
\begin{enumerate}[(i)]
\item identify the core concepts describing common behavior among the
  objects;
\item exploit this to implement generic operations that apply on all
  object having a given behavior, with appropriate specializations
  when performance calls for it.
\item design or choose a process for selecting the best implementation
  available when calling an operation on one or several objects.
\end{enumerate}

Fortunately in mathematics a lot of (i) has already been taken care
off over the centuries, in particular in the context of abstract
algebra. Our running examples of concepts in this paper will be that
of (multiplicative) \emph{magma}: a set $S$ endowed with a binary
operation $\dot: S\times S \mapsto S$ and of \emph{semigroup}: a magma
such that the binary operation is associative. Thanks to
associativity, a semigroup comes endowed with the powering operation
which can be implemented generically from the binary operation.

\ednote{NT: include here the MMT theory for magmas / semigroups}

In general, the concepts involve sets endowed with a certain number of
basic operations (addition, product, coproduct, ...) which satisfying
certain axioms (associativity, commutativity, ...). Typical concepts
include \emph{fields}, \emph{rings}, \emph{groups}, which are
naturally organized into a hierarchy according to the available
operations and axioms (a field is a ring, etc).

In practice, one wants to compute either with the elements of the
sets, with the sets themselves (called \emph{parents} in \Sage,
following the \Magma tradition), or with morphisms. Category theory
provides a convenient language, so we speak of the category of groups,
the category of fields, ...

Many selection processes for (iii) are available, including object
oriented programming with methods and/or multimethods, modular
programming and traits, composition, etc. As early as \ednote{NT: find
  the date}, the \Axiom system has based its selection process upon a
hierarchy of classes which models the hierarchy of categories, with
operations implemented as methods. Followers include the \MuPAD, or
\Fricas systems. \Sage builds on the same tradition, though using a
general purpose object oriented language (\Python). \GAP uses a custom
selection process described in Section~\ref{...}.

\ednote{NT: Could say more here about the fact that the mathematical
  categories are modeled explicitly in Sage, and not only through a
  hierarchy of classes}

Those design choices are largely motivated by another specific aspect
of mathematics: the number of fundamental concepts is actually fairly
small, and all the richness comes from the many ways the concepts can
be combined together.

To summarize, \emph{mathematical knowledge} from abstract algebra is
modeled explicitly in \Sage, and used to support genericity, control
the method selection process, structure the code and documentation,
enforce consistency, and provide generic tests.


%%% Local Variables:
%%% mode: latex
%%% TeX-master: "deleted-scenes"
%%% End:

\section{Exploring GAP types}\label{sec:gaptypes} 

\subsection{Brief introduction to GAP types and categories.}\label{gap-types-intro}

%\ednote{MP: I am not sure what I claim below about \Sage is true, it
%  also irks me a bit that we seem to conflate the idea of a type system with
%  the idea of organising mathematical hierarchies. Of course in \GAP
%  system this is intentional, in \Sage, I don't know. In my head \Sage
%  uses whatever python uses as the type system (duck typing?) and then intro-
%  duces a category system on top. We should agree on a level of description
%  that fits.}

While the \Sage type system is object-oriented, the \GAP type
system puts more of an emphasis on \emph{operations} on and between objects.

Breuer and Linton describe the \GAP type system in \cite{breuer-linton}, and
the \GAP documentation \cite{GAP4} also contains an extensive technical
description of the \GAP type system.

A type in \GAP is a pair consisting of a \emph{family} and a \emph{filter}.

Families partition the space of objects in \GAP, so every object lies in exactly one family.

A filter is a set of \emph{elementary filters}, and hence filters form a hierarchy on
objects by the subset relation on filters.
We say that an object is \emph{in a filter $F$} if its type's filter component
contains $F$ as a subset.

\emph{Operations} in \GAP are declared with an arity and for each argument with a
most general filter for which they are applicable. For instance there is an operation
for forming the direct product of two groups.
The programmer can install \emph{methods} for an operation which can carry strictly
more specific filters for the inputs.

At runtime \GAP through a very sophisticated mechanism called \emph{method selection} will
select the most appropriate method for the given arguments to an operation and execute it.

\emph{Categories} are filters that model mathematically similar objets. In terms of
algebraic structures we can think of a category as the signature of the structure.
For instance there are categories called \texttt{IsMagma}, \texttt{IsMagmaWithOne}, and
\texttt{IsMagmaWithInverses}, which we can think of as objects having signature $\{ * \}$,
$\{*,1\}$, and $\{*,^{-1}\}$ respectively. Semigroups, monoids, or groups are not
categories in \GAP.

Once an object is created, the category it is in cannot be altered.

\emph{Representations} are filters that give a way to represent mathematical
objects in different ways. One of the examples from the \GAP library are permutations
which can be represented in 2-bytes acting on at most 65536 points, or 4 bytes, acting
on at most $2^{32}-1$ points. Other examples include matrices in sparse or dense
representation, or finite field elements, where particularly matrices with entries in
the field of order 2 allow a very efficient representation.

A \emph{Property} \texttt{P} is realised by two filters \texttt{P} and \texttt{HasP} and an
operation which is also called \texttt{P}.

This models three possible states for a property: Its value can either be known or unknown,
which is reflected by the filter \texttt{HasP}, and if it is known, then the filter \texttt{P}
says whether the property holds or not.
If the value of the property is unknown, but there are methods installed for the operation
\texttt{P}, then \GAP will be attempt to compute the value of \texttt{P} using that method.

Examples of properties in \GAP are \texttt{IsAssociative}
or \texttt{IsCommutative}. A group in \GAP is an object that is in the filter
\texttt{IsMagmaWithInverses} and \texttt{IsAssociative}. An abelian group will additionally
be in the filter \texttt{IsCommutative}.

An \emph{attribute} in \GAP is a value attached to a \GAP object. There is
a filter attached with each attribute that reflects whether the value
of the attribute is known, and an operation which can be invoked to determine
the value of the attribute if it is not known.
\texttt{Size} or \texttt{Centre} are two attributes that are defined for groups.

The values of attributes and properties can be unknown on creation,
can be computed on demand, and their values can then be stored for later
reuse without the need to be recomputed. Note that in particular the knowledge
accumulated in the type of a \GAP object can influence method selection, so for
example attaining the knowledge that a group is nilpotent will allow for more
efficient methods to be run for finding its centre.

\begin{lstlisting}
gap> IsGroup;
<Filter "(IsMagmaWithInverses and IsAssociative)">
gap> IsMagmaWithInverses;
<Category "IsMagmaWithInverses">
gap> IsAssociative;
<Property "IsAssociative">
gap> IsSet;
<Property "IsSSortedList">
gap> IsFinite;
<Property "IsFinite">
gap> IsSet=IsSSortedList;
true
gap> G := Group((1,2), (2,3,4));
Group([ (1,2) ])
gap> HasSize(G);
false
gap> HasIsCommutative(G);
false
gap> Size(G);
24
gap> HasSize(G);
true
\end{lstlisting}

\ednote{TODO (???) Compare and contrast it with the Sage type system}

\subsection{Tentative approaches to exporting GAP types.}\label{gap-types-export}

Encoded in filters, categories, representations, attributes, and properties in \GAP
there is a wealth of mathematical knowledge. \GAP allows some introspection
of this knowledge after the system is loaded.

Having a clear picture of the relations between different objects is 
very helpful to GAP developers, package authors, and users.
For example one might be interested in the attributes or properties that \GAP can
compute for an object, or how it tries to compute them.

During the OpenDreamKit workshop in St~Andrews in January 2016 we developed
tools to more conveniently access mathematical knowledge encoded in \GAP,
such as introspection inside a running \GAP session, export to JSON to import
to MMT, and export as a graph for visualisation and exploration.
\ednote{picture based on \url{https://github.com/OpenDreamKit/OpenDreamKit/issues/165}?}

We will make these tools available as part of the standard \GAP distribution with the next
major release of \GAP, as they will prove useful in the development of the \GAP Jupyter interface
\url{https://github.com/gap-packages/jupyter-gap}, and possibly to do
internal consistency checks of \GAP types.

As a side-effect of the work outlined above, we fixed a number of bugs in the handling of
special categories in \GAP.

The JSON output of the \GAP object system after loading a default set of packages is currently
around 11 Megabytes in size and takes many hours to import into MMT. We did not yet attempt to
load a collection of \GAP packages that would expose even more data.

The graph exported for visualisation has 540 vertices, 759 edges and 8 connected components, if
packages are loaded this increses to 1616 vertices, 2178 edges and 17 connected components.

\ednote{NT: Do you have anything to say about the GAP-MMT formalization? 
E.g. hints on potential ways this formalization may be written?}
\ednote{MP: I have some ideas as to how I would write an MMT formalisation, unfortunately I do not
  understand MMT well enough yet to know whether my ideas make any sense. I'll think about this a bit
  more and add my comments then\\
       MP: Also, what exactly are we talking about when we talk about MMT formalisation? The description
  of an export, or how to implement structures exported from MMT in GAP?}

\subsection{An application: consistency checker for the GAP
  documentation.}\label{gap-types}

One of the immediate outcomes of the development of the tools described in the
previous section is the consistency checker for the GAP documentation. 

GAP uses special format for its main manuals. It is called GAPDoc and is 
provided by the GAP package with the same name \cite{gapdoc}. Besides main 
manuals, it is adopted by 97 out of 130 packages currently redistributed 
with GAP. Using GAPDoc, one builds text, PDF and HTML versions of the manual
from a common source given in XML.

GAPDoc defines XML constructions to specify the type of the documented object 
(function, operation, attribute, property, etc.). However, due to the 
limitations of the semi-automated conversion of GAP manuals from the \TeX-based
manuals used in GAP 4.4.12 and earlier, a number of objects had their types
stated incorrectly. 

We developed the consistency checker for the GAP documentation, which extracts
type annotations from the documented GAP objects and compares them with their
actual types. It immediately reported almost 400 inconsistencies out of 3674 
manual entries. In the subsequent cleanup, we by now have eliminated about 
75\% of them. The  consistency checker will appear in the next release of
GAP 4.8.3, and will be available via \texttt{make check-manuals}.
It also performs other useful checks: for example, it produces a list of
manual sections having no examples. Thus, the new tool helps to improve
the quality of GAP documentation, and may be useful for the similar checks
of those GAP packages which use GAPDoc-based manuals.

% \url{https://github.com/gap-system/gap/pull/675}
% \url{https://github.com/gap-system/gap/pull/538}

%%% Local Variables:
%%% mode: latex
%%% TeX-master: "paper"
%%% End:

%  LocalWords:  ednote emph breuer-linton texttt texttt itemize Jupyter formalization
%  LocalWords:  gapdoc gaptypes


\printbibliography
\end{document}



%%% Local Variables:
%%% mode: latex
%%% TeX-master: t
%%% End:

%  LocalWords:  maketitle Swinnerton-Dyer resentation desingularisation Hironaka Hironaka
%  LocalWords:  algorithmisation Villamayor oldpart emph emph ednote
