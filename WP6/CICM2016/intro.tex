\section{Introduction}
%\ednote{POD to MK: bibliography has some very long entries, lots of space could be gained there}
% \ednote{POD: I think this fits here. NT: wondering if the next three paragraph would not fit better
%   in the odk section; the reader more interested by the MitM part may
%   be put off by this long discussion}

From their earliest days, computers have been used in pure mathematics, either to make
tables, to prove theorems (famously the four colour theorem) or, as with the astronomer's
telescope, to explore new theories. Computer-aided experiments, and the use of databases
relying on computer calculations such as the Small Groups Library in GAP, the Modular
Atlas in group and representation theory, or the $L$-functions and Modular Forms Database (\LMFDB, see later), are part of the standard
toolbox of the pure mathematician, and certain areas of mathematics completely depend on
it. Computers are also increasingly used to support collaborative work and education.

The last decades witnessed the emergence of a wide ecosystem of open-source tools to
support research in pure mathematics. This ranges from specialized to general purpose
computational tools such as \GAP, \PariGP, \Linbox, \MPIR, \Sage, or \Singular, via online
databases like the \LMFDB or online services like Wikipedia,
\Arxiv, or MathOverflow. A great opportunity is the rapid emergence of key technologies,
and in particular the \Jupyter (previously \IPython) platform for interactive and
exploratory computing which targets all areas of science.

This has proven the viability and power of collaborative open-source development models,
by users and for users, even for delivering general purpose systems targeting a large
public (researchers, teachers, engineers, amateurs, \ldots). Yet some critical long term
investments, in particular on the technical side, are in order to boost the productivity
and lower the entry barrier:
\begin{compactitem}
\item Streamlining access, distribution, portability on a wide range of platforms, including
  High Performance Computers or cloud services.
\item Improving user interfaces, in particular in the promising area of collaborative
  workspaces as those provided by \SMC.
\item Lowering barriers between research communities and promoting dissemination. For example
  make it easy for a specialist of scientific computing to use tools from pure
  mathematics, and reciprocally.
\item Bringing together the developer communities to promote tighter collaboration and
  symbiosis, accelerate joint development, and share best practices.
\item Structure the development so as to outsource as much of it as possible to larger communities, and focus
  the work forces on their core specialty: the implementation of mathematical algorithms
  and databases.
\item And last but not least: Promoting collaborations at all scales to further improve
  the productivity of researchers in pure mathematics and applications.
\end{compactitem}

\ODK\, -- ``Open Digital Research Environment Toolkit for the Advancement
of Mathematics'' \cite{OpenDreamKit:on} -- is a project funded under the
European H2020 Infrastructure call \cite{EINFRA-9} on \emph{Virtual
  Research Environments}, to work on many of these problems.

In Section~\ref{sec:odk}, we will introduce the \ODK project  to establish the context for the
``Math-in-the-Middle'' (MitM) integration approach described in
Section~\ref{sec:mitm}. The remaining sections then elucidate the approach by presenting
first experiments and refinements of the chosen integration paradigm:
Section~\ref{sec:lmfdb} details how existing knowledge representation and data structures
can be represented as MitM interface theories with a case study of equipping the \LMFDB
with a MitM-based programming interface.  Section~\ref{sec:gapsage} discusses system
integration between \GAP and \Sage and how this can be routed through a MitM
ontology. Section~\ref{sec:concl} concludes the paper and discusses future work.
% \ednote{R1: Throughout the paper is would be better to reduce the use of acronyms. 
% For example, the paper is unnecessarily difficult to read because of the extensive 
% use of "VRE" and "COMs" as well as the many other acronyms. AK: I suggest to use VRE,
% but revise the others. POD: I have removed COM and many others, leaving this as a reminder to be on the lookout.}

%%% Local Variables:
%%% mode: latex
%%% TeX-master: "paper"
%%% End:

%  LocalWords:  specialized Arxiv Jupyter IPython ldots compactitem emph mitm lfmdb concl
%  LocalWords:  gaptypes
