\section{The \ODK project (2015-2019)}\label{sec:odk}

The \ODK project runs for four years, starting in September 2015, and
involves about 50 people spread over 15 sites in Europe, with a total
budget of 7.6 million euros. The largest portion of that is devoted to
employing an average of 11 researchers and developers working full
time on the project, while the other participants contribute the
equivalent of six people working full time.

\ODK's funding is directed toward \emph{Virtual Research Environments}
(VRE), that is online services enabling groups of researchers,
typically widely dispersed, to work collaboratively on a per project
basis. Rather than constructing a large monolithic VRE, we have
designed our proposal around the long-term investments listed in the
previous section, working on the large scale yet modular integration
of mathematical software. Our end goal is a modular, interoperable,
and customisable VRE toolkit built out of relatively modest
components, and our approach to work on the grease to make this
work. According to the funding scheme, the project addresses, besides
its technical goals, aspects such as outreach, dissemination, or tools
to support teaching.

\ednote{NT: The following paragraph is a political message; not sure
  it fits there; in case we need to save a bit of space, that's one of
  the first things to cut out}
An innovative aspect of the \ODK project is that its preparation and management happens,
as much as is practical and without infringing on privacy, in the open. For example, most
documents, including the proposal itself, are version controlled on public repositories
and progress on tasks and deliverables is tracked using public issues
(see~\cite{OpenDreamKit:on}). This has proven a strong feature to collaborate tightly with
the community and get early feedback.

In practice, \ODK's work plan consists of several wide breadth work packages: component architecture (modularity, packaging, distribution, deployment), user interfaces (\Jupyter interactive notebook interfaces, 3D visualization, documentation tools), high performance mathematical computing (especially on multicore/parallel architectures), a study of social aspects of collaborative software development, and a package on data/knowledge/software-bases.

The latter package focuses on the identification and extension of ontologies and standards to facilitate safe and efficient storage, reuse, interoperation and sharing of rich mathematical data, whilst taking provenance and citability into account. It will develop a component architecture for semantically sound data archival and sharing, and integrate computational software and databases. The aim is to  enable researchers to seamlessly manipulate mathematical objects across computational engines (e.g. switch algorithm implementations from one computer algebra system to another), front end interaction modes (database queries, notebooks, web, etc) and even backends (e.g. distributed vs.~local).

In this paper, we discuss the general approach chosen to develop this semantically aware component architecture. 

% The \ODK project is committed to working openly. Deliverables are tracked using public GitHub issues (see~\cite{OpenDreamKit:on}), which tightens the loop for early community feedback.

%%% Local Variables:
%%% mode: latex
%%% TeX-master: "paper"
%%% End:

%  LocalWords:  specialized Arxiv Jupyter IPython ldots compactitem emph compactenum odk
%  LocalWords:  ODKproposal organization standardization visualization citability oldpart
%  LocalWords:  organizing Swinnerton-Dyer resentation desingularisation Hironaka ednote
%  LocalWords:  Hironaka algorithmisation Villamayor
