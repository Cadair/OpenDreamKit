\section{LMFDB Knowledge and Interoperability}\label{sec:lmfdb}
%\subsection{Short description of LMFDB's goals and implementation}
The \emph{$L$-functions and modular forms database} is a project involving dozens of
mathematicians, who assemble computational data about $L$-functions, modular forms, and
related number theoretic objects. The main output of the project is a website, hosted at
\url{http://www.lmfdb.org}, that presents this data in a way that could serve as a
reference for research efforts and should be accessible at the graduate student level.
The mathematical concepts underlying the \LMFDB are extremely complex and varied, so part
of the effort has been focused on how to relay knowledge (mathematical definitions and their
relationships) to data and software. For this purpose, the \LMFDB has developed so-called
\emph{knowls}, which are a technical solution to present \LaTeX-encoded information
interactively, heavily exploiting the concept of transclusion. The end result is a
very modular and highly interlinked set of definitions in mathematical vernacular, which can be easily anchored in vastly different contexts, such as an interface to a database, to browsable data or as constituents of an encyclopedia \cite{lmfdb-definitions}.

The \LMFDB code is primarily written in \Python, with some reliance on \Sage for
the business logic. The backend
uses the NoSQL document database system \Mongo \cite{lmfdb-repo}. Again, due to the
complexity of the objects considered, many idiosyncratic encodings are used for the
data. This makes the whole data management lifecycle particularly tricky, and dependent on
different select groups of individuals for each component.
\ednote{R3: "The frontend is written in the web framework Flask" - relevant here?}

%\subsection{Tentative approach to MMT semantic layers for the LMFDB}
As the LMFDB spans the whole ``vertical'' workflow, from writing software, to producing new
data, up to presenting this new knowledge, it is a perfect test case for a large scale
case study of the MitM approach. Conversely, a semantic layer would be beneficial to its
activities across data, knowledge and software, which it would help integrate more
cohesively and systematically.

Among the components of the LMFDB, elliptic curves stand
out in the best shape, and a source of best practices for other areas. 
%
% For this reason,
% the \ODK collaboration targeted that relatively small subset of the LMFDB for its
% prototype. We plan to extend coverage in a second phase, and expect it to be relatively
% easy if we manage to demonstrate the added value of a semantic layer for our prototype.
%
We have generated MitM interface theories for LMFDB elliptic curves by (manually)
refactoring and flexiformalizing the {\LaTeX} source of knowls into \sTeX (see
Listing~\ref{stex-ec} for an excerpt), which can be converted into flexiformal OMDoc/MMT
automatically. The MMT system can already type-check the definitions, avoiding circularity
and ensuring some level of consistency in their scope and make it browsable through
\textsf{MathHub.info}, a project developed in parallel to MMT to host such formalisations.

\lstinputlisting[language={[sTeX]TeX},label={stex-ec},firstline=31,
   caption= {\protect\stex flexiformalization of an \LMFDB knowl (original: \cite{lmfdb-mwm})}]
   {examples/elliptic-curve.tex}

   The second step consisted of translating these informal definitions into progressively
   more exhaustive MMT formalisations of mathematical concepts (see
   Listing~\ref{lst:mmt-ec}). The two representations are coordinated via the theory and
   symbol names -- we can see the \sTeX representation as a human-oriented documentation
   of the MMT.

\lstinputlisting[morekeywords={namespace,theory,include},mathescape,
firstline=21,
caption= {MMT formalisation of elliptic curves and their Weierstrass models},
label=lst:mmt-ec]{examples/elliptic-curve.mmt}


Finally, we have to integrate computational data into the interface theories. Based on
recent ongoing efforts \cite{lmfdb-formats} to document the \LMFDB ``data schemata'' we
established OMDoc/MMT theories that linked the database fields to their data types (string
\emph{vs.} float \emph{vs.} integer tuple, for instance) and mathematical types (elliptic
curves or polynomials) -- the latter based on the vocabulary in the interface theories
generated from the \LMFDB knowls. This schema theory is complemented by a theory on
composable \emph{MMT codecs}, which in turn acts as a specification for a collection of
implementations in various programming languages (currently \Python, Scala, and
C\textsuperscript{++} for \Sage, MMT, and \GAP respectively) which are first instances of
a computational foundation (see Section~\ref{sec:mitm}).  For instance, one could compose
two MMT codecs, say \emph{polynomial-as-reversed-list} and
\emph{rational-as-tuple-of-int}, to signify that the data $[(2,3),(0,1),(4,1)]$ is meant
to represent the polynomial $4x^2+2/3$. Of course, these codecs could be further
decomposed (signalling which variable name to use, for instance). The initial cost of
developing these codecs is high, but the clarity gained in documentation is valuable, they
are highly reusable, and they drastically expand the range of tooling that can be built
around data management.
\ednote{R3: "composable MMT codecs" - do these satisfy any category-theoretic properties 
other than being composable?}

\paragraph{A typical application}
Based on these MitM interface theories we can generate I/O interfaces that translate
between the low-level \LMFDB API, which delivers raw \Mongo data in JSON format into MMT
expressions that are grounded in the interface theories. This ties the \LMFDB database
into the MitM architecture transparently. As a side effect, this opens up the \LMFDB to
programmatic queries via the MMT API, which can be queried and can then relay them to the
\LMFDB API directly and transparently.

%  LocalWords:  subsubsection emph knowls textsf lmfdb-repo stex-ec stex knowl mmt-ec odk
%  LocalWords:  Weierstrass emphasizing alltt tp rightarrow vdash doteq vdash doteq lst
%  LocalWords:  polynomial_equation_injectivity a_invariants_factors_injectivity colorbox
%  LocalWords:  minimality_idempotence lmfdb-formats emphasized composable standardize
%  LocalWords:  rda-typing lstinputlisting mathescape elliptic-curve.mmt lmfdb firstline

%%% Local Variables:
%%% mode: latex
%%% TeX-master: "paper"
%%% End:
%  LocalWords:  flexiformal flexiformalization flexiformalizing ednote mathhub mitm
