\ednote{This section does not work yet. I forgot to quiz Markus on how the binary operations of algebraic structures are treated. I remember GAP hard-codes two binary operations.}

\paragraph{A Logic for GAP Theories}
We can identify a logic and a group of theories that can be naturally embedded into GAP's type system.

Any GAP filter can be used as a type.

Every theory implicitly declares a fixed base type $u$ for the universe.

Then it may have two kinds of declarations:
 \begin{compactitem}
   \item includes of another theory,
   \item function symbols $f:a_1\times\ldots\times a_n\to a_0$ where each $a_i$ is a type (either $u$ or some GAP type),
   \item potential axioms: code in GAP's underlying programming language that evaluates to a boolean
 \end{compactitem}

\paragraph{Representing Theories in GAP}
A theory with name $T$, includes $T_1,\ldots,T_k$, function symbols $f_1,\ldots,f_l$, and potential axioms $p_1,\ldots,p_m$ is translated to GAP as follows:
\begin{compactitem}
 \item We declare a category with name $T$ and superfilter is $T_1\wedge \ldots \wedge T_n$.
 \item We declare an operation for each $f_i$.
 \item We declare a property for each $p_i$.
\end{compactitem}

Now the filter $T\wedge p_1$ represents the type of models of $T$ that satisfy the axiom $p_1$, and accordingly for every subset of the potential axioms.

\paragraph{Extracting Explicit Theories from GAP's}
Because GAP does not enforce an abstraction boundary between theories and types, it is not generally feasible to extract explicit theories from GAP.

A heuristic extraction might be possible by trying to identity groups of GAP declarations for which the above operation can be inverted to yield a theory.