\begin{workpackage}[id=community,wphases=5-36!.7,
<<<<<<< HEAD
title=Community Building and Engagement,
SARM=1,USHRM=8]
=======
  title=Community Building and Engagement,
  lead=PS,
  PSRM=12,SARM=1,USHRM=8]
>>>>>>> e490abbfa8a91427570f1a7695a6a95cd4610713

\begin{wpobjectives}
  The objective of this work package is to further develop the community at the
  European scale, foster cross teams collaborations, spread the
  expertise, and engage the greater community to participate to the
  definition of the needs, and the implementation and use of the
  produced solutions.
% \begin{itemize}
% \item
% \item
% \item
% \item
% \item
% \end{itemize}
\end{wpobjectives}

\begin{wpdescription}
  We will organize regular open workshops (e.g. Sage Days, Pari Days,
  summer schools, etc.); some of them will be focused on development
  and coding sprints, and others on training.

\TODO{Neil: I have a series of Gaussian process summer schools and road shows that I'rm organizing. These will also shift to more of a focus on data science across this year, I'd be happy to include these here if that's appropriate.}

  This work package will also provide general travel budget to fund
  short to long term visits between the participants, to collaborate
  on specific features. A typical such visit would bring together an
  IPython developer with a GAP developer for a couple of days to
  implement a first prototype of notebook interface to GAP.

  This work package will complement and lean on a parallel COST
  network whose role is to build and animate the greater community.


\end{wpdescription}

\begin{wpdelivs}
  \begin{wpdeliv}[due=6,id=ws1,dissem=PU,nature=O]{Workshop 1}
  \end{wpdeliv}
  \begin{wpdeliv}[due=12,id=needs,dissem=PU,nature=R]{Report on community needs}
  \end{wpdeliv}
  \begin{wpdeliv}[due=18,id=ws2,dissem=PU,nature=O]{Workshop 2}
  \end{wpdeliv}
  \begin{wpdeliv}[due=30,id=ws3,dissem=PU,nature=O]{Workshop 3}
  \end{wpdeliv}
  \begin{wpdeliv}[due=42,id=ws4,dissem=PU,nature=O]{Workshop 4}
  \end{wpdeliv}
\end{wpdelivs}
\end{workpackage}
%%% Local Variables:
%%% mode: latex
%%% TeX-master: "../proposal"
%%% End:
