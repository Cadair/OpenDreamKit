\addtocounter{wpno}{1}
\begin{Workpackage}{\thewpno}
\wplabel{wp:x}
\WPTitle{\wpname{\thewpno}}
\WPStart{Month 1}
\WPParticipant{SA}{1}

\begin{WPObjectives}
  The objective of this work package is to develop and demonstrate a
  set of API's enabling components such as database interfaces,
  computational modules, separate systems such as GAP or Sage to be
  flexibly combined and run smoothly across a wide range of
  environments (cloud, local, server, ...).
\end{WPObjectives}

\begin{WPDescription}
  This work package includes work on:
  \begin{itemize}
  \item Portability:
    \begin{itemize}
    % Jean-Pierre:
    % Should we mention port to non-x86_64 archs and non-Linuces?
    %
    % For CPUs:
    % - I guess at least ARM and ppc64 (IBM POWER*) really make sense.
    % - Sparc is less convincing though the latest sparc CPUs
    % are muche more interesting for math computation as the
    % previous ones, e.g. the GMP folk specifically added assembly
    % for them in their latest release.
    % - Itanium is dead, but it can help discovering bugs as any non
    % standard archs.
    % - Supporting any of these would mean buying (potentially very
    % expensive) hardware.
    %
    % For OSes?
    % - Should we mention OS X which is a pain at each new release?
    % - A BSD variant would be interesting, let's say FreeBSD which
    % is basically (almost) already supported
    % - Solaris? and/or OpenIndiana? Interesting if we mention sparc...
    % - Windows is already included below, my opinion is:
    %  * provide live USB, VMs and Cygwin32 first as these three are
    %  basically already working solutions
    %  * go Cygwin64 as it is still POSIX
    %  * explorate a MinGW solution, at least GAP and PARI should be
    %  problematic
    %  * try to use MSVC
    \item Sharing experience and best practices.
    \item Port to Windows (GAP, Sage, Singular).
    \item Shared multiplatform test infrastructure.
    \end{itemize}

  \item Interfaces between systems
    \begin{itemize}
    \item Self adaptation to the environment, better schemes for
      automatically selecting appropriate algorithms / components for
      a given task.
    \item Semantic-enabled handles (NT):\\
      Handles are a popular design pattern for interfaces between two
      systems A and B; instead of exchanging objects back and forth,
      only handles to those objects are exchanged, letting e.g. A
      manipulate an object which actually resides in B. Typical
      features include remote method calls, introspection, or
      documentation queries. The next step would be for A to be aware
      of the semantic of the object. For example, we would want GAP's
      categories to be mapped to Sage's categories, so that a handle
      to a GAP group would automatically appear within Sage like a
      native Sage group.
    \end{itemize}

  \item Modularization
    \begin{itemize}
    \item common architecture for module maintenance and
      distribution (related to point 1 above)
    \item Sharing experience and best practices
    \item Modularization of Sage
    \item Refactorization of GAP's package mechanism; namespaces?
    \end{itemize}

  \item Deployment and distribution
  \item High Performance Computing and Parallelism:\\
    As in all other areas of science, properly supporting of massively
    parallel architecture is a major challenge.

    Many of the computational components have already gone a long way
    in this direction. For example, grant \TODO{...} founded the
    GAP-HPC project which adapted the GAP kernel to support HPC. The
    expertise gained there was then transferred to the ongoing
    Singular-HPC project.

    Building on this, this project will:
    \begin{itemize}
    \item Foster further sharing of HPC expertise and best practices
      between computational components.
    \item Develop novel infrastructure for HPC in the context of
      combinatorics.
    \item Investigate and implement HPC-friendly ways of combining
      components together, so that calling components can benefit from
      the HPC-features of called components, with self-adaptation to
      the environment and cooperative sharing of resources.
    \end{itemize}
  \end{itemize}
\end{WPDescription}

\begin{WPDeliverables}
\begin{itemize}
\item \ref{del:virtual_machines} (Month 12): Creation, deployment, and
  distribution of preconfigured virtual machines for Pari, Sage as a
  cloud service, in particular within the StratusLab infrastructure.
\item \ref{del:sage_cygwin} (Month 12): Fully functional one-click
  install Sage distribution for Windows using a 32bits version of Cygwin.
  % Jean-Pierre: this should take a few months of work

  This 32bits version would work on Windows 64 bits, but more work
  would be required for a version working on a 64 bits of Cygwin.
  \TODO{Make this a second deliverable?}
  % JPF: I agree.

  % Comments on this by Bill Hart
  % The big problems you will have on Windows 64 on Cygwin include:
  %
  % * anything with assembly language -- the ABI is different on Windows, so
  % it'll need rewriting, or you can incur a performance penalty by using
  % generic C fallback code
  % * the memory allocator on Windows is not so great
  % * bugs exposed due to being on a different platform, e.g. segfaults due to
  % off-by-one errors that were masked by the granularity of malloc on Linux
  % * build issues, due to identifying Cygwin and using the correct header
  % files, which are often different on Cygwin than linux
  % * issues with PATH vs LD_LIBRARY_PATH
  % * Windows has a case insensitive file system
  % * EOL issues
  % * Windows is not able to rapidly create and delete files, which some
  % libraries (esp. test code) calls for
  % * memory limitations (many people using Windows are using laptops with
  % limited memory, only a portion of which is realistically available to
  % Cygwin)
  % * autotools versions that don't support Windows (usually autotools has a
  % release that is used in all the distributions, which doesn't work correctly
  % on Windows, and this is followed up by a version which has all the Windows
  % patches)
  % * building takes forever on Windows. Mingw2 has now gotten parallel build
  % working on Windows and the speed is within a factor of 5 of Linux. But I'm
  % not sure the improvements have propagated to Cygwin yet.
  % * Cygwin 64 is new, contains quite a few bugs still, and things keep
  % changing with every version as they try to get things right.
  % * Although projects will likely accept patches for Windows, they are less
  % likely to maintain support themselves. I would like to think Singular would
  % be an exception to this. And obviously flint and MPIR work on Windows (even
  % with MSVC as of the next version of flint -- or now if you use our bleeding
  % edge repo version).
  %
  % Comments by Jean-Pierre on some of the above and mor:
  % * first things first: I already completely built Sage on Cygwin64, though it
  % was surely not completely functional.
  % * assembly: that's right, note that as far as Sage and it's dependencies are
  % concerned, only a few of them actually use assembler, and yes all of them
  % provide fallback generic C code IIRC
  % * PATH vs LD_...: basically the same problem as for Cygwin32, so it's already
  % been taken care of for the Cygwin32 port
  % case issue: not a problem IIRC
  % * EOL issues: I don't thing so, Cygwin is POSIX like
  % * autotools issues: most of Sage dependencies are now updated, I used to track
  % the few problematic ones in 2013
  % * time to build: not so long, sure longer than on a POWER7 machine, but I do
  % it on a usual x86_64 laptop running Debian within a Windows VM in a few hours!
  % what we actually really need is patch/build bots to test on Cygwin 32/64!
  % * upstream cooperation: I agree Windows is often a low priority issue, but
  % most teams have welcomed my Cygwin patches
\item \ref{del:scscp_sage} Add support in Sage for the SCSCP interface
  protocol.
\item Some IPython/Jupyter deliverables here.
  \TODO{review what it can already do in term of choice of
    computational resource and storage back-end.}
\item \ref{del:hpc_configure} Configure the components of Sage's
  distribution (e.g. Atlas, Linbox, GAP, Singular, ...) to be
  systematically HPC-enabled, and make sure that, Sage's calls to such
  components indeed enables HPC.
\end{itemize}
\end{WPDeliverables}
\begin{verbatim}
Raw material:

Component Architecture
----------------------

Recomputation connection belongs here?

Collaboration with unreliable (or restricted!) networking connections
(peer-to-peer, opportunistic syncing, 3rd world). This is technically
interesting, and gets in support for non-networked working. Not sure
if it belongs here or not.

- Security concerns

Goal: Fostering collaborations/integration between components in an open source ecosystem
=============================================================================

- How to make systems "cooperate" rather than "predate each other".
- E.g. reduce the version issues
- Pushing Python bindings upstream

- How to make it easy to develop simultaneously two interdependent
  components (e.g. Sage+Singular)

- Foster communication

- Social aspect:
  Credit, Citations, Recognition
  Funding

Documentation system
====================

In which package?

Improvements to Sphinx

Sage heavily customizes the Sphinx documentation system, hacking deep
in it in some cases, with quite some duplication in some cases.
Refactor the whole thing, generalizing and contributing back upstream
as much as possible (e.g. parallel compilation).
\end{verbatim}

\end{Workpackage}
