\begin{workpackage}[id=dissem,wphases=18-48!.5,
  title=Dissemination,
  SARM=1,
  USORM=7]

\begin{wpobjectives}
  The objective of this work package is to organize and optimize the
  communication with the larger community. This includes:
  \begin{itemize}
  \item reviewing emerging technologies;
  \item disseminating research results to the scientific community;
  \item ensuring awareness of the results in the user community;
  \item raising general public awareness of the \TheProject project;
  \item defining individual exploitation plans; and,
  \item managing existing and new intellectual property.
  \end{itemize}
\end{wpobjectives}

\begin{wpdescription}
  Dissemination: software, APIs, technologies, research results, ...
\end{wpdescription}

\begin{tasklist}
\begin{task}[title=Reviewing emerging technologies]
  In this task, we will produce periodic reviews of emerging
  technologies and relevant developments elsewhere, and implications
  for our plans. This include the review of standard components and
  service for storage and sharing, computational resources,
  authentication, package management, etc. This may further include
  negotiating access or shared development when appropriate. This
  information will be fed to the other work packages, in particular
  Work Package~\WPref{component-architecture} Component Architecture.
\end{task}

\begin{task}[title=Dissemination and Communication activities]

  \TODO{scale this down as appropriate}

  This task comprises all forms of direct dissemination and public
  communication activities such as press releases, creation of the
  project web-site including visitor analysis and monitoring tools
  (\delivref{website}), scientific and technical publications, outreach
  activities (seminars, keynote talks, media interviews, press
  releases), pro- motion through social media (e.g. twitter, facebook,
  linkedin), technical workshop organisation, creation of
  advertisement materials such as flyers, posters, and electronic
  feeds as well as their distribution.

  % News articles will be produced by experienced professional staff
  % at relevant partners including ... and communicated to local,
  % national and international media, as appropriate.

  At least two press releases will be generated in the course of the
  project (\delivref{pressrelease}, \delivref{finalpressrelease}), and
  the project will organise at least one open technical workshop each
  year. % In fact many more?
\end{task}
\end{tasklist}

\begin{wpdelivs}
  \begin{wpdeliv}[due=3,id=pressrelease,dissem=??,nature=??]
      {Press release announcing start of \TheProject.}
\end{wpdeliv}%
  \begin{wpdeliv}[due=3,id=website,dissem=??,nature=??]
      {Project web site}
\end{wpdeliv}%
  \begin{wpdeliv}[due=12,id=periodic-rep-1,dissem=??,nature=??]
      {Year 1 report}
\end{wpdeliv}%
  \begin{wpdeliv}[due=24,id=periodic-rep-2,dissem=??,nature=??]
      {Year 2 report}
\end{wpdeliv}%
  \begin{wpdeliv}[due=36,id=periodic-rep-3,dissem=??,nature=??]
      {Year 3 report}
\end{wpdeliv}%
  \begin{wpdeliv}[due=48,id=periodic-rep-4,dissem=??,nature=??]
      {Year 4 report}
\end{wpdeliv}%
\end{wpdelivs}

\begin{tasklist}
\begin{task}[id=dissemination-of-oommf-nb-virtual-environment,
  title=Dissemination of OOMMF-NB virtual environment]
  % 4 months person time
  OOMMF-NB is build on top of \TheProject and demonstrates the power
  of the environment in a production environment of active OOMMF users. 
  The source code of OOMMF-NB will be made available as open source on
  public repository hosting sites (such as github/bitbucket), and
  announced to the community via the mumag mailing list. We will
  encourage participation of the micromagnetic community in the
  maintenance and development of the tools, and allow time to train
  users to join the OOMMF-NB project to drive towards a self-sustained
  OOMMF-NB project at the end of the funding period.

  To underpin this process, we will set up a publicly accessible Jenkins/Travis-CI system
  (see Task \localtaskref{oommf-nb-ve}) to (i) run regression tests routinely when the
  OOMMF-NB code changes or OOMMF core code changes and (ii) re-execute notebooks and use
  as regression test (making using of task \localtaskref{notebook-verification}). This will
  test user-contributions automatically, and can automatically create virtual environments
  for download containing the latest versions.

  We will create a manuscript for journal publication, summarising the
  approach taken and experience gathered so far. An important point of
  this publication is to provide a reference that can be cited by
  publications making use of the new OOMMF-NB, allowing tracking of
  uptake of this technology in the medium and long run.
\end{task}
\end{tasklist}

\begin{wpdelivs}
  \begin{wpdeliv}[due=15,id=OOMMF-NB-opensource,dissem=??,nature=??]
      {OOMMF-NB available online as open source}
\end{wpdeliv}
  \begin{wpdeliv}[due=18,id=OOMMF-NB-opensource-testing,dissem=??,nature=??]
      {Public continuous integration system set up}
\end{wpdeliv}
  \begin{wpdeliv}[due=36,id=OOMMF-NB-opensource-manuscript,dissem=??,nature=??]
      {Submitted OOMMF-NB manuscript}
\end{wpdeliv}

% \begin{wpdeliv}[id=workshop1,title=Report on first project workshop, year 1]
% \end{wpdeliv}
% \begin{wpdeliv}[id=dissemplan1,title=Final plan for using and disseminating knowledge]
% \end{wpdeliv}
% \begin{wpdeliv}[id=workshop2,title=Report on second project workshop, year 2]
% \end{wpdeliv}
% \begin{wpdeliv}[id=workshop3,title=Report on third project workshop, year 3]
% \end{wpdeliv}
% \begin{wpdeliv}[id=dissemplan2,title=Final plan for using and disseminating knowledge]
% \end{wpdeliv}
\end{wpdelivs}

\begin{tasklist}
\begin{task}[title=OOMMF-NB dissemination workshops,
id=dissemination-of-oommf-nb-workshops]
  % 3 months person time
  
  In this task, we run a series of workshops at major international
  meetings to disemminate the open source OOMMF-NB tool in the
  micromagnetic community.

  We will prepare the software stack for the workshops, develop teaching
  materials, and do the planning (location, advertising, registration)
  and delivery of workshops at the 4 major international meetings
  taking place in the appropriate time frame\footnote{Anticipated
    major meetings are 61st Conference
    on Magnetism and Magnetic Materials October 31-November 4, 2016,
    New Orleans, Louisiana; 62nd Conference on Magnetism and Magnetic
    Materials November 6-10, 2017, Pittsburgh, Pennsylvania; 21st
    International Conference on Magnetism (ICM 2018) July 16–20, 2018,
    San Francisco, California; 14th Joint MMM-Intermag Conference
    January 14-18, 2019, Washington, DC)} each of which tends to
  attract around 1000 attendees. Depending on demand, multiple
  workshops can be given per conference. We anticipate to travel with two
  people (Hans Fangohr and PDRA or other support staff) to be able to teach
  effectively and deliver hands-on training as part of the
  workshop. We request 500 EUR room hire per meeting.
\end{task}
\end{tasklist}

\begin{wpdelivs}
  \begin{wpdeliv}[due=48,id=OOMMF-NB-workshops,dissem=??,nature=??]
      {Delivered OOMMF-NB workshops at major international meetings}
\end{wpdeliv}
\end{wpdelivs}

Raw material:
\begin{itemize}
\item Documentation improvements: overview, cross links, overview of
  recent improvements
\item Thematic tutorials
\item Collections of pedagogical documents\\
  E.g. a complete collection of interactive class notes with computer
  lab projects for the ``Algèbre et Calcul formel'' option of the
  French math aggregation (starting from 2014-2015, only open-source
  systems will be supported, and Sage is a major player).
  % See http://nicolas.thiery.name/Enseignement/Agregation/ as a starter
  % Math labs with Sage for first year students in France (L1): http://math.univ-lyon1.fr/~omarguin/
\item Localization of the Sage user interface and key documents in
  various European languages.
\item Distribution of the documents either in the main distribution of
  Sage or through the online repository (see collaborative tools).
\item Massive online introduction course to Sage, drawing on the sage tutorial/notebooks.
Could be "First year Sage course in a box".
\item Taking the opportunity of Python courses to propose Sage as a natural extension
for mathematics; an example is French's 
% TODO: The url macro eats the accented letters. 
% It doesn't just eat it, it pukes it back!
``Classes pr\'eparatoires''\footnote{
\url{http://en.wikipedia.org/wiki/Classe_préparatoire_aux_grandes_écoles}}, 
where Python has been recently selected as the language to learn programming\footnote{See 
the ``Annexe'' at 
\url{http://www.education.gouv.fr/pid25535/bulletin_officiel.html?cid_bo=71586}}.
%\item \TODO{please expand!}
\end{itemize}

% Jeroen: About teaching: in Gent, Sage is already integrated in the
% courses (maybe you can add this, don't know if it's relevant)
% starting in the first year. It's good for the students because it
% helps in 2 ways: it helps them to understand the mathematics better
% and it helps them to learn basic down-to-earth programming (they
% also have a programming course in Java but that contains a lot of
% theory about complicated class structures)
% Same thing in Orsay
% More python centered but same in UZH

\end{workpackage}

%%% Local Variables: 
%%% mode: latex
%%% TeX-master: "../proposal"
%%% End: 
