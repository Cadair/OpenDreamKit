\begin{workpackage}[id=dissem,wphases=18-48!.5,
  short={Community Building/Dissemination},
  title={Community Building, Training, Dissemination, Exploitation, and Outreach},
  lead=PS,
  PSRM=10, % TODO!
  SARM=18,
  USORM=10,
  USHRM=8,
  USRM=24,
  UVRM=2,
  UBRM=4,
  SRRM=1
]


\begin{wpobjectives}
  The objective of this work package is to further develop the community at the
  European scale, foster cross teams collaborations, spread the
  expertise, and engage the greater community to participate to the
  definition of the needs, and the implementation and use of the
  produced solutions. This includes:

  \begin{compactitem}
  \item reviewing emerging technologies;
  \item ensuring awareness of the results in the user community;
  \item engaging cross communities discussions to foster scientific collaborations and conjoint developments;
  \item spreading the expertise through workshops and trainings;
  \item defining individual exploitation plans; and,
  \item managing existing and new intellectual property.
  \end{compactitem}
\end{wpobjectives}

\begin{wpdescription} 
  We will organize regular open workshops (e.g. Sage Days, Pari Days,
  summer schools, etc.); some of them will be focused on development
  and coding sprints, and others on training. This is also an occasion
  to organize cross communities workshops like Sage-IPython days.


  This work package will also provide general travel budget to fund
  short to long term visits between the participants, to collaborate
  on specific features. A typical such visit would bring together an
  IPython developer with a GAP developer for a couple of days to
  implement a first prototype of notebook interface to GAP.

  This work package will complement and lean on a parallel COST
  network whose role is to build and animate the greater community.
\end{wpdescription}

\begin{tasklist}
\begin{task}[title=Reviewing emerging technologies]
  In this task, we will produce periodic reviews of emerging technologies and relevant
  developments elsewhere, and implications for our plans. This include the review of
  standard components and service for storage and sharing, computational resources,
  authentication, package management, etc. This may further include negotiating access or
  shared development when appropriate. This information will be fed to the other work
  packages, in particular Work Package~\WPref{component-architecture} Component
  Architecture.
\end{task}

\begin{task}[title=Dissemination and Communication activities]

  \TODO{scale this down as appropriate}

  This task comprises all forms of direct dissemination and public communication
  activities such as press releases, creation of the project web-site including visitor
  analysis and monitoring tools , scientific and technical publications, outreach
  activities (seminars, keynote talks, media interviews, press releases), pro- motion
  through social media (e.g. twitter, Facebook, linkedin), creation of advertisement
  materials such as flyers, posters, and electronic feeds as well as their distribution.

  % News articles will be produced by experienced professional staff
  % at relevant partners including ... and communicated to local,
  % national and international media, as appropriate.

  At least two press releases will be generated in the course of the project.
  
  \TODO{neil: this coming year i will do four workshops, although the number may be
    smaller by the time we are up and running. an additional component that could be
    integrated here is a 'JMLR Monthly Notices' section that I recently gained approval
    for. This is a new section the leading machine learning journal that we'll aim to
    focus on typically smaller advances (less than a full paper). A little like a 'Nature
    Notices' section but much more informal and open. We don't yet know the best way of
    doing it, but the mechanism might be similar to ways we'd want to disseminate
    \TheProject results.}
\end{task}

\begin{task}[title=Organization of development workshops]

We will organize development workshops all throughout the project. The aim of these workshops
is to bring together developers from the different communities to implement some key aspects of \TheProject: user interface, documentation, cross compatibility, etc. These meetings will gather not only participants of \TheProject but also members of the different communities involved. Thus, they foster dissemination and  collaborations.

At least one such workshop per year will be organized in Cernay, France where similar gatherings took place before as Sage-Days. One of them will be a Sage-Sphinx day, specifically dedicated to documentation in spring 2016. Bordeaux will host one Atelier \Pari per year which usually gathers around 30 participants. Some cross-thematic workshops will happen in Grenoble and Bordeaux to work specifically on HPC (sell \WPref{hpc}), especially \taskref{hpc}{hpc-pari}. There will also be at least one specific workshop for each component of the project: \Singular, \SMC, \IPython, \Jupyter.

\TODO{Add workshop report to the deliverables}

\end{task}

%Mike Croucher and Neil Lawrence,Sheffield
\begin{task}[title=Introduce \TheProject to researchers and teachers]

\TODO{Demonstrator?}

In this task, we will develop and deliver materials that will introduce \TheProject to potential users - both researchers and teachers.

Develop a 'taster' seminar (1-2 hours) and follow-up short course (1-2 days) on \TheProject for researchers and lecturers in all disciplines. At Sheffield, this could be added to the set of courses that are offered as part of IT Services' research support department. As such, it could potentially reach all disciplines.

Elements of this work will also be integrated with the GP Summer Schools and Roadshows. The Summer School is now in its fourth edition (over 140 students educated). The Roadshows have taken place in Uganda, Colombia and are scheduled for Italy, Australia and Kenya.

These seminars and short-courses will be used to identify potential collaborators who are interested in utilising \TheProject immediately. We will act as consultants to these collaborators in two ways:

We will work with lecturers at Sheffield to introduce \TheProject to various disciplines via the production of interactive lecture notes. The focus for the student here will not necessarily be on programming but rather on interaction with the subject matter via use of \TheProject. Interactive lecture notes are an area where commercial vendors such as Maplesoft and Wolfram Research are spending a lot of time and money developing material. We will provide technical and programming expertise to lecturers - helping them to develop the interactive part of notes while they provide the subject material.

We will work as consultants with researchers at Sheffield to introduce \TheProject to their workflow. Any projects that successfully do this will be promoted as case studies for \TheProject.
\end{task}

\begin{task}[id=dissemination-of-oommf-nb-virtual-environment,
  title=Demonstrator: Open source dissemination of \OOMMFNB{} virtual environment,
  lead=USO,PM=5]
  % 4 months person time + 1 months investigator time
  \OOMMFNB{} (see \taskref{UI}{oommf-python-interface} to
  \taskref{UI}{oommf-tutorial-and-documentation}) is build on top of \TheProject and demonstrates the power
  of the environment in a production environment of active OOMMF users. 
  The source code of \OOMMFNB{} will be made available as open source on
  public repository hosting sites (such as github/bitbucket), and
  announced to the community via appropriate mailing list (mumag,
  magpar, nmag, mumax, micromagnum) and other means. We will
  encourage participation of the micromagnetic community in the
  maintenance and development of the tools, and allow time to train
  users to join the \OOMMFNB{} project to drive towards a self-sustained
  \OOMMFNB{} project at the end of the funding period.

  To underpin this process, we will set up a publicly accessible
  Jenkins/Travis continuous integration (CI) system to (i) run
  regression tests (from \taskref{UI}{oommf-python-interface} and \taskref{UI}{oommf-py-ipython-attributes}) routinely when the
  \OOMMFNB{} code changes or OOMMF core code changes and (ii)
  re-execute notebooks (from
  \taskref{UI}{oommf-tutorial-and-documentation}) and use
  as regression test (making using of the outcome of task \taskref{UI}{notebook-verification}). This will
  test user-contributions automatically, and can automatically create virtual environments
  for download that containing the latest versions (\delivref{dissem}{oommfnb-source-and-testing-setup}).

  We will create a manuscript for journal publication (\delivref{dissem}{oommfnb-publication}), summarising the
  approach taken and experience gathered so far. An important point of
  this publication is to provide a reference that can be cited by
  publications making use of the new \OOMMFNB, allowing tracking of
  uptake of this technology in the medium and long run.
\end{task}

\begin{task}[title=\OOMMFNB{} open source dissemination workshops,
id=dissemination-of-oommf-nb-workshops,lead=USO,PM=6]

  % 3 months person time, 2 months investigator time

% For reference: this line gives a local reference (such as T4,
% i.e. Task 4 in current work package):
% \localtaskref{dissemination-of-oommf-nb-workshops}\\
% This lines gives the full reference, in the style of WP8.T4 (where
% WP8 in the work package with id=dissem.
% \taskref{dissem}{dissemination-of-oommf-nb-workshops}\\

In this task, we run a series of workshops
(\delivref{dissem}{oommfnb-workshops}) during the evenings of major
international meetings to disseminate the open source \OOMMFNB{} tool
(see
%\taskref{UI}{oommf-python-interface} to
%\taskref{UI}{oommf-tutorial-and-documentation} and 
\localtaskref{dissemination-of-oommf-nb-virtual-environment}) in the
  micromagnetic community. 

  We will prepare the software stack for the workshops, develop
  teaching materials, and do the planning (location, advertising,
  registration) and delivery of workshops at the 4 most significant
  international meetings taking place in the appropriate time
  frame\footnote{Anticipated major meetings are 61st Conference on
    Magnetism and Magnetic Materials October 31-November 4, 2016, New
    Orleans, Louisiana; 62nd Conference on Magnetism and Magnetic
    Materials November 6-10, 2017, Pittsburgh, Pennsylvania; 21st
    International Conference on Magnetism (ICM 2018) July 16–20, 2018,
    San Francisco, California; 14th Joint MMM-Intermag Conference
    January 14-18, 2019, Washington, DC). Each of those meetings is
    one week long, and serves as a focal point of networking for the
    european and international research community.} each of which
  tends to attract around 1500 attendees. Depending on demand,
  multiple workshops will be given per conference, i.e. one workshop
  per evening. We expect at least 30 participants per workshop. We
  anticipate to travel with two people (Hans Fangohr and PDRA or Ian
  Hawke) to be able to teach effectively and deliver hands-on training
  as part of the workshop. The taught material will include (i) use of
  the \Jupyter-based \OOMMFNB{} workflow model to support effective and reproducible
  computational science.  To support to make the \OOMMFNB{} project
  self-sustaining as quickly as possible,
  we will also teach an (ii) introduction to the standard techniques
  for contribution to open source software (version control, pull
  requests, testing frameworks).  In addition, all teaching materials,
  including videos, will be made available on a website.

  For each workshop, we request 500 EUR room hire to support delivery of
  these workshops directly at the international meetings, and the
  travel expenses for two teachers to attend the one week
  international conference, toalling (\euro{500} + 2x\euro{2200}=\euro{4900}) per
  workshop. We do not need to organise accommodation or catering as
  these workshops attach to the international conferences where the
  participants have already organised this.
\end{task}

% Vincent Delecroix and co. (budget will be based in Bordeaux)
\begin{task}[title=workshops in developing countries]
In this task, we run a series of workshops in developing countries especially
Africa and South America. Some of these workshops will be joined to CIMPA schools.
The CIMPA is an international organization based in Nice (France) that promote
research in mathematics in developing countries. It organizes each year around
20 schools.

We request 16 K EUR to cover expenses for four workshops (each workshop involving
two persons of \TheProject).

\end{task}

\begin{task}[title=Demonstrator: interactive books,
id=ibook]
  % 2x12 _ 3x 3 months for students

One of important elements of VRE is a common flexible format which
enables the creation of large structured documents. There are many
known solutions to that problem, but they usually compromise the
interactivity of the notebook interface and quality of desktop
publishing software like LaTeX. 

Recently, few approaches tried to bring both interactivity and the
typographic features. Modestly tagged markup language implementation
DocOnce targets the problem of reusability of the document source
code. The MathBook XML is a lightweight XML application for authors of
scientific articles, textbooks and monographs extensively using Sage
single cells for interactive elements. Sphinx documentation software
has been successfully applied for creation of interactive books
containing Sage cells using sagecell plugin. 

The technical aspects of format for interactive publications is a
subject of the task ``Structured documents'' in
\taskref{UI}{structdocs}. In this task we will demonstrate usability
of the results of \taskref{UI}{structdocs} in creation of scientific
textbooks. Three interactive books will be created:

\begin{compactitem}
\item Nonlinear Processes in Biology 
\item Classical Mechanics  
\item Problems  in Physics with Sage/python    
\end{compactitem}

The choice of those particular topics has been made for the sake of maximal diversity. The ``Nonlinear Processes in Biology'' will heavily use numerical ODE and PDE integration with addition to the classical approach. Classical Mechanics will demonstrate power of CAS systems working ``on par'' with numerical tools. The last example will focus mostly on collaborative editing and modularity of content which is produced using VRE technologies. 

The main research aspect for this task will be to find a way for efficient application of computational tools in problems solving and analysis. In addition the work-flow will be optimized in order to explore the full potential of VRE environment. 

In particular we will answer following questions:

\begin{compactitem}
\item How to create a monograph at reusing independently working   building block of text and code?
\item When the interactive worksheet should be used and when  executable code cell inside interactive textbook is sufficient?
\item What are best tools and practices for using single source for with many output targets? 
\item How to collaboratively write reusable course materials?
\item How students can benefit from  using VRE as working tool?
\end{compactitem}


\end{task}

\begin{task}[title=Demonstrator: Computational mathematics resources indexing service,
id=index-librorum-salvificorum]
Beyond official documentation and tutorials, users of mathematical
software and VREs learn from a wide array of sources: university
courses, Q\& A sites, web searches, etc.  A simple web search on any
major software component yields dozens of non-official tutorials and
how-tos in many different languages. However, search engines mostly
miss the relevant metadata: how does one find a tutorial on linear
algebra in \PariGP, written at an undegraduate level, in French or
Spanish?

This need has been felt by most communities at some point, and each
has come up with its own solution: most software components (e.g.,
\GAP, \PariGP, \Sage, \dots) simply link material from their official
page; \Sage has a wiki (\url{http://wiki.sagemath.org/}) referencing
additional resources, and used to host a large number of tutorial
worksheets on \url{http://sagenb.org/}; the recent introduction of
public projects in \SMC is sparking approximately the same phenomenon
that had previously happened with \url{http://sagenb.org/}; Ipython
host the Notebook Viewer service (\url{http://nbviewer.ipython.org/}),
which renders (without hosting) community-made notebooks; teaching
institutions host or link their own collections of pedagogical
resources. The list goes on.

These collections are usually incomplete, limited in scope, hard to
search, outdated, etc.  What the community needs is a
community-curated, searchable, metadata-driven, multilingual, platform
agnostic indexing service whose goal is to reference and rank all the
community generated knowledge around a software component or VRE.  

The goal of this task is to create the tool (\delivref{dissem}{ils-tool}) powering such
service, and to host a (free) community-curated index for \TheProject related resources as
a demonstrator (\delivref{dissem}{ils-service}).

\end{task}
\end{tasklist}



Raw material:
\begin{compactitem}
\item Documentation improvements: overview, cross links, overview of
  recent improvements
\item Thematic tutorials
\item Collections of pedagogical documents\\
  E.g. a complete collection of interactive class notes with computer
  lab projects for the ``Algèbre et Calcul formel'' option of the
  French math aggregation (starting from 2014-2015, only open-source
  systems will be supported, and Sage is a major player).
  % See http://nicolas.thiery.name/Enseignement/Agregation/ as a starter
  % Math labs with Sage for first year students in France (L1): http://math.univ-lyon1.fr/~omarguin/
\item Localization of the Sage user interface and key documents in
  various European languages.
\item Distribution of the documents either in the main distribution of
  Sage or through the online repository (see collaborative tools).
\item Massive online introduction course to Sage, drawing on the sage tutorial/notebooks.
Could be "First year Sage course in a box".
\item Taking the opportunity of Python courses to propose Sage as a natural extension
for mathematics; an example is French's 
% TODO: The url macro eats the accented letters. 
% It doesn't just eat it, it pukes it back!
``Classes pr\'eparatoires''\footnote{
\url{http://en.wikipedia.org/wiki/Classe_préparatoire_aux_grandes_écoles}}, 
where Python has been recently selected as the language to learn programming\footnote{See 
the ``Annexe'' at 
\url{http://www.education.gouv.fr/pid25535/bulletin_officiel.html?cid_bo=71586}}.
%\item \TODO{please expand!}
\end{compactitem}

% Jeroen: About teaching: in Gent, Sage is already integrated in the
% courses (maybe you can add this, don't know if it's relevant)
% starting in the first year. It's good for the students because it
% helps in 2 ways: it helps them to understand the mathematics better
% and it helps them to learn basic down-to-earth programming (they
% also have a programming course in Java but that contains a lot of
% theory about complicated class structures)
% Same thing in Orsay
% More python centered but same in UZH
% We have also Sage @ Silesia from 1st semester (physics)

\begin{wpdelivs}
 \begin{wpdeliv}[due=36,id=ibook2,dissem=PU,nature=DEM]{Demonstrator: Nonlinear Processes in Biology  interactive book} \end{wpdeliv}

 \begin{wpdeliv}[due=40,id=ibook2,dissem=PU,nature=DEM]{Demo: Classical Mechanics interactive book} \end{wpdeliv}

 \begin{wpdeliv}[due=12,id=ibook3a,dissem=PU,nature=DEM]{Demo: Problems in Physics with Sage v1} \end{wpdeliv}
 \begin{wpdeliv}[due=30,id=ibook3b,dissem=PU,nature=DEM]{Demo: Problems in Physics with Sage v2} \end{wpdeliv}
 \begin{wpdeliv}[due=32,id=oommfnb-source-and-testing-setup,dissem=PU,nature=DEC]{\OOMMFNB{} public
     repository and
     continuous integration} \end{wpdeliv}
 \begin{wpdeliv}[due=36,id=oommfnb-publication,dissem=PU,nature=R,lead=USO]{\OOMMFNB{}
     publication manuscript completed} \end{wpdeliv}
 \begin{wpdeliv}[due=44,id=ibook3c,dissem=PU,nature=DEM]{Demo: Problems in Physics with Sage v3} \end{wpdeliv}
 \begin{wpdeliv}[due=42,id=oommfnb-workshops,dissem=PU,nature=OTHER,lead=USO]{\OOMMFNB{}
     workshops delivered} \end{wpdeliv}
 \begin{wpdeliv}[due=24,id=ils-tool,dissem=PU,nature=P]{Community-curated
     indexing tool (open source)} \end{wpdeliv}
 \begin{wpdeliv}[due=24,id=ils-service,dissem=PU,nature=DEM]{Community-curated
     indexing service for \TheProject} \end{wpdeliv}
\end{wpdelivs}


\end{workpackage}

%%% Local Variables: 
%%% mode: latex
%%% TeX-master: "../proposal"
%%% End: 

%  LocalWords:  workpackage dissem wphases wpobjectives wpdescription tasklist WPref nmag
%  LocalWords:  delivref linkedin organisation finalpressrelease organise wpdelivs github
%  LocalWords:  wpdeliv dissemination-of-oommf-nb-virtual-environment OOMMFNB taskref
%  LocalWords:  oommf-python-interface oommf-tutorial-and-documentation mumag magpar
%  LocalWords:  mumax micromagnum micromagnetic oommf-py-ipython-attributes summarising
%  LocalWords:  dissemination-of-oommf-nb-workshops localtaskref MMM-Intermag Fangohr
%  LocalWords:  maximise sagecell structdocs Algèbre Calcul formel eparatoires Annexe
%  LocalWords:  Jeroen
