\addtocounter{wpno}{1}
\begin{Workpackage}{\thewpno}
\label{wp:hpc}
\wplabel{wp:x}
\WPTitle{\wpname{\thewpno}}
\WPStart{Month 1}
\WPParticipant{PS}{1} % HPC for Combinatorics
\TOWRITE{SG/PB}{Evaluation of man.months for the Pythran tasks}
\WPParticipant{LL}{1} % Pythran
\WPParticipant{SA}{1} % GAP
\WPParticipant{UK}{1} % Singular
\WPParticipant{UB}{1} % Pari
\WPParticipant{UG}{1} % Linbox / Pythran

\begin{WPObjectives}
  The objective of this work package is to improve the performance of
  the computational components of \TheProject, in particular on
  massively parallel architectures. This includes notably:
  \begin{itemize}
  \item Fine grained High Performance Computing on many-cores
    architectures.
  \item Coarse grained or embarrassingly parallel computing on grids
    or on the cloud.
  \item Compilation of high level interpreted code to optimized
    parallel native code.
  \item Develop novel HPC infrastructure in the context of
    combinatorics.
  \end{itemize}
  A key aspect will be to foster further sharing expertise and best
  practices between computational components.
\end{WPObjectives}

\begin{WPDescription}
  As in all other areas of science, properly supporting massively
  parallel architecture is a major challenge. Many of the
  computational components in \TheProject have already gone a long way
  in this direction. For example, an adaptation of the \GAP kernel for
  HPC was developed during the 2009-2013 EPSRC project. The expertise
  gained there was then transferred to the ongoing \Singular-HPC
  project, in particular through the rehiring of one of the developers
  of \GAP-HPC.

  In this work package, we will build on this momentum to further
  implement HPC support in the components Tasks~\ref{task:hpc_pari},
  \ref{task:hpc_linbox}, and \ref{task:hpc_singular}.

  \TODO{transition}

  Many of the computational components of \TheProject use a high level
  interpreted language for their library. This is notably the case of
  \Sage. Performance is achieved by compiling critical sections using
  the \Cython \Python-to-C compiler, to the expense of a lower level
  implementation. In Tasks~\ref{task:pythran_cython}
  and~\ref{task:pythran_sage}, we will also boost performance by
  further developing and applying such compilation tools, while
  keeping a high-level approach.

\end{WPDescription}

\begin{task}{Pari}
  \label{task:hpc_pari}
  \TOWRITE{KB}{Task around HPC/parallelism in Pari?}

  \TODO{deliverable}
\end{task}

\begin{task}{Linbox}
  \label{task:hpc_linbox}
  \TOWRITE{JGD/CP}{Task around HPC/parallelism in Linbox}

  \TODO{deliverable}
\end{task}

\begin{task}{Singular}
  \label{task:hpc_singular}
  \TOWRITE{WD}{Task around HPC/parallelism in Singular}

  \TODO{deliverable}
\end{task}


\begin{task}{HPC infrastructure for combinatorics}
  \TOWRITE{FH}{Task around HPC infrastructure for combinatorics}
  \TODO{deliverable}
\end{task}

\TOWRITE{PB/SG}{Add references to the deliverables from the tasks; see
  e.g. UserInterfaces.tex}

\begin{task}{Pythran-Cython convergence}
  \label{task:pythran_cython}

  \Pythran is a \Python to C++ compiler for a subset of the \Python
  language. It is meant to efficiently compile scientific programs,
  and takes advantage of multi-cores and SIMD instruction units.
  Thanks to type inference, it requires little annotations. Its rutime
  supports a subset of the \Numpy package.

  \Cython is a \Python to C compiler that was originally developed for
  \Sage and is now a thriving project of its own. It can handle
  essentially any \Python code, and in particular classes, but relies
  heavily on annotations for producing optimized code.

  Therefore, \Pythran and \Cython are similar in spirit but have
  complementary feature sets: \Pythran can heavily optimize high level
  \Numpy constructs and \Cython has broader \Python support. In this
  task, we will investigate the opportunity and feasibility of a
  convergence between \Cython and \Pythran: depending on the code at
  hand, one strategy or the other would be automatically selected,
  eventually using \Pythran generated called from \Cython when
  relevant. This would result in compiler-runtime cooperation driven
  by the \Cython compiler thanks to part of the \Pythran-runtime and
  the extra typing information provided by \Cython. An effort will be
  made to improve more and more the parallelism in the
  \Pythran runtime.

  This work will be achieved through a close collaboration between the
  \Pythran developers hired for \TheProject and \Cython developers
  involved in the \Sage project. It should quicken \Sage execution
  time at least on numpy centric codes, while not putting an extra
  burden on the developpers.

\end{task}

\begin{task}{\Pythran for \Sage and \Sage Users}
  \label{task:pythran_sage}

  Currently, \Sage doesn't provide facilities to improve user written
  \Python code improvement without the modifications implied by the use
  of the \Cython compiler. As \Pythran doesn't need these these
  rewrite, a notebook interface to compile \Pythran compliant code will
  he added in \Sage to improve user kernels.

  In a similar perspective, testing and improving the integeration between
  \software{mpi4py} and \Pythran could provide an efficient toolchain for HPC
  while keeping full backward compatibility with pure \Python code.

  Internally, \Sage uses \Cython for compiling the critical sections of
  its libraries. In this task, we will explore opportunities to
  benefit from \Pythran compilation within the \Sage library, in
  particular toward better support for parallelism. A specific
  challenge is that the \Sage library uses quite heavily
  object-oriented programming.

  This task will strongly benefit from Task~\ref{task:pythran_cython},
  while providing in return a real life large-scale use case for it.

  A first step to support object-oriented programming will be to make
  \Pythran type inference more accurate, which will also improve error
  feedback provided for the user.
\end{task}

\begin{task}{Explorative task: Add support for classes in \Pythran.}
  \TOWRITE{PB/SG}{Make this into a task?}
\end{task}

\begin{WPDeliverables}
  \WPDeliverable{XX}{del:pythran_cython}{Implement \Pythran runtime support in \Cython toward a unified interface.}
  \WPDeliverable{XX}{del:pythran_runtime}{Improve \Pythran runtime support for parallelism.}
  \WPDeliverable{XX}{del:pythran_sage}{Facility to compile \Pythran compliant user kernels.}
  \TOWRITE{PB/SG}{Make the deliverables more explicit?}
  \WPDeliverable{XX}{del:pythran}{Test and improve \Pythran.}
  \WPDeliverable{XX}{del:pythran_typing}{Make \Pythran typing better to improve error information.}
\end{WPDeliverables}
\end{Workpackage}
