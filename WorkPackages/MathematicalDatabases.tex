\addtocounter{wpno}{1}
\begin{Workpackage}{\thewpno}
\wplabel{wp:x}
\WPTitle{\wpname{\thewpno}}
\WPStart{Month 1}
\WPParticipant{ZH}{1}

\begin{WPObjectives}
  The objectives of \theWP{} is to design interfaces that can be used
  for a wide range of mathematical data, facilities for accepting
  contributions while tracking provenance and credit, and definition of standard
  metadata allowing for interoperability, database discovery, 
  versioning allowing stable references, and easy recomputation.
  
\end{WPObjectives}


\begin{WPDescription}
Mathematics is the only science that has not yet benefited massively from the systematic interchange of data. At the same time, mathematics has a richer notion of data than other disciplines.
Indeed, "mathematical data" consists of tree kinds of objects:
\begin{itemize}
\item[] [D]: proper (numeric/symbolic) data
\item[] [K]:  the knowledge about the mathematical objects given as statements (definitions, theorems or proofs; either formal or rigorously informal)
\item[] [S] : software that computes (with) the mathematical objects
\end{itemize}

All three kinds of "data" are equally important for mathematics and are tightly interlinked:
\begin{itemize}
\item[] [D] serves as examples for [K] or as counterexamples for conjectures in [K];
\item[] [S] computes [D] and establishes properties of [D] (given as [K]);
\item[] [D] tests [S], [S] is verified wrt. [K];
\item[] theorems and proofs in [K] induce and justify algorithms for [S];
\item[] [D] induces conjectures and guides proofs in [K].
\end{itemize}

Many mathematical databases now exist, but their internal structure does not surface this richness. This prevents the formulation of new conjectures, the testing of new hypotheses, and generally an exploratory approach to mathematical data. The past has shown that this approach could be fruitful: 
\begin{itemize}
\item both the Riemann Hypothesis and the Birch and Swinnerton Dyer conjectures resulted from exploratory $L$-function computations, and now stand among the seven Clay Millenium Problems;
\item the Monstrous Moonshine conjecture finds its origin in a numerical co\"incidence between dimensions of representations of the Monster group and coefficients of the $j$-function, and its conclusion eventually led to Borcherds' Fields medal.
\end{itemize}

%This explains the need for infrastructure that could be used uniformly over wide ranges of mathematics, certainly poses additional challenges compared to existing solutions, but also allows for potentially groundbreaking improvements. 

Therefore we need ways to represent DKS in the same systems, and -- since DKS get huge  in current computational/experimental mathematics -- we need a new kind of "databases", which we will call MDKS-bases (Mathematical Data/Knowledge/Software-bases).

This complexity is on vivid display in the \emph{L-functions and Modular Forms database} project (\LMFDB): while the general shape of the functional equation of an $L$-function is dependent on a lot of theoretical knowledge, it also requires parameter data and the coefficients of the associated Dirichlet series. Once this is obtained, highly optimised (and heavily parallelizable) algorithms can be run to compute values of this function. 

We propose in this work package to design and build an infrastructure that would make it easy for either isolated mathematicians or a distributed collaboration to manage mathematical data. This work would provide part of the backend to Work Packages \TODO{work package on interfaces, and???}, and would draw on previous work with the \LMFDB and FindStat (which will be treated as prototypes for our purposes). Requirements should be as minimal as possible (depending on the individual's goals), and in particular would not require any background in databases to contribute new data or perform queries. 

\end{WPDescription}

\begin{task}{Survey of existing databases}
\label{task:data_assessment}
All the systems considered in this proposal (\GAP, \Sage, \Pari, \Singular) include data as part of their regular distribution. In this task, we will survey existing databases, the technology used to implement them, how they were linked to the rest of the existing infrastructure and the functionalities offered. We will also select additional external data and projects to add to this effort, aiming to maximise the impact of our work. 
\end{task}

\begin{task}{Design of new infrastructure, formulation of requirements}
\label{task:data_design}
Ontologies are the canonical method used to implement databases that require significant data interchange. However, because of extreme reification in mathematics, this is not entirely suitable for our goals. We will design a new infrastructure for \TheProject, drawing on existing emerging standards . 
\end{task}

\begin{task}{Triform Theories in OMDoc/MMT}
\label{task:data_triform}
OMDoc/MMT is a representation language for mathematical knowledge and documents. Carette and Farmer have developed the notion of biform theories (K/S) in a uniform representational approach; our work here would extend this along the data axis, which will require a specialised but integrated treatment.
\end{task}

\begin{task}{Computational Foundation for Python/Sage (or some CAS)}
\label{task:data_foundationCAS}
In the OMDoc/MMT world a foundation is a logical base language that gives the formal meaning to all objects represented/formalized in it. We have created a very initial computational foundation for Scala and implemented it in the MMT API. This can be used to execute (or verify) computations directly in OMDoc/MMT and thus forms the basis for various integration tasks for OMDoc/MMT biform theories that integrate Scala computations. Here we propose to develop a somewhat more complete computational foundation for Python and/or parts of Sage (coverage to be determined). Bi/Triform theories come in three parts:
\begin{itemize}
\item syntax: what operators/types are there, how do they nest, 
\item computation:  what does the computation relation look like (sometimes called operational semantics). The declarative semantics of a computational foundation can be given as an OMDoc/MMT theory morphism into another foundation (e.g. a set theory);
\item ??? three parts
\end{task}


\begin{task}{LMFDB Case study (Triformal Theories)}
\label{task:data_LMFDB}
In this task we would develop triformal theories for an exemplary part of the LFMDB to test the design from K6.5.  We wil identify a fragment of the LFMDB that we want to model and design the model. Then we will perform cross-validation of the three model parts against each other (essentially model-based testing of software and inference). Finally, we will pick an algorithm from the LFMDB and verify it against its specification and the computational foundation developed in K6.4. (decrease importance of verification as opposed to interoperability)

This Task will be co-developed with K6.4, it will validate the design of triformal theories and be iterated to test the design changes.
\end{task}


\begin{task}{OEIS Case Study (Coverage and automated Import)}
\label{task:data_OEIS}
In this case study we test the practical coverage of the trifunctional modules, by transforming an existing, high-profile database (the Online Sequence of Integer Sequences http://www.oeis.org) into OMDoc/MMT. The OEIS has about 250 thousand sequences, with formulae, descriptions, definitions, references, software, etc. in a structured text file (but no standardized format for formulae and references), so we expect to get 250 k theories. Having the OEIS in OMDoc/MMT form allows to do Knowledge Management services (presentation, definition lookup, formula search, ...) in MathHub.info (see WP4.?). The OEIS is a good case study, since the DKM  are licensed under a CC license which allows derived works. The large size will allow statistically significant semantic cross-validation of the heuristic transformation process and thus achieve a significant DKS community resource.
\end{task}

\begin{task}{Findstat Case Study ???}
\label{task:data_findstat}

\end{task}

\begin{task}{Memoization}
\label{task:data_memo}

\end{task}

\begin{WPDeliverables}
\begin{itemize}
\item Conversion of existing and new databases to unified interoperable system:
  \begin{itemize}
  \item Polytopes in Polymake
  \item graphs, graph properties
  \item LMFDB data
  \item FindStat data
  \item Finite groups (Max)
  \item Lattices
  \end{itemize}
  
\item \ref{del:persistent_memoization} (Month X): Shared persistent
  memoization library for Python/Sage.  Typical use case: A group of
  collaborators is using intensively a given function (in Sage, or in
  their private code). They want to memoize the results, as with
  e.g. Sage's \lstinline{cached_method}, but across sessions.  They
  further want to share the underlying growing database between
  themselves, and maybe eventually publish it.

  Features:
  \begin{itemize}
  \item Use, further extend, and contribute back to some established
    (Python?) persistent memoization infrastructure. E.g.
    \begin{itemize}
    \item \url{https://pythonhosted.org/joblib/memory.html}
    \item \url{github.com/vivekn/redis-simple-cache}
    \item \url{bitbucket.org/zzzeek/dogpile.cache}
    \end{itemize}
  \item Apply not only to user-level functions, but also to lower
    level functions, e.g. in the Sage library, so that indirect calls
    to the function also get memoized.
  \item Trivial to setup and configure for the end user: in a single
    line, the user selects an existing function, a backend (with a
    default value), maybe provide some semantic information, and
    voila. \\
    Typical interface: a decorator to be set on appropriate functions.
    \TODO{Mock code}
    % mycloud = storage("ssh:xxx@yyy/zzz")
    % memoize(sage.combinat...., storage=mycloud, input=ZZ, output=Posets(), key="catalan")
  \item Trivial to setup and configure for groups of researchers, with
    a wide range of storage backends (e.g. shared dropbox folder,
    remote directory, database, git repository, ...).
  \item Easy to setup data-bot: e.g. launching a virtual machine that
    systematically fills up the shared database.
  \item Versioning and provenance tracking (user, algorithm, software
    version, ...), for quality certification, credit, ...
  \item Recomputation?
  \item Ease of publishing, importing, ...
  \item Usual database properties: atomicity, merging (easy since the
    results are supposed to be immutable: just need to merge the
    tracking info), alerts in case of divergence.
  \end{itemize}
\end{itemize}

\begin{itemize}
\item D6.2: workshop report
\item D6.3.1: Design of Triform (DKS) Theories (Specification/RNC Schema/Examples)          # 12 PM
\item D6.3.2: Implementation of Triform Theories in the MMT API. 

\item D6.4.1: Python/Sage Syntax Foundation Module in OMDoc/MMT                                   # 12PM
\item D6.4.2: Python/Sage Computational Foundation Module in OMDoc/MMT
\item D6.4.3: Python/Sage Declarative Semantics in OMDoc/MMT

\item D6.5.1: LFMDB deep modeling: Fragment Identification & Initial Model Design              # 12PM
\item D6.5.2: LFMDB Data vs. Knowledge vs. Software Validation
\item D6.5.3: LFMDB Algorithm verification wrt. a Triformal theory
\item D.6.5.4: LMFDB full integration of algorithms, data and presentation (not so much verification)

\item D6.6.1: Heuristical Parser for the OEIS
\item D6.6.2: Cross-Validation for OEIS DKS-Theories.
\end{itemize}
\end{WPDeliverables}
\begin{verbatim}

===================8<---------------------------------


Another connection: on several occasions, we found that software was
the best way to represent certain databases of mathematical
knowledge. E.g. in Algebraic Combinatorics we have a whole zoo of Hopf
algebras. Many of them are implemented in MuPAD/Sage by specifying the
objects that index the basis together with computation rules for the
product and coproduct. When we want to retrieve information about such
algebras, it's usually much more convenient to look at the code than
to search through the literature. Especially since the code is usually
more correct than the literature because it's *tested*.



We may also think of providing an interface to LMFDB via SCSCP
protocol (http://www.symbolic-computing.org/scscp) so it may
be accessed by a variety of other systems (see their current
list at http://www.symbolic-computing.org/scscp)


database access to LMFDB as a python library


\end{verbatim}
\end{Workpackage}
