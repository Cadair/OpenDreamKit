\def\DKS{\ensuremath{\mathcal{DKS}}\xspace}

\begin{workpackage}[id=dksbases,%wphases=1-48!.5,
  title=Data/Knowledge/Software-Bases,lead=JU,
  ZHRM=12,JURM=46,UWRM=25,SARM=10,LLRM=2,PSRM=4]

\begin{wpobjectives}
  The ultimate purpose of a mathematical VRE is to create \emph{data} ($\mathcal{D}$; see
  Section~\ref{sec:innovation}), knowledge ($\mathcal{K}$), and software ($\mathcal{S}$)
  by conducting modeling world situations, computing mathematical objects, and running
  computational experiments. To be effective a VRE needs an infrastructure that supports
  the creation, management, access, and dissemination of \DKS-Structures.  All
  the systems considered in this proposal (\GAP, \Sage, \Pari, \Singular, OEIS, arXiv.org,
  \ldots) already include data, knowledge, and software modules as part of their regular
  distribution, but not in a form that is interoperable between systems, severely limiting
  the usefulness of the systems and results. The objectives of this work package are
\begin{compactenum}
\item to design metadata- and representation formats for trans-system $\mathcal{DKS}$
  structures as a basis for a math VRE, 
\item implement interfaces to existing systems for interoperability and compatibility with
  the RE, and
\item implement a joint \DKS infrastructure for, searching, documentation, traceability,
  versioning, provenance, visualisation and native dissemination of \TheProject results
  (the latter three together with \WPref{UI}).
\end{compactenum}
Concretely we will design and build an infrastructure that would make it easy for either
individual mathematicians or a distributed collaboration to manage and use such
interlinked mathematical data. This work would provide part of the backend to \WPref{UI},
and would draw on previous work with the \LMFDB and \FindStat (which will be treated as
prototypes for our purposes, to serve as exemplars to other projects) and in return will
substantially enhance their capabilities.

User prerequisites should be kept to a minimum (depending on contributors' and users'
needs and goals), and in particular would not require any background in databases to
contribute new data or perform queries.
\end{wpobjectives}

\begin{wpdescription}
  We need ways to represent \DKS in the same representational system, make the \DKS
  structures explicit and therefore machine-manageable and -- since current
  computational/experimental mathematics involve quite extensive \DKS -- we need a new
  kind of ``database'', which we will call Mathematical Data/Knowledge/Software-base
  (\textbf{\DKS base}), and which we will build in this work package.

  The starting points for this unification effort will be the system-oriented data bases
  for $\mathcal{D}$, the OMDoc (\underline{O}pen \underline{M}athematical
  \underline{Doc}uments) framework~\cite{Kohlhase:OMDoc1.2} for $\mathcal{K}$.
  OMDoc/MMT~\cite{RabKoh:WSMSML13} is a representation format for mathematical documents
  and knowledge that incorporates a metalogical framework to be foundation-independent,
  which allows interoperability between various ontologies/foundations of mathematics. For
  the integration of $\mathcal{K}$ and $\mathcal{S}$ we will build on the notion of
  \emph{biform theories} developed by Carette/Farmer~\cite{btc07} and extended to
  OMDoc/MMT by \site{JU} in~\cite{KohManRab:aumftg13}. In this setup, the programming
  language serves


 We will  The complexity is on vivid display in the \emph{L-functions and Modular Forms database}
  project (\LMFDB): while the general shape of the functional equation of an $L$-function
  is dependent on a lot of theoretical knowledge, it also requires parameter data and the
  coefficients of the associated Dirichlet series. Once this is obtained, highly optimised
  (and heavily parallelizable) algorithms can be run to compute values of this function.
\end{wpdescription}

\begin{tasklist}
\begin{task}[title={Survey of existing \DKS bases, Formulation of requirements},
  id=data-assessment,lead=ZH,partners={JU,SA,UW,US},wphases=0-3,PM=4]
  In this task, we will survey existing databases, the technology used to implement them,
  how they were linked to the rest of the existing infrastructure and the functionalities
  offered. We will also select additional external data and projects to add to this
  effort, aiming to maximise the impact of our work.

  We will organise a workshop associated to this task (\localdelivref{wsrep})
\end{task}

\begin{task}[title=Triform Theories in OMDoc/MMT,id=data-triform,
  lead=JU,partners={ZH},PM=12,wphases=0-12]
  Work here would extend OMDoc/MMT biform theories along the data axis, which will require
  a specialised but integrated treatment This integration will serve as a theoretical
  basis informing the design of a \DKS base in \localtaskref{data-design}.

  The results are reported in \localdelivref{dkstheories} (report) \localdelivref{dksimp}
  (implementation).
\end{task}

\begin{task}[id=data-design,lead=JU,partners={ZH,US,SA,UW,LL},wphases={6-12,15-18!.33},PM=12,
  title={\DKS Base Design}]
  Ontologies are the canonical method used to implement databases that require significant
  data interchange. However, because of the extreme reification present in mathematics
  (relations between objects themselves become objects of study), there are specific
  obstacles compared to the usual semantic web model of publishing.

  Drawing on semantic web/Linked Open Data experience of the \site{LL} group, specialised
  to mathematics through the OMDoc/MMT work of the Bremen group, we will design a
  decentralised infrastructure for \TheProject. This infrastructure would allow modular
  collaborations, through decentralised hosting of data without the need to merge
  everything centrally.
  
  The initial design of the \DKS base in \TheProject is reported in
  \localdelivref{design}. Conversion issues are covered in 
\end{task}

\begin{task}[title=Computational Foundation for Python/Sage,
  id=data-foundationCAS,lead=JU,partners={ZH,SA},PM=9,wphases=6-18!.66]

  In the OMDoc/MMT world a foundation is a logical base language that gives the formal
  meaning to all objects represented/formalized in it. We have created a very initial
  computational foundation for Scala and implemented it in the MMT API. This can be used
  to execute (or verify) computations directly in OMDoc/MMT and thus forms the basis for
  various integration tasks for OMDoc/MMT biform theories that integrate Scala
  computations. Here we propose to develop a somewhat more complete computational
  foundation for Python and/or parts of Sage (coverage to be determined). Bi/Triform
  theories come in three parts:
  \begin{compactitem}
  \item \emph{syntax}: what operators/types are there, how do they nest,
  \item \emph{computation}: what does the computation relation look like (sometimes called
    operational semantics). The declarative semantics of a computational foundation can be
    given as an OMDoc/MMT theory morphism into another foundation (e.g. a set theory);
  \item \emph{specification}: what are the observable properties of the computation. 
  \end{compactitem}
  The foundation (a triform theory in OMDoc/MMT) will be published as
  \localdelivref{psfoundation}.
\end{task}

\begin{task}[title=OEIS Case Study (Coverage and automated import),id=data-OEIS,lead=JU
,PM=6,wphases=12-18]
In this case study we test the practical coverage of the trifunctional modules, by
transforming an existing, high-profile database (the Online Sequence of Integer Sequences
\url[http://www.oeis.org]) into OMDoc/MMT. The OEIS has about 250 thousand sequences, with
formulae, descriptions, definitions, references, software, etc. in a structured text file
(but no standardized format for formulae and references), so we expect to get 250 k
theories. Having the OEIS in OMDoc/MMT form allows to do Knowledge Management services
(presentation, definition lookup, formula search, ...) in \MathHub (see \WPref{UI}). The
OEIS is a good case study, since the data is licensed under a Creative Commons license
which allows derived works. The large size will allow statistically significant semantic
cross-validation of the heuristic transformation process and thus achieve a significant
community resource.

The results of the import are reported in \localdelivref{oeisparser}.
\end{task}

% Michael, I think triformal theories would be easier to start with findstat.org
% There are many reasons: more consistent structure in the mathematical data, more established research patterns, more consistent database storage, tighter integration of the code with sage code (in fact copy paste), etc

\begin{task}[title=FindStat Case Study (triformal theories),id=data-findstat,
  lead=JU,partners={ZH},PM=9,wphases=18-30!.5]
  In this task we would develop triformal theories for the \FindStat project to test the
  design from \localtaskref{data-foundationCAS}.  Similarly to the previous task, in this
  case study, we first develop a thorough OMDoc/MMT model, which should only involve a
  handful of MMT theories (combinatorial collections, maps, statistics,...), each with a
  few hundred realisations. Together with   \WPref{UI}, this will again allow for
  easier knowledge management services, and in particular improved search services.

  This Task will be co-developed with \localtaskref{data-foundationCAS}, it will validate
  the design of triformal theories and be iterated to test the design changes. Results
  will be reported in \localdelivref{findstat}.
\end{task}

\begin{task}[title=\LMFDB Case study (triformal theories),id=data-LMFDB,
  lead=JU,partners={ZH,UW},PM=24,wphases={12-24!.25,24-48!.7}]
  In this task we would develop triformal theories for an exemplary part of the \LMFDB
  project to test the design from \localtaskref{data-foundationCAS}.  We will identify a
  fragment of the \LMFDB that we want to model and design the model (see
  \localdelivref{lmfmod}). 

  Then we will perform cross-validation of the three model parts against each other
  (essentially model-based testing of software and inference; see
  \localdelivref{lfmverif}). Once this has been successful for the chosen fragment, we
  will try to semi-automatically extend the import and model to the whole \LMFDB to gain
  coverage and integrate it fully into the \DKS base. We expect that this will entail
  quite a lot of work in refactoring the \LMFDB proper, which will benefit the \LMFDB
  community independently of its use in \TheProject.

  Finally, we will pick an algorithm from the \LMFDB and verify it against its
  specification and the computational foundation developed in
  \localtaskref{data-foundationCAS}, this is the final validation of the case study; the
  results are reported in \localdelivref{lfmint}.
  \end{task}

\begin{task}[title=Memoization and production of new data,id=data-memo,
  lead=SA,partners={US,PS,UW},PM=12,wphases=24-42!.6]
  Many CAS users run large and intensive computations, for which they want to collect the
  results while simultaneously working on software improvements. \GAP retains computed
  attribute values of objects within a session; \Sage currently has a limited
  \lstinline{cached_method}. Neither offers storage is not persistent across sessions or
  supports publication of the result or sharing within a collaboration. We will use,
  extend and contribute back to, an appropriate established persistent memoization
  infrastructure, such as \texttt{python-joblib}, \texttt{redis-simple-cache} or
  \texttt{dogpile.cache}, adding features needed for storage and use of results in
  mathematical research. We will design something that is simple to deploy and configure,
  and makes it easy to share results in a controlled manner, but provides enough assurance
  to enable the user to rely on the data, give proper credit to the original computation
  and rerun the computation if they with to.

  Results are reported in \localdelivref{persistent-memoization}

%Mock code:
%    \begin{lstlisting}
%       mycloud = storage("ssh:xxx@yyy/zzz")
%       memoize(sage.combinat...., storage=mycloud, input=ZZ, output=Posets(), key="catalan")
%    \end{lstlisting}
\end{task}

\begin{task}[id=mws,title=Math Search Engine,lead=JU,PM=10]
  
  \TODO{MK: write along the following lines We already have a search engine, therefore we
    only need to build harvesters -- i.e. programs that generate keywords and Formula
    URL/pairs from the contents -- \delivref{UI}{mws} and
    \delivref{UI}{notebooksearch}. For \delivref{UI}{cassearch}, this is also true in
    principle, but the formulae in code can take many more forms and the notion of a hit
    URL is not as clear.}  Outputs are reported in \localdelivref{mws},
  \localdelivref{cassearch}, \localdelivref{nbad-search}.
\end{task}

\end{tasklist}

\begin{wpdelivs}
  \begin{wpdeliv}[due=9,id=wsrep,dissem=PU,nature=R,lead=JU]{\DKS base survey and
      Requirements Workshop Report}
  \end{wpdeliv}
    \begin{wpdeliv}[id=mws,due=9,nature=OTHER,dissem=PU,lead=JU]
      {Full-text search (formulae + Keywords) over LaTeX-based documents
        (e.g. the arXiv subset)}
    \end{wpdeliv}
  \begin{wpdeliv}[due=12,id=design,dissem=PU,nature=R,lead=JU]
        {initial \DKS base Design}
   \end{wpdeliv}
  \begin{wpdeliv}[due=15,id=dkstheories,dissem=PU,nature=R,lead=JU]
        {Design of Triform (DKS) Theories (Specification/RNC Schema/Examples)}
  \end{wpdeliv}
  \begin{wpdeliv}[due=15,id=dksimp,dissem=PU,nature=OTHER,lead=JU]
        {Implementation of Triform Theories in the MMT API}
  \end{wpdeliv}
  \begin{wpdeliv}[due=18,id=lmfmod,dissem=PU,nature=R,lead=ZH]
      {\LMFDB deep modelling: Fragment Identification \& Initial Model Design}
  \end{wpdeliv}
  \begin{wpdeliv}[due=20,id=oeisparser,dissem=PU,nature=OTHER,lead=JU]
      {Heuristic Parser for the OEIS Import, Cross Validation of \DKS-Model}
  \end{wpdeliv}
  \begin{wpdeliv}[due=24,id=conv,dissem=PU,nature=DEC,lead=ZH]
        {Conversion of existing and new databases to unified interoperable system}
   \end{wpdeliv}
  \begin{wpdeliv}[due=24,id=psfoundation,dissem=PU,nature=OTHER,lead=JU]
        {Python/Sage Computational Foundation Module in OMDoc/MMT}
  \end{wpdeliv}
  \begin{wpdeliv}[due=30,id=findstat,dissem=PU,nature=OTHER,lead=JU]
      {Modelling and importing FindStat into \DKS base}
  \end{wpdeliv}
   \begin{wpdeliv}[id=cassearch,due=30,nature=OTHER,dissem=PU,lead=JU]
      {Formula search in CAS programs and Software Modules}
    \end{wpdeliv}
  \begin{wpdeliv}[due=36,id=pssem,dissem=PU,nature=OTHER,lead=JU]
      {Python/Sage Declarative Semantics in OMDoc/MMT}
  \end{wpdeliv}
  \begin{wpdeliv}[due=36,id=lfmverif,dissem=PU,nature=OTHER,lead=JU]
      {\LMFDB algorithm verification with respect to a Triformal theory}
  \end{wpdeliv}
  \begin{wpdeliv}[due=42,id=persistent-memoization,dissem=PU,nature=OTHER,lead=SA]
    {Shared persistent memoization library for Python/Sage} 
  \end{wpdeliv}
\begin{wpdeliv}[id=nbad-search,due=42,nature=OTHER,dissem=PU,lead=JU]
  {Search from Notebooks/Active Documents Interface} Often it is important to have some
  local context to inform search, therefore search from the notebook/active documents
  interface is an interesting and useful development target.
\end{wpdeliv}
  \begin{wpdeliv}[due=48,id=lfmint,dissem=PU,nature=R,lead=JU]
      {\LMFDB integration of algorithms, data and presentation}
  \end{wpdeliv}
\end{wpdelivs}

\begin{comment}
Another connection: on several occasions, we found that software was the best way to
represent certain databases of mathematical knowledge. E.g. in Algebraic Combinatorics we
have a whole zoo of Hopf algebras. Many of them are implemented in MuPAD/Sage by
specifying the objects that index the basis together with computation rules for the
product and coproduct. When we want to retrieve information about such algebras, it's
usually much more convenient to look at the code than to search through the
literature. Especially since the code is usually more correct than the literature because
it's *tested*.

We may also think of providing an interface to \LMFDB via SCSCP
protocol (http://www.symbolic-computing.org/scscp) so it may
be accessed by a variety of other systems (see their current
list at http://www.symbolic-computing.org/scscp). But it's probably as
good to access it via \Sage.

\end{comment}
\end{workpackage}
%%% Local Variables:
%%% mode: latex
%%% TeX-master: "../proposal"
%%% End:

%  LocalWords:  workpackage dksbases wphases wpobjectives standardise visualisation emph
%  LocalWords:  wpdescription Swinnerton-Dyer Millenium Borcherds optimised tasklist conv
%  LocalWords:  parallelizable maximise organise biform specialised trifunctional TOWRITE
%  LocalWords:  triformal findstat.org data-findstat localdelivref realisations texttt wrt
%  LocalWords:  Memoization python-joblib texttt redis-simple-cache texttt dogpile.cache
%  LocalWords:  lstlisting mycloud memoize sage.combinat wpdelivs wpdeliv dissem Polymake
%  LocalWords:  Recomputation wsrep dkstheories dksimp pssyntax psfoundation pssem lfmmod
%  LocalWords:  modelling lfmval lfmverif lfmint oeisparser oeisvalidation Hopf coproduct
%  LocalWords:  compactitem decentralised Logilab ensuremath xspace ldots compactenum
%  LocalWords:  textbf athematical uments RabKoh btc07 KohManRab aumftg13 localdelivref
%  LocalWords:  lmfmod lmfval findstat
