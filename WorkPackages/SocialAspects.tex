\TOWRITE{UM}{Proofread WP 7 Social aspects pass 2}
\begin{draft}
\begin{verbatim}
- [X] have all the tasks in this Work Package a lead institution?
- [X] have all deliverables in the WP a lead institution?
- [X] do all tasks list all sites involved in them? 
- [X] does the table of sites and their PM efforts match lists of sites for each task?
      (each site from the table is listed in all relevant tasks, and no site is listed
      only in the table or only at some task)
\end{verbatim}
\end{draft}


\begin{workpackage}[id=social-aspects,wphases=0-48,
  title=Social Aspects,
  lead=UO,
  UORM=23,USHRM=18,USORM=6] 

\begin{wpobjectives}

The processes by which mathematical knowledge and mathematical
software are developed, validated and applied are quite
distinctive. In other sciences, the universe provides ``ground truth''
and the scientific texts or theories can be validated against that by
experiment. In mathematics the text itself is the ground truth. The
traditional model of mathematical research is a mathematician, or a
small group of mathematicians, standing around a blackboard, producing
a proof they would ``clean up'': remove all traces of the process
that led to its discovery and then submit the ``clean'' text to
their peers for review.

Mathematicians have adopted new technology in a variety of ways: email
and shared documents are used to collaborate on problem-solving and
writing; larger ``crowdsourcing'' \cite{polymath_SIAM, PolymathBlog},
arrangements pull together diverse experts; symbolic computation
tackles huge routine calculations; and computers check proofs that are
too long and complicated for a human to comprehend. These
technologies reveal (since email messages, version control
systems and bulletin boards can be analysed) and alter the ways in
which mathematicians collaborate.

In an EPSRC funded project ``The Social Machine of Mathematics''
Martin and others are bringing together rigorous methods from the social 
sciences to study these collaborative processes. Combining this
research with the algorithmic game theory expertise of Elkind and Pasechnik,
in this work package we intend to pursue the following objectives: 

\begin{compactitem}
\item incorporate the insights from this and similar projects into the
  design of \TheProject VRE, ensuring that it supports the ways in
  which mathematicians really work, rather than the way software
  developers---or indeed mathematicians---think they do;
\item extend this work to study the collaborative processes of free
  open source (mathematical) software development so as to produce
  guidelines for best practice as well as to develop ideas for extending existing processes
  to a ``system of systems''.
\end{compactitem} 
\end{wpobjectives}

\begin{wpdescription}

``Crowdsourcing''---fine-grained collaborative development of ideas,
  proofs or software---is a common theme to both objectives. The purpose
  of a VRE is to allow effective crowdsourcing of computationally
  supported mathematical results (theorems, proofs, etc.), 
  while free software development is inherently a collaborative process, 
  and we wish to study the best ways of allowing it to scale.

  In a sense, mathematics has been a crowdsourced endeavour, dating as
  far back as the foundation of the Royal Society (UK) in the seventeenth
  century.  The first scientific journals were published collections
  of letters received, posing questions and observations and offering
  solutions.  Although limited by the speed of physical post, this
  model had much in common with the public email lists that
  underpinned collaborative software development in the 1990s.

  In recent years, the internet and critical tools such as distributed
  version control have supported much more widespread and finer-grained
  forms of crowdsourcing, first in software development, and, more recently, 
  in mathematics: examples are provided by online mathematics communities, such as Math-overflow
  \cite{mathoverflow} and Polymath Projects \cite{polymath_SIAM,
  PolymathBlog}.  Supporting and encouraging ``Mutual
  crowdsourcing'' is the main driving force for developing and
  maintaining any large-scale open-source virtual research
  environment.

  In this work package we will build on the work of Prof. Martin and her collaborators
  on the EPSRC project, and, in particular, their study of crowdsourcing, and 
  integrate their findings with tools provided by the burgeoning field 
  of algorithmic mechanism design in order to to 
  optimise crowdsourcing workflows in open-source VREs.

\end{wpdescription}

\begin{tasklist}
\begin{task}[title=Social Science Input to
    Design,id=social-input,lead=UO,PM=18, partners={UO,PS}]
The purpose of this task is to ensure that the design of \TheProject
VRE reflects the lessons learned by social scientists studying the
ways in which mathematicians actually collaborate and work. Since UO and
Martin in particular are already central in the community
working in this area, we are well placed to ensure that this happens. 

As soon as the project begins, team members at UO will combine their
own work with a review of the published literature, and identify and
meet with key research groups in this area, in order to distill relevant
current knowledge for use in the design phases of other parts of the
project. They will  present the lessons
learned at project meetings and workshops and deliver it as a report
\localdelivref{social-report} in month 3.

After that, they will monitor the further development of this area and
ensure that any new insights are communicated promptly to the rest of
the project. This will be synthesised for archival purposes into two
further reports 
\localdelivref{social-report-two} and 
\localdelivref{social-report-three} in months 24 and 42.

We will survey the data needed to assess development models of
large-scale academic open-source projects, such as the probable
correlation between the size of the atomic contribution vs. the speed
of the contribution making it into the code, and collect appropriate
statistical data, to be published as a report (and possibly a conference
publication) \localdelivref{social-datareport}. 
The latter will require non-trivial amount of
programming work, even only for the test system, \Sage.
\end{task}

\begin{task}[title=Implications of VREs for Publication,id=social-output,lead=USH,PM=12,wphases=12-42,partners={UO}]
  A key aspect of the \TheProject VRE is support for the full
  life-cycle of mathematical research, up to, and after publication of
  results. While it is necessary to support established models of
  publication, which are central to mathematical practice and academic
  life, it is also appropriate to explore whether VRE technologies may enable 
  novel models for the distribution of scientific output 
  that are more effective for new forms of mathematical results.

  The current model for dissemination of scientific output stems from
  an era when the printing press was dominant. The process has become
  formalised through peer review and publication of journals. The PDF
  format widely used for distribution of documents reflects the
  \textit{status quo} that a scientific paper is written as if for
  printing and remains an unchanging document. In scientific blogging
  we are seeing that more rapid propagation of ideas can occur when
  the constraints of the printed format are relaxed; however, these
  dissemination routes lack the formalization that ensures (usually) fair
  attribution of ideas and commentary.

  We will prototype and evaluate tools and ideas for dissemination of
  scientific knowledge that do not rely on a static format and allow
  for the full spectrum of scientific debate.  The tools will
  enable proper credit allocation by encouraging shared
  attribution of ideas, software and data. This will interact with
  work in \WPref{dksbases} concerning attribution and citeability for
  mathematical databases.

  Tools to be prototyped include live posters for distribution of
  knowledge, designed for integration with either large touch screens
  or smaller tablets \localdelivref{social-poster} as well as extensions
  of the Jupyter project that would provide facilities for commenting on
  notebooks, which we expect to encourage debate on mathematical and computational
  ideas \localdelivref{jupyter-comment}.
\end{task}


\begin{task}[title=Mechanism Design for Free Software Development,PM=15,lead=UO,
  wphases=12-42!.5,id=isocial-decisionmaking]

While crowdsourced open-source software development has become an
incredibly powerful force in recent years, it still has limitations. 
Open source projects tend to be fragile, in the community sense, and
suffer from disagreements that ultimately result in ``forks'' and the
resulting duplication of effort. We will analyse this phenomenon in the framework 
of algorithmic game theory, and try to design finely tuned systems of
incentives and rewards for contribution so as to increase the stability of
the community and its useful output.

We will focus on three areas: 
(1) prioritisation of bug fixes and feature requests to achieve reliable and useful systems; 
(2) effective cooperation among multiple collaborative projects; and 
(3) making decisions about the strategic direction of the system.  

We will use prioritisation as a testbed for designing incentives that encourage all
participants to contribute towards sustained development
of the most important parts of the system.
To this end, we will use
ideas from the burgeoning field of mechanism design \cite{AGTbook} and
in particular recent research on crowdsourcing in algorithmic
mechanism design \cite{crowds}.  While doing so, we will apply
outcomes to a case study system---\Sage.  

The reason why prioritisation poses a challenge in the 
development of open-source academic software is that this process is task-driven:
typically, tasks (also known as tickets) are posted on a website, and their
priorities are set in an ad hoc manner.  This model is usually
good enough for simple bug fixing, but for more elaborate tasks it often
leads to unacceptable delays. We will apply preference
and opinion aggregation techniques \cite{pref-aggr} to develop a
community prioritisation scheme for bugs and feature requests (which may rely on a reputation scores
technique, such as one used on MathOverflow),
and implement this scheme as a TRAC \cite{Trac} add-on 
\localdelivref{social-tracaddon}.
As \Sage is using the TRAC server \cite{trac-sagemath}, 
this will be easy to test on our testbed system.

Trusting results of computer calculations is crucial for
usability; channels for communicating bug reports and fixes need to be
carefully analysed from social point of view.  Commercial 
closed-source computer algebra and other computational systems often fail to
react to bug reports in a timely manner, and sometimes fall
into the short-sighted trap of hiding bugs from potential and current
users \cite{misfort}. Open source systems are only marginally better
in this respect, as indicated by recent computer security scares, such as the one
around Bash \cite{shellshock}.  A game-theoretic analysis of
this situation will be attempted.

A key strength of free and open-source software models is the ability
to build upon pre-existing software. \GAP, \PariGP, \Singular and
especially \Sage have made heavy use of this ability. Problems arise over
time, however, as priorities of the system developers 
diverge. For instance, bugs reported by so-called ``downstream'' systems may not be
given the same priority as bugs reported by direct
users of the ``upstream'' system, or ignored altogether; 
similarly, incompatible changes can be made as long as they are acceptable 
to the direct user community, even if they cause problems
for a dependent system. We will explore how sociological and game-theoretic 
insights can be used to reduce these problems.

The results of this task will be summarised in \localdelivref{social-gametheoretic}
and reported at relevant AI workshops and conferences.
\end{task}

\begin{task}[title=Evaluation of Micromagnetic VRE,lead=USO,PM=6,
id=oommf-nb-evaluation,partners={UO,PS},wphases=28-40!0.5]
  % 4 person months, 1 person month investigator time
  We will use the micromagnetic VRE demonstrator
  (\taskref{UI}{oommf-tutorial-and-documentation}), its dissemination
  workshops \linebreak(\taskref{dissem}{dissemination-of-oommf-nb-workshops})
  and interactions with its users and contributors in the
  micromagnetic community to evaluate, reflect and report on the project,
  taking into account technical and social aspects.

  A survey will be developed and used to gather user input and
  feedback on usefulness of the provided capabilities, with particular
  focus on the capabilities of the micromagnetic VRE to (i) enable new
  and better science, to (ii) allow to make progress effectively, to
  (iii) carry out computational science reproducibly, to (iv)
  collaboratively enable trust and to (v) become a self-sustained
  project from community contributions. Amongst other channels, we
  will target attendees of the micromagnetic VRE dissemination
  workshops (\taskref{dissem}{dissemination-of-oommf-nb-workshops}) to
  gather data.

  All results and insights will be summarised in a public document
  (\localdelivref{oommf-nb-evaluation}) and reported at appropriate
  workshops and conferences to share the lessons learned from this
  \Jupyter-based VRE for micromagnetics. We will create a manuscript
  for journal publication, summarising the demonstrator project and
  this evaluation. An important point of this publication is to
  provide a reference that can be cited by publications making use of
  the new micromagnetic VRE, to allow tracking of uptake and
  development of this VRE beyond the life time of this H2020 project.
\end{task}



\end{tasklist}

% Things to investigate?
% - User surveys. Cf. https://groups.google.com/d/msg/sage-devel/v8Kfky4p6D4/_xRM0bggCo8J
% - The discussion about Code of Conducts and the like

\begin{wpdelivs}
%   \begin{wpdeliv}[due=12,id=social-...,dissem=PU,nature=??]
%       {...}
% \end{wpdeliv}
\begin{wpdeliv}[due=3,id=social-report,dissem=PU,nature=R,lead=UO]
 {Report on relevant research in sociology of mathematics and lessons
   for design of \TheProject VRE, part I}
\end{wpdeliv}

\begin{wpdeliv}[due=18,id=social-datareport,dissem=PU,nature=R,lead=UO]
{The flow of code and patches in open source projects}
\end{wpdeliv}

\begin{wpdeliv}[due=24,id=social-tracaddon,dissem=PU,nature=OTHER,lead=UO]
{TRAC add-on to manage ticket prioritisation}
\end{wpdeliv}

\begin{wpdeliv}[due=24,id=social-report-two,dissem=PU,nature=R,lead=UO]
 {Report on relevant research in sociology of mathematics and lessons
   for design of \TheProject VRE, part II}
\end{wpdeliv}
\begin{wpdeliv}[due=24,id=jupyter-comment,dissem=PU,nature=DEM,lead=USH]
   {Demo: Mechanism for comments on posted Jupyter notebooks} 
\end{wpdeliv}

 \begin{wpdeliv}[due=36,id=social-poster,dissem=PU,nature=DEM,lead=USH]
   {Demo: Jupyter Notebook Live Poster} 
\end{wpdeliv}

\begin{wpdeliv}[due=42,id=social-report-three,dissem=PU,nature=R,lead=UO]
 {Report on relevant research in sociology of mathematics and lessons
   for design of \TheProject VRE, part III}
\end{wpdeliv}
\begin{wpdeliv}[due=42,id=social-publishing-report,dissem=PU,nature=R,lead=USH]
{Review of new publication mechanisms, including evaluation of
  demonstrator projects}
\end{wpdeliv}

\begin{wpdeliv}[due=42,id=social-gametheoretic,dissem=PU,nature=R,lead=UO]
{Game-theoretic analysis of development practices in open-source VREs}
\end{wpdeliv}

 \begin{wpdeliv}[due=48,id=oommf-nb-evaluation,dissem=PU,nature=R,lead=USO]
      {Micromagnetic VRE environment evaluation report}
\end{wpdeliv}
\end{wpdelivs}
\end{workpackage}
%%% Local Variables:
%%% mode: latex
%%% TeX-master: "../proposal"
%%% End:

%  LocalWords:  workpackage wphases TOWRITE wpobjectives analyse wpdescription AGTbook
%  LocalWords:  mathoverflow Sagemath pref-aggr prioritisation trac-sagemath Trac misfort
%  LocalWords:  analysed shellshock tasklist datacollection decisionmaking incentivised
%  LocalWords:  OOMMFNB taskref oommf-python-interface oommf-tutorial-and-documentation
%  LocalWords:  micromagnetic dissem dissemination-of-oommf-nb-virtual-environment texttt
%  LocalWords:  dissemination-of-oommf-nb-workshops summarised delivref wpdelivs wpdeliv
%  LocalWords:  oommf-nb-evaluation compactitem recomputation-style phenomenom endeavour
%  LocalWords:  localdelivref synthesised social-datareport textit WPref dksbases Elkind
%  LocalWords:  citeability isocial-decisionmaking social-tracaddon social-gametheoretic
%  LocalWords:  linebreak micromagnetics summarising Pasechnik
