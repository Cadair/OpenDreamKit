\begin{workpackage}[id=social-aspects,wphases=1-48!.5,
  title=Social Aspects,
<<<<<<< HEAD
=======
  lead=UO,
>>>>>>> e490abbfa8a91427570f1a7695a6a95cd4610713
  UORM=1,USHRM=8]

%\TOWRITE{DP/UM}{workpackage Social Aspects}

\begin{wpobjectives}
\TODO{Neil: Can we add something about scientific repeatability and sharing of mathematical models? I'm very keen to see systems where I can share a mathematical abstraction with a scientist in such a user-friendly way that they feel compelled to interact with it and develop their understanding of the modelling more. This might be an appropriate place to put this idea.}

The primary objective of this work package is to analyse the ``mutual crowdsourcing''
phenomenon occuring in the framework of development and maintainance of an open-source 
virtual research environment. 
We will focus on identifying the insentives for all
participants of the system, that would encourage sustained development of the
most important parts.  To this end, we will use ideas from the
recent reaserch on crowdsourcing in algorithmic mechanism design.
While doing so, we will apply outcomes to a case study
system.
Further, we will study development models for an academic free software ecosystem,
and analyse how they facilitate the mathematical process behind
relevant  algorithms and databases.
Finally, we will look into the questions of trust in results produced
by computations done by VREs, and ways to mitigate the issues arising there.
\end{wpobjectives}

\begin{wpdescription}
Crowdsourcing is a recent fenomenon
in software development in particular, but not only there. E.g. 
crowdsourcing of mathematical ideas occurs in online
mathematics communities, such as Mathoverflow \cite{mathoverflow} and
in Polymath Projects \cite{polymath}. 
``Mutual crowdsourcing'' is the main driving force
in developing and maintaining of any large-scale open-source 
virtual research environment. 

Open source projects tend to be fragile, in the community sense, and
suffer from disagreements that ultimately result in forks and
the resulting repetition of effort. We will analyse this in 
a setup of cooperative game theory, and try to design
a finely tuned systems of insentives and rewards for contribution, to increse
the stability of the community and its useful output.

We will focus on identifying the appropriate insentives for all
participants of the system, that would encourage sustained development of the
most important parts of the system.  To this end, we will use ideas from the
burgeoning field of meachnism design \cite{AGTbook} and in
particular on recent reaserch on crowdsourcing in algorithmic mechanism design
\cite{crowds}.  While doing so, we will apply outcomes to a case study
system---Sagemath.
We will apply preference and opinion aggregation techniques \cite{pref-aggr} to  develop
a community prioritisation scheme for Sagemath bugs and features requests, which 
are presently being mainitened on the Sagemath Trac server \cite{trac-sagemath}  and implement
them as a Trac \cite{Trac} add-on.

As well, we will study various development models for an academic free software ecosystem,
and analyse how they facilitate the mathematical process behind algorithms being
designed and implemented,  and databases of experimental data and test cases being
created and expanded.
Trusting results of computer
computations is crucial for usability; channels for communicating bug reports
and fixes need to be carefyully analysed from social point of view. 
Commercial closed source computer algebra and other computational systems often
fail to react to bug reports in a timely manner, and seemingly are falling into the
short-sighted trap of hiding bugs from potential and current users \cite{misfort},
Open source  systems are only marginally better in this way, as recent
computer security scares, such as the one around Bash \cite{shellshock}, indicate.
A game-theoretic analysis of this situation will be attempted.
\end{wpdescription}

<<<<<<< HEAD
\begin{task}[title=Modern Distribution of Scientific Output]
  \TODO{Neil: a suggested task in this domain, it's an area I've been thinking about a lot and it could fit in here. }
  The current model for distribution of scientific output stems from an era when the printing press was dominant. The process has become formalized through peer review and publication of journals. The PDF format for distribution of documents reflects the status quo, that a scientific paper is a written as if for printing and remains an unchanging document. In scientific blogging we are seeing that more rapid propagation of ideas can occur when the constrains of the printed format are released, however, there is a lack of formalization that means attribution of ideas and commentary run amiss. We will develope tools and ideas for distribution of scientific knowledge that don't rely on a static format and allow for the full spectrum of scientific debate. The tools will encourage proper credit attiribution through encouraging sharing of attribition for ideas, software and data. \TOWRITE{One deliverable simple idea would be to enable the construction of 'Living Posters' which can be interacted with, targeted at large touch screens.}
\end{task}

=======
\begin{tasklist}
\begin{task}[title=Modern Distribution of Scientific Output]
  \TODO{Neil: a suggested task in this domain, it's an area I've been thinking about a lot and it could fit in here. }
  The current model for distribution of scientific output stems from an era when the printing press was dominant. The process has become formalized through peer review and publication of journals. The PDF format for distribution of documents reflects the status quo, that a scientific paper is a written as if for printing and remains an unchanging document. In scientific blogging we are seeing that more rapid propagation of ideas can occur when the constrains of the printed format are released, however, there is a lack of formalization that means attribution of ideas and commentary run amiss. We will develope tools and ideas for distribution of scientific knowledge that don't rely on a static format and allow for the full spectrum of scientific debate. The tools will encourage proper credit attiribution through encouraging sharing of attribition for ideas, software and data.
  \TOWRITE{NL}{One deliverable simple idea would be to enable the construction of 'Living Posters' which can be interacted with, targeted at large touch screens.}
\end{task}

\begin{task}[title=Survey and collection of needed data,id=datacollection]
We will survey the data needed to assess development models of 
large-scale academic open-source projects, 
such that the probable correlation between the size of the atomic contributiion
vs. the speed of the contribution making it into the code, 
and collect appropriate statistical data. The latter will require non-trivial
amount of programming work, even only for the test system, Sagemath.
\end{task}

\begin{task}[title=Collective decision making in developement,id=decisionmaking]
Currently development of open-source academic software is task-driven, where tasks (also
known as tickets) are posted on a website, and their priorities are set in an ad hoc manner.
Whereas the latter might be good enough for simple bug fixing, for more elaborate task this
often leads to delays etc.
We would like to investigate an voting-driven approach, where the priorities are being
voted on by the developer community, and possibly the people who completed tasks
are incentivised in some form (e.g. by ``karma points'', as on Mathoverflow).
\end{task}

\end{tasklist}
>>>>>>> e490abbfa8a91427570f1a7695a6a95cd4610713

% Things to investigate?
% - User surveys. Cf. https://groups.google.com/d/msg/sage-devel/v8Kfky4p6D4/_xRM0bggCo8J
% - The discussion about Code of Conducts and the like

\begin{wpdelivs}
  \begin{wpdeliv}[due=12,id=social-...,dissem=PU,nature=??]
      {...}
\end{wpdeliv}
\end{wpdelivs}
\end{workpackage}
%%% Local Variables:
%%% mode: latex
%%% TeX-master: "../proposal"
%%% End:
