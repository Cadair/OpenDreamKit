\begin{workpackage}[id=social-aspects,wphases=1-48!.5,
  title=Social Aspects,
  lead=UO,
  UORM=1,USHRM=8, USORM=5]

%\TOWRITE{DP/UM}{workpackage Social Aspects}

\begin{wpobjectives}
  \TODO{Neil: I've provided some text below on repeatability on exchange of ideas. I don't want to edit too much into the mechanism design section, but I feel that both things are needed here to entice users into the ecosystem. I.e. a good mechanism is useless if no-one is interested in the underlying product/phenomenom. Can these two concepts be combined? I think that would strengthen this WP.}

%% From Neil (on the repeatability front if this is thought to be a good idea. Not integrated with the below though. Mechanism design is good, 
Mathematics provides a common language amongst the quantitative
sciences, however presentations of mathematics, and mathematical
models, are not always consummate with clear propagation of ideas that
the mathematics encode. By closely intertwining the application with
the mathematics, the project will develop more widespread
understanding of the utility of a mathematical abstraction and
encourage repeatability of analysis. \TheProject should enable
user-friendly presentation of the abstraction to such an extent that
even non computational/mathematical scientists feel compelled to
interact with and explore the model.



The primary objective of this work package is to analyse the ``mutual
crowdsourcing'' phenomenon occurring in the framework of development
and maintenance of an open-source virtual research environment.  We
will focus on identifying the incentives for all participants of the
system, that would encourage sustained development of the most
important parts.  To this end, we will use ideas from the recent
research on crowdsourcing in algorithmic mechanism design.  While
doing so, we will apply outcomes to a case study system.  Further, we
will study development models for an academic free software ecosystem,
and analyse how they facilitate the mathematical process behind
relevant algorithms and databases.  Finally, we will look into the
questions of trust in results produced by computations done by VREs,
and ways to mitigate the issues arising there.

\end{wpobjectives}

\begin{wpdescription}
Crowdsourcing is a recent phenomenon in software development in
particular, but not only there. E.g.  crowdsourcing of mathematical
ideas occurs in online mathematics communities, such as Math-overflow
\cite{mathoverflow} and in Polymath Projects \cite{polymath}.
``Mutual crowdsourcing'' is the main driving force in developing and
maintaining of any large-scale open-source virtual research
environment.

Open source projects tend to be fragile, in the community sense, and
suffer from disagreements that ultimately result in forks and the
resulting repetition of effort. We will analyse this in a setup of
cooperative game theory, and try to design a finely tuned systems of
incentives and rewards for contribution, to increase the stability of
the community and its useful output.

We will focus on identifying the appropriate incentives for all
participants of the system, that would encourage sustained development
of the most important parts of the system.  To this end, we will use
ideas from the burgeoning field of mechanism design \cite{AGTbook} and
in particular on recent research on crowdsourcing in algorithmic
mechanism design \cite{crowds}.  While doing so, we will apply
outcomes to a case study system --- \Sage.  We will apply preference
and opinion aggregation techniques \cite{pref-aggr} to develop a
community prioritisation scheme for \Sage bugs and features requests,
which are presently being maintained on the \Sage TRAC server
\cite{trac-sagemath} and implement them as a TRAC \cite{Trac} add-on.

As well, we will study various development models for an academic free
software ecosystem, and analyse how they facilitate the mathematical
process behind algorithms being designed and implemented, and
databases of experimental data and test cases being created and
expanded.  Trusting results of computer computations is crucial for
usability; channels for communicating bug reports and fixes need to be
carefully analysed from social point of view.  Commercial closed
source computer algebra and other computational systems often fail to
react to bug reports in a timely manner, and seemingly are falling
into the short-sighted trap of hiding bugs from potential and current
users \cite{misfort}, Open source systems are only marginally better
in this way, as recent computer security scares, such as the one
around Bash \cite{shellshock}, indicate.  A game-theoretic analysis of
this situation will be attempted.
\end{wpdescription}

\begin{tasklist}
\begin{task}[title=Modern Distribution of Scientific Output]
  \TODO{Neil: a suggested task in this domain, it's an area I've been
    thinking about a lot and it could fit in here. } The current model
  for distribution of scientific output stems from an era when the
  printing press was dominant. The process has become formalized
  through peer review and publication of journals. The PDF format for
  distribution of documents reflects the status quo, that a scientific
  paper is a written as if for printing and remains an unchanging
  document. In scientific blogging we are seeing that more rapid
  propagation of ideas can occur when the constrains of the printed
  format are released, however, there is a lack of formalization that
  means attribution of ideas and commentary run amiss. We will develop
  tools and ideas for distribution of scientific knowledge that don't
  rely on a static format and allow for the full spectrum of
  scientific debate. The tools will encourage proper credit
  attribution through encouraging sharing of attribution for ideas,
  software and data. We will develop live posters for distribution of knowledge, designed for integration with either large touch screens or smaller tablets \delivref{dissem}{social-poster}.  
\end{task}

\begin{task}[title=Survey and collection of needed data,id=datacollection]
We will survey the data needed to assess development models of
large-scale academic open-source projects,
such that the probable correlation between the size of the atomic contribution
vs. the speed of the contribution making it into the code,
and collect appropriate statistical data. The latter will require non-trivial
amount of programming work, even only for the test system, \Sage.
\end{task}

\begin{task}[title=Collective decision making in development,id=decisionmaking]
Currently development of open-source academic software is task-driven, where tasks (also
known as tickets) are posted on a website, and their priorities are set in an ad hoc manner.
Whereas the latter might be good enough for simple bug fixing, for more elaborate task this
often leads to delays etc.
We would like to investigate an voting-driven approach, where the priorities are being
voted on by the developer community, and possibly the people who completed tasks
are incentivised in some form (e.g. by ``karma points'', as on MathOverflow).
\end{task}

\begin{task}[title=OOMMF case study: Evaluation,lead=USO,PM=5]
  % 4 person months, 1 person month investigator time
  We will start to make the \OOMMFNB{} code (see
  \taskref{UI}{oommf-python-interface} to
  \taskref{UI}{oommf-tutorial-and-documentation}) available as open
  source to micromagnetic community research groups as early as
  possible
  (\taskref{dissem}{dissemination-of-oommf-nb-virtual-environment}) to
  gather feedback on its usefulness over a period as long as
  achievable. We will use this case study data to support our study of
  social aspects of mutual crowdsourcing.

  A survey will be developed and used to gather user input and
  feedback on usefulness of the provided capabilities, with particular
  focus on where \OOMMFNB{} enables new and better science, where it
  allows to do work more effectively and reproducibly, and the role of
  community contributions. We will further gather suggestions for
  improvements, and this feedback will be evaluated continuously and
  where possible re-acted on. We will \texttt{target OOMMF-NB} users
  that attend workshops (see task
  \taskref{dissem}{dissemination-of-oommf-nb-workshops}). All results
  and insights will be summarised in a report (\delivref{social-aspects}{oommf-nb-evaluation}) to share the lessons
  learned from this user interface application example, and be made
  publicly available. Where possible, we will report our findings at
  relevant workshops/conferences.
\end{task}



\end{tasklist}

% Things to investigate?
% - User surveys. Cf. https://groups.google.com/d/msg/sage-devel/v8Kfky4p6D4/_xRM0bggCo8J
% - The discussion about Code of Conducts and the like

\begin{wpdelivs}
  \begin{wpdeliv}[due=12,id=social-...,dissem=PU,nature=??]
      {...}
\end{wpdeliv}
 \begin{wpdeliv}[due=36,id=social-poster,dissem=PU,nature=DEM]{Demonstrator: Jupyter Notebook Live Poster} \end{wpdeliv}

\end{wpdelivs}
\end{workpackage}
%%% Local Variables:
%%% mode: latex
%%% TeX-master: "../proposal"
%%% End:

%  LocalWords:  workpackage wphases TOWRITE wpobjectives analyse wpdescription AGTbook
%  LocalWords:  mathoverflow Sagemath pref-aggr prioritisation trac-sagemath Trac misfort
%  LocalWords:  analysed shellshock tasklist datacollection decisionmaking incentivised
%  LocalWords:  OOMMFNB taskref oommf-python-interface oommf-tutorial-and-documentation
%  LocalWords:  micromagnetic dissem dissemination-of-oommf-nb-virtual-environment texttt
%  LocalWords:  dissemination-of-oommf-nb-workshops summarised delivref wpdelivs wpdeliv
%  LocalWords:  oommf-nb-evaluation
