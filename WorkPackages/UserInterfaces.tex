\TOWRITE{ALL}{Proofread WP 4 User Interfaces pass 2}
\begin{draft}
%\begin{verbatim}
%- [ ] do all tasks list all sites involved in them?
%- [ ] does the table of sites and their PM efforts match lists of sites for each task?
%      (each site from the table is listed in all relevant tasks, and no site is listed only in the table or only at some task)
%\end{verbatim}
%fixed: \TODO{D4.14 and D4.15 are not referenced in any task}
\end{draft}

\begin{workpackage}[id=UI,wphases=0-48,swsites,
  title=User Interfaces,
  lead=SR,
  PSRM=26,  % Sage-Jupyter interface, sphinx documentation dynamic documentation and exploration system
  UVRM=2,   % Sage-Jupyter interface
  JURM=12,  % Jacobs: active documents
  USHRM=6, % Supporting reproducible data science and sharing of models
  LLRM=12, % Help on several computer-centered tasks, dynamic SparQL in notebooks
  SARM=18, % GAP
  UKRM=2, % Singular
  UBRM=28,  % Pari
  USORM=16, % Southampton, micromagnetic VRE some contribution (1 month) to 3d visualisation
  SRRM=28,
  USRM=4, % University of Silesia, 3d without subcontracting
  swsites]    % rotate partner logos so that table fits on page.

\begin{wpobjectives}
  The objective of this work package is to provide modern, robust,
  and flexible user interfaces for computation, supporting real-time
  sharing, integration with collaborative problem-solving,
  multilingual documents, paper writing and publication, links to
  databases, etc.
\end{wpobjectives}

\begin{wpdescription}
  Project \Jupyter (formerly \IPython notebook) provides a browser
  based approach to constructing executable documents which comprise
  of code (in multiple languages), mathematics, text, diagrams (see
  Section~\ref{sec:jupyter}). The
  framework is an ideal portal through which \VREs can be operated. In
  this work package, we will add new functionality to the \Jupyter
  notebook that fosters excellence in computational science and
  research. In particular, we will push towards reproducible and
  effective science by allowing structured documents (such as reports,
  books, theses) from notebooks, and by allowing those notebooks to be
  re-executed as self-contained regression tests. We will unify the
  notebook infrastructure used in \Sage with \Jupyter, push forward
  dynamic documentation exploration capabilities, and work towards
  concurrent multi-user editing of notebooks. We will also develop
  exemplar \Jupyter notebooks for education and research
  (e.g. \taskref{dissem}{ibook}).

  To demonstrate the power of the \TheProject environment to
  accelerate computational science, deliver better value for money and
  make computational science more robust, we will put together a
  micromagnetic
  \VRE(\ref{sec:introduction-micromagnetic-vre-demonstrator}) as a
  demonstrator.

\end{wpdescription}

\begin{tasklist}
\begin{task}[title=Uniform notebook interface for all interactive
  components,id=ipython-kernels,lead=PS, partners={SR,UK,USH,USO,LL,SA,UV},
  PM=24, wphases=0-36]

  In this task, we will implement \Jupyter interfaces for the
  interactive computation components of \TheProject, including \GAP,
  \PariGP, \Sage, and Singular. A first release
  \localdelivref{ipython-kernels-basic} will focus on basic functionality,
  and a second release \localdelivref{ipython-kernels} will cover advanced
  features like 3D graphics or transparent documentation browsing (as
  live worksheets whenever relevant).

  % Note from William: my student Andrew Ohana just mostly did
  % something like that for IPython, but then stopped.  Anyway, it's
  % very do-able based on a summer project from another student and a
  % bunch of work I did with THREE.js for SMC.

  One of our objectives is to ensure the sustainability of the project
  (Objective~\ref{objective:sustainable}). The current \Sage notebook
  interface was developed alongside that of \Jupyter, but with
  slightly different goals. A notebook interface for \Sage is a vital
  integrative component, and development was fast tracked to ensure
  its availability to allow the project to move forward. However,
  \Jupyter, whilst it initially proceeded more slowly, has a larger
  developer base and has now caught up with the \Sage notebook in
  terms of functionality. In line with
  Objective~\ref{objective:sustainable} \Sage will now phase out its
  own notebook and switch focus to the \Jupyter notebook, outsourcing
  this key but non disciplinary component.

  % In charge: Jupyter dev + dev in Orsay + community?
  The \Sage and \Jupyter convergence \localdelivref{ipython-kernel-sage} will
  require:
  \begin{compactitem}
  \item Robust migration path and tools for \Sage worksheets,
  \item Support for math, 2D, and interactive 3D scene visualization,
    % \item Bundling of the \Jupyter notebook and its dependencies within
    %   the Sage distribution. DONE
  \item Import and export of ReST documents, with full support for
    \Sage's specific roles (math, ...),
  \item Support for remote \Sage kernel, typically on the cloud, or
    running with a different Python version (\Sage as a library),
  \item A migration path for interactive widgets implemented with
    \Sage's \texttt{@interact} functionality.
  \end{compactitem}

  Joint meetings and visits between the developers of \Jupyter and of
  the computing components will be a key component of this task.

\end{task}

\begin{task}[id=notebook-collab,title=Notebook improvements for collaboration,lead=SR, partners={PS,USH,JU,USO,LL}, PM=20, wphases=0-24]

  In this task, we will further improve tools for collaboration
  between authors of shared \Jupyter notebooks and draw from the
  experience of collaboration as set in Simulagora, SageMathCloud,
  etc.

  Version control tools, such as Git and Mercurial, have become an
  integral part of open and collaborative science and
  software. Version control tools allow proposed changes to be
  reviewed (`diffing') and resolve conflicts through combination of
  changes (`merging'). \Jupyter notebook documents are stored in text
  files as JSON formatted data. This makes them well suited to
  tracking in version control, but the JSON structure can make diffing
  and merging difficult. We will deploy tools to provide better
  support for visual diffing and merging of Notebook documents. These
  tools will be integrated into existing version control workflows
  \localdelivref{jupyter-collab}. The MathHub.info system already has
  a distributed Git-based versioning system, which can serve as an
  entry point here.

  Given the interactive nature of \Jupyter notebooks, live
  collaboration, where multiple authors work on the document
  simultaneously (like in Google Docs) is particularly
  desirable. However, there are particular challenges for
  collaborative editing of \emph{executable} documents. The potential
  for \emph{shared execution} adds both value and challenge to the
  live collaboration. Some attempts have been made to deal with live
  collaborative sessions (e.g. \SMC, Colaboratory) but so far these
  have been outside the core \Jupyter project. In this task we will
  explore different models of single-notebook collaboration, including
  shared or separate execution \localdelivref{jupyter-collab}. We will
  consider not only indicating authorship, but which author
  triggered which execution, and explore other challenges.  Various
  avenues for live session collaboration will be explored for
  integration into \Jupyter itself
  \localdelivref{jupyter-live-collab}.
\end{task}

\begin{task}[id=notebook-verification,title=Reproducible Notebooks,lead=SR, partners={PS,USO}, PM=4, wphases=12-24]
  In this task, we will develop tools that allow re-execution
  notebook documents with automated regression testing. The computed
  output will be compared against the stored output, and deviations
  reported as assertion errors.

  Notebooks are used in a variety of contexts, like training and
  teaching material (tutorials, documentation, books) or computer
  experimentation logbooks, where reproducibility is
  critical. Reproducibility dictates that the notebooks should remain
  functional and correct in the long run, even when the underlying
  computational software or infrastructure changes over time or across
  platforms.

  This task is a critical component of reproducibility, allowing the
  notebook author to get an immediate notice when, e.g., a backward
  incompatible change occurs. It becomes even possible to anticipate
  such situations upstream by including important notebooks directly
  in the automated test suite of the computational software, giving an
  easy way for casual users to contribute regression tests.

  Technically speaking, \Jupyter notebooks store outputs as rich
  mime-type structures, with JSON metadata. Using this metadata, it
  will be possible to express expectations of output, allowing more
  flexible and powerful tests than direct text comparison
  \localdelivref{jupyter-test}.  Prior work has been done in \Sage for
  ReST files, e.g. \lstinline{sage -t notebook.rst}, and this model
  will be extended to notebooks.
\end{task}

\begin{task}[id=sage-sphinx, title=Refactor \Sage's \Sphinx documentation system, lead=PS,PM=6, partners={SR,UV}, wphases=0-36]
  \Sage, like \Python and many other \Python based projects, uses the
  \Sphinx documentation system. Due to particularly stringent needs,
  many layers of customization and adaptations have accumulated over
  the years, in particular for proper scaling to the sheer size of the
  Sage documentation (13k pages just for the reference manual).

  A deep refactorization (\localdelivref{sage-sphinx}) is critically
  needed to get rid of multiple duplication, and foster sustainability
  by outsourcing back to \Sphinx all generic aspects (parallel
  compilation, index generation, ...).
  \TOWRITE{VP}{Viviane, this seems a little short, can we provide a little more detail of what the refactorisation will involve?}
  % In charge: dev in Orsay or Logilab + visit of Sphinx dev  + FH
\end{task}

\begin{task}[id=dynamic-inspect,title=Dynamic documentation and exploration system,lead=PS, partners={SR,USO,UV,LL}, PM=6, wphases=0-12]
  Introspection has become a critical tool in interactive computation,
  allowing user to explore, on the fly, the properties and
  capabilities of the objects under manipulation. This challenge
  becomes particularly acute in systems like \Sage where large parts
  of the class hierarchy is built dynamically, and static
  documentation builders like \Sphinx cannot anymore render all the
  available information.

  In this task, we will investigate how to further enhance the user
  experience. This will include:
  \begin{compactitem}
  \item On the fly generation of Javadoc style documentation, through
    introspection, allowing e.g. the exploration of the class
    hierarchy, available methods, etc.
  \item Widgets based on the HTML5 and web component standards to display
    graphical views of the results of SPARQL queries, as well as populating data
    structures with the results of such queries,
  \item \localdelivref{ipython-advanced-interacts} (Month 36)
    Exploratory support for semantic-aware interactive widgets
    providing views on objects of the underlying computational or
    database components. Preliminary steps are demonstrated in the
    \texttt{Larch Environment} project (see demo video on
    \url{http://www.larchenvironment.com/}) and
    \software{sage-explorer}
    (\url{https://github.com/jbandlow/sage-explorer}). The ultimate
    aim would be to automatically generate \LMFDB-style interfaces.
  \end{compactitem}
  Whenever possible, those features will be implemented generically
  for any computation kernel by extending the \Jupyter protocol with
  introspection and documentation queries.
  % In charge: \Jupyter dev + dev in Orsay + NT?
\end{task}

\begin{task}[title=Structured documents,id=structdocs,
  lead=JU,PM=22,lead=JU,partners={SR,USH,LL},wphases=0-24]

  \Jupyter notebooks consist of a sequence of cells that contain
  either text or a program (see Section~\ref{sec:jupyter}). Complex
  documents, such as books, articles or reports, require a richer
  description that covers the the structure of the document and the
  semantics of its elements. This task will investigate this problem
  and try to find a way to write these documents exploiting the
  breakthroughs achieved in the other tasks to this workpackage.

  Several technical complementary options can be explored:
  \begin{compactitem}
  \item MathHub.info is a portal for reading and interacting with
    ``active documents'' (i.e. documents that have an additional
    semantic layer that supports semantic services like definition
    lookup, type-inference, unit conversion,\ldots)
  \item \Jupyter notebooks are essentially ``programs with documentation'' and lack the
    semantical structure needed by complex documents.
  \item sTeX is a semantic variant of LaTeX that can be transformed into OMDoc/MMT, which
    is the native knowledge representation format for active documents and
    machine-actionable knowledge about math and symbolic programs.
  \end{compactitem}

  After gathering the needs and the requirements for the writing of
  complex documents in the mathematical field, we will study these
  design and build a solution that meets the expectations
  (\localdelivref{adstex}). The implementation will be achieved
  through an iterative process that incrementally improves existing
  software solutions, making them interoperable and synergistic.
  Results of this convergence will be reported
  in~\localdelivref{adcomp}, \localdelivref{ipython-kernel-sage} and
  \localdelivref{jupyter-import} and used in \taskref{dissem}{ibook}.
\end{task}

\begin{task}[id=mathhub,title=Active Documents Portal,lead=JU,PM=12,
  wphases=12-36!.5]
  We will extend the existing \url{http://mathhub.info} system to a
  portal for interacting with active/structured documents (see
  \localtaskref{structdocs}) and releasing the portal initially for
  internal use in the \TheProject and later for general
  use. \url{MathHub.info} already provides very basic sTeX editing and
  versioning. In \TheProject we will extend it on the computational
  side based on the integrated format from
  \localtaskref{structdocs}. The resulting portal will be made
  available to the consortium as~\localdelivref{mathhub-editing} and
  would be used for semantically enhanced code documentation and
  knowledge representation (see \WPref{dksbases}).
\end{task}

\begin{task}[title=Visualization system for 3D data in web-notebook
,id=vis3d,lead=SR, partners={US,PS,USO}, PM=13, wphases=0-24]
\TOWRITE{MRK,HPL}{wphases does not agree with PM. (13 vs 24}
%12 months from Simular,
% 1 month from Southampton for testing in the micromagnetic VRE demonstrator

The \Jupyter notebook provides an attractive environment for building
user interfaces for research. However, the current support for inline
visualization is limited to curve plots and 2D scalar fields. Many
scientific simulations need visualization of 3D scalar and vector
fields, as shown in Figure~\ref{fig:3d-plots}.  Experimentations in
low dimensional topology and differential geometry also relies on good
drawing capabilities
(e.g. \href{http://www.math.uic.edu/t3m/SnapPy/}{SnapPy} or
\href{http://sagemanifolds.obspm.fr/}{SageManifolds} based on \IPython
and \Sage). The amount of data can be tremendous, especially in
time-dependent problems computed in a distributed fashion over
large-scale computational clusters. Interactive inspection of such
simulations can be a valuable tool which accelerates
research. However, for inspection, one does not need to transfer and
gather the full dataset at each time step---getting selected computed
fields on user request or preprocessing certain quantities like cross
sections with some predefined frequency will mostly suffice.

In this task we will first investigate available technologies for fast
in-browser visualization of the typical structures to be displayed
(isosurfaces, streamlines, vector fields, cross sections, etc.).
There are several existing solutions which could provide basis for
further development. One of the best known, and also advanced is
\href{http://threejs.org/}{three.js} which provides basis for 3D
visualization in a web browser. Three.js is WebGL based, but also
provides canvas based rendering for system which do not support
WebGL. It has already been experimentally deployed in Sage Cell Server
and SMC projects. Another promising technologies are visualization
libraries using exclusively OpenGL. They can be deployed in browser
based systems by using of the WebGL API (which is a restricted subset
of the regular OpenGL API). This can be accomplished by visualization
executed purely in GPU. \href{http://vispy.org/}{VisPy} and
\href{http://glumpy.github.io/}{glumpy} projects have found GPU-only
solutions for common visualization objects (lines, arrows, markers,
text, iso-lines, iso-surfaces, text, etc) where data does not exit the
GPU. The VisPy project already offers an experimental interface with
Jupyter notebook that could be extended to cope with our
specifications. Through this tight collaboration with the authors,
\TheProject could benefit from both dedicated and state-of-the art
visualization techniques.

The \href{http://www.math.uic.edu/t3m/SnapPy/}{SnapPy} and
\href{http://sagemanifolds.obspm.fr/}{SageManifolds} projects will be
considered for deployment of tools we develop (see associated
deliverable \localdelivref{vis3d}).
\end{task}


\begin{task}[title=Visualization of 3D fluid dynamics data in web-notebook
,id=cfd-vis,lead=SR, partners={US,PS,USO},PM=5,wphases=12-36]

We propose to let computational fluid dynamics (CFD) be a driving
application for the development of 3D visualization in \Jupyter
notebooks (\taskref{UI}{vis3d}) since CFD is one of the most demanding
cases of scientific visualization. The same time this task
(with deliverable \localdelivref{cfd-vis}) will be
demonstrator for (\taskref{UI}{vis3d}).

Successfully handling CFD makes the tool immediately applicable to a
range of other fields such as heat transfer, electromagnetics,
material science, and 3D algebraic structures in
mathematics. Simulations would be initialised inside the notebook and
executed on HPC clusters. This approach will significantly lower the
threshold for using parallel computing codes that can be hard to
install correctly on local workstations (see also \WPref{hpc}). Such
use cases with 3D visualization will greatly extend the potential
applications of the \Jupyter notebook concept throughout science and
engineering.

As example code for a 3D live web notebook with fluid dynamics
simulations, we will use the Lattice Boltzmann solver which is under
development at the University of Silesia:
\href{http://sailfish.us.edu.pl/}{Sailfish}.  This code is an advanced
Lattice Boltzmann solver designed from the ground up for distributed
system of GPU compute clusters. It is implemented predominantly in
Python, and it uses run-time code generation techniques to
automatically build optimised code for CUDA and OpenCL devices. Since
running Sailfish requires specialised hardware, it is reasonable to
use it on dedicated HPC installations.
\end{task}

\begin{task}[lead=UB,title=Common option system for various displays
  in Sage,id=Sage-display,PM=12,wphases=0-24]
  \TOWRITE{CNRS}{There are no deliverables associated with this task
    that is listed at 12 person months. Perhaps some explanation of
    the challenges of the task would also help.}  

  Given a mathematical object, it often has various possible
  representations on a computer. From raw text to \LaTeX, from simple
  2d picture to a complicated 3d animation.

  In this task, we provide a uniform option system for displaying
  object within \Sage (raw text, \LaTeX, tikz, matplotlib, jmol,
  tachyon, \ldots). We will implement some of the missing display and
  will benefit of the work done in \taskref{UI}{cfd-vis}.
\end{task}

\begin{task}[lead=USO,title=Case study: micromagnetic VRE built from
  \TheProject,id=oommf-py-ipython-attributes,PM=6,partners={SR,USH},wphases=9-15]
  % 6 person months

  In this task, we use the \TheProject architecture to assemble a
  virtual research environment software tailored for the large
  micromagnetic research community
  (see Section \ref{sec:introduction-micromagnetic-vre-demonstrator}).

  The micromagnetic VRE will be based on the \Jupyter notebook, the
  Python interface to the micromagnetic simulation library OOMMF
  (\taskref{component-architecture}{oommf-python-interface}),
  and the additional features added to \Jupyter in this work
  package.

  The \Jupyter notebook environment allows to host, execute and
  document the Python-based OOMMF simulation in an executable
  document. In this interactive environment, objects can be displayed
  using various representations, including, for example, textual
  representation (i.e. strings), bitmap images and SVG (vector
  graphics) files. We will create functionality so that magnetisation
  vector field objects can be presented as a rendered 3d and 2d-view
  of the magnetisation field (Figure~\ref{fig:3d-plots}), and similar
  features for scalar fields such as field components and energies for
  static and time dependent data (linking to
  \localtaskref{cfd-vis}). This allows computational steering and the
  interactive exploration of the behavior of magnetic nanostructures.

  Beyond that, the \Jupyter Widgets allow the creation of graphical
  user interface (GUI) elements, and we will generate code to display
  these widgets on demand to (i) set up micromagnetic simulations
  using a GUI, and (ii) assist in common post-processing simulation
  results. Recent pilot work has shown that it is possible to make
  \Jupyter Widgets interact with the Python interpreter session and
  this allows to activate a GUI-like (widget based) interface when
  desired but to quickly return to the interpreter prompt, taking
  forward the results (data) from the GUI session
  \cite{IPython-widget-GUI-demo-youtube-2014} and providing a
  continuous path from scripting to GUI. Having the ability to mix and
  match GUI-based and command driven analysis combines the best of
  both approaches, caters for users' preferences, and provides
  significant additional value.

  The deliverable for this task is the open source micromagnetic VRE
  software (\localdelivref{oommf-nb}).
\end{task}

\begin{task}[lead=UB,title=Python/Cython bindings for Pari,PM=16,id=pari-python,wphases=0-24]
  \TOWRITE{CNRS}{The task seems a little short for a 16 PM task, more description of the difficulties of the task would help here.}

  \Pari is a C-library and GP is its standalone interpreter. Partial
  Python/Cython bindings are provided by Sage.

  The task aims to develop an independent Python/Cython library that
  would provide bindings for \PariGP and which would tightly be
  developed within the \PariGP team.

  The deliverable for this task is \localdelivref{pari-python-lib}.
\end{task}

\begin{task}[lead=USO,title=Demonstrator: micromagnetic VRE notebooks,
  id=oommf-tutorial-and-documentation,PM=6,partners={SR,PS},wphases=15-21]
  % 5 person months + 1 month co-investigator [Ian Hawke's experience]

  The purpose of the micromagnetic \VRE
  (\localtaskref{oommf-py-ipython-attributes}) is to enable excellent
  computational research. To maximise the value of this grant's
  investment for the community, we will not carry out micromagnetic
  research but instead produce a set of executable notebooks using the
  micromagnetic \VRE to demonstrate its power and applicability.

  We will create executable notebook documents
  (\localdelivref{oommf-nb-documentation}) within the micromagnetic \VRE
  including (i) a new tutorial on computational micromagnetics with
  OOMMF, (ii) the complete documentation of the \texttt{OOMMF-Py}
  library (\taskref{component-architecture}{oommf-python-interface}),
  and (iii) a set of typical micromagnetic case studies. The tutorial,
  in terms of content, will take guidance from the tutorial provided
  for Nmag \cite{Nmag-tutorial-url} and will introduce the additional
  features of the \Jupyter-driven micromagnetic \VRE. We expect this
  substantial and executable documentation of the micromagnetic \VRE to
  become the standard resource that introduces researchers to
  computational micromagnetics, in particular through the online
  portal (\localtaskref{oommf-nb-ve}).

  %% This block is about the benefits of using the notebook. It should
  %% go somewhere else in more generic form:
  %The output of this activity will deliver multiple benefits:
  %providing a systematic introduction to \texttt{OOMMF-py} suitable for both
  %those users (i) new to micromagnetic modelling and those (ii) new to
  %the \texttt{OOMMF-py} interface. Because the documentation is developed in an
  %\Jupyter notebook, the documents are executable. For new learners
  %this is a great simplification because they can skip through the
  %given document and execute the given examples there and then: at the
  %moment, this is a process of manually writing a script, or locating
  %it in the directory structure of files, then executing this,
  %subsequently opening and processing the data files, etc. In the new
  %model, this end-to-end simulation will start from specifying the
  %material parameters in the notebook (all of this is given in the
  %tutorial), to running the simulation in the notebook to processing
  %of computed data while the simulation runs (or subsequently) in the
  %notebook; thus providing one virtual research environment, with all
  %the associated benefits of making best use of the scientist's time
  %using the tool and environment.

\end{task}

\begin{task}[lead=USO,id=oommf-nb-ve,title=Online portal for
  micromagnetic VRE demonstrator,PM=3,partners={SR,JU},wphases=21-24]

  % 3 person months
  Recently, a TeMPorary \Jupyter NoteBook (TMPNB) has been made
  available (at \href{http://tmpnb.org}{http://tmpnb.org}) that allows
  anybody to open this URL and use their very own \Jupyter notebook
  for quick calculations and tests online. We will provide such a
  portal (\localdelivref{oommf-nb-tmp}) which provides the
  micromagnetic \VRE for anonymous use. This service allows users to
  execute the demonstrator tutorial and documentation notebooks
  (\localtaskref{oommf-tutorial-and-documentation}) and run the
  calculations in real time on the web server, without having to
  install any software on their own machine.  This web service will be
  based on Docker \cite{Docker} virtualisation technology and we will
  make available the scripts to create VirtualBox \cite{Virtualbox}
  images, and Docker containers. The same virtual machine images can
  also be used for Cloud hosted computing services.

  %{HF}{Do we need the resource request here? Or should it
  %just be in resources.tex: Either works, in the resources file there
  %is only the total sum mentioned and a link to here. So no
  %duplication of information, and the particular machine is maybe
  %better explained here. I'll comment this out to 'resolve' it.}

  We request \euro{6000} to purchase a machine to provide these
  services (shared memory, 64 cores, 128GB RAM, Solid-state drive (SDD)
  to make the system more responsive).
  %This machine will also support
  %the regression testing and continuous integration (see task
  %\taskref{dissem}{dissemination-of-oommf-nb-virtual-environment}).
  %Setup and
  %maintenance of the machine is part of this work task.
\end{task}

\end{tasklist}

\begin{wpdelivs}
  \begin{wpdeliv}[id=adstex,due=6,nature=R,dissem=PU,lead=JU]
    {Active/Structured Documents Requirements and existing Solutions} Presenting sTeX and
    \Jupyter to the consortium, comparing and evaluating as stepping stones.
  \end{wpdeliv}
    \begin{wpdeliv}[id=mathhub-editing,due=12,nature=DEM,dissem=PU,lead=JU]
      {Distributed, Collaborative, Versioned Editing of Active Documents in MathHub.info}
    \end{wpdeliv}
  \begin{wpdeliv}[due=14,id=ipython-kernels-basic,dissem=PU,nature=OTHER,lead=PS]
      {Basic \Jupyter interface for GAP, \PariGP, \Sage, Singular}
  \end{wpdeliv}
  \begin{wpdeliv}[due=12,id=ipython-kernels,dissem=PU,nature=OTHER,lead=PS]
      {Full featured \Jupyter interface for GAP, \PariGP, Singular}
  \end{wpdeliv}
  \begin{wpdeliv}[due=24,id=pari-python-lib,dissem=PU,nature=OTHER,lead=UB]
	  {Python/Cython bindings for \PariGP}
  \end{wpdeliv}
  \begin{wpdeliv}[due=12,id=ipython-kernel-sage,dissem=PU,nature=DEM,lead=PS]
      {\Sage notebook / \Jupyter notebook convergence}
  \end{wpdeliv}
    \begin{wpdeliv}[due=15,id=oommf-nb,dissem=PU,nature=OTHER,lead=USO]
      {Micromagnetic VRE software completed}
    \end{wpdeliv}

    \begin{wpdeliv}[id=adcomp,due=18,nature=DEM,dissem=PU,lead=JU]
      {In-place computation in active documents (context/computation)}
    \end{wpdeliv}

  \begin{wpdeliv}[due=18,id=jupyter-test,dissem=PU,nature=OTHER,lead=SR]
      {Facilities for running notebooks as verification tests}
  \end{wpdeliv}

  \begin{wpdeliv}[due=12,id=jupyter-collab,dissem=PU,nature=OTHER,lead=SR]
      {Tools for collaborating on notebooks via version-control}
  \end{wpdeliv}
    \begin{wpdeliv}[due=21,id=oommf-nb-documentation,dissem=PU,nature=DEM,lead=USO]
      {Micromagnetic VRE tutorial and documentation notebooks}
    \end{wpdeliv}
    \begin{wpdeliv}[id=jupyter-import,due=24,nature=DEM,dissem=PU,lead=JU]
      {Notebook Import into MathHub.info (interactive display)}
    \end{wpdeliv}
    \begin{wpdeliv}[due=24,id=oommf-nb-tmp,dissem=PU,nature=DEC,lead=USO]
      {Demonstrator online portal available}
    \end{wpdeliv}
  \begin{wpdeliv}[due=24,id=vis3d,dissem=PU,nature=OTHER,lead=SR]
      {\Jupyter extension for 3D visualisation}
  \end{wpdeliv}
  \begin{wpdeliv}[due=36,id=cfd-vis,dissem=PU,nature=OTHER,lead=SR]
      {Computational Fluid dynamics visualization in web notebook}
  \end{wpdeliv}
  \begin{wpdeliv}[due=36,id=jupyter-live-collab,dissem=PU,nature=OTHER,lead=SR]
      {Exploratory support for live notebook collaboration}
  \end{wpdeliv}
  \begin{wpdeliv}[due=24,id=sage-sphinx,dissem=PU,nature=OTHER,lead=PS]
      {Refactorization of \Sage's \Sphinx documentation system}
  \end{wpdeliv}
  \begin{wpdeliv}[due=36,id=ipython-advanced-interacts,dissem=PU,nature=DEM,lead=PS]
      {Exploratory support for semantic-aware interactive widgets providing views on objects
      represented and or in databases}
  \end{wpdeliv}
% communication with live computing process
% post simulation data analysis module
% visualization of vector and scalar fields
% editor for geometry and boundary conditions  on regular meshes


  % Shared \Jupyter sessions embedded in voice-over-IP or
  % teleconference calls or reciprocally.
  %
  % NOTE: This task is probably outdated by appear.in which makes
  % video-conferencing in the browser trivial
  %
  % \delivref{ipython-collaborative}
  % Eugen Dedu:
  % I think such a module can be thought of as a screen-capturing
  % module, i.e. allow Ekiga to capture the screen of a Sage user (this
  % is currently not possible).  This is not a difficult task to do.
  % Julien Puydt: ekiga can do that since something like 2008 with my
  % experimental gstreamer plugin, and I shall be able to present
  % interesting sample code to the ekiga-devel mailing-list in something
  % like two-three weeks (after I'm done with my students), which will
  % hopefully be part of the next version.
  %
  % But as Nicolas noted in his answer, some kind of interactive session
  % where people can share a sage session would be better.
  %
  % I think the feature decomposes in the following pieces:
  % - IPython should have a way to share sessions between several
  % participants using an open and standard protocol ;
  % - ekiga should implement it.
  %
  % In my opinion ekiga, because of its dependency on ptlib and opal
  % libraries and the use of complex protocols like SIP and H323, needs
  % highly technical people.  Students cannot help much, but engineers
  % are appropriate.
  \end{wpdelivs}
\end{workpackage}

%%% Local Variables:
%%% mode: latex
%%% TeX-master: "../proposal.tex"
%%% End:

%  LocalWords:  workpackage wphases Jupyter OOMMFNB wpobjectives wpdescription TOWRITE
%  LocalWords:  Paderborn IPython KBase Hackathon Quantopian Logilab Enthought Authorea
%  LocalWords:  emph Jupyther nanostructures tasklist delivref THREE.js texttt diffing
%  LocalWords:  notebook-collab jupyter-collab Colaboratory jupyter-live-collab Javadoc
%  LocalWords:  notebook.rst Knowls structdocs localtaskref Needs.rst CTypes Cython Nmag
%  LocalWords:  oommf-python-interface OOMMF-py-raw micromagnetic oommf-py magnetisation
%  LocalWords:  Micromagnetic-Standardproblem-3 oommf-py-ipython-attributes vispy taskref
%  LocalWords:  oommf-nb IPython-widget-GUI-demo-youtube-2014 oommf-nb-documentation Dedu
%  LocalWords:  oommf-tutorial-and-documentation modelling micromagnetics oommf-nb-ve mws
%  LocalWords:  TeMPorary oommf-nb-tmp oommf-nb-virtual Virtualbox Cloudhosted dissem
%  LocalWords:  dissemination-of-oommf-nb-virtual-environment wpdelivs wpdeliv Eugen tikz
%  LocalWords:  Ekiga Puydt gstreamer ekiga-devel ptlib compactitem refactorization numpy
%  LocalWords:  cfd-vis paraview ldots electromagnetics isosurfaces notebooksearch adstex
%  LocalWords:  cassearch simulagora Simulagora mathhub-editing adcomp nbad-search glumpy
%  LocalWords:  swsites visualisation introduction-micromagnetic-vre-demonstrator mathhub
%  LocalWords:  matplotlib jmol maximise virtualisation localdelivref refactorisation
%  LocalWords:  WPref dksbases Simular initialised
