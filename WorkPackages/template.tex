\begin{workpackage}[id=x,title=Template,whphases=1-48!.5,
,SARM=1]

\begin{wpobjectives}
% 5-10 lignes

The objectives of this work package are to:
\begin{itemize}
\item
\item
\item
\item
\item
\end{itemize}
\end{wpobjectives}

\begin{wpdescription}
  % Overall description, and description of 3-5 tasks
  % The description of each task can be 5 to 15 lines depending on the
  % complexity and amount of details deemed necessary.

This workpackage  ...
\end{wpdescription}

% Please see UserInterfaces.tex for now as an example

\begin{tasklist}
\begin{task}[title=Task title]
\end{task}

\begin{task}[title=Other task title]
\end{task}
\end{tasklist}

\begin{wpdelivs}
  \eucommentary{ Deliverable numbers in order of delivery
    dates. Please use the numbering convention <WP number>.<number of
    deliverable within that WP>.  For example, deliverable 4.2 would
    be the second deliverable from work package 4.

    Type:
    Use one of the following codes:
    R:
    Document, report (excluding the periodic and final reports)
    DEM: Demonstrator, pilot, prototype, plan designs
    DEC: Websites, patents filing, press \& media actions, videos, etc.
    OTHER: Software, technical diagram, etc.
    Dissemination level:
    Use one of the following codes:
    PU = Public, fully open, e.g. web
    CO = Confidential, restricted under conditions set out in Model Grant Agreement
    CI = Classified, information as referred to in Commission Decision 2001/844/EC.
    Delivery date
    Measured in months from the project start date (month 1)
  }
  \begin{wpdeliv}[due=<delivery date,type,PU>,id=xxx,dissem=??,nature=??]
      {One line deliverable description.}
\end{wpdeliv}
  \begin{wpdeliv}[due=<delivery date,type,PU>,id=yyy,dissem=??,nature=??]
      {One line deliverable description.}
\end{wpdeliv}
\end{wpdelivs}
\end{workpackage}
%%% Local Variables:
%%% mode: latex
%%% TeX-master: "../proposal"
%%% End:
