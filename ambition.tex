\subsection{Ambition}



\eucommentary{1-2 pages}

\eucommentary{-- Describe the advance your proposal would provide beyond the
state-of-the-art, and the extent the proposed work is ambitious. Your answer
could refer to the ground-breaking nature of the objectives, concepts
involved, issues and problems to be addressed, and approaches and methods to be used.\\
-- Describe the innovation potential which the proposal represents. Where relevant, refer to
products and services already available, e.g. in existing
e-Infrastructures.}

For most pure mathematicians using computational tools in their
research, the state of the art in the beginning of 2015 is still a collection of
programs each of which must be installed individually on their
desktop or laptop computer, respecting a complicated dependencies graph.
Alternatively software may be installed on a
departmental server or cluster and used via text-based remote
login. The software performs computations (using excellent
implementations of extremely sophisticated algorithms) with inputs and
outputs usually in a bespoke text-based format. 
Multiple computations involved in producing a mathematical
result must be managed by editing, naming and filing multiple scripts
or programs, and there is no automatic support for rerunning
computations to check for human or algorithmic error. The results of
computations are incorporated into publications by cut-and-paste and
collaboration is through exchange of programs and data by email,
shared general-purpose file servers or, rarely, a service such as
GitHub. Such situation creates a serious obstacle to the reproducibility
of computational experiments reported in research publications,
(even by their authors themselves after a duration of time). 
% see e.g. "Case Studies and Challenges in Reproducibility in the 
% Computational Sciences", http://arxiv.org/abs/1408.2123, submitted

There are commercial ``symbolic computation systems'' such as
\Mathematica or \Maple which offer somewhat more modern frameworks, but
they lack the specialised algorithms for research work in fields such
as algebra, number theory or algebraic geometry and are not
well-suited to support them. 
\TOWRITE{AK/MK}{This statement needs verification. It's not only the lack of algorithms
in these areas. Moreover, we want to cater for wider areas of mathematics}

The need for a more modern, more productive and less error-prone
environment for this kind of mathematical research computing is widely
acknowledged, but the separate groups developing the systems have
individually, neither the time nor the expertise to develop it. There
have been a number of interesting projects which have explored
different aspects of what is needed, in particular
\SMC (see \ref{linked-projects});
\HPCGAP (\TODO{now listed under St Andrews entry});
\scienceproject (\TODO{now listed under St Andrews entry});
\Sage and \Sage notebooks itself;
Polymath and MathOverflow (see MathSoMac entry in \ref{linked-projects});
and \software{Recomputation.org}.
We will build on the experiences, and where useful, on the software, of all of these.
\TOWRITE{AK/MK}{Cleanups in the list of projects}

Our ambitious plan in this project is to learn from, and leapfrog,
these piecemeal developments and provide a toolkit of software and
interfaces, which supports the whole mathematical research process in
a way which is \textbf{modern}, \textbf{seamless},
\textbf{collaborative}, mathematically \textbf{rigourous} and
\textbf{adaptable} to the diverse needs of different mathematical
research areas and of different mathematicians and collaborations.

\TOWRITE{AK/MK}{Explain all the highlighted words in the para above}

\TODO{Some examples here -- what will we deliver to individuals; small ``single-problem''
  collaborations; longer-lived data- or algorithm- centred teams;
  massive ``flagship'' projects. Think ``User Stories''. 
  AK: for example, a paragraph below.}
  
These will offer functionality for collaborative computational projects
of all possible scales. An individual researcher would use it for a new
experiment because it will be easier to start it in VRE than to not to;
a small multi-site team would set up a project easily shareable between
current and prospective collaborators and accessible from anywhere in the
world; providers of mathematical databases will be able to find a long-term
solution for offering mathematical data for the community maintaining
high-quality provenance and composability standards; Massive ``flagship''
algorithm-centred projects will increase their productivity from all
collaborative features that may be integrated under a single VRE.

\subsubsection{Challenges specific to  mathematics}

Mathematical research, especially pure mathematics, presents some
unique challenges to the realisation of this ambition.

\TOWRITE{AK/MK}{Evidence this in more detail, clean up the language}

\begin{itemize}
\item The community mainly comprises of individuals or \textit{very} small
  groups (say, a PI and a few students), having fewer formal or structured research
  groups such as you might find in an equipment-intensive science. There are 
  certainly examples of large scale collaborations happen (CoFSG, Polymath),
  but these are still driven by individuals, not by formal structures or funding bodies.
\item Many exemplars of top quality research have little or no formal research
  funding. In case they need computational resources, these are limited to what 
  is already available nearby, such as personal laptops or departmental clusters.
\item Many mathematical computations are highly complex and irregular. Thus,
  traditional HPC paradigms coming from numerical simulations and linear algebra do not apply.
\item Mathematical notations have been refined over many centuries to be
  used by humans with pen, paper and blackboard. Even such simple
  problems as selecting a sub-expression are hard to handle well on a
  computer. For instance $a+c$ is naturally seen as a subexpression of
  $a+b+c$ by a human.
\item The mathematical correctness of widely used algorithms hinges on
  quite complex chains of reasoning. Subtle coding errors may easily
  produce plausible, but wrong, answers.

\item Mathematical data differ in several ways from typical
  scientific data
  \begin{itemize}
  \item More often rather than not, data is the result of a computation (and
    not a measurement of the real world). The role of databases is thus primarily
    to store results for later search and reuse (persistent caching). 
    Because of this, many issues (semantics, ontologies,
    reproducibility) are to be treated upstream at the level of
    software rather than data.
  \item Extreme reification in mathematics makes classical ontologies
    techniques (such as e.g. RDF) impractical. \TODO{Someone explain this}
  \item Data are highly interlinked and based on a hierarchy of mathematical objects,
  which, in their turn, may have more than one (interchangeable) definition.
  \end{itemize}
\end{itemize}

\subsubsection{Challenges of a community built around multiple
  existing software projects}

Another source of unique challenges for this project is the need to
interact with several large and diverse ecosystems of software
developers. For instance the \GAP package development community, the
\Sage development community, the wider Python community, the developers
of key open-source libraries on which we rely and so on.

These communities exist in a delicate balance between collaboration
and competition. For instance the \scienceproject and \Sage were
simultaneously exploring two different approaches to linking
open-source mathematical software. Many technical developments (better
IO handling in \GAP, for instance) could usefully be shared, and at
the end of the day we all want to do better mathematics, but a certain
degree of competition is both natural and healthy.

In this project we need to build a sustainable ``meta-ecosystem'' in
which systems may compete to have the best designs or algorithms, but
all agree to cooperate on interfaces, bug reporting, testing, etc. to
keep the final user experience seamless and reliable.

\TODO{Promoting collaboration over competition between communities.}

\TOWRITE{All}{Describe innovation potential}
%%% Local Variables:
%%% mode: latex
%%% TeX-master: "proposal"
%%% End:

%  LocalWords:  eucommentary textsuperscript textregistered textsuperscript specialised
%  LocalWords:  textregistered recomputation textbf textbf rigourous centred flagshsip
%  LocalWords:  subsubsection realisation textit
