
The work in the \TheProject project is structured by four milestones,
which coincide with the four project meetings held at the end of each
year of the project (the other four meetings will be held in the middle
of each year). Given the nature of the project, with a
large number of essentially independent tasks, there is no need for
milestones attached to specific collections of tasks or
deliverables. Instead, given that the meetings are the main
face-to-face interaction points in the project, it's suitable to
schedule the milestones for these events, where they can be discussed
in detail, tracking the progress in each work package through status
reports on the tasks and deliverables.

We envisage that this setup will give the project the vital coherence
in spite of the broad interdisciplinary mix of various backgrounds of the
participants.

% \newcommand{\WPall}{\WPref{management}, \WPref{dissem}, \WPref{component-architecture}, \WPref{UI}, \WPref{hpc}, \WPref{dksbases}, \WPref{social-aspects}}

% \newcommand{\WPnoUI}{\WPref{management}, \WPref{dissem}, \WPref{component-architecture}, \WPref{hpc}, \WPref{dksbases}, \WPref{social-aspects}}

% \begin{center}
%   \begin{tabular}{|m{.05\textwidth}|m{.30\textwidth}|m{.15\textwidth}|m{.05\textwidth}|m{.22\textwidth}|}
%     \hline
%     Mile-stone nr. & Milestone name & Related work packages & Est. date & Means of verification \\\hline
%     M1 & Requirements study, design and prototype implementations. Start of
%          community building.
%        & \WPall 
%        & 12 
%        & 2nd Project meeting report. Completion of corresponding deliverables. \\\hline
%     M2 & First fully functional interface implementations.
%          Enhanced versions of \TheProject components.
%          Training early adopters.
%        & \WPall 
%        & 24 
%        & 4th Project meeting report. Completion of corresponding deliverables. \\\hline
%     M3 & Evaluating \TheProject software. Working with the community 
%          and building portfolio of experiments produced with \TheProject.
%        & \WPall 
%        & 36 
%        & 6th Project meeting report. Completion of corresponding deliverables. \\\hline
%     M4 & Project evaluation and final versions of all \TheProject components.
%        & \WPnoUI 
%        & 48 
%        & 8th Project meeting report. Completion of corresponding deliverables. \\\hline
%   \end{tabular}
% \end{center}

\begin{milestones}
  \milestone[id=startup,month=12,
  verif={Carrying out requirements study, design and prototype implementations. 
Start of community building. Completion of corresponding deliverables.
Reporting progress at the 2nd Project meeting.}]
  {Startup}
  {}

  \milestone[id=proto1,month=24,
  verif={Releases of first fully functional interface implementations and
training its early adopters. Releases of enhanced versions of \TheProject 
components. Completion of corresponding deliverables.
Reporting progress at the 4th Project meeting.}]
  {Prototypes}
  {}

  \milestone[id=community,month=36,
  verif={Collecting feedback and further improvement of \TheProject software. 
Expanding the community and establishing the portfolio of experiments 
produced with \TheProject.  Completion of corresponding deliverables.
Reporting progress at the 6th Project meeting.}]
  {Community/Experiments}
  {}

  \milestone[id=eval,month=48,
  verif={Project evaluation and release of final versions of all \TheProject
components. Completion of corresponding deliverables.
Reporting progress at the 8th Project meeting.}]
  {Evaluation}
  {}
\end{milestones}

%%% Local Variables:
%%% mode: latex
%%% TeX-master: "proposal"
%%% End:

%  LocalWords:  verif ldots
