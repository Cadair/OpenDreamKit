\begin{draft}
\section*{Outline of Project (for Proposers)}

\TODO{This is the place for various READMEs not included in the final submission}

\subsection*{Vision}

An internal attempt at specifying our vision through short
(unsubstantiated) answers.

\begin{verbatim}
> 1) Who are we?

Lead or core developers of some of the major open source components
for pure mathematics and applications:

- Computational components: GAP, Linbox, MPIR, Pari, Sage, Singular
- Databases: LMFDB (findstat as well)
- Knowledge management: MathHub

Together with, in a larger scientific domain, lead developers for:

- Collaborative user interfaces (IPython, SageMathCloud)
- Database and Scientific Computing for the industry (Logilab)
- Numerical code optimization/parallelisation (Pythran)

> 2) What is our goal?

Building blocks with a sustainable development model that can be
seamlessly combined together to build versatile high performance
VRE's, each tailored to a specific need in pure mathematics and
application.

> 2.5) What is our strategy?

Maximize sustainability and impact by reusing and improving existing
building blocks, and reaching toward larger communities whenever possible.
E.g. factoring out our common user interface needs at the level
of IPython/Jupyter will save us time (sustainability), and impact
the larger scientific computing community.
The improvements to the building blocks will impact all their users,
whether they use the VRE or not.

> 3) From where do we start?

- Building blocks with a sustainable development model
- Proof-of-concept prototypes of VRE (SMC, Simulagora)
- Experience on combining together some of the building blocks

> 4) How do we connect or differ from other projects?

The other projects focus on either one or a few of the building
blocks, or on a specific VRE.

We articulate our work with each of them.

> 5) Why are we excellent?

The consortium puts together recognized experts in all
areas and most building blocks that are relevant to the goal. There is
simultaneously a variety of point of views and a record of past
experiences collaborating together at smaller scale
(e.g. GAP-Singular). The approach is bottom up.  Most joint tasks
consist in bringing together people with a common need. There is
experience in community building.  Most participants are
simultaneously users and developers of their tools.

All of this makes me confident that we will indeed be able to
productively collaborate. And do stuff that is first class and useful.

On Sat, Dec 13, 2014 at 11:18:10PM +0100, Wolfram Decker wrote:
> 0) What precisely is our starting point and why are we the right people to
> achieve what we promise to do? Are we leaders in the area touched
> by the proposal? How do we connect? Is there some past
> collaborative success?
> 1) You still do not say what we actually will provide. What precisely will
> the VRE offer to its users?

I more or less answered those points above. Let me know if I should
elaborate.

> Who will be its users? Will those already familiar with the involved
> CAS use it? Will it make the CAS more attractive for a much larger
> community?

One objective is definitely to make CAS and others more attractive by
lowering a lot the entry barrier to access the soft (and db, ...). A
typical situation that most of us ran into is, when collaborating with
other less tech-savvy mathematicians, to have trouble sharing code,
data, and in-the-writing papers with them. SMC was launched with this
idea in mind, and the success proves the concept.

At the same time, the improvements in the building blocks will also
impact CAS users that are happy with their current user interface /
work-flow.

Improvements to IPython will impact a much larger community.

> 2) You motivate what we wish to do by the success of SageMathCloud.
> But why do we than need another VRE? How do we differ from
> SageMathCloud?

There is no one-size-fits-all VRE. One might want to run a VRE on
one's own computer resources for a variety of reason (speed of access,
specific resources, privacy, independence, ...). One might want a
different combination of software (e.g. a lightweight VRE with only
Singular).  One might want to focus on data with LMFDB-style database
searches, or on interactive computing, or on document writing, or some
combination thereof.

> Do we have a chance to compete? Or will we rather join forces? In
> which way?

We join forces (the plan is to have William/UW in the consortium, as
non funded participant). SMC focuses on one specific cloud based
VRE. We focus on the building blocks and the glue. Both project are
mutually beneficial. See the language p. 14 of the proposal.

> 3) You motivate what we wish to do by the success of LMFDB. But what
> are our connections to this database? Will we enhance it? Will we connect
> it to other stuff we do? Will we create other databases?

LMFDB is a prototype of large scale database. We want to make it
easier for other groups of mathematicians to set up similar databases
in their area. Reciprocally, like SMC, the LMFDB with benefit back
from the improved building blocks.

> 4) Why is Europe in the lead if there is already SageMathCloud?
> Where precisely is Europe in the lead?

Europe is the lead in many of the building blocks.
\end{verbatim}

% \subsection*{Mission statement for the grant}

% Our mission is to promote the next generation of community-developed
% open source software, databases, and services adapted to the needs of
% collaborative research in pure mathematics and applications.

% Our research will cover a wide variety of aspects, ranging from
% software development models, user interfaces \TODO{virtual
%   environments?}, deployment frameworks and novel collaborative tools,
% component architecture, design, and standardization of software
% \TODO{system?} and databases, to links to publication, data archival
% and reproducibility of experiments, development models and tools, and
% social aspects.

% It will consolidate Europe's leading position in computational
% mathematics and build on the remarkable success of the ecosystem of
% projects GAP, Python/Sage, Pari, Singular, LMFDB.

\subsection*{Description of the call}

\verbatiminput{call_description}

% \TODO{What do we mean by ``new generation''}.

\renewcommand{\thepage}{\arabic{page}}
\setcounter{page}{1}
\black
\cleardoublepage
\end{draft}

%%% Local Variables: 
%%% mode: latex
%%% TeX-master: "proposal"
%%% End: 

%  LocalWords:  verbatiminput renewcommand thepage setcounter cleardoublepage
