\TOWRITE{NT}{Rework this section}

\subsubsection{Travel, dissemination and outreach}
\label{sect:budget-details}

This community building nature of this grant proposal requires a large
number of staff exchanges, work shops with project partners and
workshops engaging the wider community in addition to the usual
management and project review meetings. For dissemination, we need to
target the computer science and focussed community and their conferences
as well as the domains benefitting from \TheProject, such as 
Mathematics and Science.

\subsubsection{Travel, dissemination}
\label{sect:budget-details-travel}

We use the following guidelines for expected travel expenses:
\euro{2200} for attendance of a typical one week international
conference outside Europe (including travel, subsistence,
accommodation and registration), \euro{1200} for a corresponding
conference in Europe, \euro{750} for a one-week visit to project
partners, for example for coding sprints and one-to-one visits for
particular research. We expect a similar cost per week when hosting
visitors. For the 6-monthly project meetings, we expect on average a
cost of \euro{400} for travel, accommodation and subsistence.

For a partner site with one investigator and one full time researcher,
we expect that that both will attend all of the 9 project meetings that take 
place every 6 months (cost of 9 * 2 * 400 =
\euro{7200}), and that the site spends \euro{2000} per year to host
external visitors contributing to the project (total \euro{8000}). We
expect the investigator and research together to do 4 one-week visits
to other sites (each at \euro{750}, totals \euro{12000}). 


For dissemination, we expect the researcher to attend on average 1
international conference and 2 European meetings per year (totals
\euro{18400}) and the investigator to attend one international and one
european gathering (totals \euro{13600}).

Where there are multiple investigators per site, they will share the
travel and associated costs outlined above. Where there are multiple
researchers, or researchers not employed for the full 48 months, the
travel budget is reduced accordingly.

\subsubsection{Outreach costs}
\label{sect:budget-outreach-publication-charges}
We also request \euro{1000} per year per partner to pay for open
access publication charges.

\label{sect:budget-outreach-workshops}
We request funds for outreach activities such as workshops that
facilitate community building, disseminate best practice and
encourages sustained contributions of the community to the project and
beyond the lifetime of the funding. For a one-week workshops reaching
out to the community, we cost these at about 400 EUR per participant
to provide accommodation and catering. A workshop for 15 people will
thus cost about \euro{6000}. Participants donate their time will fund
their own travel. The particular budgeted cost will depend on the
local availability of accommodation and will tus vary from workshop to
workshop. We use creative means to increase value and improve
community building where possible, for example by cooking food
ourselves as done in for this recent workshop
\href{http://wiki.sagemath.org/days57}{http://wiki.sagemath.org/days57}.

Details are given in the tables below and in the work packages.






\TODO{Give details in work packages on what workshops are planned} 

\TODO{List workshop expenses and travel expenses in tables below}.


\eucommentary{Will get help from Orsay's grant services}

\eucommentary{Please provide the following:
\begin{itemize}
\item
a table showing number of person/months required (table 3.4a)
\item
a table showing 'other direct costs' (table 3.4b) for participants where
those costs exceed 15\% of the personnel costs (according to the budget
table in section 3 of the administrative proposal forms)
\end{itemize}}

%\newpage

%\landscape

\fbox{\begin{minipage}{\textwidth}

\begin{center}\Large\bf
Summary of staff effort
\end{center}
\end{minipage}}

\eucommentary{Please indicate the number of person/months over the whole 
duration of the planned work, for each work package, for each participant. 
Identify the work-package leader for each WP by showing the relevant 
person-month figure in bold.}

\bigskip




\subsection{Resources }

%%%%%%%%%%%%%%%%%%%%%%%%%%%%%%%%%%%%%%%%%%%%%%%%%%%%%%%%%%%%%%%%
%
% Guidelines for completion of partner specific resource summary:
%
%
% Please explain how many person months for each person are
% requested. Say who is the local lead. Say anything that helps to
% understand why people are recruited as you plan, in particular if
% this deviates from having one research for 48 months.  We can also
% use this bit of the proposal (and the table, see below) to address
% any other unusual arrangements.
%
%
% The table should contain all non-staff costs (the EU requests that
% this table must be present if the non-staff costs exceed
% 15% of the total cost, but it is good practice and will show
% openness and transparency that we show the data for all partners).
%
% Link back from the table to the work packages and tasks for which
% the expenses are required. Add information that makes it easier to
% understand why the expenses are justified.
%
%     To refer to a task in a work package, use "\taskref{WP-ID}{TASK-ID}" where 
%     WP-ID is the ID of the work package:
%        WP#: WP-ID - full title
%        ----------------------
%        WP1: 'management' - Management
%        WP2: 'community' - Community Building and Engagement
%        WP3: 'component-architecture' - Component Architecture
%        WP4: 'UI' - User interfaces
%        WP5: 'hpc' - High Performance Computing
%        WP6: 'dksbases' - Data/Knowledge/Software-Bases
%        WP7: 'social-aspects' - Social Aspects
%        WP8: 'dissem' - Dissemination
%        
%     
%     and "TASK-ID" is the ID of the task. You can set this using
%     
%       \begin{task}[id=TASK-ID,title=Math Search Engine,lead=JU,PM=10,lead=JU]
%
%     To refer to deliverables, use "\delivref{WP-ID}{DELIV-ID}" where DELIV-ID is
%     the ID of the deliverable that can be set like this:
%     
%       \begin{wpdeliv}[due=36,id=DELIV-ID,dissem=PU,nature=DEM]
%           {Exploratory support for semantic-aware interactive widgets providing views on objects
%           represented and or in databases}
%       \end{wpdeliv}
%     
%
% The table is pre-populated with entries most sites are likely
% to need. If a line does not apply to you, just delete it. If you need
% an extra line, then add it. Use common sense: the number of rows should not 
% be very big, but at the same time it is useful to give some breakdown/explanation
% of costs.
% 
%
% Eventually, try to create you entry similar in style to the others.
% (The Southampton entry is fully populated, so use this as guidance
% if in doubt.)
%
%
%%%%%%%%%%%%%%%%%%%%%%%%%%%%%%%%%%%%%%%%%%%%%%%%%%%%%%%%%%%%%%%%

\subsubsection{Resources Universit\'{e} Paris Sud}

\TODO{UPS}{Please complete section and table. See ``Guidelines for completion of partner specific resource summary'' in latex comments in H2020/resources.tex.} 
% See guidance above ("Guidelies for completion of partner specific resource summary") for what to add here.

% update table as is appropriate for your contribution. Please remove unused lines.
\bigskip
\begin{table}[h!]
\begin{tabular}{|r|r|p{9cm}|}
\hline
\textbf{} & \textbf{Cost (\euro)} & \textbf{Justification} \\\hline
\textbf{Travel} & ?,??? & Travel (see \ref{sect:budget-details-travel})\\\hline
\textbf{Publication charges} & ?,??? & Open access publication charges (see \ref{sect:budget-outreach-publication-charges})\\\hline
\textbf{Equipment} & ?,??? &  \\\hline    %\taskref{WP-ID}{TASK-ID} 

\textbf{Other goods and services} & ?,??? & Workshops \\\hline   %\taskref{WP-ID}{TASK-ID} \delivref{WP-ID}{DELIV-ID}
\textbf{Audit} & ?,??? & Audit cost \\\hline
\textbf{Total} & ?,???\\\cline{1-2}
\end{tabular}
\caption{Overview: Non-staff resources to be committed at UPS (all in \texteuro)}\vspace*{-1em}
\end{table}



\subsubsection{Resources Logilab}

\TODO{Logilab}{XXX0} XXX1 Logilab XXX2

\subsubsection{Resources Universit\'{e} de Versailles Saint-Quentin}

\TODO{UVSQ}{XXX0} XXX1 UVSQ XXX2


\subsubsection{Resources Universit\'{e} Joseph Fourier}

\TODO{UJF}{XXX0} XXX1 UJF XXX2

\subsubsection{Resources Universit\'{e} Bordeaux}

\TODO{UB}{XXX0} XXX1 UB XXX2

\subsubsection{Resources University of Oxford}

\TODO{UO}{XXX0} XXX1 UO XXX2

\subsubsection{Resources University of Sheffield}

\TODO{USHEF}{XXX0} XXX1 USHEF XXX2 


\subsubsection{Resources Southampton}

Southampton requests 32 person months for a researcher (expected to start in month 4 of the project), 6 person months for the lead PI (Hans Fangohr) and 2 person months for the co investigator (Ian Hawke). The lead PI will take on all management responsibilities. The researcher will not be employed for the whole project duration, and the PIs will carry out all tasks for the project in the remaining period. 

\bigskip
\begin{table}[h!]
\begin{tabular}{|r|r|p{9cm}|}
\hline
\textbf{} & \textbf{Cost (\euro)} & \textbf{Justification} \\\hline
\textbf{Travel} & 51,500& Travel (see \ref{sect:budget-details-travel})\\\hline
\textbf{Publication charges} & 4,000 & Open access publication charges (see \ref{sect:budget-outreach-publication-charges})\\\hline
\textbf{Equipment} & 10,000 & HPC Workstation (6k) to host \OOMMFNB{} webserver, \taskref{UI}{oommf-nb-ve}, and two high performance laptops (2x2k)\\\hline
\textbf{Other goods and services} & 19,600 & 4 Dissemination workshops (travel \& room hire), \taskref{dissem}{dissemination-of-oommf-nb-workshops}\\\hline
\textbf{Audit} & 5,200 & Audit cost \\\hline
\textbf{Total} & 86,300\\\cline{1-2}
\end{tabular}
\caption{Overview: Non-staff resources to be committed at Southampton (all in \texteuro)}\label{tab:resources-non-staff-southampton}\vspace*{-1em}
\end{table}


%--------------

\subsubsection{Resources University of St Andrews}

\TODO{USTAN}{XXX0} XXX1 USTAN XXX2 

\subsubsection{Resources University of Warwick}

\TODO{UW}{XXX0} XXX1 UW XXX2 

\subsubsection{Resources Jacobs University Bremen}

\TODO{JacU}{XXX0} XXX1 JacU XXX2 

\subsubsection{Resources University of Kaiserslautern}

\TODO{UK}{XXX0} XXX1 UK XXX2 

\subsubsection{Resources University of Silesia}

\TODO{US}{XXX0} XXX1 US XXX2 

\subsubsection{Resources University of Z\"{u}rich}

\TODO{UZ}{XXX0} XXX1 UZ XXX2 

\subsubsection{Simula Research Labortaory}

\TODO{Simula}{XXX0} XXX1 Simula XXX2 

\subsubsection{University of Washington at Seattle}

\TODO{Simula}{UWS} XXX1 UWS XXX2 














\subsection{Resources partner Summary}

\begin{table}[ht]\centering
  \TODO{Table 3.4.a: insert here table from Figure 3, and transpose; see
    Table 3.4.a in the word template}
  \TODO{The work package leader will usually have the largest investment}
\caption{Overview: Resources to be committed (all in \texteuro)}\label{tab:resources}\vspace*{-1em}
\end{table}
\fbox{\begin{minipage}{\textwidth}

\eucommentary{Please complete the table below for each participant if the sum of the costs for’ travel’, ‘equipment’,
and ‘goods and services’ exceeds 15% of the personnel costs for that participant (according to the
budget table in section 3 of the proposal administrative forms).}

\begin{center}\Large\bf
Other direct cost items
\end{center}
\end{minipage}}

\bigskip

\begin{tabular}{|r|l|p{9cm}|}
\hline
\textbf{} & \textbf{Cost (\euro)} & \textbf{Justification} \\\hline
\textbf{Travel} & & \\\hline
\textbf{Equipment} & & \\\hline
\textbf{Other goods and services} & & \\\hline
\textbf{Total} & \\\cline{1-2}
\end{tabular}




%%% Local Variables:
%%% mode: latex
%%% TeX-master: "proposal"
%%% End:

%  LocalWords:  newpage fbox minipage textwidth eucommentary bigskip rr rr rr hline
%  LocalWords:  vspace texteuro textbf textbf textbf
